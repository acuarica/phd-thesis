\section{The Risks of Compromising Safety}
\label{sec:unsafe:background}

We outline the risks of \unsafe{} by illustrating how the improper use of
\unsafe{} violates \java{}'s safety guarantees.

In \java{}, the unsafe capabilities are provided as instance methods of
class \smu{}.
Access to the class has been made less than straightforward.
Class \smu{} is final, and its constructor is not public.
Thus, creating an instance requires some tricks.
For example, one can invoke the private constructor via reflection.
This is not the only way to get hold of an unsafe object,
but it is the most portable.

\begin{listing}
\begin{minted}{java}
Constructor<Unsafe> c = Unsafe.class.getDeclaredConstructor();
c.setAccessible(true);
Unsafe unsafe = c.newInstance();
\end{minted}
\caption{Instantiating an Unsafe object}
\end{listing}
 
Given the unsafe object, one can now simply invoke any of its methods to
directly perform unsafe operations.

\subsection*{Violating Type Safety}

In \java{}, variables are strongly typed.
For example, it is impossible to store an int value inside a variable of
a reference type.
\unsafe{} can violate that guarantee:
it can be used to store a value of any type in a field or array element.

\begin{listing}
\begin{minted}{java}
class C {
  private Object f = new Object();
}
long fieldOffset = unsafe.objectFieldOffset(
        C.class.getDeclaredField("f") );
C o = new C();
unsafe.putInt(o, fieldOffset, 1234567890);      // f now points to nirvana
\end{minted}
\caption{\smu{} can violate type safety}
\end{listing}

\subsection*{Crashing the Virtual Machine}

A quick way to crash the VM is to free memory that is in a protected
address range, for example by calling \code{freeMemory} as follows.

\begin{listing}
\begin{minted}[linenos=false]{java}
unsafe.freeMemory(1);
\end{minted}
\caption{\smu{} can crash the VM}
\end{listing}

In \java{}, the normal behavior of a method to deal with such situations
is to throw an exception.
Being unsafe, instead of throwing an exception,
this invocation of \code{freeMemory} crashes the VM.

\subsection*{Violating Method Contracts}

In \java{}, a method that does not declare an exception cannot throw any
checked exceptions.
\unsafe{} can violate that contract:
it can be used to throw a checked exception that the surrounding method
does not declare or catch.

\begin{listing}
\begin{minted}{java}
void m() {
  unsafe.throwException(new Exception());
}
\end{minted}
\caption{\smu{} can violate a method contract}
\end{listing}

\subsection*{Uninitialized Objects}

\java{} guarantees that an object allocation also initializes the object
by running its constructor.
\unsafe{} can violate that guarantee:
it can be used to allocate an object without ever running its
constructor.
This can lead to objects in states that the objects' classes would
not seem to admit.

\begin{listing}
\begin{minted}{java}
class C {
  private int f;
  public C() { f = 5; }
  public int getF() { return f; }
}

C c = (C)unsafe.allocateInstance(C.class);
assert c.getF()==5; // violated
\end{minted}
\caption{\smu{} can lead to uninitialized objects}
\end{listing}

\subsection*{Monitor Deadlock}

\java{} provides synchronized methods and synchronized blocks.
These constructs guarantee that monitors entered at the beginning
of a section of code are exited at the end.
\unsafe{} can violate that contract:
it can be used to asymmetrically enter or exit a monitor,
and that asymmetry might be not immediately obvious.

\begin{listing}
\begin{minted}{java}
void m() {
  unsafe.monitorEnter(o);
  if (c) return;
  unsafe.monitorExit(o);
}
\end{minted}
\caption{\smu{} can lead to monitor deadlocks}
\end{listing}

The above examples are just the most straightforward violations of
\java{}'s safety guarantees.
The \smu{} class provides a multitude of methods that can be used
to violate most guarantees \java{} provides.

To sum it up: \unsafe{} is dangerous.
But should anybody care?
In the next sections we present a study to determine whether and how
\unsafe{} is used in real-world third-party \java{} libraries,
and to what degree real-world applications directly and indirectly
depend on it.