\section{Is Unsafe Used?}
\label{sec:unsafe:overview}

To answer~\ref{unsafe:rq1} (\emph{\urqA})
we need to determine whether and how Unsafe is actually used in real-world third-party \java{} libraries,
and to what degree real-world applications directly and indirectly depend on such unsafe libraries.
To achieve our goal, several elements are needed.

\textbf{Code Repository.}
As a code base representative of the ``real world'',
we have chosen the \mavencentral{} software repository.
The rationale behind this decision is that a large number of well-known \java{} projects deploy to \mavencentral{} using Apache Maven.
Besides code written in \java{},
projects written in \lang{Scala} are also deployed to \mavencentral{} using the \scala{} Build Tool (sbt).
Moreover, \mavencentral{} is the largest \java{} repository,%
\footnote{\url{http://www.modulecounts.com/}}
and it contains projects from the most popular source code management repositories, like \github{} and \sourceforge{}.

\textbf{Artifacts.}
In Maven, an artifact is the output of the build procedure of a project.
An artifact can be any type of file, ranging from a \emph{.pdf} to a \emph{.zip} file.
However, Artifacts are usually \emph{.jar} files,
which archive compiled \java{} bytecode stored in \emph{.class} files.

\textbf{Bytecode Analysis.}
We examine these kinds of artifacts to analyse how they use \code{sun.misc.Unsafe}.
We use a bytecode analysis library to search for method call sites and field accesses of the \code{sun.misc.Unsafe} class~\citep{brunetonRecursiveDynamicSoftware2002,brunetonASMCodeManipulation2002,kuleshovUsingASMFramework2007}.

However, our first attempt was to use \jnif{},
our own bytecode analysis library~\citep{mastrangeloJNIFJavaNative2014}.
\jnif{} is described in Appendix~\ref{ap:jnif}.
Due to its own limitations,
we decided to use the aforementioned analysis library.

\textbf{Dependency Analysis.}
We define the impact of an artifact as how many artifacts depend on it,
either directly or indirectly.
This helps us to define the impact of artifacts that use \code{sun.misc.Unsafe},
and thus the impact \code{sun.misc.Unsafe} has on real-world code overall.

\textbf{Usage Pattern Detection.}
After all call sites and field accesses are found,
we analyse this information to discover usage patterns.
It is common that an artifact exhibits more than one pattern.
Our list of patterns is not exhaustive.
We have manually investigated the source code of the 100 highest-impact artifacts using \code{sun.misc.Unsafe} to understand why and how they are using it.

Our analysis found $48,490$ uses of \code{sun.misc.Unsafe}---$48,139$ call sites and $351$ field accesses---distributed over $817$ different artifacts.
This initial result shows that Unsafe is indeed used in third-party code.

We use the dependency information to determine the impact of the artifacts that use \code{sun.misc.Unsafe}.
We rank all artifacts according to their impact (the number of artifacts that directly or indirectly depend on them).
High-impact artifacts are important;
a safety violation in them can affect any artifact that directly or indirectly depends on them.
We find that while overall about $1\%$ of artifacts directly use Unsafe,
for the top-ranked $1000$ artifacts, $3\%$ directly use Unsafe.
Thus, Unsafe usage is particularly prevalent in high-impact artifacts, artifacts that can affect many other artifacts.

Moreover, we found that $21,297$ artifacts ($47\%$ of the $47,127$ artifacts with dependency information, or $25\%$ of the $86,479$ artifacts we downloaded) directly or indirectly depend on \code{sun.misc.Unsafe}.
Excluding language artifacts, numbers do not change much:
Instead of $21,297$ artifacts, we found $19,173$ artifacts,
$41\%$ of the artifacts with dependency information, or $22\%$ of artifacts downloaded.
Thus, \code{sun.misc.Unsafe} usage in third-party code indeed impacts a large fraction of projects.


\subsection*{Which Features of \unsafe{} Are Actually Used?}

\begin{figure}[p]
\centering
\includegraphics[width=0.54\columnwidth]{chapters/unsafe/usage-maven-methods}
\caption{\smu{} method usage on \mavencentral{}}
\label{fig:overview}
\end{figure}

\begin{figure}[!ht]
\centering
\includegraphics[width=0.7\columnwidth]{chapters/unsafe/usage-maven-fields}
\caption{\smu{} field usage on \mavencentral{}}
\label{fig:overview-field}
\end{figure}

Figures~\ref{fig:overview} and~\ref{fig:overview-field} show all instance methods and static fields of the \smu{} class.
For each member we show how many call sites or field accesses we found across the artifacts. The class provides $120$ public instance methods and $20$ public fields (version 1.8 update 40). The figure only shows $93$ methods because the $18$ methods in the \smugroup{Heap Get} and \smugroup{Heap Put} groups, and \member{staticFieldBase} are overloaded, and we combine overloaded methods into one bar.

We show two columns, \smugroup{Application} and \smugroup{Language}.
The \smugroup{Language} column corresponds to language implementation artifacts while the \smugroup{Application} column corresponds to the rest of the artifacts.

We categorized the members into groups, based on the functionality they provide:

\begin{itemize}
  
\item The \smugroup{Alloc} group contains only the  \member{allocateInstance} method, 
which allows the developer to allocate a \java{} object without executing a constructor.
This method is used 181 times: 180 in \smugroup{Application} and 1 in \smugroup{Language}.

\item The \smugroup{Array} group contains methods and fields for computing relative addresses of array elements.
The fields were added as a simpler and potentially faster alternative in a more recent version of \unsafe{}.
The value of all fields in this group are constants initialized with the result of a call to either \member{arrayBaseOffset} or \member{arrayIndexScale} in the \smugroup{Array} group.
The figures show that the majority of sites still invoke the methods instead of accessing the corresponding constant fields.

\item The \smugroup{CAS} group contains methods to atomically compare-and-swap a \java{} variable.
These methods are implemented using processor-specific atomic instructions.
For instance, on \emph{x86} architectures,
\member{compareAndSwapInt} is implemented using the \texttt{CMPXCHG} machine instruction.
Figure~\ref{fig:overview} shows that these methods represent the most heavily used feature of \unsafe{}.

\item Methods of the \smugroup{Class} group are used to dynamically load and check \java{} classes. They are rarely used, with \member{defineClass} being used the most.

\item 
The methods of the \smugroup{Fence} group provide memory fences to ensure loads and stores are visible to other threads.
These methods are implemented using processor-specific instructions.
These methods were introduced only recently in \java{} 8, which explains their limited use in our data set.
We expect that their use will increase over time and that other operations, such as those in the \smugroup{Ordered Put}, or \smugroup{Volatile Put} groups will decrease as programmers use the lower-level fence operations.

\item The \smugroup{Fetch \& Add} group, like the \smugroup{CAS} group, allows the programmer to atomically update a \java{} variable. This group of methods was also added recently in \java{} 8. We expect their use to increase as programmers replace some calls to methods in the \smugroup{CAS} group with the new functionality.

\item The \smugroup{Heap} group methods are used to directly access memory in the \java{} heap. The \smugroup{Heap Get} and \smugroup{Heap Put} groups allow the developer to load and store a Java variable. These groups are among the most frequently used ones in \unsafe{}.

\item The \smugroup{Misc} group contains the method \member{getLoadAverage}, to get the load average in the operating system run queue assigned to the available processors. It is not used.

\item The \smugroup{Monitor} group contains methods to explicitly manage \java{} monitors.
The \member{tryMonitorEnter} method is never used.

\item The \smugroup{Off-Heap} group members provide access to unmanaged memory,
enabling explicit memory management.
Similarly to the \smugroup{Heap Get} and \smugroup{Heap Put} groups,
the \smugroup{Off-Heap Get} and \smugroup{Off-Heap Put} groups allow the developer to load and store values in Off-Heap memory.
The usage of these methods is non-negligible,
with \member{getByte} and \member{putByte} dominating the rest.
The value of the \member{ADDRESS\_SIZE} field is the result of the method \member{addressSize()}.

\item Methods of the \smugroup{Offset} group are used to compute the location of fields within \java{} objects.
The offsets are used in calls to many other \smu{} methods,
for instance those in the \smugroup{Heap Get}, \smugroup{Heap Put}, and the \smugroup{CAS} groups.
The method \member{objectFieldOffset} is the most called method in \smu{} due to its result being used by many other \smu{} methods.
The \member{fieldOffset} method is deprecated,
and indeed, we found no uses.
The \member{INVALID\_FIELD\_OFFSET} field indicates an invalid field offset;
it is never used because code using \member{objectFieldOffset} is not written in a defensive style
(given that \unsafe{} is used when performance matters,
and extra checks might negatively affect performance).

\item The \smugroup{Ordered Put} group has methods to store to a \java{} variable without emitting any memory barrier but guaranteeing no reordering across the store.

\item The \member{park} and \member{unpark} methods are contained in the \smugroup{Park} group. With them, it is possible to block and unblock a thread's execution.

\item The \member{throwException} method is contained in the \smugroup{Throw} group, and allows one to throw checked exceptions without declaring them in the \texttt{throws} clause.

\item Finally, the \smugroup{Volatile Get} and \smugroup{Volatile Put} groups allow the developer to store a value in a \java{} variable with \texttt{volatile} semantics.

\end{itemize}

It is interesting to note that despite our large corpus of code,
there are several \unsafe{} methods that are never actually called.
If \unsafe{} is to be used in third-party code,
then it might make sense to extract those methods into a separate class to be only used from within the runtime library.

\subsection*{Beyond Maven Central}

While \mavencentral{} is a large repository,
we wanted to check whether our results generalize to other common repositories.
thus performed a similar analysis of method usage using the \boa{}~\cite{dyerBoaLanguageInfrastructure2013,dyerMiningSourceCode2013} infrastructure.
\boa{} allows the developer to mine ASTs of \java{} projects in \sourceforge{}.

The usage profile of \unsafe{} methods we obtained from \boa{} was similar in shape,
but at a different scale, compared to the one obtained from \mavencentral{}.
Using \boa{}'s \sourceforge{} dataset, for instance,
the most called method, \code{objectFieldOffset}, 
is called 200 times in $50$ projects.
This is two orders of magnitude lower than the count we found on \mavencentral{}.
Although \boa{} enables the mining of source code in a convenient way, 
the \code{} data it analyses probably is not current enough to include the more recent \java{} code that uses \smu{} more heavily.

In recent versions~\citep{boa-github},
\boa{} added support to conduct studies on open source projects from \github{} and Qualitas Corpus~\citep{temperoQualitasCorpusCurated2010}.
However, at the time we conducted our study on \unsafe{},
this support was not included yet.
