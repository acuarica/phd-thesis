\section{Is Unsafe Used?}\label{sec:unsafe:overview}

To answer~\ref{unsafe:rq1} (\emph{\urqA})
we need to determine whether and how Unsafe is actually used in real-world third-party \java{} libraries,
and to what degree real-world applications directly and indirectly depend on such unsafe libraries.
To achieve our goal, several elements are needed.

\textbf{Code Repository.}
As a code base representative of the ``real world'',
we have chosen the Maven Central software repository.
% The rationale behind this decision is that a large number of well-known \java{} projects deploy to Maven Central using Apache Maven.
% Besides code written in \java{}, projects written in \lang{Scala} are also deployed to Maven Central using the Scala Build Tool (sbt).
% Moreover, Maven Central is the largest \java{} repository\footnote{\url{http://www.modulecounts.com/}}
% , and it contains projects from the most popular source code management repositories, like \github{} and \sourceforge{}.

\textbf{Artifacts.}
In Maven, an artifact is the output of the build procedure of a project.
% An artifact can be any type of file, ranging from a \emph{.pdf} to a \emph{.zip} file.
% However,
Artifacts are usually \emph{.jar} files,
which archive compiled \java{} bytecode stored in \emph{.class} files.

\textbf{Bytecode Analysis.}
% We examine these kinds of artifacts to analyze how they use \code{sun.misc.\-Unsafe}.
We use a bytecode analysis library to search for method call sites and field accesses of the \code{sun.misc.Unsafe} class.

\textbf{Dependency Analysis.}
We define the impact of an artifact as how many artifacts depend on it,
either directly or indirectly.
This helps us to define the impact of artifacts that use \code{sun.misc.Unsafe},
and thus the impact \code{sun.misc.Unsafe} has on real-world code overall.

% \textbf{Usage Pattern Detection.}
% After all call sites and field accesses are found,
% we analyze this information to discover usage patterns.
% It is common that an artifact exhibits more than one pattern.
% Our list of patterns is not exhaustive.
% We have manually investigated the source code of the 100 highest-impact artifacts using \code{sun.misc.Unsafe} to understand why and how they are using it.

Our analysis found $48,490$ uses of \code{sun.misc.Unsafe} --- $48,139$ call sites and $351$ field accesses --- distributed over $817$ different artifacts.
This initial result shows that Unsafe is indeed used in third-party code.

We use the dependency information to determine the impact of the artifacts that use \code{sun.misc.Unsafe}.
We rank all artifacts according to their impact (the number of artifacts that directly or indirectly depend on them).
High-impact artifacts are important;
a safety violation in them can affect any artifact that directly or indirectly depends on them.
We find that while overall about $1\%$ of artifacts directly use Unsafe,
for the top-ranked $1000$ artifacts, $3\%$ directly use Unsafe.
Thus, Unsafe usage is particularly prevalent in high-impact artifacts, artifacts that can affect many other artifacts.

Moreover, we found that $21,297$ artifacts ($47\%$ of the $47,127$ artifacts with dependency information, or $25\%$ of the $86,479$ artifacts we downloaded) directly or indirectly depend on \code{sun.misc.Unsafe}.
Excluding language artifacts, numbers do not change much:
Instead of $21,297$ artifacts, we found $19,173$ artifacts,
$41\%$ of the artifacts with dependency information, or $22\%$ of artifacts downloaded.
Thus, \code{sun.misc.Unsafe} usage in third-party code indeed impacts a large fraction of projects.
