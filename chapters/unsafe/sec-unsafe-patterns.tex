
%\newcommand{\artitem}[3]{ \art{##1\}{##2\}{\} }

\newcommand{\sep}{

}
%\newcommand{\artitemurl}[4]{#1\footnote{#2}}
\newcommand{\artitem}[3]{\artexp{#2}{#3}}
%\newcommand{\artitemurl}[4]{\art{#2}{#3}{#4}}
\newcommand{\artitemurl}[4]{\artexp{#2}{#3}}
%\newcommand{\artitemurl}[4]{ ##1 ##2 ##3 \textbf{##4} }

\newenvironment{artlist}{}{}


\newcommand{\allocmost}{
\begin{artlist}
\artitemurl{springframework}{org.springframework}{spring-core}{http://projects.spring.io/spring-framework/}\sep{}
\artitemurl{objenesis}{org.objenesis}{objenesis}{http://objenesis.org/}\sep{}

\artitemurl{mockito}{org.mockito}{mockito-all}{https://github.com/mockito/mockito}
\end{artlist}
}

\newcommand{\probytemost}{
\begin{artlist}
\artitemurl{guava}{com.google.guava}{guava}{https://github.com/google/guava}\sep{}
\artitemurl{gwt-dev}{com.google.gwt}{gwt-dev}{http://www.gwtproject.org/}\sep{}

\artitemurl{lz4}{net.jpountz.lz4}{lz4}{https://code.google.com/p/lz4/}
\end{artlist}
}

\newcommand{\lockfreemost}{
\begin{artlist}
\artitemurl{scala-lang}{org.scala-lang}{scala-library}{http://scala-lang.org/}\sep{}
\artitemurl{apache-hadoop}{org.apache.hadoop}{hadoop-hdfs}{https://hadoop.apache.org/}\sep{}
\artitemurl{glassfish}{org.glassfish.grizzly}{grizzly-framework}{https://grizzly.java.net/}
\end{artlist}
}

\newcommand{\fencemost}{
\begin{artlist}
\artitem{scala-lang}{org.scala-lang}{scala-library}\sep{}

\artitemurl{jruby}{org.jruby}{jruby-core}{http://jruby.org/}\sep{}

\artitemurl{hazelcast}{com.hazelcast}{hazelcast-all}{http://hazelcast.com/}
\end{artlist}
}

\newcommand{\parkmost}{
\begin{artlist}
\artitem{scala-lang}{org.scala-lang}{scala-library}\sep{}

\artitemurl{jsr166}{org.codehaus.jsr166-mirror}{jsr166y}{http://xircles.codehaus.org/projects/jsr166-mirror}\sep{}

\artitemurl{netflix-servo}{com.netflix.servo}{servo-internal}{https://github.com/Netflix/servo}
\end{artlist}
}

\newcommand{\finalfieldmost}{
\begin{artlist}
\artitemurl{groovy}{org.codehaus.groovy}{groovy-all}{http://groovy-lang.org/}\sep{}
\artitemurl{jodd}{org.jodd}{jodd-core}{http://jodd.org/}\sep{}

\artitemurl{disruptor}{com.lmax}{disruptor}{http://lmax-exchange.github.io/disruptor/}
\end{artlist}
}

\newcommand{\monitormost}{
\begin{artlist}
\artitemurl{jboss}{org.jboss.modules}{jboss-modules}{http://www.jboss.org/}\sep{}
\artitemurl{cassandra}{org.apache.cassandra}{cassandra-all}{http://cassandra.apache.org/}\sep{}
\artitemurl{gridgain}{org.gridgain}{gridgain-core}{http://www.gridgain.com/}
\end{artlist}
}

\newcommand{\serializationmost}{
\begin{artlist}
\artitem{hazelcast}{com.hazelcast}{hazelcast-all}{}\sep{}
\artitemurl{kryo}{com.esotericsoftware.kryo}{kryo}{https://github.com/EsotericSoftware/kryo}\sep{}
\artitemurl{xstream}{com.thoughtworks.xstream}{xstream}{http://xstream.codehaus.org/}
\end{artlist}
}

\newcommand{\marshallingmost}{
\begin{artlist}
\artitemurl{eu.stratosphere}{eu.stratosphere}{stratosphere-core}{http://stratosphere.eu/}\sep{}

\artitemurl{jnr}{com.github.jnr}{jffi}{https://github.com/jnr/jffi}\sep{}

\artitemurl{jython}{org.python}{jython}{http://www.jython.org/}
\end{artlist}
}

\newcommand{\throwmost}{
\begin{artlist}
\artitemurl{io.netty}{io.netty}{netty-all}{http://netty.io/}\sep{}

\artitemurl{openhft}{net.openhft}{lang}{https://github.com/OpenHFT/Java-Lang}\sep{}

\artitemurl{ai.h20}{ai.h2o}{h2o-core}{https://github.com/h2oai/h2o-dev}
\end{artlist}
}

\newcommand{\sizemost}{
\begin{artlist}
\artitemurl{ehcache}{net.sf.ehcache}{ehcache}{http://ehcache.org/}\sep{}

\artitemurl{jamm}{com.github.jbellis}{jamm}{https://github.com/jbellis/jamm}\sep{}

\artitemurl{openjdk.jol}{org.openjdk.jol}{jol-core}{http://openjdk.java.net/projects/code-tools/jol/}
\end{artlist}
}

\newcommand{\largearraysmost}{
\begin{artlist}
\artitemurl{neo4j}{org.neo4j}{neo4j-primitive-collections}{http://neo4j.com/}\sep{}

\artitemurl{orientdb}{com.orientechnologies}{orientdb-core}{https://github.com/orientechnologies/orientdb}\sep{}

\artitemurl{mapdb}{org.mapdb}{mapdb}{http://www.mapdb.org/}
\end{artlist}
}

\newcommand{\pagemost}{
\begin{artlist}
\artitem{apache-hadoop}{org.apache.hadoop}{hadoop-common}{}\sep{}

\artitem{openhft}{net.openhft}{lang}\sep{}

\artitemurl{xerial}{org.xerial.larray}{larray-mmap}{https://github.com/xerial/larray}
\end{artlist}
}

\newcommand{\classmost}{
\begin{artlist}
\artitemurl{elasticserach}{org.elasticsearch}{elasticsearch}{https://github.com/elastic/elasticsearch}\sep{}

\artitemurl{mvel2}{org.apache.geronimo.ext.openejb}{core}{http://geronimo.apache.org/}\sep{}

\artitem{openhft}{net.openhft}{lang}
\end{artlist}
}


\newcommand{\javaclass}[1]{\emph{#1}}

\newcommand{\patternrow}[1]{
  \expandafter\newcommand\csname row#1\endcsname{\csname foundin#1\endcsname & \csname usedby#1\endcsname & \csname mostused#1\endcsname}
}

\newcommand{\patterntext}[6]{
    \expandafter\newcommand\csname desc#1\endcsname{#2}
    \expandafter\newcommand\csname alt#1\endcsname{#3}
    \expandafter\newcommand\csname impl#1\endcsname{#4}
    \expandafter\newcommand\csname rationale#1\endcsname{#5}
    \expandafter\newcommand\csname issues#1\endcsname{#6}
    \patternrow{#1}
}

\newcommand{\patternsection}[1]{

\expandafter\subsection{\csname name#1\endcsname}
\expandafter\label{sec:#1}

\noindent \textbf{\em Description.} \expandafter\csname desc#1\endcsname
%\smallskip

\noindent \textbf{\em Rationale.} \expandafter\csname rationale#1\endcsname
%\smallskip

\noindent \textbf{\em Implementation.} \expandafter\csname impl#1\endcsname
%\smallskip

\noindent \textbf{\em Issues.} \expandafter\csname issues#1\endcsname
%\smallskip

}

\newcommand\foundinalloc{88}
\newcommand\usedbyalloc{14794}
\newcommand\mostusedalloc{\allocmost}
\newcommand\membersalloc{\member{allocate\-Instance}}
\newcommand\namealloc{Allocate an Object without Invoking a Constructor}

\patterntext{alloc}%
{With this pattern an object can be allocated on the heap
without executing its constructor.}
{}
{The \member{allocate\-Instance} method takes a \code{java.lang.Class} object as parameter,
and returns a new instance of that class.
Unlike allocating an
object directly, or through the reflection \api{}, the object's constructor is not invoked.}
{This pattern is useful for creating mock objects for testing and in deserializing
serialized objects.}
{If the constructor is not invoked, the object might be left uninitialized
  and its invariants might not hold.
  Users of \member{allocate\-Instance} must take care to properly
  initialize the object before it is used by other code. This is often done in conjunction with other methods
  of \unsafe{}, for instance those in the \smugroup{Heap Put} group, or by using
the \java{} reflection API.
}

\newcommand\foundinprobyte{44}
\newcommand\usedbyprobyte{12274}
\newcommand\mostusedprobyte{\probytemost}
\newcommand\membersprobyte{\member{array\-Base\-Offset}, \member{getLong}, and optionally \member{array\-Index\-Scale} (to assert that the size of byte is equal to 1)}
\newcommand\nameprobyte{Process Byte Arrays in Block}

\patterntext{probyte}%
{When processing the elements of a byte array, better performance
can be achieved by processing the elements 8 bytes at a time, treating it as a
long array, rather than one byte at a time.}
{The JVM's runtime compiler can be extended with optimizations for vectorizing
byte array accesses.}
{The \member{array\-Base\-Offset} method is invoked to get the base offset of the byte array.
Then \member{getLong} is used to fetch and process 8 bytes of the array at a time.}
{The pattern is used for fast byte array processing, for instance,
when comparing two byte arrays lexicographically.}
{The pattern assumes that bytes in an array are stored contiguously. This may
  not be true for some VMs, \eg{} those implementing large arrays using
  discontinuous arrays or
  arraylets~\cite{siebertEliminatingExternalFragmentation2000,baconRealtimeGarbageCollector2003}. Users of
the pattern should be aware of the endianness of the underlying hardware.
In one \stackoverflow{} discussion, this pattern is
discouraged since it is non-portable and, on many JVMs, results in slower
code.\footnote{\url{http://stackoverflow.com/questions/12226123}} }

\newcommand\foundinlockfree{84}
\newcommand\usedbylockfree{10259}
\newcommand\mostusedlockfree{\lockfreemost}
\newcommand\memberslockfree{Either \member{object\-Field\-Offset} or \member{array\-Base\-Offset} in conjunction
  with \member{array\-Index\-Scale}, plus methods of the \smugroup{CAS} group
or the \smugroup{Fetch \& Add} group.
}
\newcommand\namelockfree{Atomic Operations}


\patterntext{lockfree}%
{
  This pattern is used to implement
  non-blocking concurrent data structures and synchronization primitives.
  Hardware-specific atomic operations
  provided by \smu{} are used.
%Some of them include following classes developed by Cliff Click, author of
  %the HotSpot Server Compiler: \javaclass{ConcurrentAutoTable}, \javaclass{NonBlockingHashMap}.
%\javaclass{org.vmmagic}~\cite{Frampton:2009:DMH:1508293.1508305}.
}
{The Java standard library provides classes for some concurrent data structures.
The library also provides classes
(\javaclass{Atomic\-Field\-Reference\-Updater}, \javaclass{AtomicIntegerArray}, etc.)
for safely performing atomic operations on fields and array elements, as well
as several synchronizer classes. These
can be used instead of the \unsafe{} atomic operations.}
{To get the offset of a \java{} variable either \member{object\-Field\-Offset} or
  \member{array\-Base\-Offset}/\member{array\-Index\-Scale} can be used.
With this offset, the methods from the \smugroup{CAS} or \smugroup{Fetch \&
Add} groups are used to perform atomic operations on the variable.
Other methods of \unsafe{} are often used in the implementation of concurrent
data structures, including \smugroup{Volatile Get/Put}, \smugroup{Ordered Put}, and \smugroup{Fence} methods.
}
{Non-blocking algorithms often scale better than algorithms that use locking.}
{Non-blocking algorithms can be difficult to implement correctly. Programmers
must understand the \java{} memory model and how the \unsafe{} methods interact
with the memory model.}

\newcommand\foundinfence{198}
\newcommand\usedbyfence{9795}
\newcommand\mostusedfence{\fencemost}
\newcommand\membersfence{Methods of the \smugroup{Fence} group, or methods of
the \smugroup{Get/Put Volatile} groups or \smugroup{Put Ordered} group}
\newcommand\namefence{Strongly Consistent Shared Variables}


\patterntext{fence}%
{Because of \java{}'s weak memory
  model, when implementing concurrent code,
  it is often necessary to ensure that
  writes to a shared variable by one thread become visible to other threads,
  or to prevent
  reordering of loads and stores.
  Volatile variables can be used for this purpose, but
  \smu{} can be used instead with better performance.
  Additionally, because \java{} does not allow array elements to be declared volatile,
  there is no possibility other than to use \unsafe{} to ensure visibility of
  array stores. The methods of the \smugroup{Ordered Put} groups
  and the \smugroup{Volatile Get/Put} groups can be used for these purposes.
  In addition, the \smugroup{Fence} methods were introduced in \java{} 8 expressly
to provide greater flexibility for this use case.}
{Memory fence operations can be added to the standard library. The language
can be changed to make volatile variables more flexible.}
{To ensure a write is visible to another thread, \smugroup{Volatile Put}
  methods or \smugroup{Ordered Put} methods can be used, even on non-volatile variables.
  Alternatively, a \member{storeFence} or \member{fullFence} can be used.
  \smugroup{Volatile Get} methods ensure other loads and stores are not reordered
  across the load. A \member{loadFence} could also be used before a read of a
  shared variable.
}{This pattern is useful for implementing concurrent algorithms or shared
  variables in concurrent settings. For instance, JRuby uses a \member{fullFence}
  to ensure visibility of writes to object fields.
}{Fences can replace volatile variables in some situations, offering better
  performance. Most of the uses of the pattern use the \smugroup{Ordered Put}
  and \smugroup{Volatile Put} methods. Since they were added to Java only recently, there are currently few instances
of the pattern that use the \smugroup{Fence} methods.}

\newcommand\foundinpark{62}
\newcommand\usedbypark{7330}
\newcommand\mostusedpark{\parkmost}
\newcommand\memberspark{\member{park}, \member{unpark}}
\newcommand\namepark{Park/Unpark Threads}


\patterntext{park}%
{
To implement locks and other blocking synchronization constructs,
the \member{park} and \member{unpark} methods are used.
With these methods, the developer can block and unblock threads.
}
{The standard library class \javaclass{java.\-util.\-concurrent.\-locks.\-LockSupport} provides \member{park} and \member{unpark}
methods to be used for implementing locks.}
{The \member{park} method blocks the current thread while \member{unpark}
unblocks a thread given as an argument.}
{The alternative to parking a thread is to busy-wait, which uses CPU
resources and does not allow other threads to proceed.}{Users of
  \member{park} must be careful to avoid deadlock.}

\newcommand\foundinfinalfield{11}
\newcommand\usedbyfinalfield{7281}
\newcommand\mostusedfinalfield{\finalfieldmost}
\newcommand\membersfinalfield{\member{object\-Field\-Offset}; and, at least one method of the \smugroup{Heap Put} or \smugroup{Put Volatile} groups.}
\newcommand\namefinalfield{Update Final Fields}

\patterntext{finalfield}%
{This pattern is used to update a final field.  }
{The reflection API can be used to implement the same functionality.}
{The \member{object\-Field\-Offset} methods and one of the \smugroup{Put} methods work in conjunction to directly modify the memory where a final field resides.}
{Although it is possible to use reflection to implement the same behavior,
  updating a final field is easier and more efficient using \smu{}.
  Some applications update final fields when cloning objects
  or when deserializing objects.
}{
  There are numerous security and safety issues with modifying final
  fields. The update should be done only on newly created objects
  (perhaps also using \member{allocate\-Instance} to avoid
  invoking the constructor) before the object becomes visible to
  other threads. The \java{} Language
  Specification (Section~17.5.3)~\cite{Gosling:2013:JLS:2462622}
  recommends that final fields not be read until all updates are
  complete. In addition, the language permits compiler optimizations
  with final fields that can prevent updates to the field from being
  observed.
  Since final fields can be cached by other threads, one instance of
  the pattern uses \member{putObject\-Volatile} to update the field rather than
  simply \member{putObject}.
  Using this method ensures that any cached copy in other threads
  is invalidated.}

\newcommand\foundinmonitor{14}
\newcommand\usedbymonitor{7015}
\newcommand\mostusedmonitor{\monitormost}
\newcommand\membersmonitor{\member{monitor\-Enter}, \member{monitor\-Exit}}
\newcommand\namemonitor{Non-Lexically-Scoped Monitors}

\patterntext{monitor}%
{In this pattern, monitors are explicitly acquired and released without using
\texttt{synchronized} blocks.}
{One can extend the language to support non-lexically-scoped
monitors.}
{One usage of the pattern is to temporarily release monitor locks acquired
  in client code (e.g., through a synchronized block or method) and
  then to reenter the monitor before returning
  to the client. The \member{monitor\-Exit} method is used to exit the
  synchronized block. Because monitors are reentrant, the pattern
  uses the method \javaclass{Thread.\-holds\-Lock} to implement a loop
  that repeatedly exits the monitor
  until the lock is no longer held. When reentering the monitor,
  \member{monitor\-Enter} is called the same number of times
  as \member{monitor\-Exit} was called to release the lock.
}{The pattern is used in some situations to avoid deadlock, releasing a monitor
temporarily, then reacquiring it.}{Care must be taken to balance calls to
\member{monitor\-Enter} and \member{monitor\-Exit}, or else the lock might
not be released or an \texttt{Illegal\-Monitor\-State\-Exception} might be
thrown.}

\newcommand\foundinserialization{32}
\newcommand\usedbyserialization{5689}
\newcommand\mostusedserialization{\serializationmost}
\newcommand\membersserialization{\member{object\-Field\-Offset} and methods of the \smugroup{Heap Get} and \smugroup{Heap Put} groups}
\newcommand\nameserialization{Serialization/Deserialization}

\patterntext{serialization}%
{In this pattern, \smu{} is used to persist and subsequently load objects to and from secondary memory dynamically.
Serialization in \java{} is so important that it has a \javaclass{Serializable} interface to automatically serialize objects that implement it.
Although this kind of serialization is easy to use, it does not offer good
performance and is inflexible.
It is possible to implement serialization using the reflection API. This is
also expensive in terms of performance. Therefore, fast serialization frameworks
often use \unsafe{} to get and set fields of objects.
% Project \project{jon} however invoke the \member{allocate\-Instance} method.
% These projects use one of the following classes:
% \javaclass{com.lts.xmlser.SerializationUtils},
% \javaclass{Sun14ReflectionProvider}, or
% \javaclass{JSX.magic.MagicClass14}, which are third-party classes that implement fast serialization.
Some of these projects use reflection to check if \smu{} is available, falling
back on a slower implementation if not.}
{Reflection can be used for accessing fields, more safely although less
  efficiently. Java's supports serialization of objects using
  \javaclass{java.io.Object\-Output\-Stream} and related classes. These serialization features could be extended with support for
user-defined serialization formats.}
{Methods of \smugroup{Heap Get} and \smugroup{Heap Put} are used to read and
  write fields and array elements. Deserialization may use
\member{allocate\-Instance} to create objects without invoking the constructor.}
{De/serialization requires reading and writing fields to save and restore
objects. Some of these fields may be final or private.}{
Using \unsafe{} for serialization and deserialization has many of the same issues
as using \unsafe{} for updating final fields (Section~\ref{sec:finalfield}) and for creating objects without
invoking a constructor (Section~\ref{sec:alloc}). Objects must not escape before being completely
deserialized. Type safety can be violated by using methods of the
\smugroup{Heap Put} group. In addition, care must be taken when deserializing
some data structures. For instance, data structures that use
\javaclass{System.identity\-Hash\-Code} or \javaclass{Object.hash\-Code} may need to rehash objects on
deserialization because the deserialized object might have a different
hash code than the original serialized object.}

\newcommand\foundinmarshalling{8}
\newcommand\usedbymarshalling{3690}
\newcommand\mostusedmarshalling{\marshallingmost}
\newcommand\membersmarshalling{Methods of the \smugroup{Off-Heap} and \smugroup{Off-Heap Get/Put} groups}
\newcommand\namemarshalling{Foreign Data Access and Object Marshaling}

\patterntext{marshalling}%
{In this pattern \smu{} is used to share data between Java code and code
  written in another language, usually C or C++.
% The \project{osfree} and \project{snarej} projects are peculiar because they use only \smugroup{Get} and \smugroup{Put} methods without calling \smugroup{Offset}.
% This is explained by the fact that these kinds of projects are implementing
% operating systems on top of the JVM.
% They obtain the memory address for the \smugroup{Get} and \smugroup{Put} methods using JNI.
% The \project{janux} project implements a native loader that needs to load class files from native memory.
% The \project{javapayload} project is about running exploits in pure \java{}, it has the ability to load a native library and execute an exploit.
% Stdlib: bytebuffer, datainputstream
}
{\javaclass{java.nio.Byte\-Buffer} and related classes can be used for marshaling data instead of \unsafe{}.}
{The methods of the \smugroup{Off-Heap} group are used to access memory off
the Java heap. Often a buffer is allocated using \member{allocate\-Memory},
which is then passed to the other language using JNI. Alternatively, the
native code can allocate a buffer in a JNI method. The \smugroup{Off-Heap Get}
and \smugroup{Off-Heap Put} methods are used to access the buffer.}{This
pattern is needed to efficiently pass data, especially structures and arrays, back and forth between Java and
native code. Using this pattern can be more efficient than using native
methods and JNI.}{Use of \unsafe{} here is inherently not type-safe. Care must be
  taken especially with native pointers, which are represented as
\texttt{long} values in Java code.}

\newcommand\foundinthrow{59}
\newcommand\usedbythrow{3566}
\newcommand\mostusedthrow{\throwmost}
\newcommand\membersthrow{\member{throw\-Exception}}
\newcommand\namethrow{Throw Checked Exceptions without Being Declared}

\patterntext{throw}%
{This pattern allows the programmer to throw checked exceptions without being
declared in the method's \texttt{throws} clause.}
{The issue can be avoided by not requiring \texttt{throws} declarations at
  all. Indeed, there is a long-running
  debate\footnote{\url{http://www.ibm.com/developerworks/library/j-jtp05254/}} about the
  software-engineering benefits of checked exceptions.
C\#, for instance, does not require that exceptions be declared in method signatures at all.
One alternative proposed in a \stackoverflow
discussion is to use Java generics
instead.\footnote{\url{http://stackoverflow.com/questions/11410042}} Because
of type erasure, a checked exception can be coerced unsafely into an unchecked exception and thrown.
}
{This pattern is implemented using the \member{throw\-Exception} method.}
{In testing and mocking frameworks, the pattern is used to circumvent declaring
  the exception to be thrown, which is often unknown.
It is used in the \java{} Fork/Join framework to save the generic exception of a thread to be re-thrown later.}
{This method can violate \java{}'s subtyping
  relation, because it is not expected
  for a method that does not declare an exception to actually throw it. At
  run time, this can manifest as an uncaught exception.}

\newcommand\foundinsize{4}
\newcommand\usedbysize{3003}
\newcommand\mostusedsize{\sizemost}
\newcommand\memberssize{\member{array\-Base\-Offset}, \member{array\-Index\-Scale},
\member{object\-Field\-Offset}}
\newcommand\namesize{Get the Size of an Object or an Array}

\patterntext{size}%
{This pattern uses \smu{} to estimate the size of an object or an array in
memory.}
{A sizeof feature could be introduced into the language or
into the standard library to make the implementation portable.}
{To compute the size of an array, add \member{array\-Base\-Offset} and
\member{array\-Index\-Scale} (for the given array base type) times the array
length. For objects, use \member{object\-Field\-Offset} to compute the offset
of the last instance field. In both cases, a VM-dependent fudge factor is
added to account for the object header and for object alignment and padding.}{The object
size can be useful for making manual memory management decisions. For
instance, when implementing a cache, object sizes can be used to implement
code to limit the cache size.}{Object
size is very implementation dependent. Accounting for the object header and
alignment requires adding VM-dependent constants for these parameters.}

\newcommand\foundinlargearrays{12}
\newcommand\usedbylargearrays{487}
\newcommand\mostusedlargearrays{\largearraysmost}
\newcommand\memberslargearrays{\member{allocate\-Memory}, \member{free\-Memory}, \member{setMemory}, \member{getInt}, \member{getLong}, \member{putInt}, \member{putLong}}
\newcommand\namelargearrays{Large Arrays and Off-Heap Data Structures}

\patterntext{largearrays}%
{This pattern uses off-heap memory to create large arrays or data structures with manual memory management.}
{This functionality could be provided with a language feature or library.}
{A block of memory is allocated with \member{allocate\-Memory} and then
accessed using \smugroup{Off-Heap Get} and \smugroup{Off-Heap Put} methods.
The block is freed with \member{free\-Memory}.}
{Java's arrays are indexed by \texttt{int} and are thus limited to $2^{31}$
elements. Using \unsafe{}, larger buffers can be allocated outside the heap.}{This pattern has all the issues of manual memory
management: memory leaks, dangling pointers, double free, etc.
One issue, mentioned on
\stackoverflow, is that the memory returned by \member{allocate\-Memory} is
uninitialized and may contain
garbage.\footnote{\url{http://stackoverflow.com/questions/16723244}}
Therefore, care must be taken to initialize allocated memory before use.
The \unsafe{} method \member{set\-Memory} can be used for this purpose.
}

\newcommand\foundinpage{11}
\newcommand\usedbypage{359}
\newcommand\mostusedpage{\pagemost}
\newcommand\memberspage{\member{pageSize}}
\newcommand\namepage{Get Memory Page Size}


\patterntext{page}%
{\smu{} is used to determine the size of a page in memory.}
{This functionality could be added to the standard library, perhaps in
the \javaclass{java.nio} package.}
{Call \member{pageSize}.}{The page size is needed to allocate buffers or
  access memory by page.
  A common use case is to round up a buffer size,
  typically a \javaclass{java.nio.ByteBuffer}, to the
  nearest page size. Hadoop uses the page size to track memory usage of
  cache files mapped directly into memory
using \javaclass{java.nio.MappedByteBuffer}. Another use is to process a
buffer page-by-page.
Some native libraries require or recommend allocating buffers on page-size
boundaries.\footnote{\url{http://stackoverflow.com/questions/19047584}}
}{Some platforms on which the JVM runs do not have
  virtual memory, so requesting the page size is non-portable.
}

\newcommand\foundinclass{21}
\newcommand\usedbyclass{294}
\newcommand\mostusedclass{\classmost}
\newcommand\membersclass{\member{define\-Class}}
\newcommand\nameclass{Load Class without Security Checks}

\patterntext{class}%
{\smu{} is used to load a class from an array containing its bytecode. Unlike
with the \javaclass{ClassLoader} API, security checks are not performed.}
{This feature could be added to the standard library, with a
  \javaclass{SecurityManager} used to explicitly relax the Java security model.}
{The pattern is implemented using the \member{define\-Class} method, which
takes a byte array containing the bytecode of the class to load.}{This pattern is useful for implementing lambdas, dynamic
class generation, and dynamic class rewriting. It is also useful in application
frameworks that do not interact well with user-defined class loaders.}{The
pattern violates the Java security model. Untrusted code could be introduced
into the same protection domain as trusted code.}

\section{Usage Patterns of \smu{}}
\label{sec:unsafe:patterns}

This section presents the patterns we have found during our study.
We present them sorted by how many artifacts depend on them, as computed from
the Maven dependency graph described in Section~\ref{sec:unsafe:overview}.

A summary of the patterns is shown in Table~\ref{table:unsafe:patterns}.
The \textbf{\em Pattern} column indicates the name of the pattern.
\textbf{\em Found in}
indicates the number of artifacts in \mavencentral{} that contain the pattern.
\textbf{\em Used by}
indicates the number of artifacts that transitively depend on the artifacts with the
pattern.
\textbf{\em Most used artifacts}
presents the three most used artifacts containing the pattern, that is, the 
artifact with the most other artifacts that transitively depend upon it.
Artifacts are shown using their Maven identifier, \ie{}
\artexp{$\langle$groupId$\rangle$}{$\langle$artifactId$\rangle$}.

\begin{table*}[t!]
\scriptsize
\centering
\caption{Patterns and their occurrences in the Maven Central repository}
\label{table:unsafe:patterns}
\begin{tabularx}{\linewidth}{rp{5.0cm}ccX}
\hdr    & \textbf{Pattern}              & \textbf{Found In} & \textbf{Used by} & \textbf{Most used artifacts} \\
\alt  1 & \namealloc                    & \rowalloc             \\
\row  2 & \nameprobyte                  & \rowprobyte           \\
\alt  3 & \namelockfree                 & \rowlockfree          \\
\row  4 & \namefence                    & \rowfence             \\
\alt  5 & \namepark                     & \rowpark              \\
\row  6 & \namefinalfield               & \rowfinalfield        \\
\alt  7 & \namemonitor                  & \rowmonitor           \\
\row  8 & \nameserialization            & \rowserialization     \\
\alt  9 & \namemarshalling              & \rowmarshalling       \\
\row 10 & \namethrow                    & \rowthrow             \\
\alt 11 & \namesize                     & \rowsize              \\
\row 12 & \namelargearrays              & \rowlargearrays       \\
\alt 13 & \namepage                     & \rowpage              \\
\row 14 & \nameclass                    & \rowclass             \\
\hline
\end{tabularx}
\end{table*}

We present each pattern using the following template.
\medskip

{

\noindent \textbf{\em Description.}
{What is the purpose of the pattern? What does it do?}
%\smallskip

\noindent \textbf{\em Rationale.}
{What problem is the pattern trying to solve? In what contexts is it used?}
%\smallskip

% \noindent \textbf{\em Unsafe members.}
% {Methods and fields of \smu{} used in the pattern.
% In some cases, there are multiple alternatives to implementing the pattern.
% These are explained when necessary in turn for each pattern. }
% %\smallskip

\noindent \textbf{\em Implementation.}
{How is the pattern implemented using \smu{}?}
%\smallskip

% \noindent \textbf{\em Found in.}
% {Number of artifacts in \mavencentral{} that contain the pattern.}
% %\smallskip

% \noindent \textbf{\em Used by.}
% {Number of artifacts that transitively depend on the artifacts with the
% pattern.}
% %\smallskip

% \noindent \textbf{\em Most used artifacts.}
% {The three most used artifacts containing the pattern, that is, the
% artifact with the most other artifacts that transitively depend upon it.
% Artifacts are shown using their Maven identifier, \ie{}
% \artexp{$\langle$groupId$\rangle$}{$\langle$artifactId$\rangle$}.
% %\smallskip

\noindent \textbf{\em Issues.}
{Issues to consider when using the pattern.
In addition, we present the problems discussed in the
\stackoverflow{} question/answer database based on our previous work~\citep{mastrangeloUseYourOwn2015}.}
%\smallskip

% \noindent \textbf{\em Alternatives.}
% {Potential alternatives to using the \unsafe{} for solving the same
% problem.}


}



\patternsection{alloc}
\patternsection{probyte}
\patternsection{lockfree}
\patternsection{fence}
\patternsection{park}
\patternsection{finalfield}
\patternsection{monitor}
\patternsection{serialization}
\patternsection{marshalling}
\patternsection{throw}
\patternsection{size}
\patternsection{largearrays}
\patternsection{page}
\patternsection{class}
