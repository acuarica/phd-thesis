\section{Conclusions}
\label{sec:jnif-conclusions}

Until now,
full-coverage dynamic instrumentation in production \jvm{}s required performing the code rewriting in a separate \jvm{}, 
because of the lack of a native bytecode rewriting framework.
This paper introduces \jnif{}, the first full-coverage in-process dynamic instrumentation framework for \java{}.
It discusses the key issues of creating such a framework for \java{}---such as stack-map generation---and
it evaluates the performance of \jnif{} against the most prevalent \java{}-level framework: \asm{}.
We find that \jnif{} is faster than using out-of-process \asm{} in most cases.
We hope that thanks to \jnif{}, and this paper, a broader number of researchers and developers will
be enabled to develop native \jvm{} agents that analyse and rewrite \java{} bytecode without limitations.
