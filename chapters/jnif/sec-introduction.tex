\section{Introduction}
\label{sec:jnif-introduction}

Program analysis tools are important in software engineering tasks 
such as comprehension, verification and validation, profiling, debugging, and optimization.
They can be broadly categorized either as static or dynamic,
based on the input that they take.
Static analysis tools carry out their task using as input only a program
in a given representation, \eg{}, source code, abstract syntax tree,
bytecode, or binary code.
In contrast, dynamic analysis tools observe the program being analyzed 
by collecting runtime information.
Many dynamic analysis tools rely on instrumentation to achieve their goals.

In the context of the JVM, 
static analysis and instrumentation for dynamic analysis often happens on the level of Java bytecode.
Analysis tools thus need to decode and analyze---and in the case of instrumentation also edit and encode---Java bytecode.
Given the relative complexity of the Java class file format,
a diverse set of libraries (see Section~\ref{sec:jnif-relatedwork}) has been created for this purpose.
All those libraries are implemented in Java.

Instrumentation at bytecode level can be done in two ways:
using a \java{} instrumentation agent or using a native \jvmti{} agent.%
\footnote{\url{http://docs.oracle.com/javase/7/docs/technotes/guides/jvmti/index.html}}
A Java instrumentation agent is written in Java and runs in the same JVM as the application.
This leads to two main problems: poor isolation and poor coverage.
It provides \emph{poor isolation} because to instrument the VM, 
the agent's classes must be loaded in the same VM,
and this can lead to perturbation in the VM.
It provides \emph{poor coverage} because 
an instrumentation agent (implemented in Java) will require some runtime library classes to be loaded before it can start instrumenting,
and those runtime classes thus cannot be instrumented at load time. 

A native \jvmti{} agent can instrument every class that the VM loads, including runtime classes. 
The main issue when using \jvmti{} is that instrumentation must be done in a native language, usually C or C++. 
Using C/C++ as the instrumentation language can be problematic, 
because of the lack of a C/C++ library for Java bytecode rewriting. 
Therefore developers have been using an extra JVM as an ``instrumentation server''
in which they could use Java-based bytecode rewriting libraries.
The C/C++ \jvmti{} agent thus only has to send code to the server,
and no native bytecode rewriting library is needed.
However, this approach has a drawback: it requires an additional \jvm{}, 
and it causes IPC traffic between the observed JVM and the instrumentation server.

We created \jnif{} to overcome this problem. 
To the best of our knowledge, \jnif{} is the first native Java bytecode rewriting library.
\jnif{} is a C++ library for decoding, analyzing, editing, and encoding Java bytecode.
The main benefit of \jnif{} is that it can be plugged into a \jvmti{} agent 
for instrumenting all classes in a JVM transparently, i.e., 
without connecting to another \jvm{} and without perturbing the observed \jvm{}.

Starting with \java{} 6, class files can include stack maps to
simplify bytecode verification for the \jvm{}.
\java{} 7 made those stack maps mandatory.
Thus, unless one wants to disable the \jvm{}'s verifier,
code rewriting tools need to also generate stack maps.
Stack maps contain, for each basic block,
type information for each local variable and operand stack slot.
To generate stack maps, a bytecode rewriting tool needs to perform a static analysis.
Due to the fact that bytecode does not contain type declarations of variable slots and local variables,
these types have to be inferred using an intra-procedural data flow analysis.
For reference types, 
computing the least upper bound of two types in a join point of a control flow graph 
even requires access to the class hierarchy of the program.
Thus, the seemingly innocuous requirement for stack maps
significantly complicates the creation of a bytecode rewriting library.
\jnif{} solves these issues, also thanks to the fact that it can be
used in-process in a \jvmti{} agent,
and thus can determine the necessary subtyping relationships 
by requesting the bytes of arbitrary classes loaded or loadable at any given point in time.
This works for classes loaded via user-defined class loaders as well as
for classes generated dynamically on-the-fly.

Overall, the main contributions of this paper are:

\begin{itemize}
	\item We present \jnif{}, a C++ library for decoding, analyzing, editing, and encoding Java class files.
	\item \jnif{} includes a data-flow analysis for stack map generation, 
	      a complication necessary for any library that provides editing and encoding support for modern \jvm{}s with split-time verification.
	\item We evaluate \jnif{} by comparing its performance against the most prevalent Java bytecode rewriting library, ASM.
\end{itemize}

The rest of this chapter is organized as follows:
Section~\ref{sec:jnif-relatedwork} presents related work.
In Section~\ref{sec:jnif-usage} we show how to use the \jnif{} API.
Section~\ref{sec:jnif-implementation} describes the design of \jnif{}.
Section~\ref{sec:jnif-test} explains how we validated \jnif{}.
Section~\ref{sec:jnif-evaluation} evaluates \jnif{}'s performance against the mainstream bytecode manipulator, ASM.
Section~\ref{sec:jnif-limitations} discusses limitations, and 
Section~\ref{sec:jnif-conclusions} concludes.