\section{Limitations}\label{sec:jnif-limitations}

\jnif{} still has some limitations.

\textbf{jsr/ret.} \jnif{} does not support stack map generation for \code{jsr} and \code{ret}.
Class files requiring stack maps do not include \code{jsr}/\code{ret}.
The \code{jsr}/\code{ret} instructions make the control flow graph generation difficult, because a \code{ret} instruction can jump to multiple targets instead of a predefined one.

\textbf{invokedynamic.} \jnif{}'s support for \code{invokedynamic} is not yet fully tested
but our initial tests with JRuby have been successful.
Dacapo bach was released in 2009,
before the creation of \java{} 7,
which introduces the \code{invoke\-dynamic} instruction.
Thus it does not contain any benchmark with \code{invoke\-dynamic}.
Instead we use JRuby 1.7 in order to create a self-contained jar file. 
This jar file does not contain any \code{invoke\-dynamic} instruction, 
but it does contain the JRuby compiler, 
that when specified via \code{-Djruby.compile.invoke\-dynamic=true} it will generate class files with \code{invokedynamic}. 
We tested our parser and writer with this settings with successful results.

\textbf{Stack map generation with full coverage.}
When the \jvm{} loads the first few runtime library classes,
and calls the \jvmti{} agent to have those classes instrumented,
it is still too early to use \jni{} for loading classes needed for computing least upper bounds for stack map generation.
For this reason, we do not generate stack maps for runtime library classes.
This no problem, because the \jvm{} does not verify the runtime library classes by default,
and thus it does not need stack maps for those classes.
However, should developers decide to explicitly turn on the verification of runtime library classes
(with \code{-Xverify:all}), the verifier would complain because \jnif{} would not have generated stack maps.

To get full coverage for the instrumentation inside a \jvmti{} agent, 
it is necessary to instrument every class, 
even the whole java class library.
If the instrumentation needs to change or add branch targets, 
the compute frames option must be used, 
but it cannot be used against the class library,
because to compute frames, 
the class hierarchy must be known,
and this imposes a dependency with a classloader which is not yet available.

Luckily, by default the \java{} library classes are not verified,
because they are trusted. 
Thus the instrumentation only needs to compute frames on classes not belonging to the \java{} library.
