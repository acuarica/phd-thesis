\section{\jnif{} Design and Implementation}
\label{sec:jnif-implementation}

\jnif{} is written in C++11~\citep{ISO:2012:III}, 
in an object-oriented style similar to \java{}-based class rewriting \api{}s.

\subsection{Design}

\jnif{} consists of five main modules: model, parser, writer, printer, and analysis.
%
\emph{Model} implements \jnif{}'s intermediate representation.
It is centered around its \code{ClassFile} class. 
It is possible to create and manipulate class files from scratch.
%
\emph{Parser} implements the parsing of class files from a given byte array.
The parser parses a byte array and translates it to the model's IR.
Once a \texttt{ClassFile} is created by the parser,
it can be manipulated with the methods available in the model.
%
\emph{Writer} and \emph{printer} represent two back-ends for the model.
%
\emph{Writer} serializes the entire \texttt{ClassFile} into a byte array ready to be loaded inside a JVM.
%
\emph{Printer} instead serializes the \texttt{ClassFile} into a textual format useful for debugging.
%
Finally, \emph{analysis} implements the static analyses necessary for computing stack maps.


\subsubsection*{JVM-Independence}

\jnif{} is a stand-alone C++ library that can be used outside a JVM.
It does not depend on \jvmti{} or JNI.
However, for the purpose of stack map generation,
it may need to determine the common super class of two classes.
For this it will need to retrieve a class file given the name of an arbitrary class. 
This functionality is provided by a plugin that implements \jnif{}'s \texttt{IClassPath}.
\jnif{} comes with such a plugin that uses JNI in case it is running inside a JVM.


\subsubsection*{Explicit Constant Pool Management}

Unlike some other class rewriting libraries, \jnif{} exposes the constant pool instead of hiding it.
Our reasons for this design decision were two-fold:
(1) We wanted to fully control the structure of the class file, 
and for that it is necessary to expose the constant pool. 
To reduce the additional complexity, we provide a rich set of methods that facilitate constant pool management. 
(2) We wanted to preserve, whenever possible, the original structure of the class file. 
This means that if one parses and then writes a class file, 
the original bytes will be obtained. 
This decreases the perturbation done by the instrumentation and allows for better testing.


\subsubsection*{Memory Management}

Given that \jnif{} is implemented in an unmanaged language,
we have to worry about memory deallocation.
Our API follows a simple ownership model
where all IR objects are owned by their enclosing objects.
This means, that the \texttt{ClassFile} object owns the complete IR of a class.
Our API design enforces this ownership model
by requiring IR objects to be created by their enclosing objects.
For example, to create a \texttt{Method}, 
one has to use the \texttt{ClassFile::addMethod()} factory method
instead of directly allocating a new \texttt{Method} object.


\subsection{Stack Map Generation}

When encoding a \texttt{ClassFile} into a byte array, \jnif{} needs to generate stack maps.
The necessary static analyses are implemented in the analyzer module.
This module uses data flow analysis and abstract interpretation to determine the types of operand stack slots and local variables.
The analysis module first builds a control flow graph of the method.
The data flow analysis associates to each basic block an input and output stack frame, 
which represents the types of the local variables and operand stack slots at that point in the code.
The input frame represents the type before any instruction in the basic block is executed.
The output frame is computed by abstract interpretation of each instruction in the basic block.
The entry basic block has an empty stack and each entry in its local variable table is set to top.
Then the algorithm starts from the entry block and follows each edge.
If a basic block is reachable from multiple edges, then a merge is involved.

Merging involves finding the least upper bound of multiple incoming types.
While this is trivial for primitive types,
it can require access to the class hierarchy for reference types.
This requirement represents a severe complication for binary rewriting tools:
when rewriting a single class, they may require access to many other classes in the program.
\jnif{} solves this problem by providing the \texttt{IClassPath} interface.
Different \texttt{IClassPath} implementations can provide different ways for getting access to classes.
For example, a static instrumentation tool may use a user-defined class path to find classes,
while a dynamic instrumentation tool may use JNI to request the bytes of a class given that class' name.

%\todo{Problem of finding common super class. 
%How this issue is broken in ASM, 
%because it will cause a lot of class loading and hence static initializer to be run in an order that was not the one assumed by the programmer.}

\subsection{Running \jnif{} Inside a \jvmti{} Agent}
When using \jnif{} inside a \jvmti{} agent, 
\jnif{} uses an \texttt{IClassPath} implementation
that uses \jni{} to load the bytes of classes
required for least upper bound computations during stack map generation.

\subsubsection*{Avoiding Premature Static Initialization}
Using JNI to load a class (with \texttt{ClassLoader.loadClass()}), however, will call that class' static initializer.
This is a side effect that may change the observable behavior of the program under analysis.
To avoid this, one can request the bytes of the class (with \texttt{ClassLoader.getResourceAsStream())} instead of loading the class.
It can then parse the bytes of the class into its IR to determine that class' supertypes.

\subsubsection*{Avoiding the Loading of the Class Being Instrumented}
If during the instrumentation of a class \code{X} \jnif{} needs to
perform a least upper bound computation involving type \code{X},
then using \code{ClassLoader.loadClass()} to load class \code{X} would
cause an infinite recursion.
The above solution with \code{get\-ResourceAsStream()} also prevents this problem.

\subsubsection*{Avoiding Premature ClassNotFoundException}
If during the instrumentation of a class \texttt{X}
\jnif{} needs to perform a least upper bound computation involving a type \texttt{Y},
and if class \texttt{Y} cannot be found,
then throwing a \texttt{ClassNotFoundException} at that time would be premature
(because without instrumentation, such an exception would only be thrown later).
We solve that problem by assuming a least upper bound of \verb=java.lang.Object= in that case.



%\subsection{Features}
%
%The \jnif{} also provides a convenient way to output dot files from generated control flow graph in order to better understand and analyze complex methods.
%
%Architecture and design picture.?
%
%\subsection{Portability?}
%
%The library uses only C++11 and STL constructions, thus there are no dependencies with any third-party library. Currently the \jnif{} compiles and runs on Linux and MacOS.
%
%
%\begin{lstlisting}[caption=Dots,label=usage-parse2]
%> make dots
%\end{lstlisting}
%