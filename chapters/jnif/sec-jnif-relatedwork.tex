\section{Related Work}\label{sec:jnif-relatedwork}

We now discuss low-level \java{} bytecode rewriting libraries,
JVM hooks for dynamic bytecode rewriting,
high-level dynamic bytecode rewriting frameworks,
and how they relate to JNIF.
 
\subsection*{Low-level Rewriting Libraries}

\jnif{} certainly is not the first \java{} bytecode analysis and instrumentation framework.
The probably earliest is BCEL\footnote{\url{http://commons.apache.org/bcel/}},
a well-designed \java{} library with a tree-based \api{}.
\asm{}\footnote{\url{http://asm.ow2.org/}}~\citep{brunetonASMCodeManipulation2002,kuleshovUsingASMFramework2007}
is probably the most prevalent,
which aims to be more efficient, especially due to the addition of a visitor-based streaming \api{}, 
but which has a somewhat less encapsulated design.

SOOT\footnote{\url{http://www.sable.mcgill.ca/soot/}}~\citep{vallee-raiSootJavaBytecode1999}
is a \java{} bytecode optimization framework supporting whole-program analysis
with four different intermediate representations:
Baf, which is simple to manipulate,
Jimple, which is easy to optimize, 
Shimple, an SSA-based variant of Jimple, and
Grimp, focused on decompilation. 

WALA\footnote{\url{http://wala.sourceforge.net/}} is a framework for static analysis, 
which also includes SHRIKE\footnote{\url{http://wala.sourceforge.net/wiki/index.php/Shrike_technical_overview}}, 
a library for instrumenting bytecode using a patch-based approach.

Unlike the above libraries, 
Javassist\footnote{\url{http://www.javassist.org/}}~\citep{chibaEasytoUseToolkitEfficient2003}
provides an API for editing class files like they were Java source code,
thereby enabling developers who do not understand bytecode to instrument class files.

\subsection*{Dynamic Instrumentation Hooks}

The most limited way for dynamically rewriting \java{} classes at run time
is the use of a custom class loader.
This requires modifications to the application,
so that it uses that class loader.
This can be problematic for applications,
especially for large programs based on powerful frameworks,
that already use their own class loaders.
This limitation can be circumvented by using dynamic instrumentation
hooks provided by the \jvm{}~\citep{lindholmJavaVirtualMachine}.
Java provides two such hooks: \java{} agents and \jvmti{}.
Java agents%
\footnote{\url{http://docs.oracle.com/javase/7/docs/api/java/lang/instrument/package-summary.html}} 
are supported via the \texttt{-javaagent} \jvm{} command line argument.
They are implemented in \java{} and use the \code{java.lang.instrument} package to interact with the \jvm{}.
This allows them to get notified when classes are about to get loaded,
and it allows them to modify the class bytecode.
They can also modify and reload already loaded classes,
however the kinds of transformations allowed with class reloading are severely limited.
JVMTI (the \emph{Java Virtual Machine Tool Interface}) is a native API into the JVM that, 
amongst many other things, provides hooks that allow the
rewriting of bytecode.
The advantage of JVMTI over Java agents is that JVMTI allows the instrumentation of \emph{all} Java classes, including the entire runtime library.
Java also provides JDI%
\footnote{\url{https://docs.oracle.com/javase/7/docs/jdk/api/jpda/jdi/}}
(the \emph{Java Debug Interface}), 
a high-level interface on top of JVMTI to control a running application in a remote JVM.

\subsection*{High-level Dynamic Analysis Frameworks}

We now discuss dynamic analysis frameworks that are built on top of the previously mentioned rewriting libraries
and use the above instrumentation hooks.
These frameworks do not allow arbitrary code transformations 
and they shield the developer from the necessary instrumentation effort.
%
Sofya\footnote{\url{http://sofya.unl.edu}}~\citep{kinneerSofyaSupportingRapid2007} 
is a dynamic analysis framework that runs the analysis in a separate JVM from the observed application.
It provides analysis developers with a set of observable events, to which the analyses can subscribe.
Sofya combines bytecode instrumentation using BCEL with the use of JDI for capturing events.
FERRARI~\citep{binderReengineeringStandardJava2007} is a dynamic bytecode instrumentation framework
that combines static instrumentation of runtime library classes with
dynamic instrumentation of application classes to achieve full coverage.
FERRARI hooks into the JVM using a Java agent.
DiSL~\citep{marekDiSLDomainspecificLanguage2012,marekDiSLExtensibleLanguage2012}
is a domain-specific aspect language for dynamic analysis.
It eliminates the need for static instrumentation from FERRARI
by using a separate JVM for instrumentation.
It uses JVMTI to hook into the JVM and 
forwards loaded classes to an instrumentation server,
where it performs instrumentation using the ASM rewriting library.
Turbo DiSL~\citep{zhengTurboDiSLPartial2012} significantly improves the performance of DiSL 
by partially evaluating analysis code at instrumentation time.
RoadRunner\footnote{\url{http://dept.cs.williams.edu/~freund/rr/}}~\citep{flanaganRoadRunnerDynamicAnalysis2010}
is a high-level framework for creating dynamic analyses focusing on concurrent programs.
An analysis implemented on top of RoadRunner simply consists of analysis code
in the form of a class that can handle the various event types 
(such as method calls or field accesses) that RoadRunner can track.
RoadRunner uses a custom classloader to be able to rewrite classes at load time,
and it uses \asm{} for bytecode rewriting.
Btrace\footnote{\url{https://kenai.com/projects/btrace}} is an instrumentation tool
that allows developers to inject probes based on a predefined set of probe types
(such as method entry, or bytecode for a specific source line number).
Btrace uses the Java agent hooks and builds on top of ASM for instrumentation.
Chord\footnote{\url{http://pag.gatech.edu/chord/}}~\citep{naik11} is a static analysis framework based on Datalog.
It uses joeq\footnote{\url{http://joeq.sourceforge.net}} to decode classes and 
convert bytecode into a three-address quadcode internal representation for static analysis.
Chord also supports dynamic analysis, 
for which it instruments programs using Javassist.

\subsection*{How \jnif{} Differs}

Similar to BCEL, \jnif{} is a low-level library that uses a clean object model to represent java class files.
However, unlike all the libraries described above,
\jnif{} is not implemented in Java, but in C++.
This allows \jnif{} to be used directly inside a JVMTI agent. 
Java-based libraries do not allow dynamic instrumentation in this way:
they either are limited to Java agents (which only provide limited coverage),
or they require out-of-process instrumentation inside a second \jvm{} (a so-called instrumentation server),
and inter-process communication between the JVMTI agent and the instrumentation server.

\jnif{} simplifies the development and deployment of full-coverage dynamic analysis tools,
because one does not need to run an instrumentation server in a separate \jvm{} process.
The fact that this is essential is demonstrated by the HPROF\footnote{\url{http://docs.oracle.com/javase/7/docs/technotes/samples/hprof.html}} profiling agent coming with the JVM.
HPROF does not use Java libraries for rewriting bytecode,
but implements (a limited form of) class file instrumentation as native code inside a JVMTI agent.

The high-level frameworks described above all abstract away from the underlying instrumentation approach.
Thus, they could make use of JNIF to provide their users with full-coverage 
while eliminating the need for a separate instrumentation server.

%Low overhead agent. 
%Modular parser, you pay what you need. 
%Full power of templates for better performance.

%\textbf{Supporting Class File Verification.}
%Java byte code verification~\cite{Leroy:2003:JBV:872715.872719}, 
%its abstract: "Bytecode verification is a crucial security component for Java applets, 
%on the Web and on embedded devices such as smart cards. 
%This paper reviews the various bytecode verification algorithms that have been proposed, 
%recasts them in a common framework of dataflow analysis, and surveys the use of proof assistants to specify bytecode verification and prove its correctness."
