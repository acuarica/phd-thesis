
Casting operations provide the means to escape the static type system.
\emph{But do they pose a problem for developers?}
Several studies~\citep{kechagiaUndocumentedUncheckedExceptions2014,coelhoUnveilingExceptionHandling2015,zhitnitskyTop10Exception2016} show that \code{ClassCastException} is in top 10 of exceptions being
thrown when analysing stack traces.

To illustrate the sort of problem developers have when applying casting conversions, we performed a simple search for commits and issues including the term \code{ClassCastException} on \github{} within projects marked as using the \java{} language.
The search returns about 171K%
\footnote{\url{https://github.com/search?l=Java&q=ClassCastException&type=Commits}}
and 73K%
\footnote{\url{https://github.com/search?l=Java&q=ClassCastException&type=Issues}}
results respectively at the time of writing.
At first glance, these results indicate that indeed \code{ClassCastException} represents a source for problems to developers.
We have included here a few commit results as an example.%
\footnote{To easily spot what the developer has changed to fix the \code{ClassCastException}, we present each source code excerpt using the Git commit \emph{diff} as reported by \github{}.}

\textbf{Forgotten Guard.}
The following listing%
\footnote{\url{https://github.com/jenkinsci/extra-columns-plugin/commit/02d10bd1fcbb2e656da9b1b4ec54208b0cc1cbb2}}
shows a cast applied to the variable \code{job} (in line 6) that throws \code{ClassCastException} because the developer forgot to include a guard.
In this case, the developer fixed the error by introducing a guard on the cast with \code{instanceof}.

\begin{lstlisting}[style=java]
@@ -41,6 +41,8 @@ public SCMTypeColumn() {
   }
       public String getScmType(@SuppressWarnings("rawtypes") Job job) {
+        if(!(job instanceof AbstractProject<?, ?>))
+            return "";
       AbstractProject<?, ?> project = (AbstractProject<?, ?>) job;
       return project.getScm().getDescriptor().getDisplayName();
   }
\end{lstlisting}

\textbf{Wrong Cast Target.}
In the next example%
\footnote{\url{https://github.com/GoldenGnu/jeveassets/commit/5f4750bc8cfa7eed8ad01efd8add2cd2cc9bd831}}
the \code{CustomFileFilter} is an inner static class inside \code{JCustomFileFilter}.
Notice that the cast happens inside an \code{equals} method, where this idiom is well known.
But the developer has used the outer --- wrong --- class to cast to.

\begin{lstlisting}[style=java]
@@ -156,7 +156,7 @@ public boolean equals(Object obj) {
  if (getClass() != obj.getClass()) {
      return false;
  }
- final JCustomFileChooser other = (JCustomFileChooser) obj;
+ final CustomFileFilter other = (CustomFileFilter) obj;
  if (!Objects.equals(this.extensions, other.extensions)) {
      return false;
  }
\end{lstlisting}

\textbf{Generic Type Inference Mismatch.}
In the following listing,%
\footnote{\url{https://github.com/ethereum/ethereumj/commit/224e65b9b4ddcb46198a6f8faf69edc65d34d382}}
the \emph{dynamic} property \code{"peer.p2p.pingInterval"} (lines 5 and 6) has type \code{int}.
To fix the error, the developer only changed the type of the
literal 5: from \code{long} to \code{int}.

\begin{lstlisting}[style=java]
@@ -281,7 +281,7 @@ private void startTimers() {
        } catch (Throwable t) {
            logger.error("Unhandled exception", t);
        }
-   }, 2, config.getProperty("peer.p2p.pingInterval", 5L), TimeUnit.SECONDS);
+   }, 2, config.getProperty("peer.p2p.pingInterval", 5), TimeUnit.SECONDS);
}
\end{lstlisting}

Looking at the definition of the \code{getProperty} method below,%
\footnote{\url{https://github.com/ethereum/ethereumj/blob/224e65b9b4ddcb46198a6f8faf69edc65d34d382/ethereumj-core/src/main/java/org/ethereum/config/SystemProperties.java\#L312}}
it obtains a dynamic property given a property name.
If it finds a value, return it.
Otherwise, returns the default value (second argument).
But the return type of \code{getProperty} is a generic type inferred
by the type of the default value, in this case, \code{long}.
The \code{ClassCastException} is then thrown in line $5$,
when casting \code{java.lang.Integer} to \code{java.lang.Long}.
To then fix the bug, the developer changed the type of the literal:
from \code{long} to \code{int}.

\begin{lstlisting}[style=java]
public <T> T getProperty(String propName, T defaultValue) {
    if (!config.hasPath(propName)) return defaultValue;
    String string = config.getString(propName);
    if (string.trim().isEmpty()) return defaultValue;
    return (T) config.getAnyRef(propName);
}
\end{lstlisting}

\textbf{Compiler Bug.}
One issue\footnote{\url{https://github.com/mockito/mockito/issues/357}} shows bad things happen when abusing the type system.
A bug in the \textsf{javac} compiler\footnote{\url{https://bugs.openjdk.java.net/browse/JDK-8058199}} was causing \jvm{}'s \code{checkcast} instructions to be skipped.
This bug was fixed in JDK 9, breaking Mockito answer strategies.

This indicates that casts represents a source of errors for developers.
We present here our partial results for the cast study.
First we give an overview of the study in \S\ref{sec:casts:overview}, while \S\ref{sec:casts:stats} gives an estimation of how often a cast operator is used.
Finally, \S\ref{sec:casts:methodology} introduces the methodology we plan to use to discover cast usage patterns.
