\begin{pattern}{GetByClassLiteral}
A cast is using an application-specific guard,
but the guard depends on a class literal.

\instances{}
The following example,%
\footnote{\url{http://bit.ly/elastic_elasticsearch_2SSgsFV}}
shows an instance of the \thisp{} pattern.
A cast is performed to the \code{field} variable (line 22),
based whether the runtime class of the variable is actually \code{Short.class}.

%https://lgtm.com/projects/g/elastic/elasticsearch/snapshot/dist-1916470085-1524814812150/files/server/src/main/java/org/elasticsearch/common/lucene/Lucene.java?sort=name&dir=ASC&mode=heatmap#L478
\begin{minted}[highlightlines=22]{java}
Class type = field.getClass();
if (type == String.class) {
    out.writeByte((byte) 1);
    out.writeString((String) field);
} else if (type == Integer.class) {
    out.writeByte((byte) 2);
    out.writeInt((Integer) field);
} else if (type == Long.class) {
    out.writeByte((byte) 3);
    out.writeLong((Long) field);
} else if (type == Float.class) {
    out.writeByte((byte) 4);
    out.writeFloat((Float) field);
} else if (type == Double.class) {
    out.writeByte((byte) 5);
    out.writeDouble((Double) field);
} else if (type == Byte.class) {
    out.writeByte((byte) 6);
    out.writeByte((Byte) field);
} else if (type == Short.class) {
    out.writeByte((byte) 7);
    out.writeShort((Short) field);
} else if (type == Boolean.class) {
    out.writeByte((byte) 8);
    out.writeBoolean((Boolean) field);
} else if (type == BytesRef.class) {
    out.writeByte((byte) 9);
    out.writeBytesRef((BytesRef) field);
} else {
    throw new IOException("Can't handle sort field value of type ["+type+"]");
}
\end{minted}

\detection{}

\discussion{}

\related{}

\end{pattern}