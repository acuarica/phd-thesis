\section{Discussion}
\label{sec:casts:discussion}

A summary of the patterns is shown in Table~\ref{table:casts:categories}.
The table consists of the following columns:
\newcommand{\gh}[1]{\emph{\textbf{#1}}}
The \gh{Pattern} column indicates the name of the pattern.
\gh{Guarded?} The patterns in this category are guarded casts.
A guarded cast is a cast such that before the cast is applied,
some condition---the \emph{guard}---needs to be verified.
The condition to be verified guarantees that the cast will not fail at runtime (unless there is a bug in the application), \ie,
the cast will not throw a \code{ClassCastException}.
Some kind of guards ensure that the cast will not fail at the language-level,
while others only can guarantee it at the application-level.
\gh{Lang} These casts could be removed if there is enough language support.
\gh{Tools} The casts in this group could be checked with new analysis or refactoring tools.
\gh{Gen} These casts are related to generated or boilerplate code.
\gh{Dev} These casts can be removed by the developer with no or little refactoring.
or suggest a code smell in the source code.
\gh{Generic} The casts in this category are related to generic or reified generics.

\begin{table*}[t!]
\scriptsize
\centering
\caption{Categorization of Cast Usage Patterns}
\label{table:casts:categories}
\begin{tabularx}{\linewidth}{|r|X||c||c|c|c||c|}
\hdr \hline
    & \multicolumn{1}{|c||}{\textbf{Pattern}}
    & \textbf{Guarded?}
    & \textbf{Lang}
    & \textbf{Tools}
    & \textbf{Dev}
    & \textbf{Generic}
    \\ \hline
\row 1 & \nameref{pat:Typecase} & \cmark & \cmark & \cmark &  &  &  \\
\alt 2 & \nameref{pat:Stash} &  &  & \cmark &  &  &  \\
\row 3 & \nameref{pat:Factory} &  &  & \cmark &  &  &  \\
\alt 4 & \nameref{pat:Family} &  & \cmark &  &  &  &  \\
\row 5 & \nameref{pat:UseRawType} &  &  &  &  & \cmark & \cmark \\
\alt 6 & \nameref{pat:Equals} & \cmark &  &  & \cmark &  &  \\
\row 7 & \nameref{pat:KnownReturnType} &  &  & \cmark &  & \cmark &  \\
\alt 8 & \nameref{pat:Redundant} &  &  &  &  & \cmark &  \\
\row 9 & \nameref{pat:SelectOverload} &  & \cmark &  &  &  &  \\
\alt 10 & \nameref{pat:Deserialization} &  &  & \cmark &  &  &  \\
\row 11 & \nameref{pat:VariableSupertype} &  &  &  &  & \cmark &  \\
\alt 12 & \nameref{pat:NewDynamicInstance} &  &  & \cmark &  &  &  \\
\row 13 & \nameref{pat:SoleSubclassImplementation} &  & \cmark &  &  &  &  \\
\alt 14 & \nameref{pat:ObjectAsArray} &  &  &  &  & \cmark &  \\
\row 15 & \nameref{pat:ImplicitIntersectionType} &  & \cmark &  &  &  &  \\
\alt 16 & \nameref{pat:CovariantReturnType} &  & \cmark &  &  &  &  \\
\row 17 & \nameref{pat:RemoveWildcard} &  & \cmark &  &  &  & \cmark \\
\alt 18 & \nameref{pat:OperandStack} & \cmark & \cmark &  &  &  &  \\
\row 19 & \nameref{pat:ReflectiveAccessibility} &  & \cmark &  &  &  &  \\
\alt 20 & \nameref{pat:FluentAPI} &  & \cmark &  &  &  & \cmark \\
\row 21 & \nameref{pat:CovariantGeneric} &  & \cmark &  &  &  & \cmark \\
\alt 22 & \nameref{pat:Composite} &  & \cmark &  &  &  &  \\
\row 23 & \nameref{pat:GenericArray} &  & \cmark &  &  &  & \cmark \\
\alt 24 & \nameref{pat:AccessSuperclassField} &  &  &  &  & \cmark &  \\
\row 25 & \nameref{pat:UnoccupiedTypeParameter} &  & \cmark &  &  &  & \cmark \\

\hline
\end{tabularx}
\end{table*}

Many programming languages provide features to ameliorate the more common use cases of casts.
For instance,
\kotlin{}'s smart casts couple together the \code{instanceof} operator and cast operation on value, 
providing direct support for the \nameref{pat:Typecase} pattern.
More generally, ML-style pattern matching subsumes this pattern.
Other language features that might at least partially obviate the need for some of the patterns are
intersection types (cf. \nameref{pat:ImplicitIntersectionType}),
and self types or associated types (cf. 
\nameref{pat:Factory},
\nameref{pat:KnownReturnType},
\nameref{pat:Deserialization},
\nameref{pat:CovariantReturnType},
\nameref{pat:FluentAPI}).
Virtual classes~\citep{gbeta, scalaIndependentlyExtensible} and languages that support family polymorphism~\citep{ernstFamilyPolymorphism2001}
would help with casts in the \nameref{pat:Family} pattern.

Many cast patterns (\eg, 
\nameref{pat:RemoveWildcard},
\nameref{pat:GenericArray},
\nameref{pat:CovariantGeneric},
\nameref{pat:UnoccupiedTypeParameter})
are used either to 
workaround---or to take advantage of---the erasure of generic type parameters in \java{}.
Reified generics or definition-site, rather than use-site, variance annotations~\citep{altidorTamingWildcardsCombining2011}
would reduce the need for these patterns.

Our study also suggests analyses could be performed to improve code quality and eliminate some cast usages,
for instance removing redundant casts,
finding opportunities to use generics instead,
or locating code smells (cf.
\nameref{pat:UseRawType},
\nameref{pat:KnownReturnType},
\nameref{pat:VariableSupertype}).