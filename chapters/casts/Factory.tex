\begin{pattern}{Factory}
Creates an object based on some arguments either to the method call or constructor.
Since the arguments are known at compile-time, cast to the specific type.

Cast factory method result to subtype (special case of family polymorphism).
Usually Logger.getLogger.

The method is declared to return URLConnection but can return a more specific type based on the URL string.
Cast to that.
We should generalize this pattern.

\instances{}

\footnote{\url{http://bit.ly/2HvRlUX}}

\footnote{\url{https://docs.oracle.com/javase/8/docs/api/java/security/KeyPair.html\#getPrivate()}}

%https://lgtm.com/projects/b/connect2id/oauth-2.0-sdk-with-openid-connect-extensions/snapshot/dist-1311020143-1524814812150/files/src/test/java/com/nimbusds/oauth2/sdk/jose/jwk/RemoteJWKSetTest.java?sort=name&dir=ASC&mode=heatmap#L242
\begin{minted}[highlightlines=10]{java}
KeyPairGenerator pairGen = KeyPairGenerator.getInstance("RSA");
pairGen.initialize(1024);
KeyPair keyPair = pairGen.generateKeyPair();
RSAKey rsaJWK1 = new RSAKey.Builder((RSAPublicKey) keyPair.getPublic())
        .privateKey((RSAPrivateKey) keyPair.getPrivate())
        .keyID("1")
        .build();
keyPair = pairGen.generateKeyPair();
RSAKey rsaJWK2 = new RSAKey.Builder((RSAPublicKey) keyPair.getPublic())
        .privateKey((RSAPrivateKey) keyPair.getPrivate())
        .keyID("2")
        .build();
\end{minted}

\footnote{\url{http://bit.ly/2E6KY6T}}

\footnote{\url{https://docs.oracle.com/javase/8/docs/api/java/net/URL.html\#openConnection--}}

%https://lgtm.com/projects/g/apache/hadoop/snapshot/dist-956730001-1524814812150/files/hadoop-yarn-project/hadoop-yarn/hadoop-yarn-server/hadoop-yarn-server-resourcemanager/src/test/java/org/apache/hadoop/yarn/server/resourcemanager/webapp/TestRMWebServicesHttpStaticUserPermissions.java?sort=name&dir=ASC&mode=heatmap#L138
\begin{minted}[highlightlines=2]{java}
URL url = new URL("http://localhost:8088/ws/v1/cluster/apps");
HttpURLConnection conn = (HttpURLConnection) url.openConnection();
\end{minted}

\detection{}

\discussion{}

\related{}

\end{pattern}