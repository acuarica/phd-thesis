\begin{pattern}{Factory}
\todo{Also Logger}
Creates an object based on some arguments either to the method call or constructor.
Since the arguments are known at compile-time, cast to the specific type.

\todo{Nate: Move to "Related Patterns"}
Cast factory method result to subtype (special case of family polymorphism).
Usually Logger.getLogger.

The method is declared to return URLConnection but can return a more specific type based on the URL string.
Cast to that.
We should generalize this pattern.
%
\todo{Nate: Create with type descriptor "Tag"}

\instances{}
The following snippet%
\footnote{\url{http://bit.ly/connect2id_oauth-2-0-sdk-with-_2HvRlUX}}
shows an instance of the \thisp{} pattern.
The cast is applied to the result of invoking \code{keyPair.getPrivate}
(line 10).
The variable \code{keyPair} is assigned the result of \code{pairGen.generateKeyPair} (line 3).
At the same time, \code{pairGen} is assigned the value of \code{KeyPairGenerator.getInstance("RSA")}.
The argument \code{"RSA"} indicates the algorithm to use.
%
\done{Describe}
%
The method%
\footnote{\url{https://docs.oracle.com/javase/8/docs/api/java/security/KeyPair.html\#getPrivate()}}
will return a reference to the private key component,
and this is determined by the algorithm argument described above.

%https://lgtm.com/projects/b/connect2id/oauth-2.0-sdk-with-openid-connect-extensions/snapshot/dist-1311020143-1524814812150/files/src/test/java/com/nimbusds/oauth2/sdk/jose/jwk/RemoteJWKSetTest.java?sort=name&dir=ASC&mode=heatmap#L242
\begin{minted}[highlightlines=10]{java}
KeyPairGenerator pairGen = KeyPairGenerator.getInstance("RSA");
pairGen.initialize(1024);
KeyPair keyPair = pairGen.generateKeyPair();
RSAKey rsaJWK1 = new RSAKey.Builder((RSAPublicKey) keyPair.getPublic())
        .privateKey((RSAPrivateKey) keyPair.getPrivate())
        .keyID("1")
        .build();
keyPair = pairGen.generateKeyPair();
RSAKey rsaJWK2 = new RSAKey.Builder((RSAPublicKey) keyPair.getPublic())
        .privateKey((RSAPrivateKey) keyPair.getPrivate())
        .keyID("2")
        .build();
\end{minted}

\done{Also URLOpenConnection.}
Similar to the above snippet, the next example%
\footnote{\url{http://bit.ly/apache_hadoop_2E6KY6T}}
shows an instance of the \thisp{} pattern where a cast is performed on the result of the \code{openConnection} method%
\footnote{\url{https://docs.oracle.com/javase/8/docs/api/java/net/URL.html\#openConnection--}}
(line 2).
The \code{openConnection} method is applied to the \code{url} variable,
which is assigned in line 1 using the \code{URL} constructor.
The argument to the constructor is an \code{http} URL,
thus the result is cast to \code{HttpURLConnection}.

%https://lgtm.com/projects/g/apache/hadoop/snapshot/dist-956730001-1524814812150/files/hadoop-yarn-project/hadoop-yarn/hadoop-yarn-server/hadoop-yarn-server-resourcemanager/src/test/java/org/apache/hadoop/yarn/server/resourcemanager/webapp/TestRMWebServicesHttpStaticUserPermissions.java?sort=name&dir=ASC&mode=heatmap#L138
\begin{minted}[highlightlines=2]{java}
URL url = new URL("http://localhost:8088/ws/v1/cluster/apps");
HttpURLConnection conn = (HttpURLConnection) url.openConnection();
\end{minted}

The following example%
\footnote{\url{http://bit.ly/bcgit_bc-java_2TEVScM}}
\done{Luis: Get bitly link}
shows how a cast (line 3) is being determined by the argument to the \code{CertificateFactory.getInstance} method (line 1).
The argument is the string \code{"X.509"},
therefore the method \code{generateCRL} will return a value of type \code{X509CRL}.

%https://lgtm.com/projects/g/bcgit/bc-java/snapshot/dist-20740003-1524814812150/files/prov/src/test/java/org/bouncycastle/jce/provider/test/X509LDAPCertStoreTest.java?sort=name&dir=ASC&mode=heatmap#L241
\begin{minted}[highlightlines=1]{java}
CertificateFactory cf = CertificateFactory.getInstance("X.509", "BC");
// [...]
X509CRL crl = (X509CRL)cf.generateCRL(new ByteArrayInputStream(directCRL));
\end{minted}

In our last example%
\footnote{\url{http://bit.ly/JSQLParser_JSqlParser_2TecMyB}}
the cast instance (line 2) is applied to the \code{parse} method.
The \code{parse} method (defined in line 10) returns a value of type \code{Statement}.
The method is using the argument \code{statement},
assigned in line 1.
Since the statement is a \code{SELECT} statement,
the value returned by the \code{parse} method is of type \code{Select}
(declared in line 4).

%https://lgtm.com/projects/g/JSQLParser/JSqlParser/snapshot/dist-43250114-1524814812150/files/src/test/java/net/sf/jsqlparser/test/select/SelectTest.java?sort=name&dir=ASC&mode=heatmap#L437
\begin{minted}[highlightlines=2]{java}
statement = "SELECT * FROM mytable WHERE mytable.col = 9 LIMIT :param_name";
select = (Select) parserManager.parse(new StringReader(statement));

public class Select implements Statement {
        // [...]
}

public class CCJSqlParserManager implements JSqlParser {
    @Override
    public Statement parse(Reader statementReader) throws JSQLParserException {
        // [...]
    }
}
\end{minted}


\detection{}

\discussion{}
Almost a third of this cast instances appear in test code.

\related{}
The \nameref{pat:KnownReturnType} pattern is similar to the \thisp{} pattern,
since both depends on the knowledge that a method returns a more specific type.

\end{pattern}