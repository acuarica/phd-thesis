\begin{pattern}{LookupById}
This pattern is used to extract stashed values from a generic container.

Lookup an object by ID, tag or name and cast the result
(it is used often in Android code).
It accesses a collection that holds values of different types
(usually implemented as \code{Collection<Object>} or as \code{Map<K, Object>}).

\instances

%https://lgtm.com/projects/g/loopj/android-async-http/snapshot/dist-1879340034-1518514025554/files/library/src/main/java/com/loopj/android/http/AsyncHttpClient.java?sort=name&dir=ASC&mode=heatmap&excluded=false#L258

In the example shown in listing,
% \footnote{\url{d}},
the \texttt{getAttribute} method returns \texttt{Object}.
The variable \texttt{context} is of type \texttt{BasicHttpContext},
which is implemented with \texttt{HashMap}.

\lstset{language=java,label=orga7c88d3,caption={Example of the \pname{} pattern.},captionpos=b,numbers=none,style=java}
\begin{lstlisting}
AuthState authState =
        (AuthState) context.getAttribute(ClientContext.TARGET_AUTH_STATE);
\end{lstlisting}

\discussion{}
%
\todo{Cut/Move to future work.}
%
This pattern suggests heterogeneous dictionary.
Given our manual inspection,
we believe that all dictionary keys and resulting types are known at
compile-time, \ie, by the programmer.
%
\done{Nate: Replace "restriction" for "inexpressiveness"}
%
But in any case a cast is needed given the inexpressiveness of the type system.
As a complementary analysis,
it would be interesting to check whether all call sites to
\code{getAttribute} receives a constant (\code{final static} field).

Notice that this pattern is not guarded by an \code{instanceof}.
However, the cast involved does not fail at runtime.
This means that the source of the cast is known to the programmer.
This raises the following questions:
\begin{itemize}
\item \emph{What kind of analysis is needed to detect the source of the cast?}
\item \emph{Is worth to have it?}
\item \emph{Is better to change API?}
\item \emph{How other --- statically typed --- languages support this kind of idiom?}
\item \emph{Could generative programming a.k.a. templates solve this problem?}
\end{itemize}

\end{pattern}