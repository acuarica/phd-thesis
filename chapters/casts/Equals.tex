
\begin{pattern}{Equals}
This pattern is a common pattern to implement the \code{equals} method.

A cast expression is guarded by either an \code{instanceof} test or
a \code{getClass} comparison (to the same target type as the cast);
in an \code{equals}\footnote{\url{https://docs.oracle.com/javase/8/docs/api/java/lang/Object.html\#equals-java.lang.Object-}}
method implementation.
This is done to check if the argument has same type as the receiver
(\code{this} argument).

Notice that a cast in an \code{equals} method is needed because it
receives an \code{Object} as a parameter.

\instances

%https://lgtm.com/projects/g/neo4j/neo4j/snapshot/dist-15760049-1519892555006/files/community/kernel/src/main/java/org/neo4j/kernel/impl/api/CountsRecordState.java?sort=name&dir=ASC&mode=heatmap&excluded=false#L182
%http://bit.ly/2vJw94J

The following listing\footnote{\url{http://bit.ly/2vJw94J}}
shows an example of the \pname{} pattern.
In this case, \code{instanceof} is used to guard for
the same type as the receiver.

\begin{lstlisting}[style=java,caption={\pname{} pattern using \code{instanceof} as a guard.}]
@Override
public boolean equals(Object obj) {
    if ( this == obj ) {
        return true;
    }
    if ( (obj instanceof Difference) ) {
        Difference that = (Difference) obj;
        return actualFirst == that.actualFirst
          && expectedFirst == that.expectedFirst
          && actualSecond == that.actualSecond 
          && expectedSecond == that.expectedSecond
          && key.equals( that.key );
    }
    return false;
}
\end{lstlisting}

%https://lgtm.com/projects/g/neo4j/neo4j/snapshot/dist-15760049-1519892555006/files/community/bolt/src/main/java/org/neo4j/bolt/v1/messaging/infrastructure/ValuePath.java?sort=name&dir=ASC&mode=heatmap&excluded=false#L278
%http://bit.ly/2vKP0MW

Alternatively,
listing \ref{lst:equals}\footnote{\url{http://bit.ly/2vKP0MW}} shows
another example of the \pname{} pattern.
But in this case,
a \code{getClass} comparison is used to guard for the same type
as the receiver.

\begin{lstlisting}[style=java,label=lst:equals,caption=\pname{} pattern guarded by a \code{getClass} comparison]
@Override
public boolean equals( Object o ) {
    if ( this == o ) return true;
    if ( o == null || getClass() != o.getClass() )
        return false;

    ValuePath that = (ValuePath) o;
    return nodes.equals(that.nodes) &&
        relationships.equals(that.relationships);
}
\end{lstlisting}

\detection

The detection query looks for a cast expression inside an
\code{equals} method implementation.
Moreover, the cast needs to be guarded by either an
\code{instanceof} test or a \code{getClass} comparison.

\discussion

The pattern for an \code{equals} method implementation is well-known.
% We have started looking at all \code{equals} methods with the following \ql{} query:

% \begin{lstlisting}[style=ql,caption=Fetching all \code{equals} method implementations.]
% import java
% from EqualsMethod eqm select eqm
% \end{lstlisting}

We found out that, with respect to cast,
most \code{equals} methods are implemented with the same structure.
Maybe avoid boilerplate code by providing code generation,
like in \haskell{} (with \code{deriving}).

\cite{vaziriDeclarativeObjectIdentity2007} propose a declarative approach
to avoid boilerplate code when implementing both
the \code{equals} and \code{hashCode} methods.
They manually analyzed several applications,
and found many issues while implementing \code{equals()} and
\code{hashCode()} methods.
It would be interesting to check whether these issues happen in real
application code.

There is an exploratory document%
\footnote{\url{http://cr.openjdk.java.net/\~briangoetz/amber/datum.html}}
by Brian Goetz --- \java{} Language Architect ---
addressing these issues from a more general perspective.
It is definitely a starting point towards improving the \java{} language.

%\footnote{\url{https://lgtm.com/projects/g/square/sqlbrite/snapshot/3a9916985485ba5922097fe59a18230500f02df4/files/sample/build/generated/source/apt/debug/com/example/sqlbrite/todo/ui/$AutoValue_ListsItem.java?sort=name&dir=ASC&mode=heatmap&showExcluded=false#L47}}

\end{pattern}