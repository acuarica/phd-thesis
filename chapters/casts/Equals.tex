\begin{pattern}{Equals}
This pattern is a common pattern to implement the \code{equals} method (declared in \code{java.lang.Object}).
A cast expression is guarded by either an \code{instanceof} test or a \code{getClass} comparison (usually to the same target type as the cast);
in an \code{equals}%
\footnote{\url{https://docs.oracle.com/javase/8/docs/api/java/lang/Object.html\#equals-java.lang.Object-}} method implementation.
This is done to check if the argument has same type as the receiver
(\code{this} argument).

Notice that a cast in an \code{equals} method is needed because it
receives an \code{Object} as a parameter.

\instances{}
The following listing\footnote{\url{http://bit.ly/2vJw94J}} shows an example of the \pname{} pattern.
In this case, \code{instanceof} is used to guard for the same type as the receiver.

%https://lgtm.com/projects/g/neo4j/neo4j/snapshot/dist-15760049-1519892555006/files/community/kernel/src/main/java/org/neo4j/kernel/impl/api/CountsRecordState.java?sort=name&dir=ASC&mode=heatmap&excluded=false#L182
\begin{minted}[highlightlines=7]{java}
@Override
public boolean equals(Object obj) {
    if ( this == obj ) {
        return true;
    }
    if ( (obj instanceof Difference) ) {
        Difference that = (Difference) obj;
        return actualFirst == that.actualFirst
          && expectedFirst == that.expectedFirst
          && actualSecond == that.actualSecond 
          && expectedSecond == that.expectedSecond
          && key.equals( that.key );
    }
    return false;
}
\end{minted}

Alternatively, the following listing%
\footnote{\url{http://bit.ly/2vKP0MW}}
shows another example of the \thisp{} pattern.
But in this case, a \code{getClass} comparison is used to guard for the same type as the receiver.

%https://lgtm.com/projects/g/neo4j/neo4j/snapshot/dist-15760049-1519892555006/files/community/bolt/src/main/java/org/neo4j/bolt/v1/messaging/infrastructure/ValuePath.java?sort=name&dir=ASC&mode=heatmap&excluded=false#L278
\begin{minted}[highlightlines=7]{java}
@Override
public boolean equals( Object o ) {
    if ( this == o ) return true;
    if ( o == null || getClass() != o.getClass() )
        return false;

    ValuePath that = (ValuePath) o;
    return nodes.equals(that.nodes) &&
        relationships.equals(that.relationships);
}
\end{minted}

In some situations, the type casted to is not same as the enclosing class.
Instead, the type casted to is the super class of the enclosing class.
The following example%
\footnote{\url{http://bit.ly/2HmHMYE}}
shows this scenario.
This usually happens when the Google AutoValue library%
\footnote{\url{https://github.com/google/auto/tree/master/value}}
is used.
The AutoValue is a code generator for value classes.

%https://lgtm.com/projects/g/square/sqlbrite/snapshot/3a9916985485ba5922097fe59a18230500f02df4/files/sample/build/generated/source/apt/debug/com/example/sqlbrite/todo/ui/$AutoValue_ListsItem.java?sort=name&dir=ASC&mode=heatmap&showExcluded=false#L52
\begin{minted}[highlightlines=13]{java}
@AutoValue
abstract class ListsItem implements Parcelable {
    // [...]
}

abstract class $AutoValue_ListsItem extends ListsItem {
    @Override
    public boolean equals(Object o) {
      if (o == this) {
        return true;
      }
      if (o instanceof ListsItem) {
        ListsItem that = (ListsItem) o;
        return (this.id == that.id())
             && (this.name.equals(that.name()))
             && (this.itemCount == that.itemCount());
      }
      return false;
    }
}
\end{minted}

\footnote{\url{http://bit.ly/2SM5pOw}}
%https://lgtm.com/projects/g/bndtools/bnd/snapshot/dist-930051-1524814812150/files/biz.aQute.bndlib/src/aQute/bnd/osgi/resource/CapReq.java?sort=name&dir=ASC&mode=heatmap#L73
\begin{minted}[highlightlines=12]{java}
@Override
public boolean equals(Object obj) {
    if (this == obj)
            return true;
    if (obj == null)
            return false;
    if (obj instanceof CapReq)
            return equalsNative((CapReq) obj);
    if ((mode == MODE.Capability) && (obj instanceof Capability))
            return equalsCap((Capability) obj);
    if ((mode == MODE.Requirement) && (obj instanceof Requirement))
            return equalsReq((Requirement) obj);
    return false;
}
\end{minted}

\detection{}
The detection query looks for a cast expression inside an \code{equals} method implementation.
Moreover, the cast needs to be guarded by either an \code{instanceof} test or a \code{getClass} comparison.
And the type being casted to needs to be either the same as the enclosing class or a superclass of it.

\discussion{}
The pattern for an \code{equals} method implementation is well-known.

We found out that, with respect to cast,
most \code{equals} methods are implemented with the same structure.
Maybe avoid boilerplate code by providing code generation,
like in \haskell{} (with \code{deriving}).

\cite{vaziriDeclarativeObjectIdentity2007} propose a declarative approach to avoid boilerplate code when implementing both the \code{equals} and \code{hashCode} methods.
They manually analyzed several applications, and found there are many issues while implementing \code{equals()} and \code{hashCode()} methods.
It would be interesting to check whether these issues happen in real application code.

There is an exploratory document%
\footnote{\url{http://cr.openjdk.java.net/\~briangoetz/amber/datum.html}}
by Brian Goetz --- \java{} Language Architect --- addressing these issues from a more general perspective.
It is definitely a starting point towards improving the \java{} language.

\related{}
This pattern can be seen as a special instance of the \nameref{pat:PatternMatching} pattern.
\end{pattern}