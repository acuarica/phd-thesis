
The main goal of a \emph{static} type system
is to prevent certain kinds of errors from happening at run time.
A type system is formulated as a set of constraints that gives any expression or term in a program a well-defined type.
Any program not satisfying the constraints specified by the type system is
simply rejected by the compiler.

Nevertheless, often a static type system is insufficiently precise.
The type checker is necessarily conservative:
it must not accept invalid programs,
but it may reject programs that are valid but whose validity cannot be ensured at compile time.
However, there are situations when the developer has more information
about the program than can be encoded---or encoded easily---into the types.
To that end, programming languages often provide mechanisms to make the typing constraints less strict,
allowing more valid programs at the expense of more errors at run time.

One common extension of type systems is \emph{subtyping} --- or \emph{subtype polymorphism} --- usually seen in \emph{object-oriented} programming languages.
The subtype mechanism allows the interoperability of two different related types.
As~\cite{pierceTypesProgrammingLanguages2002} states, "[\ldots] $S$ is a subtype of $T$, written $S <: T$, to mean that any term of type $S$ can safely be used in a context where a term of type $T$ is expected.
This view of subtyping is often called the \emph{principle of safe substitution}."

A common mechanism for relaxing static typing in object-oriented languages is \emph{casting}.
In programming languages with subtyping such as \java{} or \cpp{}, casting
allows
an expression to be viewed at a different type than the one at which it was defined.
Casts are checked dynamically to ensure that the object
being cast is an instance of the desired type.

We aim to understand why developers use casts.
Why is the static type system insufficient,
requiring an escape hatch into dynamic type checking?
Specifically, we attempt to answer the following three research questions:

\begin{enumerate}[label=$RQ/C\arabic*:$,ref=$RQ/C\arabic*$,leftmargin=3.4\parindent]
\item\label{casts:rq1}{\bf \crqA} \crqAdesc
\item\label{casts:rq2}{\bf \crqB} \crqBdesc
\item\label{casts:rq3}{\bf \crqC} \crqCdesc
\end{enumerate}

To answer these research questions, we devise
\emph{usage patterns}.
Usage patterns are \emph{recurrent programming idioms} used by developers to solve a specific issue.
Usage patterns enable the categorization of different kinds of cast usages and
thus provide insights into how the language is being used by developers in real-world applications.
Our cast usage patterns can be:
\begin{inparaenum}[(1)]
\item a reference for current and future language designers
to make more informed decisions about programming languages,
\eg{},
the addition of \emph{smart casts} in \lang{Kotlin},\footnote{\url{https://kotlinlang.org/docs/reference/typecasts.html\#smart-casts}}
\item a reference for tool builders, \eg{}, by providing more precise or new
  refactoring or code smell analyses,
\item a guide for researchers to test new language features, \eg{}, \cite{wintherGuardedTypePromotion2011} or to carry out controlled
  experiments about programming, \eg{}, \cite{stuchlikStaticVsDynamic2011}, and
\item a guide for developers for best or better practices.
\end{inparaenum}
To answer our research questions,
we empirically study how casts
are used by developers.
We focus on \java{} given its wide usage and relevance for both
research and industry.\footnote{\url{https://www.tiobe.com/tiobe-index/}}

\subsection*{Casts in Java}

While casts should be familiar to most programmers of object-oriented languages, 
because casts have different semantics in different programming languages,
we briefly summarize the meaning of casts in \java{} and the terminology used in rest of this paper.

A cast operation, written \code{(T)}~\code{e} in \java{}
consists of a \emph{target type} \code{T} and an \emph{operand} \code{e}.
The operand evaluates to a \emph{source value} which has a run-time
\emph{source type}.
In \java{}, a source reference type is always a class type.
For a particular cast evaluated at run time, 
the \emph{source} of the cast is the expression in the program that
created the source value.
For reference casts, the source is an object allocation.
The source may or may not be known statically.

An \emph{upcast} occurs when the cast is from a source reference type $S$ to a target reference type $T$,
where $T$ is a supertype of $S$.
In our terminology, upcasts include identity casts where the target type
is the same as the type of the operand.
An upcast does not require a run-time check.

A \emph{downcast}, on the other hand,
occurs when converting from a source reference type $S$ to a target reference
type $T$, where $T$ is a proper subtype of $S$.
Listing~\ref{lst:casts:simple} shows how to use the cast operator (line 2) to treat a reference (the variable \code{o}) as a different type (\code{String}) as it was defined (\code{Object}).

\begin{listing}
\caption{Variable \code{o} (defined as \code{Object}) is cast to \code{String}.}
\label{lst:casts:simple}
\begin{minted}{java}
Object o = "foo"; 
String s = (String)o;
\end{minted}
\end{listing}

In type-safe OO languages, downcasts require
a run-time check to ensure that the source value is an instance of the target type.
The above snippet is compiled into the following \java{} bytecode.
The \code{aload\_1} instruction (line 3) pushes the local variable \code{o} into the operand stack.
The \code{checkcast} instruction (line 4) then checks at run-time that the top of the stack has the specified type (\code{java.lang.String} in this example).

\begin{lstlisting}[style=bytecode]
ldc           #2     // String foo
astore_1
aload_1
checkcast     #3     // class java/lang/String
astore_2
\end{lstlisting}


This run-time check can either \emph{succeed} or \emph{fail}.
A \code{ClassCastException} is thrown when a downcast fails.
This exception is an unchecked exception, \ie,
the programmer is required neither to handle it
nor to specify the exception in the method signature.
Listing~\ref{lst:catchcce} shows how to detect whether a cast failed by catching this exception.

\begin{listing}
\caption{Catching \code{ClassCastException}}
\label{lst:catchcce}
\begin{minted}{java}	
try {
	Object x = new Integer(0);
	System.out.println((String)x); 
} catch (ClassCastException e) { 
	System.out.println(""); 
}
\end{minted}
\end{listing}


A \emph{guard} is a conditional expression
on which a (down)cast is control-dependent and that ensures that the cast
is evaluated only if it will succeed.

Guards are often implemented using the \code{instanceof} operator, which tests
if an expression is an instance of a given reference type.
If an \code{instanceof} guard returns true, the guarded cast should not throw
a \code{ClassCastException}.
Listing~\ref{lst:instanceoftest} shows a usage of the \code{instanceof} operator together with a cast expression.

\begin{listing}
\caption{Runtime type test using \code{instanceof} before applying a cast.}
\label{lst:instanceoftest}
\begin{minted}{java}
if (x instanceof Foo) {
	((Foo)x).doFoo();
}
\end{minted}
\end{listing}

In \java{}, an object's type can also be checked using reflection:
the \code{getClass} method%
\footnote{\url{https://docs.oracle.com/javase/8/docs/api/java/lang/Object.html\#getClass--}}
returns the run-time class of an object.
This \code{Class} object can be then compared against a class literal, \eg,
\code{x.getClass()}~\code{==}~\code{C.class}.
This test is more precise than an \code{x}~\code{instanceof}~\code{C} test since it succeeds only when the operand's class is exactly \code{C},
rather than any subclass of \code{C}.
Listing~\ref{lst:getclasstest} shows how to use the \code{getClass} method to test for an object's type.

\begin{listing}
\caption{Runtime type test using \code{getClass} before applying a cast.}
\label{lst:getclasstest}
\begin{minted}{java}
if (x.getClass() == Foo.class) {
	((Foo)x).doFoo(); 
}
\end{minted}
\end{listing}

Because they can fail,
downcasts pose potential threats.
Unguarded downcasts in particular are
worrisome because the developer is essentially telling the compiler
\emph{``Trust me, I know what I'm doing.''}
Because downcasts are an escape-hatch from the static type system---they
permit dynamic type errors---a cast is often seen as a design flaw or code
smell in an object-oriented system~\citep{tufanoWhenWhyYour2015}.

A cast can also fail at compile time if the cast operand and the target type are incompatible.
For instance, in the expression \code{(String) new Integer(1)} a value of type
\code{Integer} can \emph{never} be converted to \code{String}, so the compiler
rejects the cast expression.

Another form of casts in \java{} are \emph{primitive conversions}, or more specifically
\emph{numeric conversions}. These are conversions from
one primitive (non-reference) type, usually a numeric type, to another. These conversions can result
in loss of precision of the numeric value, although they do not fail with a
run-time exception.

\emph{Boxing} and \emph{unboxing} occur when casting from a primitive type to
a reference type or vice versa, \eg, \code{(Integer) 3} converts the primitive
\code{int} 3 into a boxed \code{java.lang.Integer}.
Like downcasts, unboxing casts can fail at
run time if the source value cannot be converted to the target type.
\java{} supports \emph{autoboxing} and \emph{autounboxing} between primitives and their corresponding
boxed type in the \code{java.lang} package.

Generics were introduced into \java{} to provide more static type safety.
For instance, the type \code{List<T>} contains only elements of type \code{T}.
The underlying implementation of generics, however, erases the actual type
arguments when compiling to bytecode. To ensure type safety in the generated
bytecode, the compiler inserts cast instructions into the generated code.
Improper use of generic types or mixing of generic and raw types can lead
to dynamic type errors---\ie, \code{ClassClassException}.
Our study, however, does not consider these compiler-inserted casts. We are only concerned with
programmer-inserted casts in the source code, not in the generated bytecode.

\subsection*{Issues Developers have Applying the Cast Operator}

Several studies~\citep{kechagiaUndocumentedUncheckedExceptions2014,coelhoUnveilingExceptionHandling2015,zhitnitskyTop10Exception2016}
suggest that in \java{},
the \code{ClassCastException} is in the top ten of exceptions being thrown when analysing stack traces.
%These studies have analyzed the exceptions thrown in stack traces.
%The exceptions come from third-party libraries \api{}s and the Android \api{},
%indicating a misuse of such \api{}s.
%\code{ClassCastException} is in the top 10 of exceptions thrown,
%thus it represents a problem for developers.
%
To illustrate the sort of problems developers have when applying casting conversions,
we performed a search for commits and issues including the term
\code{ClassCastException} within projects marked as using the \java{} language
on \github{},
the largest host of source code in the world~\citep{gousiosLeanGHTorrentGitHub2014}.
Our searches returned about 171K commits%
\footnote{\url{https://github.com/search?l=Java&q=ClassCastException&type=Commits}}
and 73K issues,%
\footnote{\url{https://github.com/search?l=Java&q=ClassCastException&type=Issues}}
respectively, at the time of this writing.
At first glance, these results indicate that \code{ClassCastException} indeed
represents a source for problems for developers.

Typical classes of bugs
encountered when using a cast are using the wrong cast target type, or using
the wrong operand, or failing to guard a cast.
More subtle, however, is the interaction between casting and generics.
For example, the following call,%
\footnote{\url{https://github.com/ethereum/ethereumj/commit/224e65b9b4ddcb46198a6f8faf69edc65d34d382}}
throws a \code{ClassCastException}.
\begin{minted}[linenos=false]{java}
config.getProperty("peer.p2p.pingInterval", 5L)
\end{minted}
The method has the following signature:%
\footnote{\url{https://github.com/ethereum/ethereumj/blob/224e65b9b4ddcb46198a6f8faf69edc65d34d382/ethereumj-core/src/main/java/org/ethereum/config/SystemProperties.java\#L312}}
\begin{minted}[linenos=false]{java}
public <T> T getProperty(String propName, T defaultValue)
\end{minted}
The first argument to the method is the name of a property, used to lookup a
value in a table. The second argument is a default value to use if the
property is not in the table. If the lookup is successful, the method casts
the value found to type \code{T}.
%
In the call, the given property
\code{"peer.p2p.pingInterval"} is in the table and mapped to an \code{Integer}.
However, \java{} uses the type of the \code{defaultValue} argument, in this
case \code{Long}, to instantiate the type parameter \code{T}.

Note, however, that the cast inside \code{getProperty}, which in this context
should cast from \code{Integer} to \code{Long}, \emph{does not fail}.
This is because the \java{} compiler erases the type
parameters like \code{T} and so dynamic type tests are not performed on them.
Instead, the compiler inserts a cast where the return value
of \code{getProperty} is used later with type \code{Long}.
It is this cast that fails at run time and that is reported at run time.

The fix for this bug is to change the default value argument from \code{5L}
to just \code{5}. This causes the call's return type is inferred to be
\code{Integer}, and the compiler-inserted cast succeeds.
As this example shows, problems with casts are not always obvious.
In this thesis we aim to uncover the many different ways in which developers use casts
by manually analyzing a large sample of cast usages in open source software.

\subsection*{Outline}

In Section~\ref{sec:casts:methodology} we introduce the methodology we used to analyze casts and to devise cast usage patterns.
Sections~\ref{sec:casts:overview} and \ref{sec:casts:patterns} present the cast usage patterns and answers our research questions,
while Section~\ref{sec:casts:conclusions} concludes.