\section{Casts in \java{}}
\label{sec:casts:background}

While casts should be familiar to most developers of object-oriented languages,
because casts have different semantics in different programming languages,
we briefly summarize the meaning of casts in \java{} and the terminology used in the rest of this chapter.

One common extension of type systems is \emph{subtyping},
usually seen in \emph{object-oriented} programming languages like \java{}.
The subtype mechanism allows the interoperability of two different but related types.
As~\cite{pierceTypesProgrammingLanguages2002} states,
"[\ldots] $S$ is a subtype of $T$, written $S <: T$,
to mean that any term of type $S$ can safely be used in a context where a term of type $T$ is expected.
This view of subtyping is often called the \emph{principle of safe substitution}."
Conversely, if $S$ is a subtype of $T$, we say that $T$ is a supertype of $S$.

A cast operation, written \code{(T) e} in \java{}
consists of a \emph{target type} \code{T} and an \emph{operand} \code{e}.
The operand evaluates to a \emph{source value} which has a run-time
\emph{source type}.
In \java{}, a source reference type is always a class type.
For a particular cast evaluated at run time, 
the \emph{source} of the cast is the expression in the program that
created the source value.
For reference casts, the source is an object allocation.
The source may or may not be known statically.

An \emph{upcast} occurs when the cast is from a source reference type $S$ to a target reference type $T$,
where $T$ is a supertype of $S$.
In our terminology, upcasts include identity casts where the target type
is the same as the type of the operand.
An upcast does not require a run-time check.

A \emph{downcast}, on the other hand,
occurs when converting from a source reference type $S$ to a target reference
type $T$, where $T$ is a proper subtype of $S$.
Listing~\ref{lst:casts:simple} shows how to use the cast operator (line 2) to treat a reference (the variable \code{o}) as a different type (\code{String}) as it was defined (\code{Object}).

\begin{listing}
\begin{minted}{java}
Object o = "foo"; 
String s = (String)o;
\end{minted}
\caption{Variable \code{o} (defined as \code{Object}) cast to \code{String}.}
\label{lst:casts:simple}
\end{listing}

In type-safe OO languages, downcasts require
a run-time check to ensure that the source value is an instance of the target type.
The above snippet is compiled into the \java{} bytecode shown in listing~\ref{lst:casts:bytecode}.
The \code{aload\_1} instruction (line 4) pushes the local variable \code{o} into the operand stack.
The \code{checkcast} instruction (line 5) then checks at run-time that the top of the stack has the specified type (\code{java.lang.String} in this example).

\begin{listing}
\begin{minted}[escapeinside=||]{gas}
|\bcbox|
ldc           #2     // String foo 
astore_1
aload_1
checkcast     #3     // class java/lang/String
astore_2
\end{minted}	
\caption{Compiled bytecode to the \code{checkcast} instruction.}
\label{lst:casts:bytecode}
\end{listing}

This run-time check can either \emph{succeed} or \emph{fail}.
A \code{ClassCastException} is thrown when a downcast fails.
This exception is an unchecked exception, \ie{},
the programmer is neither required to handle nor to specify the exception in the method signature.
Listing~\ref{lst:casts:catchcce} shows how to detect whether a cast failed by catching this exception.

\begin{listing}
\begin{minted}{java}	
try {
	Object x = new Integer(0);
	System.out.println((String)x); 
} catch (ClassCastException e) { 
	System.out.println(""); 
}
\end{minted}
\caption{Catch \code{ClassCastException} when a cast fails.}
\label{lst:casts:catchcce}
\end{listing}

A \emph{guard} is a conditional expression on which a cast---usually a downcast--- is control-dependent and that ensures that the cast is evaluated only if it will succeed.
Guards are often implemented using the \code{instanceof} operator, which tests
if an expression is an instance of a given reference type.
If an \code{instanceof} guard returns true, the guarded cast should not throw
a \code{ClassCastException}.
Listing~\ref{lst:casts:instanceoftest} shows a usage of the \code{instanceof} operator together with a cast expression.

\begin{listing}
\begin{minted}{java}
if (x instanceof Foo) {
	((Foo)x).doFoo();
}
\end{minted}
\caption{Runtime type test using \code{instanceof} before applying a cast.}
\label{lst:casts:instanceoftest}
\end{listing}

In \java{}, an object's type can also be checked using reflection:
the \code{getClass} method%
\footnote{\url{https://docs.oracle.com/javase/8/docs/api/java/lang/Object.html\#getClass--}}
returns the run-time class of an object.
This \code{Class} object can be then compared against a class literal, \eg,
\code{x.getClass()}~\code{==}~\code{C.class}.
This test is more precise than an \code{x}~\code{instanceof}~\code{C} test since it succeeds only when the operand's class is exactly \code{C},
rather than any subclass of \code{C}.
Listing~\ref{lst:casts:getclasstest} shows how to use the \code{getClass} method to test for an object's type.

\begin{listing}
\begin{minted}{java}
if (x.getClass() == Foo.class) {
	((Foo)x).doFoo(); 
}
\end{minted}
\caption{Runtime type test using \code{getClass} before applying a cast.}
\label{lst:casts:getclasstest}
\end{listing}

Because they can fail,
downcasts pose potential threats.
Unguarded downcasts in particular are
worrisome because the developer is essentially telling the compiler
\emph{``Trust me, I know what I'm doing.''}
Because downcasts are an escape-hatch from the static type system---they
permit dynamic type errors---a cast is often seen as a design flaw or code
smell in an object-oriented system~\citep{tufanoWhenWhyYour2015}.

A cast can also fail at compile time if the cast operand and the target type are incompatible.
For instance, in the expression \code{(String) new Integer(1)} a value of type
\code{Integer} can \emph{never} be converted to \code{String}, so the compiler
rejects the cast expression.

Another form of casts in \java{} are \emph{primitive conversions}, or more specifically
\emph{numeric conversions}. These are conversions from
one primitive (non-reference) type, usually a numeric type, to another. These conversions can result
in loss of precision of the numeric value, although they do not fail with a
run-time exception.

\emph{Boxing} and \emph{unboxing} occur when casting from a primitive type to the corresponding reference type or vice versa,
\eg{}, \code{(Integer) 3} converts the primitive \code{int} 3 into a boxed \code{java.lang.Integer}.
Unlike downcasts, unboxing casts never \code{throw} a \code{ClassCastException}.
However, an unboxing conversion throws a \code{NullPointerException} when 
the cast operand is \code{null}, \eg{}, \code{(double) (Double) null}.%
\urlnote{https://docs.oracle.com/javase/specs/jls/se8/html/jls-5.html\#jls-5.1.8}
\java{} supports \emph{autoboxing} and \emph{autounboxing} between primitives and their corresponding boxed type in the \code{java.lang} package.

Generics were introduced into \java{} to provide more static type safety.
For instance, the type \code{List<T>} contains only elements of type \code{T}.
The underlying implementation of generics, however,
erases the actual type arguments when compiling to bytecode.
To ensure type safety in the generated bytecode,
the compiler inserts cast instructions into the generated code.
Improper use of generic types or mixing of generic and raw types can lead
to dynamic type errors---\ie, \code{ClassClassException}.
Our study, however, does not consider these compiler-inserted casts.
Moreover, upcasts inserted by the developer in the source code are completely removed from the generated bytecode by the compiler.
Our first attempt to conduct this study was to use our bytecode library
analysis~\citep{mastrangeloJNIFJavaNative2014} described in Appendix~\ref{ap:jnif}.
Nevertheless, unlike our \unsafe{} study---which targeted \java{} bytecode level---in this chapter we are only concerned with programmer-inserted casts in the source code,
not in the generated bytecode---given the discrepancy between them.
