\section{Research Questions}

The examples in the previous section indicates that type casts represent a source of errors for developers.
Therefore we want to understand why developers need to use type casts.
To this end, we propose to answer the following question:
\emph{How and when do developers need to escape the type system?}
The cast operator in \java{} provides the means to view a reference as a different type as it was declared.
Upcasts are done automatically by the compiler.
Nevertheless, as we shall see later, in some situations a developer is forced to insert upcasts.
In the case of downcasts, a check is inserted at run-time to verify that the conversion is sound, thus escaping the static type system.

Therefore, we believe we should care about how the casting operations are used in the wild.
Specifically, we want to answer the following research questions:

\begin{enumerate}[label=$RQ/C\arabic*:$,ref=$RQ/C\arabic*$,leftmargin=3.4\parindent]
\item\label{enum:rq1}{\bf \crqA}
We want to understand to what extent application code actually uses cast operations.
\item\label{enum:rq2}{\bf \crqB}
If casts are actually used in application code, we want to know how and why developers need to escape the type system.
\item\label{enum:rq3}{\bf \crqC}
In addition to understand how and when casts are used, we want to measure how often developers need to resort to certain idioms to solve a particular problem.
\end{enumerate}

To answer the above questions, we need to determine whether and how cast operations are actually used in real-world \java{} applications.
In \S\ref{sec:casts:stats} we first give an estimation of how often the cast operator is used in real-world applications to answer~\ref{casts:rq1}.
In \S\ref{sec:casts:methodology} we introduce the methodology we have used to devise cast usage patterns.
Then, \S\ref{sec:casts:patterns} presents the cast usage patterns we have devised to answer both~\ref{casts:rq2} and \ref{casts:rq3}.
