\begin{pattern}{Deserialization}
Used to deserialize an object.

\instances{}
The following example%
\footnote{\url{http://bit.ly/internetarchive_heritrix3_2SF4j7k}}
shows how the \thisp{} pattern is used to create objects from a file system.

%https://lgtm.com/projects/g/internetarchive/heritrix3/snapshot/dist-12140105-1524814812150/files/engine/src/test/java/org/archive/crawler/datamodel/CrawlURITest.java?sort=name&dir=ASC&mode=heatmap#L83
\begin{minted}[highlightlines=19]{java}
final public void testSerialization()
        throws IOException, ClassNotFoundException {
    File serialize = new File(getTmpDir(),
            this.getClass().getName() + ".serialize");
    try {
        FileOutputStream fos = new FileOutputStream(serialize);
        ObjectOutputStream oos = new ObjectOutputStream(fos);
        oos.writeObject(this.seed);
        oos.reset();
        oos.writeObject(this.seed);
        oos.reset();
        oos.writeObject(this.seed);
        oos.close();
        // Read in the object.
        FileInputStream fis = new FileInputStream(serialize);
        ObjectInputStream ois = new ObjectInputStream(fis);
        CrawlURI deserializedCuri = (CrawlURI)ois.readObject();
        deserializedCuri = (CrawlURI)ois.readObject();
        deserializedCuri = (CrawlURI)ois.readObject();
        assertEquals("Deserialized not equal to original",
                this.seed.toString(), deserializedCuri.toString());
        String host = this.seed.getUURI().getHost();
        assertTrue("Deserialized host not null",
                host != null && host.length() >= 0);
    } finally {
        serialize.delete();
    }
}
\end{minted}

\detection{}

\discussion{}

\related{}

\end{pattern}