
\begin{pattern}{ReflectiveAccessibility}

This pattern accesses a field of an object by means of reflection.
It uses reflection because at compile time the field is unaccesible.
Usually the method \code{setAccessible(true)} is invoked on the field
before actually getting the value from an object.

\instances{}
The following cast%
\footnote{\url{http://bit.ly/loopj_android-async-http_2SOISRr}}
uses this pattern:

%https://lgtm.com/projects/g/loopj/android-async-http/snapshot/dist-1879340034-1529316783166/files/library/src/main/java/com/loopj/android/http/AsyncHttpClient.java?sort=name&dir=ASC&mode=heatmap&showExcluded=false#L445
\begin{listing}
\caption{Using \code{Field::get} to gain access to a field.}
\begin{minted}[highlightlines=2]{java}
f.setAccessible(true);
HttpEntity wrapped = (HttpEntity) f.get(entity);
\end{minted}
\end{listing}

\end{pattern}
