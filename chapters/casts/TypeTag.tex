\begin{pattern}{TypeTag}
%
\done{Matthias: Your first example doesn't have the tag in the *same* object, but passes it as a separate variable.}
%
A cast instance belonging to the \thisp{} pattern is guarded by an application-specific test instead of using an \code{instanceof} test.

\instances{}
The following example%
\footnote{\url{http://bit.ly/JesusFreke_smali_2Ho8bVL}}
shows an instance of the \thisp{} pattern.
The cast type of the parameter \code{reference} is determined by the value of the parameter \code{referenceType}.

%https://lgtm.com/projects/b/JesusFreke/smali/snapshot/dist-1306230039-1524814812150/files/dexlib2/src/main/java/org/jf/dexlib2/writer/InstructionWriter.java?sort=name&dir=ASC&mode=heatmap#L492
\begin{minted}[highlightlines=8]{java}
private int getReferenceIndex(int referenceType, Reference reference) {
    switch (referenceType) {
        case ReferenceType.FIELD:
            return fieldSection.getItemIndex((FieldRefKey) reference);
        case ReferenceType.METHOD:
            return methodSection.getItemIndex((MethodRefKey) reference);
        case ReferenceType.STRING:
            return stringSection.getItemIndex((StringRef) reference);
        case ReferenceType.TYPE:
            return typeSection.getItemIndex((TypeRef) reference);
        case ReferenceType.METHOD_PROTO:
            return protoSection.getItemIndex((ProtoRefKey) reference);
        default:
            throw new ExceptionWithContext(
                "Unknown reference type: %d",  referenceType);
    }
}
\end{minted}

In the next example,%
\footnote{\url{http://bit.ly/FenixEdu_fenixedu-academic_2SUNOUJ}}
instead of a \code{switch} statement,
an \code{if} statement is used to guard the cast (in line 6).

%https://lgtm.com/projects/g/FenixEdu/fenixedu-academic/snapshot/dist-29270029-1524814812150/files/src/main/java/org/fenixedu/academic/ui/renderers/student/curriculum/StudentCurricularPlanRenderer.java?sort=name&dir=ASC&mode=heatmap#L853
\begin{minted}[highlightlines=6]{java}
for (final IEnrolment enrolment : dismissal.getSourceIEnrolments()) {
    if (enrolment.isExternalEnrolment()) {
        generateExternalEnrolmentRow(mainTable, (ExternalEnrolment) enrolment,
                level + 1, true);
    } else {
        generateEnrolmentRow(mainTable, (Enrolment) enrolment,
                level + 1, false, true, true);
    }
}
\end{minted}

\done{Nate: Why is this not pattern matching?}
%
In the next case%
\footnote{\url{http://bit.ly/apache_poi_2FW5SXU}}
a type test is performed --- through a method call --- before actually applying the cast to the variable \code{props} (line 3).
Note that the type test is internally using the \code{instanceof} operator (line 8).
Although in this case the type test is using an \code{instanceof} operator,
it is not considered \nameref{pat:PatternMatching} because the \code{instanceof} is located in a method call.

%https://lgtm.com/projects/g/apache/poi/snapshot/dist-1790760597-1524814812150/files/src/ooxml/java/org/apache/poi/xslf/usermodel/XSLFPropertiesDelegate.java?sort=name&dir=ASC&mode=heatmap#L1367
\begin{minted}[highlightlines=3]{java}
@Override
public CTSolidColorFillProperties getSolidFill() {
    return isSetSolidFill() ? (CTSolidColorFillProperties)props : null;
}

@Override
public boolean isSetSolidFill() {
    return (props instanceof CTSolidColorFillProperties);
}
\end{minted}

\detection{}
The detection of this pattern is similar to the \nameref{pat:PatternMatching} detection, but instead of looking for an \code{instanceof} guarded cast, we look for an application-specific guard.

\discussion{}
In some cases, the \thisp{} could be replaced by \nameref{pat:PatternMatching}.

\related{}
%
\todo{Nate: PatternMatching}
Related to \nameref{pat:PatternMatching}

\end{pattern}