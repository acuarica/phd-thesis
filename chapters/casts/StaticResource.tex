\begin{pattern}{StaticResource}
A cast to a method access to that reads a static resource file, \eg,
XML, HTML or \java{} properties file.
The file is static since its contents are known at compile-time.
Usually this file is build with a third-party tool, like a GUI designer.

\instances{}
In the following example,%
\footnote{\url{http://bit.ly/pwittchen_NetworkEvents_2HGbrMq}}
a cast is applied to a \code{findViewById} method invocation.
The \code{findViewById} method looks up for the given ID in a XML resource file to retrieve the specified view. 

%https://lgtm.com/projects/g/pwittchen/NetworkEvents/snapshot/dist-2032650416-1524814812150/files/example/src/main/java/com/github/pwittchen/networkevents/app/MainActivity.java?sort=name&dir=ASC&mode=heatmap#L65
\begin{minted}[highlightlines=6]{java}
@Override
protected void onCreate(Bundle savedInstanceState) {
    super.onCreate(savedInstanceState);
    setContentView(R.layout.activity_main);
    connectivityStatus = (TextView) findViewById(R.id.connectivity_status);
    mobileNetworkType = (TextView) findViewById(R.id.mobile_network_type);
    accessPoints = (ListView) findViewById(R.id.access_points);
    busWrapper = getOttoBusWrapper(new Bus());
    networkEvents = new NetworkEvents(getApplicationContext(), busWrapper)
        .enableInternetCheck()
        .enableWifiScan();
}
\end{minted}

The next listing,%
\footnote{\url{http://bit.ly/pentaho_pentaho-kettle_2TswNSf}}
shows a cast to a GUI component (\code{XulListbox}) using the \code{getElementById} method (lines 12 and 13).
In this case the developer is using the XUL language.%
\footnote{\url{https://developer.mozilla.org/en-US/docs/Mozilla/Tech/XUL}}

%https://lgtm.com/projects/g/pentaho/pentaho-kettle/snapshot/dist-1815472020-1524814812150/files/ui/src/main/java/org/pentaho/di/ui/repository/controllers/RepositoriesController.java?sort=name&dir=ASC&mode=heatmap#L115
\begin{minted}[highlightlines=12-13]{java}
private void createBindings() {
    loginDialog = (XulDialog) document
                    .getElementById( "repository-login-dialog" );
    repositoryEditButton = (XulButton) document
                    .getElementById( "repository-edit" );
    repositoryRemoveButton = (XulButton) document
                    .getElementById( "repository-remove" );
    username = (XulTextbox) document
                    .getElementById( "user-name" );
    userPassword = (XulTextbox) document
                    .getElementById( "user-password" );
    availableRepositories = (XulListbox) document
                    .getElementById( "available-repository-list" );
    showAtStartup = (XulCheckbox) document
                    .getElementById( "show-login-dialog-at-startup" );
    okButton = (XulButton) document
                    .getElementById( "repository-login-dialog_accept" );
    cancelButton = (XulButton) document
                    .getElementById( "repository-login-dialog_cancel" );
    // [...]
}
\end{minted}


\detection{}
To detect this pattern, we need to identify well-known frameworks that use static resource.
Using our methodology,
we have identified the Android API and Mozilla XUL language.

\discussion{}
%
\done{Nate: Could be determined at compile-time.}
%
These casts could be solved by using code generation,
or partial classes like in \csharp{}.
Since the contents of the resource file are known at compile-time,
code generation could be used to generate the corresponding \java{} code.

This is a pattern most often seen when using the Android platform.
The Butter Knife framework%
\footnote{\url{http://jakewharton.github.io/butterknife/}}
make use of annotations to avoid the ``manual'' casting.
Instead, code is generated the cast the result of \code{findViewById}.
%
\done{Luis: Mention frameworks to avoid this cast using annotation.}

\related{}
This pattern is similar to \nameref{pat:LookupById},
since both use a key or ID to look up in a collection and cast the result.
However, the difference is how the content value was generated.
In the \nameref{pat:LookupById} pattern,
the developer ensures in another class the return value,
whereas in the \thisp{} pattern the content is given by a static resource file.
%
\done{Nate: Why not LookupById?}
%

\end{pattern}