\begin{pattern}{StaticResource}
%
\todo{Nate: Why not LookupById?}
%
A cast to a method access to \code{findViewById} or a method that reads a static resource.
This is a pattern seen when using the Android platform. 

\instances{}

\footnote{\url{http://bit.ly/2HGbrMq}}

%https://lgtm.com/projects/g/pwittchen/NetworkEvents/snapshot/dist-2032650416-1524814812150/files/example/src/main/java/com/github/pwittchen/networkevents/app/MainActivity.java?sort=name&dir=ASC&mode=heatmap#L65
\begin{minted}[highlightlines=6]{java}
@Override
protected void onCreate(Bundle savedInstanceState) {
    super.onCreate(savedInstanceState);
    setContentView(R.layout.activity_main);
    connectivityStatus = (TextView) findViewById(R.id.connectivity_status);
    mobileNetworkType = (TextView) findViewById(R.id.mobile_network_type);
    accessPoints = (ListView) findViewById(R.id.access_points);
    busWrapper = getOttoBusWrapper(new Bus());
    networkEvents = new NetworkEvents(getApplicationContext(), busWrapper)
        .enableInternetCheck()
        .enableWifiScan();
}
\end{minted}

\detection{}

\discussion{}
%
\todo{Nate: Could be determined at compile-time.}
%
These casts could be solved by using code generation,
or partial classes as in \csharp{}.

\end{pattern}