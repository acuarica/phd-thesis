\section{Overview of our Study}
\label{sec:casts:overview}

We propose to answer the following question:
\emph{How and when do developers need to escape the type system?}
The cast operator in \java{} provides the means to view a reference at a different type as it was declared.
Upcasts conversions are done automatically by the compiler.
Nevertheless, as we shall see later, in some situations a developer is forced to insert upcasts.
In the case of downcasts, a check is inserted at run-time to verify that the conversion is sound, thus escaping the type system.
\emph{Why is so?}
Therefore, we believe we should care about how the casting operations are used in the wild.
Specifically, we want to answer the following research questions:

\begin{enumerate}[label=$CRQ\arabic*:$,ref=$CRQ\arabic*$,leftmargin=3.4\parindent]
\item\label{enum:rq1}{\bf \crqA}
We want to understand to what extent application code actually uses casting operations.
\item\label{enum:rq2}{\bf \crqB}
If casts are actually used in application code, we want to know how and why developers need to escape the type system.
\item\label{enum:rq3}{\bf \crqC}
In addition to understand how and why casts are used, we want to measure how often developers need to resort to certain idioms to solve a particular problem.
\end{enumerate}

To answer the above questions, we need to determine whether and how casting operations are actually used in real-world \java{} applications.
To achieve our goal, several elements are needed.

\textbf{Source Code Analysis.}
We have implemented our study using the \ql{} query language:
``a declarative, object-oriented logic programming language for querying complex, potentially recursive data structures encoded in a relational data model''~\citep{avgustinovQLObjectorientedQueries2016}.
\ql{} allows us to analyze programs at the source code level by abstracting the code sources into a Datalog model.
Besides providing structural data for programs, \ie{}, ASTs,
\ql{} has the ability to query static types and perform data-flow analysis.
To run our \ql{} queries, we have used the service provided by Semmle.\footnote{\url{https://lgtm.com/}} 

\textbf{Projects.} 
As a code base representative of the ``real world'',
we have chosen open-source projects hosted in 
\github{},
the world-most popular source code management repository.
% , \ie{},
% \github{},
%\footnote{\url{https://github.com/}},
% \gitlab{},
%\footnote{\url{https://gitlab.com/}},
% \bitbucket{},
%\footnote{\url{https://bitbucket.org/}}.
So far, we have analyzed \nproject{} \java{} projects in \lgtm{}.
We plan to scale up our analysis to the whole \lgtm{} project database.

\textbf{Usage Pattern Detection.}
After all cast instances are found, we analyze this information to discover usage patterns.
\ql{} allows us to automatically categorize cast use cases into patterns.
This methodology is described in section~\ref{sec:casts:methodology}.

Our list of patterns is not exhaustive.
Due to the nature of the cast operator, some casts were uncategorized as they would need a whole program analysis, \eg{}, including libraries in the analysis.