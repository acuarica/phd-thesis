
\section{Overview of the Sampled Casts}
\label{sec:casts:overview}

The casts we sampled are summarized in Table~\ref{table:casts:guarded}.
In our sample of \nSize{} casts,
we found \nPrimitivePattern{} primitive conversions.
Moreover, we found \nBrokenLink{} links that were not accessible during our
analysis, making manual code inspection impossible.
We also found one cast that was clearly a bug,
a downcast using the wrong cast operand.
The remaining \nReference{} casts are either reference upcasts, downcasts, boxing
casts, or unboxing casts.

\begin{table*}[h]
\scriptsize
\centering
\caption{Statistics on Sampled Casts}
\label{table:casts:guarded}
\begin{tabular}{|l|r|r|}
  \hline
  All sampled casts & \nSize{} & 100\% \\
  \hline
  Reference casts & \nReference{} & \pReference\% \\
  Primitive casts & \nPrimitivePattern{} & \pPrimitivePattern\% \\
%  Bug & \nBug{} & \pBug\% \\
%  Source code inaccessible & \nBrokenLink{} & \pBrokenLink\% \\
  \hline
  Upcasts & \nUpcast{} & \pUpcast\% \\
  Downcasts & \nDowncast{} & \pDowncast\% \\
  \hline
  Boxing casts & \nToRemoveBoxingSubpattern{} & \pToRemoveBoxingSubpattern\% \\
  Unboxing casts & \nToRemoveUnboxingSubpattern{} & \pToRemoveUnboxingSubpattern\% \\
  \hline
  Guarded by \code{instanceof} & \nTypecaseGuardByInstanceOfSubpattern{} & \pTypecaseGuardByInstanceOfSubpattern\% \\
  Guarded by \code{getClass} & \nTypecaseGuardByClassLiteralSubpattern{} & \pTypecaseGuardByClassLiteralSubpattern\% \\
  Guarded by type tag & \nTypecaseGuardByTypeTagSubpattern{} & \pTypecaseGuardByTypeTagSubpattern\% \\
  Unguarded or possibly unguarded & \nUnguarded{} & \pUnguarded\% \\
  \hline
\end{tabular}
\end{table*}

Casts can be classified as either \emph{guarded} or \emph{unguarded} casts.
A guard is a conditional expression on which the cast is control dependent,
which, if successful, ensures the cast will not fail.
Guards are typically implemented using the \code{instanceof} operator or using
a test of the source value's class (retrieved using the
\code{Object::getClass} method) against a subtype of the cast target type.
Guards can also be implemented in an application-specific manner, for instance
by associating a ``type tag'' with the source value that can be used to
distinguish the run-time type.

Of the \nReference{} analyzed reference casts,
we found that \nGuarded{} (\pGuarded\%) were guarded by a
guard in the same method as the cast and \nUnguarded{} (\pUnguarded\%)
were either unguarded or had a guard in another method.
In the latter case, which we refer to as \emph{possibly unguarded},
determining by manual inspection if a guard is actually
present is often infeasible. The possibly unguarded casts are cases where the application developer
has some reason for believing the cast will succeed, but it is not immediately
apparent in the source code.

As we describe in the next section, nearly all guarded casts fit into just a
few patterns. Unguarded or possibly unguarded casts account for most of the
patterns.