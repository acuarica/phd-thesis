\begin{pattern}{NewDynamicInstance}
%
\todo{Nate: Why not Creational?}
%
Dynamically creates an object or array by means of reflection.
The \code{newInstance} method family declared in the \code{Class},%
\footnote{\url{https://docs.oracle.com/javase/8/docs/api/java/lang/Class.html\#newInstance--}}
\code{Array}\footnote{\url{https://docs.oracle.com/javase/8/docs/api/java/lang/reflect/Array.html\#newInstance-java.lang.Class-int-}}\(^{,}\)
\footnote{\url{https://docs.oracle.com/javase/8/docs/api/java/lang/reflect/Array.html\#newInstance-java.lang.Class-int...-}}
and \code{Constructor}%
\footnote{\url{https://docs.oracle.com/javase/8/docs/api/java/lang/reflect/Constructor.html\#newInstance-java.lang.Object...-}}
classes creates an object or array dynamically by means of reflection, \ie, the type of object being created is not known at compile-time.
This pattern consists of casting the result of these methods to the appropriate target type.

\instances{}
The following example%
\footnote{\url{http://bit.ly/2HC3IPg}}
shows a cast to the \code{Class.newInstance()}
method.

%https://lgtm.com/projects/g/apache/hadoop/snapshot/6bedbef6c5f2d937a6cbc268300ce2a39609d06c/files/hadoop-hdfs-project/hadoop-hdfs/src/main/java/org/apache/hadoop/hdfs/server/namenode/FSNamesystem.java?sort=name\&dir=ASC\&mode=heatmap\&showExcluded=false#L1039
\begin{minted}[highlightlines=1]{java}
logger = (AuditLogger) Class.forName(className).newInstance();
\end{minted}

The following example%
\footnote{\url{http://bit.ly/2Hp5Hqc}}
shows how to dynamically create an array, using the \texttt{Array} class.

%https://lgtm.com/projects/g/neo4j/neo4j/snapshot/27aaa67633e4d26446e38125d04fbbd27f938b75/files/community/collections/src/main/java/org/neo4j/helpers/collection/Iterables.java?sort=name\&dir=ASC\&mode=heatmap\&showExcluded=false#L403
\begin{minted}[highlightlines=1]{java}
return list.toArray( (T[]) Array.newInstance( componentType, list.size()));
\end{minted}

Whenever a constructor other than the default constructor is needed,
the \code{newInstance} method declared in the \code{Constructor} class
should be used to select the appropriate constructor,
as shown in the following example.
\footnote{\url{http://bit.ly/2HsUgOo}}

%https://lgtm.com/projects/g/gradle/gradle/snapshot/209c3175e75af6ac30cb66c02eda15b0f8b6a616/files/subprojects/internal-integ-testing/src/main/groovy/org/gradle/integtests/fixtures/executer/OutputScrapingExecutionFailure.java?sort=name\&dir=ASC\&mode=heatmap\&showExcluded=false#L174
\begin{minted}[highlightlines=1]{java}
return (Exception) Class
                       .forName(className)
                       .getConstructor(String.class)
                       .newInstance(message);
\end{minted}

The following example%
\footnote{\url{http://bit.ly/2HC33xg}}
shows a guarded instance of the \thisp{} pattern.
This seems rather unusual, as this pattern is not guarded.

%https://lgtm.com/projects/g/alibaba/LuaViewSDK/snapshot/dist-2037250419-1524814812150/files/Android/LuaViewSDK/src/com/taobao/luaview/global/LuaViewManager.java?sort=name&dir=ASC&mode=heatmap#L373
\begin{minted}[highlightlines=6]{java}
private static List<String> getMapperMethodNames(final Class clazz) {
    try {
        if (clazz != null) {
            Object obj = clazz.newInstance();
            if (obj instanceof BaseMethodMapper) {
                return ((BaseMethodMapper) obj).getAllFunctionNames();
            }
        }
    } catch (Exception e) {
        e.printStackTrace();
    }
    return null;
}
\end{minted}

There are cases when the cast is not directly applied to the result of the \code{newInstance} method.
The following snippet shows such a case.%
\footnote{\url{http://bit.ly/2HJtXUn}}
The cast is used to convert from \code{Class<?>} to \code{Class<ConfigFactory>} (line 4).
The \code{newInstance} invocation then does not need a direct cast (line 8) given the definition of the \code{clazz} variable (line 2).
Nevertheless, the cast is unchecked, and a \code{checkcast} instruction is going to be emitted anyway for the result of the \code{newInstance} invocation.

%https://lgtm.com/projects/g/pac4j/pac4j/snapshot/dist-4840350-1524814812150/files/pac4j-core/src/main/java/org/pac4j/core/config/ConfigBuilder.java?sort=name&dir=ASC&mode=heatmap#L25
\begin{minted}[highlightlines=4]{java}
ClassLoader tccl = Thread.currentThread().getContextClassLoader();
final Class<ConfigFactory> clazz;
if (tccl == null) {
    clazz = (Class<ConfigFactory>) Class.forName(factoryName);
} else {
    clazz = (Class<ConfigFactory>) Class.forName(factoryName, true, tccl);
}
final ConfigFactory factory = clazz.newInstance();
\end{minted}


\detection{}
This detection query looks for casts,
where the expression being cast is a call site to methods mentioned above.

\discussion{}
The cast here is needed because of the dynamic essence of reflection.
This pattern is mostly unguarded, that is,
the application programmer knows what target type is being created.

The following two code snippets:

\begin{minted}{java}
Class<?> c = Class.forName("java.lang.String");
String pf = (String) c.newInstance();
\end{minted}

\begin{minted}{java}
Class<String> c = (Class<String>) Class.forName("java.lang.String");
String pf = c.newInstance();
\end{minted}

compile to the same bytecode below.

\begin{lstlisting}[style=bytecode]
ldc           #24    // String java.lang.String
invokestatic  #26    // Method java/lang/Class.forName
astore_1
aload_1
invokevirtual #32    // Method java/lang/Class.newInstance
checkcast     #36    // class java/lang/String
\end{lstlisting}

\related{}
Reflection.

\end{pattern}