\begin{pattern}{NewDynamicInstance}
Dynamically creation of object by means of reflection.
These are the casts that can not be avoidable.

The \code{newInstance} method family declared in the
\code{Class}\footnote{\url{https://docs.oracle.com/javase/8/docs/api/java/lang/Class.html\#newInstance--}},
\code{Array}\footnote{\url{https://docs.oracle.com/javase/8/docs/api/java/lang/reflect/Array.html\#newInstance-java.lang.Class-int-}}\(^{,}\)
\footnote{\url{https://docs.oracle.com/javase/8/docs/api/java/lang/reflect/Array.html\#newInstance-java.lang.Class-int...-}} and
\code{Constructor}\footnote{\url{https://docs.oracle.com/javase/8/docs/api/java/lang/reflect/Constructor.html\#newInstance-java.lang.Object...-}}
classes creates an object or array by means of reflection.

This pattern consists of casting the result of these methods to the appropriate target type.

\instances

%https://lgtm.com/projects/g/apache/hadoop/snapshot/6bedbef6c5f2d937a6cbc268300ce2a39609d06c/files/hadoop-hdfs-project/hadoop-hdfs/src/main/java/org/apache/hadoop/hdfs/server/namenode/FSNamesystem.java?sort=name\&dir=ASC\&mode=heatmap\&showExcluded=false#L1039

The following example shows a cast from the \code{Class.newInstance()}
method.
% \footnote{\url{d}}

\begin{lstlisting}[style=java,caption={The \pname{} pattern using the \texttt{Class} class.}]
logger = (AuditLogger) Class.forName(className).newInstance();
\end{lstlisting}

%https://lgtm.com/projects/g/neo4j/neo4j/snapshot/27aaa67633e4d26446e38125d04fbbd27f938b75/files/community/collections/src/main/java/org/neo4j/helpers/collection/Iterables.java?sort=name\&dir=ASC\&mode=heatmap\&showExcluded=false#L403
The following example shows how to dynamically create an array.
%\footnote{\url{d}}

\begin{lstlisting}[style=java,caption={Example of the \pname{} pattern using the \texttt{Array} class.}]
return list.toArray( (T[]) Array.newInstance( componentType, list.size()));
\end{lstlisting}

%https://lgtm.com/projects/g/gradle/gradle/snapshot/209c3175e75af6ac30cb66c02eda15b0f8b6a616/files/subprojects/internal-integ-testing/src/main/groovy/org/gradle/integtests/fixtures/executer/OutputScrapingExecutionFailure.java?sort=name\&dir=ASC\&mode=heatmap\&showExcluded=false#L174

Whenever a constructor other than the default constructor is needed,
the \code{newInstance} method declared in the \code{Constructor} class
should be used to select the appropriate constructor,
as shown in the following example.
%\footnote{\url{d}}

\begin{lstlisting}[style=java,caption=Example of the \pname{} pattern using the \code{Constructor} class.]
return (Exception) Class
                       .forName(className)
                       .getConstructor(String.class)
                       .newInstance(message);
\end{lstlisting}

\detection

This detection query looks for casts,
where the expression being cast is a call site to methods mentioned above.

\lstset{language=sql,label=orgc5e83de,caption={Fetching all casts to \texttt{newInstance()}.},captionpos=b,numbers=none,style=ql}
\begin{lstlisting}
import java

predicate isByReflection(string qname) {
  qname = "java.lang.reflect.Array" or 
  qname = "java.lang.Class<?>" or 
  qname = "java.lang.reflect.Constructor<?>"
}

from CastExpr ce, MethodAccess ma, Method m
where ma = ce.getExpr()
  and m = ma.getMethod()
  and m.getName() = "newInstance"
  and isByReflection(m.getDeclaringType().getQualifiedName())
select ce, m.getDeclaringType().getQualifiedName()
\end{lstlisting}

\discussion

The cast here is needed because of the dynamic essence of reflection.
This pattern is unguarded, that is,
the application programmer knows what is the target type being created.
\end{pattern}