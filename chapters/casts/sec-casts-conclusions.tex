\section{Conclusions}
\label{sec:casts:conclusions}

The cast operator in \java{} bridges the gap between compile-time and run-time safety.
We have discovered several cast usage patterns.
We found the rationale behind some cast patterns is due to the inexpressiveness of \java{}'s type system.
On the other hand,
there are patterns that abuse or misuse it.

Many of the patterns we found should be unsurprising to most object-oriented programmers.
That nearly 45\% of casts are (possibly) unguarded 
suggests that developers use application-specific knowledge that cannot be easily encoded in
the type system to ensure the absence of run-time type errors.

Our study provides insight on the boundary between static and dynamic typing,
which may inform research on both static and dynamic,
as well as gradual type systems~\citep{Siek06gradualtyping}.
Conversely, this research can inform the design of extensions of the \java{} type system to reduce the need for casting.
