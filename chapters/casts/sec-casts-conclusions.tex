\section{Conclusions}
\label{sec:casts:conclusions}

Many of the patterns we found should be unsurprising to most object-oriented programmers.
That nearly 45\% of casts are (possibly) unguarded 
suggests that developers use application-specific knowledge that cannot be easily encoded in
the type system to ensure the absense of run-time type errors.

Our study provides insight on the boundary between static and dynamic typing, which may inform
research on both static and dynamic, as well as gradual type systems~\cite{Siek06gradualtyping}.
Conversely, this research can inform the design of extensions of the \java{} type system to reduce the need for casting.
Many programming languages provide features to ameliorate the more common use cases of casts.
For instance,
\kotlin{}'s smart casts couple together \code{instanceof} and cast operation on value, 
providing direct support for the \nameref{pat:Typecase} pattern.
More generally, ML-style pattern matching subsumes this pattern.
Other language features that might at least partially obviate the need for some of the patterns are
intersection types (cf. \nameref{pat:ImplicitIntersectionType}),
and
self types or associated types (cf. 
\nameref{pat:Factory},
\nameref{pat:KnownReturnType},
\nameref{pat:Deserialization},
\nameref{pat:CovariantReturnType},
\nameref{pat:FluentAPI}).
Virtual classes~\cite{gbeta, scalaIndependentlyExtensible} and languages that support
family polymorphism~\cite{ernstFamilyPolymorphism2001}
would help with casts in the \nameref{pat:Family} pattern.

Many cast patterns (\eg, 
\nameref{pat:RemoveWildcard},
\nameref{pat:GenericArray},
\nameref{pat:CovariantGeneric},
\nameref{pat:UnoccupiedTypeParameter})
are used either to 
workaround---or to take advantage of---the erasure of generic type parameters in \java{}.
Reified generics or definition-site, rather than use-site, variance annotations~\cite{altidorTamingWildcardsCombining2011}
would reduce the need for these patterns.

Our study also suggests analyses could be performed 
to improve code quality and eliminate some cast usages, for instance removing redundant casts,
finding opportunities to use generics instead,
or locating code smells (cf. \nameref{pat:UseRawType}, \nameref{pat:KnownReturnType}, \nameref{pat:VariableSupertype}).
We are currently working to define static analyses to detect some of these patterns automatically.
With these analyses, tools can be developed to identify instances of the pattern and to
ensure that they are being implemented properly.