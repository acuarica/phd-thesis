\section{Issues Developers have Applying the Cast Operator}
\label{sec:casts:issues}

\emph{Do cast operations pose a problem for developers?}
Several studies~\citep{kechagiaUndocumentedUncheckedExceptions2014,coelhoUnveilingExceptionHandling2015,zhitnitskyTop10Exception2016}
suggest that in \java{},
the \code{ClassCastException} is in the top ten of exceptions being thrown when analysing stack traces.
These studies have analyzed the exceptions thrown in stack traces.
The exceptions come from third-party libraries \api{}s and the Android \api{},
indicating a misuse of such \api{}s.
\code{ClassCastException} is in the top 10 of exceptions thrown,
thus it represents a problem for developers.

To illustrate the sort of problems developers have when applying casting conversions,
we performed a search for commits and issues including the term
\code{ClassCastException} within projects marked as using the \java{} language
on \github{},
the largest host of source code in the world~\citep{gousiosLeanGHTorrentGitHub2014}.
Our searches returned about 171K commits%
\footnote{\url{https://github.com/search?l=Java&q=ClassCastException&type=Commits}}
and 73K issues,%
\footnote{\url{https://github.com/search?l=Java&q=ClassCastException&type=Issues}}
respectively, at the time of this writing.
At first glance, these results indicate that \code{ClassCastException} indeed
represents a source for problems for developers.

Typical classes of bugs encountered when using a cast are using the wrong cast target type,
or using the wrong operand, or failing to guard a cast.
The following snippet%
\footnote{\url{https://github.com/jenkinsci/extra-columns-plugin/commit/02d10bd1fcbb2e656da9b1b4ec54208b0cc1cbb2}}
shows a cast applied to the variable \code{job} (in line 3) that throws \code{ClassCastException} because the developer forgot to include a guard.
In this case, the developer fixed the error by introducing an \code{instanceof} guard to the cast (lines 1 and 2).

\begin{listing}
\begin{minted}[highlightlines=3]{java}
if(!(job instanceof AbstractProject<?, ?>))
  return "";
AbstractProject<?, ?> project = (AbstractProject<?, ?>) job;
\end{minted}
\caption{Cast throws \code{ClassCastException} because of a forgotten guard.}
\end{listing}

In the next example%
\footnote{\url{https://github.com/GoldenGnu/jeveassets/commit/5f4750bc8cfa7eed8ad01efd8add2cd2cc9bd831}}
the developer made a mistake by choosing a wrong class for the cast target,
\ie, \code{JCustomFileChooser} instead of \code{CustomFileFilter} (line 9).
The \code{CustomFileFilter} is an inner static class inside the \code{JCustomFileFilter} class.
There is no subclass relationship between these two classes.
The cast happens inside an \code{equals} method
---where this idiom is well known---
within the \code{CustomFileFilter} class.
But the developer made a typo, using the outer class (\code{JCustomFileFilter}), instead of the inner class (\code{CustomFileFilter}).

\begin{listing}
\begin{minted}[highlightlines=9]{java}
public final class JCustomFileChooser extends JFileChooser {
  /* [...] */
  public static class CustomFileFilter extends FileFilter {
    /* [...] */
    public boolean equals(Object obj) {
      if (getClass() != obj.getClass()) {
          return false;
      }
      final JCustomFileChooser other = (JCustomFileChooser) obj;
      if (!Objects.equals(this.extensions, other.extensions)) {
          return false;
      }
    }
  }
}
\end{minted} 
\caption{Cast throws \code{ClassCastException} because of wrong cast target.}
\end{listing}

More subtle, however, is the interaction between casting and generics.
For example, the following call to the \code{getProperty} method (line 1),%
\footnote{\url{https://github.com/ethereum/ethereumj/commit/224e65b9b4ddcb46198a6f8faf69edc65d34d382}}
throws a \code{ClassCastException}.
The method definition is shown in line 3.%
\footnote{\url{https://github.com/ethereum/ethereumj/blob/224e65b9b4ddcb46198a6f8faf69edc65d34d382/ethereumj-core/src/main/java/org/ethereum/config/SystemProperties.java\#L312}}

\begin{listing}
\begin{minted}[highlightlines=1]{java}
config.getProperty("peer.p2p.pingInterval", 5L)

public <T> T getProperty(String propName, T defaultValue) {
    if (!config.hasPath(propName)) return defaultValue;
    String string = config.getString(propName);
    if (string.trim().isEmpty()) return defaultValue;
    return (T) config.getAnyRef(propName);
}
\end{minted}
\caption{Cast throws \code{ClassCastException} because of generic inference.}
\end{listing}

The first argument to the method is the name of a property,
used to lookup a value in a table.
The second argument is a default value to use if the property is not in the table.
If the lookup is successful,
the method casts the value found to type \code{T}.
In the call, the given property
\code{"peer.p2p.pingInterval"} is in the table and mapped to an \code{Integer}.
However, \java{} uses the type of the \code{defaultValue} argument, in this
case \code{Long}, to instantiate the type parameter \code{T}.

Note, however, that the cast inside \code{getProperty}, which in this context
should cast from \code{Integer} to \code{Long}, \emph{does not fail}.
This is because the \java{} compiler erases the type
parameters like \code{T} and so dynamic type tests are not performed on them.
Instead, the compiler inserts a cast where the return value
of \code{getProperty} is used later with type \code{Long}.
It is this cast that fails at run time and that is reported at run time.

The fix for this bug is to change the default value argument from \code{5L}
to just \code{5}.
This causes the call's return type is inferred to be
\code{Integer}, and the compiler-inserted cast succeeds.

As these examples show, problems with casts are not always obvious.
In this thesis we aim to uncover the many different ways in which developers use casts
by manually analyzing a large sample of cast usages in open source software.