\section{Cast Usage Patterns}
\label{sec:casts:patterns}

Using the methodology described in the above section, we have devised \npattern{} casts usage patterns.%
\footnote{We have excluded casts that represents primitive numeric type conversions, as they do not represent any pattern.
However, during our manual analysis we found \nprim{} primitive conversions.
Moreover, we found \nbrokenlinks{} links that were not accessible during our analysis.}
In this section we present the cast usage patterns we found.
To ease the patterns presentation, we have clustered them in \ngroup{} groups according to their purpose.
Figure~\ref{fig:patterns} shows our patterns and their occurrences sort by frequency.
The column on the right correspond to the group the pattern belongs to.

\begin{figure}[ht!]
\centering
\includegraphics[width=\textwidth]{analysis/table-patterns-5000.pdf}
\caption{Cast Patterns Occurrences} \label{fig:patterns}
\end{figure}

Each pattern is described using the following template:

\begin{itemize}
\item \textbf{Description.}
Tells what is this pattern about.
It gives a general overview of the structure of the pattern.
\item \textbf{Instances.}
Gives one or more concrete examples found in real code.%
\footnote{Please notice that the snippets presented here were slightly
modified for formatting purposes.
Moreover, to facilitate some snippet presentations,
we remove irrelevant code and replace it with the
comment \code{// [...]}.}
For each instance presented here, we provide the link to the source code repository in \lgtm{}.
We provide the link in case the reader wants to do further inspection
of the snippet presented.%
\footnote{Instead of presenting \lgtm{} long URLs,
we have used the URL shortening service \url{https://bitly.com/}
for an easier reading.}
\item \textbf{Detection.}
Describes briefly how this pattern was detected in terms of the tags introduced in the previous section.
\item \textbf{Discussion.}
Presents suggestions, flaws, or comments about the pattern.
\item \textbf{Related Patterns.}
How the pattern being described relates to other patterns?
\end{itemize}

\group{Guarded}

The patterns in this group are guarded casts.


\begin{pattern}{PatternMatching}
This pattern is composed of a guard (\code{instanceof}) followed by a
cast on known subtypes of the static type.
Often there is just one case and the default case, \ie,
\code{instanceof} fails, does a no-op or reports an error.
Another common approach is to have several cases,
usually one \emph{per} subtype.

\instances{}
The following listing shows an example of the \thisp{} pattern.%
\footnote{\url{http://bit.ly/2FzYYHq}}
In this example, there is only one case

% https://lgtm.com/projects/g/OpenMods/OpenBlocks/snapshot/dist-2040060754-1524814812150/files/build/sources/java/openblocks/common/tileentity/TileEntityImaginary.java?sort=name&dir=ASC&mode=heatmap#L268
\begin{minted}[highlightlines=3]{java}
Item item = helmet.getItem();
if (item instanceof ItemImaginationGlasses)
	return ((ItemImaginationGlasses)item).checkBlock(what, helmet, this);
\end{minted}

Double typecase example
\footnote{\url{http://bit.ly/2FDN9Rd}}

% https://lgtm.com/projects/g/bbossgroups/bboss/snapshot/dist-2025970729-1524814812150/files/bboss-util/src/org/frameworkset/util/ObjectUtils.java?sort=name&dir=ASC&mode=heatmap#L228
\begin{minted}[highlightlines=25]{java}
public static boolean nullSafeEquals(Object o1, Object o2) {
	if (o1 == o2) {
		return true;
	}
	if (o1 == null || o2 == null) {
		return false;
	}
	if (o1.equals(o2)) {
		return true;
	}
	if (o1.getClass().isArray() && o2.getClass().isArray()) {
		if (o1 instanceof Object[] && o2 instanceof Object[]) {
			return Arrays.equals((Object[]) o1, (Object[]) o2);
		}
		if (o1 instanceof boolean[] && o2 instanceof boolean[]) {
			return Arrays.equals((boolean[]) o1, (boolean[]) o2);
		}
		if (o1 instanceof byte[] && o2 instanceof byte[]) {
			return Arrays.equals((byte[]) o1, (byte[]) o2);
		}
		if (o1 instanceof char[] && o2 instanceof char[]) {
			return Arrays.equals((char[]) o1, (char[]) o2);
		}
		if (o1 instanceof double[] && o2 instanceof double[]) {
			return Arrays.equals((double[]) o1, (double[]) o2);
		}
		if (o1 instanceof float[] && o2 instanceof float[]) {
			return Arrays.equals((float[]) o1, (float[]) o2);
		}
		if (o1 instanceof int[] && o2 instanceof int[]) {
			return Arrays.equals((int[]) o1, (int[]) o2);
		}
		if (o1 instanceof long[] && o2 instanceof long[]) {
			return Arrays.equals((long[]) o1, (long[]) o2);
		}
		if (o1 instanceof short[] && o2 instanceof short[]) {
			return Arrays.equals((short[]) o1, (short[]) o2);
		}
	}
	return false;
}
\end{minted}


\detection{}
To detect this pattern, we look

\discussion{}
The \thisp{} pattern can be seen as an \adhoc{}
alternative to pattern matching.
This construct can be seen in several other languages, \eg,
\haskell{}, \scala{}, and \cs{}.
There is an ongoing proposal%
\footnote{\url{http://openjdk.java.net/jeps/305}} to add pattern
matching to the \java{} language.

As a workaround, alternatives to the \thisp{} pattern can be the
visitor pattern or polymorphism.
But in some cases, the chain of \code{instanceof}s is of boxed types.
Thus no polymorphism can be used.

% Maybe this should be called properly instanceof-guarded cast, to be more specific.
% This pattern checks whether a parameter in an overridden method has a more specific type.
% A cast to a variable guarded by an \code{instanceof}.
% A variable is \emph{guarded} by a condition when the condition controls
% that access to the variable, and there is no assignment after the
% condition and before the access to that variable.

The \thisp{} pattern consists of testing the runtime type of a variable against several related types.
Based on rule taken from:
It was taken from a \lgtm{} rule\footnote{\url{https://lgtm.com/rules/910065/}}.

It is a technique that allows a developer to take different actions according to the runtime type of an object.
Depending on the --- runtime --- type of an object, different cases, usually one for each type will follow.

\end{pattern}

\begin{pattern}{TypeTag}
Lookup in a collection using a application-specific type tag or a
\code{java.lang.Class}.

A cast guarded by a test on a field from the same object instead of
using \code{instanceof}.

\instances{}
The following example%
\footnote{\url{http://bit.ly/2Ho8bVL}}
shows an instance of the \thisp{} pattern.
The cast type of the parameter \code{reference} is determined by the value of the parameter \code{referenceType}.

%https://lgtm.com/projects/b/JesusFreke/smali/snapshot/dist-1306230039-1524814812150/files/dexlib2/src/main/java/org/jf/dexlib2/writer/InstructionWriter.java?sort=name&dir=ASC&mode=heatmap#L492
\begin{minted}[highlightlines=8]{java}
private int getReferenceIndex(int referenceType, Reference reference) {
    switch (referenceType) {
        case ReferenceType.FIELD:
            return fieldSection.getItemIndex((FieldRefKey) reference);
        case ReferenceType.METHOD:
            return methodSection.getItemIndex((MethodRefKey) reference);
        case ReferenceType.STRING:
            return stringSection.getItemIndex((StringRef) reference);
        case ReferenceType.TYPE:
            return typeSection.getItemIndex((TypeRef) reference);
        case ReferenceType.METHOD_PROTO:
            return protoSection.getItemIndex((ProtoRefKey) reference);
        default:
            throw new ExceptionWithContext(
                "Unknown reference type: %d",  referenceType);
    }
}
\end{minted}

In the next case%
\footnote{\url{http://bit.ly/2FW5SXU}}
a type test is performed --- through a method call --- before actually applying the cast to the variable \code{props}.
Note that the type test is using the \code{instanceof} operator (line 8).

%https://lgtm.com/projects/g/apache/poi/snapshot/dist-1790760597-1524814812150/files/src/ooxml/java/org/apache/poi/xslf/usermodel/XSLFPropertiesDelegate.java?sort=name&dir=ASC&mode=heatmap#L1367
\begin{minted}[highlightlines=3]{java}
@Override
public CTSolidColorFillProperties getSolidFill() {
    return isSetSolidFill() ? (CTSolidColorFillProperties)props : null;
}

@Override
public boolean isSetSolidFill() {
    return (props instanceof CTSolidColorFillProperties);
}
\end{minted}

\detection{}
The detection of this pattern is similar to the \nameref{pat:PatternMatching} detection, but instead of looking for an \code{instanceof} guarded cast, we look for an application-specific guard.

\discussion{}

\related{}

\end{pattern}
\begin{pattern}{Equals}
This pattern is a common pattern to implement the \code{equals} method (declared in \code{java.lang.Object}).
A cast expression is guarded by either an \code{instanceof} test or a \code{getClass} comparison (usually to the same target type as the cast);
in an \code{equals}%
\footnote{\url{https://docs.oracle.com/javase/8/docs/api/java/lang/Object.html\#equals-java.lang.Object-}} method implementation.
This is done to check if the argument has same type as the receiver
(\code{this} argument).

Notice that a cast in an \code{equals} method is needed because it
receives an \code{Object} as a parameter.

\instances{}
The following listing\footnote{\url{http://bit.ly/2vJw94J}} shows an example of the \pname{} pattern.
In this case, \code{instanceof} is used to guard for the same type as the receiver.

%https://lgtm.com/projects/g/neo4j/neo4j/snapshot/dist-15760049-1519892555006/files/community/kernel/src/main/java/org/neo4j/kernel/impl/api/CountsRecordState.java?sort=name&dir=ASC&mode=heatmap&excluded=false#L182
\begin{minted}[highlightlines=7]{java}
@Override
public boolean equals(Object obj) {
    if ( this == obj ) {
        return true;
    }
    if ( (obj instanceof Difference) ) {
        Difference that = (Difference) obj;
        return actualFirst == that.actualFirst
          && expectedFirst == that.expectedFirst
          && actualSecond == that.actualSecond 
          && expectedSecond == that.expectedSecond
          && key.equals( that.key );
    }
    return false;
}
\end{minted}

Alternatively, the following listing%
\footnote{\url{http://bit.ly/2vKP0MW}}
shows another example of the \thisp{} pattern.
But in this case, a \code{getClass} comparison is used to guard for the same type as the receiver.

%https://lgtm.com/projects/g/neo4j/neo4j/snapshot/dist-15760049-1519892555006/files/community/bolt/src/main/java/org/neo4j/bolt/v1/messaging/infrastructure/ValuePath.java?sort=name&dir=ASC&mode=heatmap&excluded=false#L278
\begin{minted}[highlightlines=7]{java}
@Override
public boolean equals( Object o ) {
    if ( this == o ) return true;
    if ( o == null || getClass() != o.getClass() )
        return false;

    ValuePath that = (ValuePath) o;
    return nodes.equals(that.nodes) &&
        relationships.equals(that.relationships);
}
\end{minted}

In some situations, the type casted to is not same as the enclosing class.
Instead, the type casted to is the super class of the enclosing class.
The following example%
\footnote{\url{http://bit.ly/2HmHMYE}}
shows this scenario.
This usually happens when the Google AutoValue library%
\footnote{\url{https://github.com/google/auto/tree/master/value}}
is used.
The AutoValue is a code generator for value classes.

%https://lgtm.com/projects/g/square/sqlbrite/snapshot/3a9916985485ba5922097fe59a18230500f02df4/files/sample/build/generated/source/apt/debug/com/example/sqlbrite/todo/ui/$AutoValue_ListsItem.java?sort=name&dir=ASC&mode=heatmap&showExcluded=false#L52
\begin{minted}[highlightlines=13]{java}
@AutoValue
abstract class ListsItem implements Parcelable {
    // [...]
}

abstract class $AutoValue_ListsItem extends ListsItem {
    @Override
    public boolean equals(Object o) {
      if (o == this) {
        return true;
      }
      if (o instanceof ListsItem) {
        ListsItem that = (ListsItem) o;
        return (this.id == that.id())
             && (this.name.equals(that.name()))
             && (this.itemCount == that.itemCount());
      }
      return false;
    }
}
\end{minted}

\footnote{\url{http://bit.ly/2SM5pOw}}
%https://lgtm.com/projects/g/bndtools/bnd/snapshot/dist-930051-1524814812150/files/biz.aQute.bndlib/src/aQute/bnd/osgi/resource/CapReq.java?sort=name&dir=ASC&mode=heatmap#L73
\begin{minted}[highlightlines=12]{java}
@Override
public boolean equals(Object obj) {
    if (this == obj)
            return true;
    if (obj == null)
            return false;
    if (obj instanceof CapReq)
            return equalsNative((CapReq) obj);
    if ((mode == MODE.Capability) && (obj instanceof Capability))
            return equalsCap((Capability) obj);
    if ((mode == MODE.Requirement) && (obj instanceof Requirement))
            return equalsReq((Requirement) obj);
    return false;
}
\end{minted}

\detection{}
The detection query looks for a cast expression inside an \code{equals} method implementation.
Moreover, the cast needs to be guarded by either an \code{instanceof} test or a \code{getClass} comparison.
And the type being casted to needs to be either the same as the enclosing class or a superclass of it.

\discussion{}
The pattern for an \code{equals} method implementation is well-known.

We found out that, with respect to cast,
most \code{equals} methods are implemented with the same structure.
Maybe avoid boilerplate code by providing code generation,
like in \haskell{} (with \code{deriving}).

\cite{vaziriDeclarativeObjectIdentity2007} propose a declarative approach to avoid boilerplate code when implementing both the \code{equals} and \code{hashCode} methods.
They manually analyzed several applications, and found there are many issues while implementing \code{equals()} and \code{hashCode()} methods.
It would be interesting to check whether these issues happen in real application code.

There is an exploratory document%
\footnote{\url{http://cr.openjdk.java.net/\~briangoetz/amber/datum.html}}
by Brian Goetz --- \java{} Language Architect --- addressing these issues from a more general perspective.
It is definitely a starting point towards improving the \java{} language.

\related{}
This pattern can be seen as a special instance of the \nameref{pat:PatternMatching} pattern.
\end{pattern}
\begin{pattern}{GetByClassLiteral}
A cast is using an application-specific guard,
but the guard depends on a class literal.

\instances{}
The following example%
\footnote{\url{http://bit.ly/elastic_elasticsearch_2SSgsFV}}
shows an instance of the \thisp{} pattern.
A cast is performed to the \code{field} variable (line 22),
based whether the runtime class of the variable is actually \code{Short.class}.

%https://lgtm.com/projects/g/elastic/elasticsearch/snapshot/dist-1916470085-1524814812150/files/server/src/main/java/org/elasticsearch/common/lucene/Lucene.java?sort=name&dir=ASC&mode=heatmap#L478
\begin{minted}[highlightlines=22]{java}
Class type = field.getClass();
if (type == String.class) {
    out.writeByte((byte) 1);
    out.writeString((String) field);
} else if (type == Integer.class) {
    out.writeByte((byte) 2);
    out.writeInt((Integer) field);
} else if (type == Long.class) {
    out.writeByte((byte) 3);
    out.writeLong((Long) field);
} else if (type == Float.class) {
    out.writeByte((byte) 4);
    out.writeFloat((Float) field);
} else if (type == Double.class) {
    out.writeByte((byte) 5);
    out.writeDouble((Double) field);
} else if (type == Byte.class) {
    out.writeByte((byte) 6);
    out.writeByte((Byte) field);
} else if (type == Short.class) {
    out.writeByte((byte) 7);
    out.writeShort((Short) field);
} else if (type == Boolean.class) {
    out.writeByte((byte) 8);
    out.writeBoolean((Boolean) field);
} else if (type == BytesRef.class) {
    out.writeByte((byte) 9);
    out.writeBytesRef((BytesRef) field);
} else {
    throw new IOException("Can't handle sort field value of type ["+type+"]");
}
\end{minted}

In the following listing,%
\footnote{\url{http://bit.ly/smartdevicelink_sdl_android_2EjJiaq}}
a cast is applied to the result of the \code{getObject} method (line 2).
The target type of the cast, \code{MyKey}, corresponds to the class literal argument, \code{MyKey.class}.
Essentially, \code{getObject} is using the \code{isInstance} method%
\footnote{\url{https://docs.oracle.com/javase/8/docs/api/java/lang/Class.html\#isInstance-java.lang.Object-}}
of the class \code{java.lang.Class} to check whether an object is from a certain type.

%https://lgtm.com/projects/g/smartdevicelink/sdl_android/snapshot/dist-2037360569-1524814812150/files/sdl_android/src/main/java/com/smartdevicelink/proxy/rpc/GetVehicleDataResponse.java?sort=name&dir=ASC&mode=heatmap#L236
\begin{minted}[highlightlines=2]{java}
public MyKey getMyKey() {
    return (MyKey) getObject(MyKey.class, KEY_MY_KEY);
} 
\end{minted}

Similar to the first example, the next snippet%
\footnote{\url{http://bit.ly/OPCFoundation_UA-Java-Legacy_2Fb2xmZ}}
contains several type cases.
Each type case is guarded by an \code{equals} comparison between a class literal and the \code{clazz} parameter.
The cast is applied to the type parameter \code{T} only if the guard succeeds.

%https://lgtm.com/projects/g/OPCFoundation/UA-Java-Legacy/snapshot/dist-1804064538-1524814812150/files/src/main/java/org/opcfoundation/ua/encoding/binary/BinaryDecoder.java?sort=name&dir=ASC&mode=heatmap#L214
\begin{minted}[highlightlines=9]{java}
@Override
@SuppressWarnings("unchecked")
public <T> T get(String fieldName, Class<T> clazz) throws DecodingException {
    if (clazz.equals(Boolean.class)) {
        return (T) getBoolean(fieldName);
    }
    // [...]
    if (clazz.equals(ExtensionObject.class)) {
        return (T) getExtensionObject(fieldName);
    }
    // [...]
}
\end{minted}


\detection{}
This pattern is characterized by the use of a class literal.%
\footnote{\url{https://docs.oracle.com/javase/specs/jls/se8/html/jls-15.html\#jls-15.8.2}}
The class literal needs to be used in a comparison such that guards the cast instance.

\discussion{}
This pattern may be used instead of \nameref{pat:PatternMatching} when the developer wants to match exactly the runtime class.
The \code{instanceof} operator%
\footnote{\url{https://docs.oracle.com/javase/specs/jls/se8/html/jls-15.html\#jls-15.20.2}}
returns \code{true} if the expression could be cast to the specified type,
whereas using a class literal comparison returns \code{true} if the expression is exactly the runtime class.

\related{}
This pattern can be seen as a particular instance of the \nameref{pat:TypeTag} pattern,
where the tag is given by the class literal.
As discussed above, it is related to \nameref{pat:PatternMatching} but this pattern uses an exact match of the runtime class.

\end{pattern}
\begin{pattern}{ClassForName}

\instances{}

\detection{}

\discussion{}

\related{}

\end{pattern}

\group{Creational}

These patterns creates to how they are created.

\begin{pattern}{Family}
Family polymorphism.

\instances{}

\detection{}

\discussion{}
\cite{ernstFamilyPolymorphism2001}

\related{}

\end{pattern}
\begin{pattern}{Factory}
Creates an object based on some arguments either to the method call or constructor.
Since the arguments are known at compile-time, cast to the specific type.

Cast factory method result to subtype (special case of family polymorphism).
Usually Logger.getLogger.

The method is declared to return URLConnection but can return a more specific type based on the URL string.
Cast to that.
We should generalize this pattern.

\instances{}

\footnote{\url{http://bit.ly/2HvRlUX}}

\footnote{\url{https://docs.oracle.com/javase/8/docs/api/java/security/KeyPair.html\#getPrivate()}}

%https://lgtm.com/projects/b/connect2id/oauth-2.0-sdk-with-openid-connect-extensions/snapshot/dist-1311020143-1524814812150/files/src/test/java/com/nimbusds/oauth2/sdk/jose/jwk/RemoteJWKSetTest.java?sort=name&dir=ASC&mode=heatmap#L242
\begin{minted}[highlightlines=10]{java}
KeyPairGenerator pairGen = KeyPairGenerator.getInstance("RSA");
pairGen.initialize(1024);
KeyPair keyPair = pairGen.generateKeyPair();
RSAKey rsaJWK1 = new RSAKey.Builder((RSAPublicKey) keyPair.getPublic())
        .privateKey((RSAPrivateKey) keyPair.getPrivate())
        .keyID("1")
        .build();
keyPair = pairGen.generateKeyPair();
RSAKey rsaJWK2 = new RSAKey.Builder((RSAPublicKey) keyPair.getPublic())
        .privateKey((RSAPrivateKey) keyPair.getPrivate())
        .keyID("2")
        .build();
\end{minted}

\footnote{\url{http://bit.ly/2E6KY6T}}

\footnote{\url{https://docs.oracle.com/javase/8/docs/api/java/net/URL.html\#openConnection--}}

%https://lgtm.com/projects/g/apache/hadoop/snapshot/dist-956730001-1524814812150/files/hadoop-yarn-project/hadoop-yarn/hadoop-yarn-server/hadoop-yarn-server-resourcemanager/src/test/java/org/apache/hadoop/yarn/server/resourcemanager/webapp/TestRMWebServicesHttpStaticUserPermissions.java?sort=name&dir=ASC&mode=heatmap#L138
\begin{minted}[highlightlines=2]{java}
URL url = new URL("http://localhost:8088/ws/v1/cluster/apps");
HttpURLConnection conn = (HttpURLConnection) url.openConnection();
\end{minted}

\detection{}

\discussion{}

\related{}

\end{pattern}
\input{chapters/casts/KnownLibraryMethod}
\begin{pattern}{NewDynamicInstance}
Dynamically creation of object by means of reflection.
These are the casts that can not be avoidable.

The \code{newInstance} method family declared in the
\code{Class}\footnote{\url{https://docs.oracle.com/javase/8/docs/api/java/lang/Class.html\#newInstance--}},
\code{Array}\footnote{\url{https://docs.oracle.com/javase/8/docs/api/java/lang/reflect/Array.html\#newInstance-java.lang.Class-int-}}\(^{,}\)
\footnote{\url{https://docs.oracle.com/javase/8/docs/api/java/lang/reflect/Array.html\#newInstance-java.lang.Class-int...-}} and
\code{Constructor}\footnote{\url{https://docs.oracle.com/javase/8/docs/api/java/lang/reflect/Constructor.html\#newInstance-java.lang.Object...-}}
classes creates an object or array by means of reflection.

This pattern consists of casting the result of these methods to the appropriate target type.

\instances{}

%https://lgtm.com/projects/g/apache/hadoop/snapshot/6bedbef6c5f2d937a6cbc268300ce2a39609d06c/files/hadoop-hdfs-project/hadoop-hdfs/src/main/java/org/apache/hadoop/hdfs/server/namenode/FSNamesystem.java?sort=name\&dir=ASC\&mode=heatmap\&showExcluded=false#L1039

The following example shows a cast from the \code{Class.newInstance()}
method.
% \footnote{\url{d}}

\begin{lstlisting}[style=java,caption={The \pname{} pattern using the \texttt{Class} class.}]
logger = (AuditLogger) Class.forName(className).newInstance();
\end{lstlisting}

%https://lgtm.com/projects/g/neo4j/neo4j/snapshot/27aaa67633e4d26446e38125d04fbbd27f938b75/files/community/collections/src/main/java/org/neo4j/helpers/collection/Iterables.java?sort=name\&dir=ASC\&mode=heatmap\&showExcluded=false#L403
The following example shows how to dynamically create an array.
%\footnote{\url{d}}

\begin{lstlisting}[style=java,caption={Example of the \pname{} pattern using the \texttt{Array} class.}]
return list.toArray( (T[]) Array.newInstance( componentType, list.size()));
\end{lstlisting}

%https://lgtm.com/projects/g/gradle/gradle/snapshot/209c3175e75af6ac30cb66c02eda15b0f8b6a616/files/subprojects/internal-integ-testing/src/main/groovy/org/gradle/integtests/fixtures/executer/OutputScrapingExecutionFailure.java?sort=name\&dir=ASC\&mode=heatmap\&showExcluded=false#L174

Whenever a constructor other than the default constructor is needed,
the \code{newInstance} method declared in the \code{Constructor} class
should be used to select the appropriate constructor,
as shown in the following example.
%\footnote{\url{d}}

\begin{lstlisting}[style=java,caption=Example of the \pname{} pattern using the \code{Constructor} class.]
return (Exception) Class
                       .forName(className)
                       .getConstructor(String.class)
                       .newInstance(message);
\end{lstlisting}

\detection{}
This detection query looks for casts,
where the expression being cast is a call site to methods mentioned above.

\discussion{}
The cast here is needed because of the dynamic essence of reflection.
This pattern is unguarded, that is,
the application programmer knows what is the target type being created.

\related{}
Reflection.

\end{pattern}
\begin{pattern}{Tag}
Used 

\instances{}

\footnote{\url{http://bit.ly/UniTime_cpsolver_2HUmGki}}

%https://lgtm.com/projects/g/UniTime/cpsolver/snapshot/dist-4860376-1524814812150/files/src/org/cpsolver/ifs/assignment/context/AssignmentContextHolderMap.java?sort=name&dir=ASC&mode=heatmap#L47
\begin{minted}[highlightlines=9]{java}
protected Map<Integer,AssignmentContext> iContexts =
                new HashMap<Integer, AssignmentContext>();

@Override
@SuppressWarnings("unchecked")
public <U extends AssignmentContext> U getAssignmentContext(
                Assignment<V, T> assignment,
                AssignmentContextReference<V, T, U> reference) {
    U context = (U) iContexts.get(reference.getIndex());
    if (context != null) return context;
    
    context = reference.getParent().createAssignmentContext(assignment);
    iContexts.put(reference.getIndex(), context);
    return context;
}
\end{minted}

\footnote{\url{http://bit.ly/ggp-org_ggp-base_2SAEXHu}}

%https://lgtm.com/projects/g/ggp-org/ggp-base/snapshot/dist-59800051-1524814812150/files/src/main/java/org/ggp/base/apps/player/match/MatchPanel.java?sort=name&dir=ASC&mode=heatmap#L66
\begin{minted}[highlightlines=16]{java}
public final class MatchPanel extends JPanel implements Observer {
    private final JZebraTable matchTable;
    public MatchPanel() {
        super(new GridBagLayout());
        DefaultTableModel model = new DefaultTableModel();
        // [...]
        matchTable = new JZebraTable(model) {
            @Override
            public boolean isCellEditable(int rowIndex, int colIndex) {
                return false;
            }
        };
    }
    // [...]
    private void observe(GamerCompletedMatchEvent event) {
        DefaultTableModel model = (DefaultTableModel) matchTable.getModel();
        model.setValueAt("Inactive", model.getRowCount() - 1, 4);
    }
}
\end{minted}

\detection{}

\discussion{}

\related{}
Related to \nameref{pat:VariableLessSpecificType}.
\nameref{pat:LookupById}.

\end{pattern}
\begin{pattern}{Deserialization}
Used to deserialize an object.

\instances

\footnote{\url{}}

\begin{minted}[highlightlines=2]{java}

\end{minted}

\detection{}

\discussion{}

\related{}

\end{pattern}
% \begin{pattern}{StackSymbol}

\instances{}

\footnote{\url{http://bit.ly/2HF6nrF}}

%https://lgtm.com/projects/g/fabioz/Pydev/snapshot/dist-20832102-1524814812150/files/plugins/org.python.pydev.parser/src/org/python/pydev/parser/grammar27/TreeBuilder27.java?sort=name&dir=ASC&mode=heatmap#L231
\begin{minted}[highlightlines=1]{java}
            case JJTASSERT_STMT:
                exprType msg = arity == 2 ? ((exprType) stack.popNode()) : null;
                test = (exprType) stack.popNode();
                return new Assert(test, msg);
\end{minted}

\detection{}

\discussion{}

\related{}

\end{pattern}
\begin{pattern}{CreateByClassLiteral}
    
\instances{}

\detection{}

\discussion{}

\related{}

\end{pattern}

\group{Tuples}

Tuples patterns.

\begin{pattern}{LookupById}
This pattern is used to extract stashed values from a generic container.

Lookup an object by ID, tag or name and cast the result
(it is used often in Android code).
It accesses a collection that holds values of different types
(usually implemented as \code{Collection<Object>} or as \code{Map<K, Object>}).

\instances

%https://lgtm.com/projects/g/loopj/android-async-http/snapshot/dist-1879340034-1518514025554/files/library/src/main/java/com/loopj/android/http/AsyncHttpClient.java?sort=name&dir=ASC&mode=heatmap&excluded=false#L258

In the example shown in listing,
% \footnote{\url{d}},
the \texttt{getAttribute} method returns \texttt{Object}.
The variable \texttt{context} is of type \texttt{BasicHttpContext},
which is implemented with \texttt{HashMap}.

\lstset{language=java,label=orga7c88d3,caption={Example of the \pname{} pattern.},captionpos=b,numbers=none,style=java}
\begin{lstlisting}
AuthState authState =
        (AuthState) context.getAttribute(ClientContext.TARGET_AUTH_STATE);
\end{lstlisting}

\discussion{}
%
\todo{Cut/Move to future work.}
%
This pattern suggests heterogeneous dictionary.
Given our manual inspection,
we believe that all dictionary keys and resulting types are known at
compile-time, \ie, by the programmer.
%
\done{Nate: Replace "restriction" for "inexpressiveness"}
%
But in any case a cast is needed given the inexpressiveness of the type system.
As a complementary analysis,
it would be interesting to check whether all call sites to
\code{getAttribute} receives a constant (\code{final static} field).

Notice that this pattern is not guarded by an \code{instanceof}.
However, the cast involved does not fail at runtime.
This means that the source of the cast is known to the programmer.
This raises the following questions:
\begin{itemize}
\item \emph{What kind of analysis is needed to detect the source of the cast?}
\item \emph{Is worth to have it?}
\item \emph{Is better to change API?}
\item \emph{How other --- statically typed --- languages support this kind of idiom?}
\item \emph{Could generative programming a.k.a. templates solve this problem?}
\end{itemize}

\end{pattern}
\begin{pattern}{StaticResource}
%
\todo{Nate: Why not LookupById?}
%
A cast to a method access to \code{findViewById} or a method that reads a static resource.
\todo{Luis: Mention frameworks to avoid this cast using annotation.}
This is a pattern seen when using the Android platform. 

\instances{}

\footnote{\url{http://bit.ly/pwittchen_NetworkEvents_2HGbrMq}}

%https://lgtm.com/projects/g/pwittchen/NetworkEvents/snapshot/dist-2032650416-1524814812150/files/example/src/main/java/com/github/pwittchen/networkevents/app/MainActivity.java?sort=name&dir=ASC&mode=heatmap#L65
\begin{minted}[highlightlines=6]{java}
@Override
protected void onCreate(Bundle savedInstanceState) {
    super.onCreate(savedInstanceState);
    setContentView(R.layout.activity_main);
    connectivityStatus = (TextView) findViewById(R.id.connectivity_status);
    mobileNetworkType = (TextView) findViewById(R.id.mobile_network_type);
    accessPoints = (ListView) findViewById(R.id.access_points);
    busWrapper = getOttoBusWrapper(new Bus());
    networkEvents = new NetworkEvents(getApplicationContext(), busWrapper)
        .enableInternetCheck()
        .enableWifiScan();
}
\end{minted}

\detection{}

\discussion{}
%
\todo{Nate: Could be determined at compile-time.}
%
These casts could be solved by using code generation,
or partial classes as in \csharp{}.

\end{pattern}

\begin{pattern}{ObjectAsArray}
An array used as an untyped object.
A cast applied to an array slot, \eg, \code{(String) array[1]}.

\end{pattern}


\group{Code Smell}

These are code smells.

\begin{pattern}{Literal}
Cast a numeric literal or constant --- defined as \code{static final} ---
to a primitive type of
\code{byte}, \code{char}, \code{short}

\instances

The following listing shows an example of the \pname{Literal} pattern.%
\footnote{https://lgtm.com/projects/g/kaitoy/pcap4j/snapshot/dist-29675155-1524814812150/files/pcap4j-core/src/main/java/org/pcap4j/packet/namednumber/TcpPort.java?sort=name\&dir=ASC\&mode=heatmap\#L2329}

\begin{lstlisting}[style=java,caption=Literal example]
public static final TcpPort POWERBURST =
    new TcpPort((short)485, "Air Soft Power Burst");
\end{lstlisting}

\discussion

This pattern is related with \ref{pat:Prim}

\end{pattern}
\begin{pattern}{RawTypes}
When a generic method is not used as such.
The expression of this cast is a method invocation,
but the declaration differs from the usage.

\instances

\end{pattern}
\begin{pattern}{Redundant}
A cast that is not necessary for compilation.

\instances{}
The following example%
\footnote{\url{http://bit.ly/2FWXw2e}}

%https://lgtm.com/projects/g/vladmihalcea/high-performance-java-persistence/snapshot/dist-1813180502-1524814812150/files/core/src/test/java/com/vladmihalcea/book/hpjp/hibernate/schema/flyway/FlywayTest.java#L40
\begin{minted}[highlightlines=1]{java}
transactionTemplate.execute((TransactionCallback<Void>) transactionStatus -> {
    Post post = new Post();
    entityManager.persist(post);
    return null;
});
\end{minted}

\detection{}

\discussion{}

\related{}
    
\end{pattern}
\begin{pattern}{VariableLessSpecificType}
This pattern occurs when a cast is applied to a variable (local variable,
parameter, or field),
that is usually being assigned once and
is declared with a less specific type than the type of the value 
that is being assigned to.
The type of the value being assigned to can be determined locally
either within the enclosing method or class.

\instances{}
The following example\footnote{\url{http://bit.ly/2FuDeO7}}
shows an instance of the \thisp{} pattern.
We can see that the field \code{uncompressedDirectBuf} is being cast to
the \code{java.nio.ByteBuffer} class (line $13$) but it is declared as
\code{java.nio.Buffer} (line $3$).
Nevertheless, the field is assigned only once in the constructor (line $7$)
with a value of type \code{java.nio.ByteBuffer}.
The value assigned is returned by the method
\code{ByteBuffer.allocateDirect}.%
\footnote{\url{https://docs.oracle.com/javase/7/docs/api/java/nio/ByteBuffer.html\#allocateDirect(int)}}
Inspecting the enclosing class, there is no other assignment to the
field \code{uncompressedDirectBuf}.
Therefore, the cast pattern in line $13$ will always succeed.
Note that in this case the variable \code{uncompressedDirectBuf}
could have been declared as \code{final}.

% https://lgtm.com/projects/g/facebookarchive/hadoop-20/snapshot/dist-1802091768-1524814812150/files/src/core/org/apache/hadoop/io/compress/snappy/SnappyCompressor.java?sort=name&dir=ASC&mode=heatmap#L134
\begin{minted}[highlightlines=13]{java}
public class SnappyCompressor implements Compressor {
    // [...]
    private Buffer uncompressedDirectBuf = null;
    // [...]
    public SnappyCompressor(int directBufferSize) {
        // [...]
        uncompressedDirectBuf = ByteBuffer.allocateDirect(directBufferSize);
        // [...]
    }
    // [...]
    synchronized void setInputFromSavedData() {
        // [...]
        ((ByteBuffer) uncompressedDirectBuf).put(userBuf, userBufOff,
            uncompressedDirectBufLen);
        // [...]
    }
    // [...]
}
\end{minted}

\detection{}
To detect this pattern, a cast needs to be applied to a variable whose
value can be determined simply by looking at
the enclosing method or class.

\discussion{}
In most the cases this can be considered as a bad practice or
code smell.
This is because by only changing the declaration of the variable
to a more specific type type, the cast can be simply eliminated.

\related{}
This pattern is related to the \nameref{pat:Redundant} pattern.
Although \thisp{} is not redundant,
by only changing the declaration of the variable to a more specific type,
the cast becomes redundant.

\end{pattern}


\begin{pattern}{SelectOverload}

This pattern is used to select the appropiate version of an overloaded
method\footnote{Using ad-hoc polymorphism~\cite{stracheyFundamentalConceptsProgramming2000}}
where two or more of its implementations differ \emph{only} in some argument type.

A cast to \code{null} is often used to select against different versions of a method,
\ie{}, to resolve method overloading ambiguity.
Whenever a \code{null} value needs to be an argument of an a cast is needed to select
the appropriate implementation.
This is because the type of \code{null} has the special type
\emph{null}\footnote{\url{https://docs.oracle.com/javase/specs/jls/se8/html/jls-4.html\#jls-4.1}}
which can be treated as any reference type.
In this case, the compiler cannot determine which method implementation to select.

\instances
%https://lgtm.com/projects/g/loopj/android-async-http/snapshot/dist-1879340034-1518514025554/files/library/src/main/java/com/loopj/android/http/JsonHttpResponseHandler.java?sort=name\&dir=ASC\&mode=heatmap\&excluded=false#L150

Listing \ref{lst:selection} \footnote{\url{asdf}}
shows an example of \pname{} pattern.
Another use case is to select the appropriate the right argument when calling a method with variable arguments.

\begin{lstlisting}[style=ql,label=lst:selection,caption=Example of \pname{} pattern.]
onSuccess(statusCode, headers, (String) null);
\end{lstlisting}

In this example, there are three versions of the \texttt{onSuccess} method, as shown in listing \ref{orgaa1b3bd}.
The cast \texttt{(String) null} is used to select the appropriate version (line 7), based on the third parameter.

	\lstset{language=java,label=orgaa1b3bd,caption={Overloaded methods that differ only in their argument type (the third one).},captionpos=b,numbers=none,style=java}
	\begin{lstlisting}
public void onSuccess(
      int statusCode, Header[] headers, JSONObject response) {...}

public void onSuccess(
      int statusCode, Header[] headers, JSONArray response) {...}

public void onSuccess(
      int statusCode, Header[] headers, String responseString) {...}
\end{lstlisting}

\detection

Listing \ref{org9e0faf3} shows how to detect this pattern.
This pattern shows up when a cast is directly applied to the \texttt{null} constant.

\lstset{language=sql,label=org9e0faf3,caption={Detection of the \pname{} pattern.},captionpos=b,numbers=none,style=ql}
\begin{lstlisting}
import java

from CastExpr ce, NullLiteral nl
where ce.getExpr() = nl
select ce
\end{lstlisting}

\discussion

Casting the \code{null} constant seems rather artificial.
This pattern shows either a lack of expressiveness in \java{} or
a bad \api{} design.
Several other languages support default parameters, \eg{},
\scala{}, \cs{} and \cpp{}.
Adding default parameters might be a partial solution.
% \todo{Relate null as theorical point of view in the TAPL book.}

\end{pattern}
\begin{pattern}{Clone}
A cast to a \code{clone} method.

\instances

\end{pattern}

\begin{pattern}{ReflectiveAccessibility}

This pattern accesses a field of an object by means of reflection.
It uses reflection because at compile time the field is unaccesible.
Usually the method \code{setAccessible(true)} is invoked on the field
before actually getting the value from an object.

\instances{}
The following cast%
\footnote{\url{http://bit.ly/loopj_android-async-http_2SOISRr}}
uses this pattern:

%https://lgtm.com/projects/g/loopj/android-async-http/snapshot/dist-1879340034-1529316783166/files/library/src/main/java/com/loopj/android/http/AsyncHttpClient.java?sort=name&dir=ASC&mode=heatmap&showExcluded=false#L445
\begin{listing}
\caption{Using \code{Field::get} to gain access to a field.}
\begin{minted}[highlightlines=2]{java}
f.setAccessible(true);
HttpEntity wrapped = (HttpEntity) f.get(entity);
\end{minted}
\end{listing}

\end{pattern}

\begin{pattern}{ImplicitIntersectionType}
Cast a reference $v$ of type --- class or interface --- $T$ to an
interface type $I$ whether $T$ does not implement $I$.
The cast succeeds at runtime because all possible runtime types of $v$
actually implement the interface $I$.
For instance, in \code{(Comparable)(Number)4}, \code{Number} does not
implement the \code{Comparable} interface, but class \code{Integer} does.

\instances

\begin{lstlisting}[style=java,caption=From \url{http://bit.ly/2FQOt4v}]
final Comparable max = (Comparable) properties.getMaxValue();
\end{lstlisting}
\end{pattern}
\begin{pattern}{ExceptionSoftening}
We can throw CheckedExceptions even on methods that don't declare them (via Exception softening).

\instances

\end{pattern}
\begin{pattern}{SoleSubclassImplementation}

\instances{}
The following example%
\footnote{\url{http://bit.ly/immutables_immutables_2S4BoJs}}

%https://lgtm.com/projects/g/immutables/immutables/snapshot/dist-43930039-1524814812150/files/value-processor/target/generated-sources/annotations/org/immutables/value/processor/encode/ImmutableEncodedElement.java?sort=name&dir=ASC&mode=heatmap#L1734
\begin{minted}{java}
public final EncodedElement.Builder addAllThrown( Iterable<? extends Type> elements) {
    this.thrown.addAll(elements);
    return (EncodedElement.Builder) this;
}
\end{minted}

\detection{}

\discussion{}

\related{}

\end{pattern}

* keep reference \cite{altidorTamingWildcardsCombining2011}

- CovariantReturn
%https://lgtm.com/projects/g/apache/cayenne/snapshot/dist-1890050-1524814812150/files/modeler/cayenne-modeler/src/main/java/org/apache/cayenne/modeler/undo/TextCompoundEdit.java?sort=name&dir=ASC&mode=heatmap#L74
\begin{minted}{java}
EditorView editorView = ((CayenneModelerFrame) Application.
        getInstance().getFrameController().getView())
        .getView();
\end{minted}