\begin{pattern}{ReflectiveAccessibility}
This pattern accesses a field of an object by means of reflection.
Typically reflection is used because the field is private and therefore
inaccessible at compile time and the developer cannot change the field
declaration itself.
In this case, the method \code{Field::setAccessible(true)} is invoked on the field
before getting the value of the field.
The cast is needed because \code{Field::get} returns an \code{Object}.

\instances{}
The following two snippets show how this pattern is used:

%https://lgtm.com/projects/g/loopj/android-async-http/snapshot/dist-1879340034-1529316783166/files/library/src/main/java/com/loopj/android/http/AsyncHttpClient.java?sort=name&dir=ASC&mode=heatmap&showExcluded=false#L445
\def\urlvar{http://bit.ly/loopj_android_async_http_2SOISRr}
\begin{minted}[highlightlines=2]{java}
f.setAccessible(true);
HttpEntity wrapped = (HttpEntity) f.get(entity);
#\urlbox#
\end{minted}

%https://lgtm.com/projects/g/joel-costigliola/assertj-db/snapshot/dist-890344-1524814812150/files/src/test/java/org/assertj/db/navigation/ToChange_ChangeOfModification_Integer_Test.java?sort=name&dir=ASC&mode=heatmap#L199
\def\urlvar{http://bit.ly/joel_costigliola_assertj_db_2Ip1Rho}
\begin{minted}[highlightlines=5]{java}
Field fieldPosition=ChangesOutputter.class.getDeclaredField("changesPosition");
fieldPosition.setAccessible(true);
ChangesOutputter changesDisplayBis = output(changes);
PositionWithChanges<ChangesAssert, ChangeAssert> positionBis = 
            (PositionWithChanges) fieldPosition.get(changesDisplayBis);
#\urlbox#
\end{minted}


\detection{}
The Query~\ref{lst:ql:ReflectiveAccessibilityCast} detects this pattern.
The query looks for a cast applied to a \code{get} or \code{invoke} method in an object \code{o} of type \code{Field} or \code{Method} respectively.
Moreover, it checks that \code{setAccessible(true)} has been invoked in \code{o}.
However, this query does not check that \code{setAccessible(true)} has been invoked \emph{before} the cast.

\begin{listing}
\begin{minted}{java}	
class ReflectiveAccessibilityCast extends #\qlref{Cast}# { #\qlbox#
	Variable fieldVariable;
    #\qlref{ReflectiveMethodAccess}# reflectiveMethodAccess;
	#\qlref{SetAccessibleTrueMethodAccess}# satma;
	ReflectiveAccessibilityCast() {
		reflectiveMethodAccess = getExprOrDef() and
		fieldVariable.getAnAccess() = 
                getExprOrDef().(MethodAccess).getQualifier().(VarAccess) and
		fieldVariable.getAnAccess() = satma.getQualifier()
	}
}
\end{minted}
\caption{Detection of the \thisp{} pattern.}
\label{lst:ql:ReflectiveAccessibilityCast}
\end{listing}


\issues{}
Using reflection to access a field is a common workaround to tight access control restrictions.
However, it should generally be regarded as a code smell.

As with \nameref{pat:Deserialization}, this pattern is necessary because
a library method can return values of many different types at run time,
and so is declared to return \code{Object}.

\end{pattern}
