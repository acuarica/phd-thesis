\begin{pattern}{Composite}
The \thisp{} pattern is characterized by a cast to another element of a
composite data structure, typically a tree, where the target type is known because of its
position in the data structure.

\instances{}
The following example%
\footnote{\url{http://bit.ly/flyingsaucerproject_flyingsaucer_2N2nYbY}}
shows a cast from a \code{Box} to a \code{TableSectionBox}.
The programmer reasons that the cast will succeed because the 
the source of the cast is a sibling of another \code{TableSectionBox}.

%https://lgtm.com/projects/g/flyingsaucerproject/flyingsaucer/snapshot/dist-26624048-1524814812150/files/flying-saucer-core/src/main/java/org/xhtmlrenderer/newtable/TableBox.java#L711
\begin{minted}[highlightlines=3]{java}
public class TableBox extends BlockBox {
    protected TableSectionBox sectionAbove(TableSectionBox section, boolean skipEmptySections) {
        TableSectionBox prevSection=(TableSectionBox)section.getPreviousSibling();
    }
}

public abstract class Box implements Styleable {
    // [...]
    public Box getPreviousSibling() {
        // [...]
    }
}
\end{minted}

\discussion{}
The pattern is typical of hierarchical data structures such as abstract syntax
trees, document models, or UI layouts. Based on the grammar of 
the data structure, the types of adjacent objects in the structure can be known.
The cast succeeds if the data structure is well-formed.

More precise typing of the links in the the data structure could 
eliminate the need for the casts. For example, in the above example,
the sibling of a \code{TableSectionBox} might be declared to have type
\code{TableSectionBox}. However, this may require the programmer to override
methods to refine return types covariantly.
Language features available in other languages like generalized algebraic data types
  (GADTs)~\cite{gadts} or self types~\cite{bruceChallengingTypingIssues2003,scalaIndependentlyExtensible} could also be 
used to provide a more precise typing.

The pattern can be thought of as a more dynamic variant of the
\nameref{pat:Family} pattern. Rather than
reasoning that the cast will succeed because of the source type's relative position in the 
class hierarchy, the cast will succeed because of the source value's position
in a composite data structure.

\end{pattern}
