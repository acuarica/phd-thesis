\begin{pattern}{LookupById}
This pattern is used to extract values from a heterogenous container.
It looks up an object by a compile-time constant identifier, tag, or name and casts the result to an appropriate type.
It accesses a collection that holds values of different types
(usually implemented as \code{Collection<Object>} or as \code{Map<K, Object>}).
The actual run-time type returned from the lookup is determined by the value of the identifier.

\instances{}
In the example shown below,%
\footnote{\url{http://bit.ly/loopj_android-async-http_2SUzY4E}}
the \code{getAttribute} method returns \code{Object}.
The variable \texttt{context} is of type \code{BasicHttpContext},
which is implemented with \code{HashMap}.

%https://lgtm.com/projects/g/loopj/android-async-http/snapshot/dist-1879340034-1518514025554/files/library/src/main/java/com/loopj/android/http/AsyncHttpClient.java?sort=name&dir=ASC&mode=heatmap&excluded=false#L258
\begin{minted}[highlightlines=1,linenos=false]{java}
AuthState authState = (AuthState) context.getAttribute(ClientContext.TARGET_AUTH_STATE);
\end{minted}

The next snippet%
\footnote{\url{http://bit.ly/skerit_cmusphinx_2HGgL1D}}
shows a call site to the \code{getComponent} method cast to the \code{ActiveListManager} class (line 15).
The \code{getComponent} method in this cast instance uses as argument the \code{PROP\_ACTIVE\_LIST\_MANAGER} constant.
Looking at the definition of this constant (line 3),
we can see there is a companion attribute (\code{@S4Component}) whose argument is the \code{ActiveListManager} class, the target of the cast instance.

% https://lgtm.com/projects/g/skerit/cmusphinx/snapshot/dist-1506204046584-1524814812150/files/sphinx4/src/sphinx4/edu/cmu/sphinx/decoder/search/WordPruningBreadthFirstSearchManager.java?sort=name&dir=ASC&mode=heatmap#L207
\begin{minted}[highlightlines=15]{java}
/** The property that defines the type of active list to use */
@S4Component(type = ActiveListManager.class)
public final static String PROP_ACTIVE_LIST_MANAGER = "activeListManager";

@Override
public void newProperties(PropertySheet ps) throws PropertyException {
    super.newProperties(ps);
    logMath = (LogMath) ps.getComponent(PROP_LOG_MATH);
    logger = ps.getLogger();
    linguist = (Linguist) ps.getComponent(PROP_LINGUIST);
    pruner = (Pruner) ps.getComponent(PROP_PRUNER);
    scorer = (AcousticScorer) ps.getComponent(PROP_SCORER);
    activeListManager = 
            (ActiveListManager) ps.getComponent(PROP_ACTIVE_LIST_MANAGER);
    // [...]
}
\end{minted}

A common version of this pattern is to retrieve a value instantiated from
static resource file, \eg,
an XML, HTML or \java{} properties file.
The file contents are (in theory) known at compile-time and the file is included in the binary distribution
of the application. These files are often built using tools such as GUI
builders.

In the following example from an Android application,%
\footnote{\url{http://bit.ly/pwittchen_NetworkEvents_2HGbrMq}}
a cast is applied to the \code{findViewById} method invocation.
View classes are instantiated by the application framework using an XML resource file.
The \code{findViewById} method looks up the view by its ID.

%https://lgtm.com/projects/g/pwittchen/NetworkEvents/snapshot/dist-2032650416-1524814812150/files/example/src/main/java/com/github/pwittchen/networkevents/app/MainActivity.java?sort=name&dir=ASC&mode=heatmap#L65
\begin{minted}[highlightlines=1,linenos=false]{java}
mobileNetworkType = (TextView) findViewById(R.id.mobile_network_type);
\end{minted}

\discussion{}
This pattern suggests a heterogeneous dictionary.
In our manual inspection,
all dictionary keys and the resulting types are known at
compile time, however
a cast is needed because the dictionary type does not encode the
relationship between key values and the result type.
Casts in this pattern are typically not guarded indicating that the programmer
knows the source of the cast based on the value of the key.

This pattern is often seen in Android applications.
The Butter Knife framework%
\footnote{\url{http://jakewharton.github.io/butterknife/}}
uses annotations to avoid the ``manual'' casting.
Instead, code is generated that casts the result of \code{findViewById} to the
appropriate type.


But in any case a cast is needed given the inexpressiveness of the type system.
As a complementary analysis,
it would be interesting to check whether all call sites to
\code{getAttribute} receives a constant (\code{final static} field).

Notice that this pattern is not guarded by an \code{instanceof}.
However, the cast involved does not fail at runtime.
This means that the source of the cast is known to the programmer.
This raises the following questions:
\begin{itemize}
\item \emph{What kind of analysis is needed to detect the source of the cast?}
\item \emph{Is worth to have it?}
\item \emph{Is better to change API?}
\item \emph{How other --- statically typed --- languages support this kind of idiom?}
\item \emph{Could generative programming a.k.a. templates solve this problem?}
\end{itemize}

This pattern retrieves a stashed a value from an heterogeneous collection (or dictionary).
A cast is needed because the return type depends on the ID to be retrieved.
Since this pattern casts a value to a known type from a method invocation,
it can be seen as a kind of \nameref{pat:KnownReturnType} pattern.






A cast to a method access to that reads a static resource file, \eg,
XML, HTML or \java{} properties file.
The file is static since its contents are known at compile-time.
Usually this file is build with a third-party tool, like a GUI designer.

In the following example,%
\footnote{\url{http://bit.ly/pwittchen_NetworkEvents_2HGbrMq}}
a cast is applied to a \code{findViewById} method invocation.
The \code{findViewById} method looks up for the given ID in a XML resource file to retrieve the specified view. 

%https://lgtm.com/projects/g/pwittchen/NetworkEvents/snapshot/dist-2032650416-1524814812150/files/example/src/main/java/com/github/pwittchen/networkevents/app/MainActivity.java?sort=name&dir=ASC&mode=heatmap#L65
\begin{minted}[highlightlines=6]{java}
@Override
protected void onCreate(Bundle savedInstanceState) {
    super.onCreate(savedInstanceState);
    setContentView(R.layout.activity_main);
    connectivityStatus = (TextView) findViewById(R.id.connectivity_status);
    mobileNetworkType = (TextView) findViewById(R.id.mobile_network_type);
    accessPoints = (ListView) findViewById(R.id.access_points);
    busWrapper = getOttoBusWrapper(new Bus());
    networkEvents = new NetworkEvents(getApplicationContext(), busWrapper)
        .enableInternetCheck()
        .enableWifiScan();
}
\end{minted}

The next listing,%
\footnote{\url{http://bit.ly/pentaho_pentaho-kettle_2TswNSf}}
shows a cast to a GUI component (\code{XulListbox}) using the \code{getElementById} method (lines 12 and 13).
In this case the developer is using the XUL language.%
\footnote{\url{https://developer.mozilla.org/en-US/docs/Mozilla/Tech/XUL}}

%https://lgtm.com/projects/g/pentaho/pentaho-kettle/snapshot/dist-1815472020-1524814812150/files/ui/src/main/java/org/pentaho/di/ui/repository/controllers/RepositoriesController.java?sort=name&dir=ASC&mode=heatmap#L115
\begin{minted}[highlightlines=12-13]{java}
private void createBindings() {
    loginDialog = (XulDialog) document
                    .getElementById( "repository-login-dialog" );
    repositoryEditButton = (XulButton) document
                    .getElementById( "repository-edit" );
    repositoryRemoveButton = (XulButton) document
                    .getElementById( "repository-remove" );
    username = (XulTextbox) document
                    .getElementById( "user-name" );
    userPassword = (XulTextbox) document
                    .getElementById( "user-password" );
    availableRepositories = (XulListbox) document
                    .getElementById( "available-repository-list" );
    showAtStartup = (XulCheckbox) document
                    .getElementById( "show-login-dialog-at-startup" );
    okButton = (XulButton) document
                    .getElementById( "repository-login-dialog_accept" );
    cancelButton = (XulButton) document
                    .getElementById( "repository-login-dialog_cancel" );
    // [...]
}
\end{minted}

These casts could be solved by using code generation,
or partial classes like in \csharp{}.
Since the contents of the resource file are known at compile-time,
code generation could be used to generate the corresponding \java{} code.

This is a pattern most often seen when using the Android platform.
The Butter Knife framework%
\footnote{\url{http://jakewharton.github.io/butterknife/}}
make use of annotations to avoid the ``manual'' casting.
Instead, code is generated the cast the result of \code{findViewById}.

This pattern is similar to \nameref{pat:LookupById},
since both use a key or ID to look up in a collection and cast the result.
However, the difference is how the content value was generated.
In the \nameref{pat:LookupById} pattern,
the developer ensures in another class the return value,
whereas in the \thisp{} pattern the content is given by a static resource file.








This pattern is used to stash an application-specific value,
typically a ``tag'' value in a GUI object or message payload.

In the following snippet,%
\footnote{\url{http://bit.ly/proninyaroslav_libretorrent_2TxpZCM}}
a cast is applied to a \code{getSerializable} invocation (lines 15 and 16).
This method gets a \code{Serializable} value given the specified key, \code{TAG\_CUR\_DIR} in this case.
To set a value with a specified key, the \code{putSerializable} method is used.
The mentioned cast succeeds because a value of the appropriate type is set in line 28 using the \code{putSerializable} method.

%https://lgtm.com/projects/g/proninyaroslav/libretorrent/snapshot/dist-1506096326216-1524814812150/files/app/src/main/java/org/proninyaroslav/libretorrent/fragments/DetailTorrentFilesFragment.java?sort=name&dir=ASC&mode=heatmap#L166
\begin{minted}[highlightlines=15-16]{java}
private TorrentContentFileTree curDir;

@Override
public void onActivityCreated(@Nullable Bundle savedInstanceState) {
    super.onActivityCreated(savedInstanceState);
    if (activity == null)
        activity = (AppCompatActivity) getActivity();
    if (savedInstanceState != null) {
        files = (ArrayList<BencodeFileItem>) savedInstanceState
                                .getSerializable(TAG_FILES);
        priorities = (ArrayList<FilePriority>) savedInstanceState
                                .getSerializable(TAG_PRIORITIES);
        fileTree = (TorrentContentFileTree) savedInstanceState
                                .getSerializable(TAG_FILE_TREE);
        curDir = (TorrentContentFileTree) savedInstanceState
                                .getSerializable(TAG_CUR_DIR);
    } else {
        makeFileTree();
    }
    // [...]
}

@Override
public void onSaveInstanceState(Bundle outState) {
    outState.putSerializable(TAG_FILES, files);
    outState.putSerializable(TAG_PRIORITIES, priorities);
    outState.putSerializable(TAG_FILE_TREE, fileTree);
    outState.putSerializable(TAG_CUR_DIR, curDir);
}
\end{minted}

In the following example,%
\footnote{\url{http://bit.ly/UniTime_cpsolver_2HUmGki}}
a cast is applied to the result of looking up by index in the \code{iContexts} map (line 9).
In case there is no value for the given index,
a value of the corresponding type is stored using the same index (line 13),
thus guaranteeing the success of the cast.

%https://lgtm.com/projects/g/UniTime/cpsolver/snapshot/dist-4860376-1524814812150/files/src/org/cpsolver/ifs/assignment/context/AssignmentContextHolderMap.java?sort=name&dir=ASC&mode=heatmap#L47
\begin{minted}[highlightlines=9]{java}
protected Map<Integer,AssignmentContext> iContexts =
                new HashMap<Integer, AssignmentContext>();

@Override
@SuppressWarnings("unchecked")
public <U extends AssignmentContext> U getAssignmentContext(
                Assignment<V, T> assignment,
                AssignmentContextReference<V, T, U> reference) {
    U context = (U) iContexts.get(reference.getIndex());
    if (context != null) return context;
    
    context = reference.getParent().createAssignmentContext(assignment);
    iContexts.put(reference.getIndex(), context);
    return context;
}
\end{minted}

In the last example,%
\footnote{\url{http://bit.ly/ggp-org_ggp-base_2SAEXHu}}
the cast is applied to a \code{getModel} invocation on the \code{matchTable} field (line 16).
Looking how \code{matchTable} is initialized (line 7),
the \code{model} variable (line 5) is used as an argument to the constructor.
This argument is the value returned by \code{getModel},
and since they are both of the same type,
the mentioned cast is guaranteed to succeed.

%https://lgtm.com/projects/g/ggp-org/ggp-base/snapshot/dist-59800051-1524814812150/files/src/main/java/org/ggp/base/apps/player/match/MatchPanel.java?sort=name&dir=ASC&mode=heatmap#L66
\begin{minted}[highlightlines=16]{java}
public final class MatchPanel extends JPanel implements Observer {
    private final JZebraTable matchTable;
    public MatchPanel() {
        super(new GridBagLayout());
        DefaultTableModel model = new DefaultTableModel();
        // [...]
        matchTable = new JZebraTable(model) {
            @Override
            public boolean isCellEditable(int rowIndex, int colIndex) {
                return false;
            }
        };
    }
    // [...]
    private void observe(GamerCompletedMatchEvent event) {
        DefaultTableModel model = (DefaultTableModel) matchTable.getModel();
        model.setValueAt("Inactive", model.getRowCount() - 1, 4);
    }
}
\end{minted}

Usually, the client application knows at compile-time the kind of values that the \thisp{} will use.

This pattern is closely related to the \nameref{pat:LookupById} pattern.
The \thisp{} pattern can also be used to fetch a value from a collection (as in \nameref{pat:LookupById}).
The main difference is ``locality''.
That is, in the \thisp{} pattern the cast value is set ``locally'', \ie,
in the same method or class,
whereas the cast value in the \nameref{pat:LookupById} pattern is usually set in another class.

The \nameref{pat:VariableSupertype} pattern resembles this pattern,
since a value is stashed using a less specific type.





\end{pattern}