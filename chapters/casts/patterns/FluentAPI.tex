\begin{pattern}{FluentAPI}
A fluent \api{} is an \api{} that allows the developer to operate on the same
object using method chaining.
This pattern is exhibited when the receiver (\code{this} reference) is cast to a type parameter which is itself bounded by the self type.


\instances{}
In the following snippet,
the receiver (\code{this}) is cast to a type parameter (\code{B}) (line 5).
This allows subclasses to reuse the methods in the base class without overriding them just to change the return type.

%https://lgtm.com/projects/g/HanSolo/Medusa/snapshot/dist-1798710811-1524814812150/files/src/main/java/eu/hansolo/medusa/ClockBuilder.java?sort=name&dir=ASC&mode=heatmap#L293
\def\urlvar{http://bit.ly/HanSolo_Medusa_2TyBObH}
\begin{minted}[highlightlines=5]{java}
public class ClockBuilder <B extends ClockBuilder<B>> {
    // [...]
    public final B alarms(final Alarm... ALARMS) {
        properties.put("alarmsArray", new SimpleObjectProperty<>(ALARMS));
        return (B) this;
    }
} #\urlbox#
\end{minted}

The following example implements \thisp{} by directly casting the receiver
(\code{this}) in line 3.
Similarly to the \code{addAllThrown} method,
the rest of methods in the enclosing class perform a cast to \code{this}.
Although there is a lot of boilerplate code,
this instance happens in generated code.
The cast succeeds because there is a guard (line 8) in the constructor that 
guarantees the receiver is of the appropriate type (\cf{} \nameref{pat:Typecase}).

% https://lgtm.com/projects/g/immutables/immutables/snapshot/dist-43930039-1524814812150/files/value-processor/target/generated-sources/annotations/org/immutables/value/processor/encode/ImmutableEncodedElement.java?sort=name&dir=ASC&mode=heatmap#L1734
\def\urlvar{http://bit.ly/immutables_immutables_2S4BoJs}
\begin{minted}[highlightlines=3]{java}
public final EncodedElement.Builder addAllThrown(
            Iterable<? extends Type> elements) {
    this.thrown.addAll(elements);
    return (EncodedElement.Builder) this;
}

public Builder() {
    if (!(this instanceof EncodedElement.Builder)) {
        throw new UnsupportedOperationException("/* [...] */");
    }
} #\urlbox#
\end{minted}


\detection{}
The Query~\ref{lst:ql:FluentAPICast} detects the \thisp{} pattern.
The query detects the case like the first example.

\begin{listing}
\begin{minted}{\qllexer}
class FluentAPICast extends #\qlref{Cast}# { #\qlbox#
	TypeVariable x;
	GenericType enclosingClass;
	FluentAPICast() {
		getExpr() instanceof ThisAccess and
		getParent() instanceof ReturnStmt and
		x = getTargetType() and
		enclosingClass = getExpr().getType() and
		x.hasTypeBound() and
		x.getFirstTypeBound().getType() = enclosingClass and
		x.getFirstTypeBound().getType().(GenericType).getATypeParameter() = x
	}
}
\end{minted}
\caption{Detection of the \thisp{} pattern.}
\label{lst:ql:FluentAPICast}
\end{listing}

\issues{}
In most cases,
this pattern is concerned with a particular implementation of fluent \api{}s
where recursive generics are used to mimic self types~\citep{bruceChallengingTypingIssues2003}.
Other implementations of fluent \api{}s simply return \code{this} without a cast, but these are less extensible.

\end{pattern}
