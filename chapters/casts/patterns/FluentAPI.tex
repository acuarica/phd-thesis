\begin{pattern}{FluentAPI}
A fluent \api{} is an \api{} that allows the developer to operate on the same
object using method chaining.
This pattern is exhibited when the receiver (\code{this} reference) is cast to a type parameter which is itself bounded by the self type.


\instances{}
In the following snippet,
the receiver (\code{this}) is cast to a type parameter (\code{B}) (line 5).
This allows subclasses to reuse the methods in the base class without overriding them just to change the return type.

%https://lgtm.com/projects/g/HanSolo/Medusa/snapshot/dist-1798710811-1524814812150/files/src/main/java/eu/hansolo/medusa/ClockBuilder.java?sort=name&dir=ASC&mode=heatmap#L293
\def\urlvar{http://bit.ly/HanSolo_Medusa_2TyBObH}
\begin{minted}[highlightlines=5]{java}
public class ClockBuilder <B extends ClockBuilder<B>> {
    // [...]
    public final B alarms(final Alarm... ALARMS) {
        properties.put("alarmsArray", new SimpleObjectProperty<>(ALARMS));
        return (B) this;
    }
} #\urlbox#
\end{minted}

The following example implements \thisp{} by directly casting the receiver (\code{this}).
Similarly to the \code{addAllThrown} method, the rest of methods in the enclosing class cast \code{this}.

% https://lgtm.com/projects/g/immutables/immutables/snapshot/dist-43930039-1524814812150/files/value-processor/target/generated-sources/annotations/org/immutables/value/processor/encode/ImmutableEncodedElement.java?sort=name&dir=ASC&mode=heatmap#L1734
\def\urlvar{http://bit.ly/immutables_immutables_2S4BoJs}
\begin{minted}[highlightlines=3]{java}
public final EncodedElement.Builder addAllThrown(
            Iterable<? extends Type> elements) {
    this.thrown.addAll(elements);
    return (EncodedElement.Builder) this;
} #\urlbox#
\end{minted}


\issues{}
In most cases,
this pattern is concerned with a particular implementation of fluent \api{}s
where recursive generics are used to mimic self types~\cite{bruceChallengingTypingIssues2003}.
Other implementations of fluent \api{}s simply return \code{this} without a cast, but these are less extensible.

\end{pattern}
