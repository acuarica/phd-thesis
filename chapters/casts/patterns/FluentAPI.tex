\begin{pattern}{FluentAPI}
A fluent \api{} is an \api{} that allows the developer to operate on the same
object using method chaining.
This pattern is exhibited when the receiver (\code{this} reference) is cast to a type parameter
  which is itself bounded by the self type.


\instances{}
In the following snippet,%
\def\urlvar{http://bit.ly/HanSolo_Medusa_2TyBObH}
the receiver (\code{this}) is cast to a type parameter (\code{B}) (line 5).
This allows subclasses to reuse the methods in the base class without overriding them just to change the return type.

%https://lgtm.com/projects/g/HanSolo/Medusa/snapshot/dist-1798710811-1524814812150/files/src/main/java/eu/hansolo/medusa/ClockBuilder.java?sort=name&dir=ASC&mode=heatmap#L293
\begin{minted}[highlightlines=5]{java}
public class ClockBuilder <B extends ClockBuilder<B>> {
    // [...]
    public final B alarms(final Alarm... ALARMS) {
        properties.put("alarmsArray", new SimpleObjectProperty<>(ALARMS));
        return (B) this;
    }
} #\urlbox#
\end{minted}

\issues{}
  This pattern is concerned with a particular implementation of fluent \api{}s
  where recursive generics are used to mimic self
  types~\cite{bruceChallengingTypingIssues2003}.
  Other implementations of fluent \api{}s simply return \code{this} without a
  cast, but these are less extensible.

\end{pattern}
