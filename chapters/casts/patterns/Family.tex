\begin{pattern}{Family}
The \thisp{} pattern implements casts to provide a sort of
\emph{family polymorphism}~\citep{ernstFamilyPolymorphism2001}.
A ``family'' consists of multiple mutually-dependent types designed to
collaborate with each other.
Each type has a role in the family.
Deriving from a base family to form another family requires subclassing all the members of the base family,
with the subclasses in the new family retaining their roles in the new family.
Because method parameter types are invariant in \java{} and because covariant parameter types are unsound in general,
the method parameter types in the derived family are the same as in the base family.
Casts are therefore necessary for one member of a derived family to access
another member using its derived family type rather that its base family
type.

To detect this pattern, the cast needs to be applied to a family.
A family is distinguished by a covariant usage of a field or parameter in an overriding method.

\instances{}
The following example shows an instance of the \thisp{} pattern.
In this case, the interfaces \code{StepInterface}, \code{StepMetaInterface},
and \code{StepDataInterface} are part of a base family and the
\code{stopRunning} method has parameters of these types.
In the derived family the roles of these three interfaces are
implemented by the classes
\code{DynamicSQLRow},
\code{DynamicSQLRowMeta}, and
\code{DynamicSQLRowData}.
A cast is applied to the parameter \code{smi} of \code{stopRunning} in
\code{DynamicSQLRow} (line 12).
This cast is necessary to convert the method parameter,
of the base family type \code{StepDataInterface},
into the derived family type with the same role.
%https://lgtm.com/projects/g/pentaho/pentaho-kettle/snapshot/dist-1815472020-1524814812150/files/engine/src/main/java/org/pentaho/di/trans/steps/dynamicsqlrow/DynamicSQLRow.java?sort=name&dir=ASC&mode=heatmap#L281
\def\urlvar{http://bit.ly/pentaho_pentaho_kettle_2FN59J8}
\begin{minted}[highlightlines=12]{java}
public interface StepInterface extends VariableSpace, HasLogChannelInterface {
  // [...]
  public void stopRunning( StepMetaInterface stepMetaInterface,
          StepDataInterface stepDataInterface ) throws KettleException;
}
public class DynamicSQLRow extends BaseStep implements StepInterface {
  private DynamicSQLRowMeta meta;
  private DynamicSQLRowData data;
  // [...]
  public void stopRunning( StepMetaInterface smi, StepDataInterface sdi )
          throws KettleException {
    meta = (DynamicSQLRowMeta) smi;
    data = (DynamicSQLRowData) sdi;
    // [...]
  }
} #\urlbox#
\end{minted}

The next example is similar to the previous one.
The \code{masked} parameter is cast to \code{DoubleColumnVector} (line 5).
It is so because the \code{masked} variable is expected to hold an instance of \code{DoubleColumnVector} when the \code{maskData} method is applied to an object of type \code{DoubleIdentity}.
%https://lgtm.com/projects/g/apache/orc/snapshot/dist-1506201906740-1524814812150/files/java/core/src/java/org/apache/orc/impl/mask/DoubleIdentity.java?sort=name&dir=ASC&mode=heatmap#L32
\def\urlvar{http://bit.ly/apache_orc_2SE4C2m}
\begin{minted}[highlightlines=5]{java}
public class DoubleIdentity implements DataMask {
  @Override
  public void maskData(ColumnVector original, ColumnVector masked, int start,
                       int length) {
    DoubleColumnVector target = (DoubleColumnVector) masked;
    DoubleColumnVector source = (DoubleColumnVector) original;
    // [...]
  }
}
public interface DataMask {
  // [...]
  void maskData(ColumnVector original, ColumnVector masked,
                int start, int length);
} #\urlbox#
\end{minted}

In both previous examples,
cast instances were applied to a parameter in a overriding method.
In the next example, the cast instance is applied to super class field (line 12).
The field is declared in the \code{BaseExchange} class (line 20).
However, the field is initialized with a \code{BitflyerMarketDataService} value in line~5.

%https://lgtm.com/projects/g/knowm/XChange/snapshot/dist-4990076-1524814812150/files/xchange-bitflyer/src/main/java/org/knowm/xchange/bitflyer/BitflyerExchange.java?sort=name&dir=ASC&mode=heatmap#L52
\def\urlvar{http://bit.ly/knowm_XChange_2UPPDj9}
\begin{minted}[highlightlines=12]{java}
public class BitflyerExchange extends BaseExchange implements Exchange {
  // [...]
  @Override
  protected void initServices() {
    this.marketDataService = new BitflyerMarketDataService(this);
    // [...]
  }
  // [...]
  @Override
  public void remoteInit() throws IOException, ExchangeException {
    BitflyerMarketDataServiceRaw dataService =
        (BitflyerMarketDataServiceRaw) this.marketDataService;
    List<BitflyerMarket> markets = dataService.getMarkets();
    exchangeMetaData = BitflyerAdapters.adaptMetaData(markets);
  }
}
public abstract class BaseExchange implements Exchange {
  // [...]
  protected MarketDataService marketDataService;
  // [...]
} #\urlbox#
\end{minted}

\issues{}
\java{} itself does not support statically type-safe family polymorphism directly and so casts are often necessary.
Various proposals have been made to better support family polymorphism 
(and the related ``expression problem''~\citep{Wadler98ExpressionProblem}) 
in object-oriented languages,
including the use of design patterns~\citep{WangOliveira16ExpressionProblem, oliveiraExtensibilityMasses2012, nystromPolyglotExtensibleCompiler2003},
and type systems~\citep{gbeta, scalaIndependentlyExtensible,
Myers2006SoftwareCW, olivieraDisjointIntersectionTypes,
funWithTypeFunctionsKiselyov09} that
permit some restricted form of covariant method parameters.

\end{pattern}