\begin{pattern}{CovariantReturnType}
The \thisp{} pattern is used to cast a call to a method that returns
an instance of a type that is covariant with the receiver type.
Commonly the method returns a instance of the receiver type itself.

\instances{}
A common instance of this pattern is for calls to the \code{clone} method of \code{java.lang.Object} (\nCovariantReturnTypeCloneSubpattern{} instances),
which returns an object of the same type as the receiver,
but whose static type is \code{Object}.
The following snippet shows a cast to the \code{clone} method.

%https://lgtm.com/projects/g/aws/aws-sdk-java/snapshot/dist-6950065-1524814812150/files/aws-java-sdk-elasticache/src/main/java/com/amazonaws/services/elasticache/model/ListTagsForResourceResult.java?sort=name&dir=ASC&mode=heatmap#L156
\def\urlvar{http://bit.ly/aws_aws_sdk_java_2GvHhYt}
\begin{minted}[highlightlines=4]{java}
@Override
public ListTagsForResourceResult clone() {
    try {
        return (ListTagsForResourceResult) super.clone();
    } catch (CloneNotSupportedException e) {
        throw new IllegalStateException(/* [...] */);
    }
} #\urlbox#
\end{minted}

In the following example,
the \code{unmarshall} method overrides a superclass method with a covariant return type.
A cast is used on the call to the superclass method to change the type of the return value to match the more precise return type.

%https://lgtm.com/projects/g/aws-amplify/aws-sdk-android/snapshot/dist-2970378-1524814812150/files/aws-android-sdk-autoscaling/src/main/java/com/amazonaws/services/autoscaling/model/transform/ResourceContentionExceptionUnmarshaller.java?sort=name&dir=ASC&mode=heatmap#L39
\def\urlvar{http://bit.ly/aws_amplify_aws_sdk_android_2FVWl13}
\begin{minted}[highlightlines=11]{java}
public class ResourceContentionExceptionUnmarshaller extends StandardErrorUnmarshaller {
    public ResourceContentionExceptionUnmarshaller() {
        super(ResourceContentionException.class);
    }
    public AmazonServiceException unmarshall(Node node) throws Exception {
        // Bail out if this isn't the right error code that this
        // marshaller understands.
        String errorCode = parseErrorCode(node);
        if (errorCode == null || !errorCode.equals("ResourceContention"))
            return null;
        ResourceContentionException e = (ResourceContentionException) super.unmarshall(node);
        return e;
    }
} #\urlbox#
\end{minted}

The \code{initCause} method---from the \code{java.lang.Throwable} class---has return type \code{Throwable}.
Nevertheless, this method returns the receiver (after setting the cause exception).
Therefore a cast is needed to recover the original exception type,
as shown in the following example.
This use case resembles the \nameref{pat:FluentAPI} pattern.

%https://lgtm.com/projects/g/apache/activemq/snapshot/dist-11730660-1524814812150/files/activemq-client/src/main/java/org/apache/activemq/ActiveMQConnectionFactory.java?sort=name&dir=ASC&mode=heatmap#L235
\def\urlvar{http://bit.ly/apache_activemq_2EnSivc}
\begin{minted}[highlightlines=1]{java}
throw (IllegalArgumentException)
        new IllegalArgumentException("Invalid broker URI: " + brokerURL)
        .initCause(e);
#\urlbox#
\end{minted}


\issues{}
The case of returning \code{this} could be avoided if \java{} supported self types~\citep{bruceChallengingTypingIssues2003}.
More generally, associated types~\citep{chakravartyAssociatedTypeSynonyms2005} can provide a statically typed solution,
for instance in the second example above.

\end{pattern}