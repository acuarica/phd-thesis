\begin{pattern}{TypeTag}
%
\done{Matthias: Your first example doesn't have the tag in the *same* object, but passes it as a separate variable.}
%
A cast instance belonging to the \thisp{} pattern is guarded by an application-specific test instead of using an \code{instanceof} test.

\instances{}
The following example%
\footnote{\url{http://bit.ly/JesusFreke_smali_2Ho8bVL}}
shows an instance of the \thisp{} pattern.
The cast type of the parameter \code{reference} is determined by the value of the parameter \code{referenceType}.

%https://lgtm.com/projects/b/JesusFreke/smali/snapshot/dist-1306230039-1524814812150/files/dexlib2/src/main/java/org/jf/dexlib2/writer/InstructionWriter.java?sort=name&dir=ASC&mode=heatmap#L492
\begin{minted}[highlightlines=8]{java}
private int getReferenceIndex(int referenceType, Reference reference) {
    switch (referenceType) {
        case ReferenceType.FIELD:
            return fieldSection.getItemIndex((FieldRefKey) reference);
        case ReferenceType.METHOD:
            return methodSection.getItemIndex((MethodRefKey) reference);
        case ReferenceType.STRING:
            return stringSection.getItemIndex((StringRef) reference);
        case ReferenceType.TYPE:
            return typeSection.getItemIndex((TypeRef) reference);
        case ReferenceType.METHOD_PROTO:
            return protoSection.getItemIndex((ProtoRefKey) reference);
        default:
            throw new ExceptionWithContext(
                "Unknown reference type: %d",  referenceType);
    }
}
\end{minted}

In the next example,%
\footnote{\url{http://bit.ly/FenixEdu_fenixedu-academic_2SUNOUJ}}
instead of a \code{switch} statement,
an \code{if} statement is used to guard the cast (in line 6).

%https://lgtm.com/projects/g/FenixEdu/fenixedu-academic/snapshot/dist-29270029-1524814812150/files/src/main/java/org/fenixedu/academic/ui/renderers/student/curriculum/StudentCurricularPlanRenderer.java?sort=name&dir=ASC&mode=heatmap#L853
\begin{minted}[highlightlines=6]{java}
for (final IEnrolment enrolment : dismissal.getSourceIEnrolments()) {
    if (enrolment.isExternalEnrolment()) {
        generateExternalEnrolmentRow(mainTable, (ExternalEnrolment) enrolment,
                level + 1, true);
    } else {
        generateEnrolmentRow(mainTable, (Enrolment) enrolment,
                level + 1, false, true, true);
    }
}
\end{minted}

\done{Nate: Why is this not pattern matching?}
%
In the next case%
\footnote{\url{http://bit.ly/apache_poi_2FW5SXU}}
a type test is performed --- through a method call --- before actually applying the cast to the variable \code{props} (line 3).
Note that the type test is internally using the \code{instanceof} operator (line 8).
Although in this case the type test is using an \code{instanceof} operator,
it is not considered \nameref{pat:PatternMatching} because the \code{instanceof} is located in a method call.

%https://lgtm.com/projects/g/apache/poi/snapshot/dist-1790760597-1524814812150/files/src/ooxml/java/org/apache/poi/xslf/usermodel/XSLFPropertiesDelegate.java?sort=name&dir=ASC&mode=heatmap#L1367
\begin{minted}[highlightlines=3]{java}
@Override
public CTSolidColorFillProperties getSolidFill() {
    return isSetSolidFill() ? (CTSolidColorFillProperties)props : null;
}

@Override
public boolean isSetSolidFill() {
    return (props instanceof CTSolidColorFillProperties);
}
\end{minted}

In some cases, the type to be cast is the same in every branch.
The following snippet%
\footnote{\url{http://bit.ly/loopj_android-async-http_2IpIULk}}
shows an instance of this case.
The cast is applied to the \code{message.obj} field to (line 13),
according to the value of the \code{message.what} field (line 1).
However, a similar cast is applied in the first branch (line 3).
In both branches \code{message.obj} is of type \code{Object[]},
but in the case of \code{FAILURE\_MESSAGE},
the array contains one more element (line 16).
This suggests that the \code{(Object[]) message.obj} array denotes two different objects,
but are not distinguishable from the type system perspective.

%https://lgtm.com/projects/g/loopj/android-async-http/snapshot/dist-1879340034-1549372228293/files/library/src/main/java/com/loopj/android/http/AsyncHttpResponseHandler.java?sort=name&dir=ASC&mode=heatmap#L359
\begin{minted}[highlightlines=13]{java}
switch (message.what) {
    case SUCCESS_MESSAGE:
        response = (Object[]) message.obj;
        if (response != null && response.length >= 3) {
            onSuccess((Integer) response[0], (Header[]) response[1],
                    (byte[]) response[2]);
        } else {
            AsyncHttpClient.log.e(LOG_TAG, 
                    "SUCCESS_MESSAGE didn't got enough params");
        }
        break;
    case FAILURE_MESSAGE:
        response = (Object[]) message.obj;
        if (response != null && response.length >= 4) {
            onFailure((Integer) response[0], (Header[]) response[1],
                    (byte[]) response[2], (Throwable) response[3]);
        } else {
            AsyncHttpClient.log.e(LOG_TAG,
                    "FAILURE_MESSAGE didn't got enough params");
        }
        break;
    // [...]
}
\end{minted}

In the next example,%
\footnote{\url{http://bit.ly/groovy_groovy-core_2SGzK16}}
the parameter \code{args} is cast to \code{Object[]} (line 13).
The ``type tag'' is given by the fact that the cast is executed in a \code{catch} block,
and that \code{value} is an instance of \code{Closure} (line 9).
The \code{args} parameter flows into two methods,
\code{invokeMethod(String name, Object args)}
and
\code{call(Object... args)}%
.
Thus, \code{args} is treated as an \code{Object} or \code{Object[]} depending on the type tag,
resembling an union type.

%https://lgtm.com/projects/g/groovy/groovy-core/snapshot/dist-45390050-1524814812150/files/src/main/groovy/util/Expando.java?sort=name&dir=ASC&mode=heatmap#L103
\begin{minted}[highlightlines=13]{java}
public Object invokeMethod(String name, Object args) {
    try {
        return super.invokeMethod(name, args);
    }
    catch (GroovyRuntimeException e) {
        // br should get a "native" property match first.
        // getProperty includes such fall-back logic
        Object value = this.getProperty(name);
        if (value instanceof Closure) {
            Closure closure = (Closure) value;
            closure = (Closure) closure.clone();
            closure.setDelegate(this);
            return closure.call((Object[]) args);
        } else {
            throw e;
        }
    }
}
\end{minted}


\detection{}
The detection of this pattern is similar to the \nameref{pat:PatternMatching} detection, but instead of looking for an \code{instanceof} guarded cast, we look for an application-specific guard.
The guard needs to determine either the resulting type of the cast instance, or
the subsequent operations applied to the result of the cast instance if the types in every branch are the same.

\discussion{}
In some cases, the \thisp{} pattern can be replaced by \nameref{pat:PatternMatching}.
However, if the application-specific tag is a numeric value,
the \thisp{} could perform better than the \nameref{pat:PatternMatching} using \code{instanceof}.
Moreover, there are situation where the \thisp{} can not be avoid since the types to be cast are the same.

\related{}
%
\done{Nate: PatternMatching}
This pattern is related to \nameref{pat:PatternMatching} since both denoted guarded casts.
The difference is that \thisp{} uses an application-specific test.
\nameref{pat:GetByClassLiteral} could be seen as a special case of \thisp{} where the tag is a class literal.

\end{pattern}