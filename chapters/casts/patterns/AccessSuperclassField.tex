\begin{pattern}{AccessSuperclassField}
Perform an upcast to access a field of a superclass of the cast operand.

\instancesgen{}
The following snippet shows an instance of this pattern.

%https://lgtm.com/projects/g/FenixEdu/fenixedu-academic/snapshot/dist-29270029-1524814812150/files/target/generated-test-sources/dml-maven-plugin/org/fenixedu/academic/domain/residence/StudentsPerformanceReport_Base.java?sort=name&dir=ASC&mode=heatmap#L12
\def\urlvar{http://bit.ly/FenixEdu_fenixedu_academic_2SQxlkC}
\begin{minted}[highlightlines=4]{java}
public abstract class StudentsPerformanceReport_Base extends QueueJobWithFile {
    // [...]
    public ExecutionSemester getValue(StudentsPerformanceReport o1) {
        return ((StudentsPerformanceReport_Base)o1).executionSemester.get();
    }
    private OwnedVBox<ExecutionSemester> executionSemester;
}
public class StudentsPerformanceReport extends StudentsPerformanceReport_Base {
    // [...]
} #\urlbox#
\end{minted}


\detection{}
The Query~\ref{lst:ql:AccessSuperclassFieldCast} detects the two variants of this pattern.
The first variant, as shown in the example below, is when an upcast is performed to access a field when the subclass does not have access privileges to access the field.
In this case, the field is declared as either \code{private} or \code{protected} in the superclass.
On the other variant---not found in our manual sample---an upcast is performed to access a field declared in a superclass,
when the subclass declares a field with the same name.

\begin{listing}
\begin{minted}{\qllexer}
class AccessSuperclassFieldCast extends #\qlref{Cast}# { #\qlbox#
	FieldAccess fa;
	AccessSuperclassFieldCast() {
		this instanceof Upcast and
		fa.getQualifier().getProperExpr() = this and (
		getExpr().getType().(RefType).declaresField(fa.getField().getName()) or
		( fa.getField().isPrivate() or fa.getField().isProtected() )
		)
	}
}
\end{minted}
\caption{Detection of the \thisp{} pattern.}
\label{lst:ql:AccessSuperclassFieldCast}
\end{listing}

As in our first example,
the following snippet shows an example of the second variant mentioned above.

%https://lgtm.com/projects/g/mockito/mockito/snapshot/da68900466a17e21fef3e27690f4cef4b5c240ea/files/src/test/java/org/mockito/internal/util/reflection/LenientCopyToolTest.java?sort=name&dir=ASC&mode=heatmap&showExcluded=false#L120
\def\urlvar{http://bit.ly/mockito_mockito_2vF51Em}
\begin{listing}
\begin{minted}[highlightlines=4]{java}
private SomeObject from = new SomeObject(100);

assertThat(((InheritMe) to).privateInherited)
				.isNotEqualTo(((InheritMe) from).privateInherited);

static class InheritMe {
	protected String protectedInherited = "protected";
	private String privateInherited = "private";
}
public static class SomeObject extends InheritMe {
} #\urlbox#
\end{minted}
\end{listing}


\issues{}
The particular instance we encountered has a method whose parameter is a subclass of the current class.
The cast is needed to access a private field of the current class.
Being an upcast, the cast is always safe.
More problematic is the strong coupling between the base class and the derived class,
however the base class is generated code;
possibly, a manually written version would just combine the two classes.

Another use of the pattern, not found in our sample however,
is to upcast a value to access a field of a superclass which is shadowed by another field of the same name in the subclass.

The \nameref{pat:ReflectiveAccessibility} pattern is also used to access private or protected fields,
albeit fields of unrelated classes that cannot be accessed simply by casting to another type.
Like \nameref{pat:SoleSubclassImplementation},
this pattern occurs when there is high cohesion between super and subclass.

Both the \nameref{pat:SelectOverload} and this pattern are used to select class members.
While this pattern is used to select a field in a superclass,
the \nameref{pat:SelectOverload} is used to select the appropriate overloaded method.

\end{pattern}
