\begin{pattern}{Typecase}
The \thisp{} pattern consists of dispatching to different cases
depending on the run-time type of the source value.
The run-time type is tested against known subtypes of the operand type,
with each test followed by a cast to that type.
The guard may be implemented using one of three variants:
an \code{instanceof} operator (\variant{GuardByInstanceOf}),
a comparison of the runtime class against a class literal (\variant{GuardByClassLiteral}),
or an application-specific type tag (\variant{GuardByTypeTag}).

\instances{}
\thisp{} is by far the most common pattern.
Figure~\ref{fig:casts:typecase} shows the different variants of the pattern.
The \variant{GuardByInstanceOf} is the most used variant.
Often there is just one case and the default case, \ie,
when the guard fails, performs a no-op or reports an error.

\plot{patterns/table-pattern-Typecase-features}{fig:casts:typecase}{\thisp{} Variant Occurrences}

The following listing shows an example of the \thisp{} pattern,
using the \variant{GuardByInstanceOf} variant.

%https://lgtm.com/projects/g/PenguinSquad/Enchiridion/snapshot/dist-19218583-1524814812150/files/build/sources/main/java/joshie/enchiridion/helpers/StackHelper.java?sort=name&dir=ASC&mode=heatmap#L27
\def\urlvar{http://bit.ly/PenguinSquad_Enchiridion_2HnNwB7}
\begin{minted}[highlightlines=6]{java}
if (object instanceof Item) {
	return getStringFromStack(new ItemStack((Item) object));
} else if (object instanceof Block) {
	return getStringFromStack(new ItemStack((Block) object));
} else if (object instanceof ItemStack) {
	return getStringFromStack((ItemStack) object);
} else if (object instanceof String) {
	return (String) object;
} else if (object instanceof List) {
	return getStringFromStack((ItemStack) ((List) object).get(0));
} else return ""; #\urlbox#
\end{minted}

In the next case a type test is performed---through a method call---before actually applying the cast to the variable \code{props} (line 3).
Note that the type test is internally using the \code{instanceof} operator (line 8).

%https://lgtm.com/projects/g/apache/poi/snapshot/dist-1790760597-1524814812150/files/src/ooxml/java/org/apache/poi/xslf/usermodel/XSLFPropertiesDelegate.java?sort=name&dir=ASC&mode=heatmap#L1367
\def\urlvar{http://bit.ly/apache_poi_2FW5SXU}
\begin{minted}[highlightlines=3]{java}
@Override
public CTSolidColorFillProperties getSolidFill() {
    return isSetSolidFill() ? (CTSolidColorFillProperties) props : null;
}
@Override
public boolean isSetSolidFill() {
    return (props instanceof CTSolidColorFillProperties);
} #\urlbox#
\end{minted}

Another common scenario is when several cases are used to re-\code{throw}
an exception of the right type, as shown below.
The cast instance is applied to a variable of type \code{Throwable}
(line 13).
Nevertheless, the enclosing method is only allowed to throw \code{NamingException} by the \code{throws} declaration (line 3).
Since an exception of type \code{Throwable} is checked,
a cast to \code{VirtualMachineError} (subclass of \code{Error}) is needed.

%https://lgtm.com/projects/g/codefollower/Tomcat-Research/snapshot/dist-13734061-1524814812150/files/java/org/apache/naming/factory/DataSourceLinkFactory.java?sort=name&dir=ASC&mode=heatmap#L85
\def\urlvar{http://bit.ly/codefollower_Tomcat_Research_2SGDUG5}
\begin{minted}[highlightlines=13]{java}
protected Object wrapDataSource(
			Object datasource, String username, String password)
			throws NamingException {
	try {
		// [...]
	}catch (Exception x) {
		if (x instanceof InvocationTargetException) {
			Throwable cause = x.getCause();
			if (cause instanceof ThreadDeath) {
				throw (ThreadDeath) cause;
			}
			if (cause instanceof VirtualMachineError) {
				throw (VirtualMachineError) cause;
			}
			if (cause instanceof Exception) {
				x = (Exception) cause;
			}
		}
		if (x instanceof NamingException) throw (NamingException)x;
		else {
			// [...]
		}
	}
} #\urlbox#
\end{minted}

The next example shows that \thisp{} can also be used to filter elements by type within a stream.
The cast is applied to stream operations (line 1) over the \code{caseAssignments} collection.
The \code{instanceof} guard is tested in line 2.

%https://lgtm.com/projects/g/kiegroup/jbpm/snapshot/dist-1810050-1524814812150/files/jbpm-case-mgmt/jbpm-case-mgmt-impl/src/main/java/org/jbpm/casemgmt/impl/wih/StartCaseWorkItemHandler.java?sort=name&dir=ASC&mode=heatmap#L212
\def\urlvar{http://bit.ly/kiegroup_jbpm_2ENCL8a}
\begin{minted}[highlightlines=1-4]{java}
user = (User) caseAssignments
                .stream().filter(oe -> oe instanceof User)
                .findFirst()
                .orElseThrow(() -> new IllegalArgumentException());
#\urlbox#
\end{minted}

Rather than using an \code{instanceof} guard, in the following example
the target type of the parameter \code{reference} is determined by the value
of the parameter \code{referenceType},
which acts as a \emph{type tag} for \code{reference}.

%https://lgtm.com/projects/b/JesusFreke/smali/snapshot/dist-1306230039-1524814812150/files/dexlib2/src/main/java/org/jf/dexlib2/writer/InstructionWriter.java?sort=name&dir=ASC&mode=heatmap#L492
\def\urlvar{http://bit.ly/JesusFreke_smali_2Ho8bVL}
\begin{minted}[highlightlines=7]{java}
switch (referenceType) {
    case ReferenceType.FIELD:
        return fieldSection.getItemIndex((FieldRefKey) reference);
    case ReferenceType.METHOD:
        return methodSection.getItemIndex((MethodRefKey) reference);
    case ReferenceType.STRING:
        return stringSection.getItemIndex((StringRef) reference);
    case ReferenceType.TYPE:
        return typeSection.getItemIndex((TypeRef) reference);
    case ReferenceType.METHOD_PROTO:
        return protoSection.getItemIndex((ProtoRefKey) reference);
    default:
        throw new ExceptionWithContext("/* [...] */", referenceType);
} #\urlbox#
\end{minted}

In some cases, the target types of the casts are the same in every branch.
In the following snippet,
the cast is applied to the \code{message.obj} field to (line 11),
according to the value of the tag \code{message.what} field (line 1).
However, a similar cast is applied in the first branch (line 3).
In both branches \code{message.obj} is of type \code{Object[]},
but with different lengths.
The casts in the calls to \code{onSuccess} and
\code{onFailure} (lines 5, 13--14)
are instances of the \nameref{pat:ObjectAsArray} pattern.

%https://lgtm.com/projects/g/loopj/android-async-http/snapshot/dist-1879340034-1549372228293/files/library/src/main/java/com/loopj/android/http/AsyncHttpResponseHandler.java?sort=name&dir=ASC&mode=heatmap#L359
\def\urlvar{http://bit.ly/loopj_android_async_http_2IpIULk}
\begin{minted}[highlightlines=10]{java}
switch (message.what) {
    case SUCCESS_MESSAGE:
        response = (Object[]) message.obj;
        if (response != null && response.length >= 3) {
            onSuccess((Integer) response[0], (Header[]) response[1],
					(byte[]) response[2]);
        } else { /* [...] */ }
        break;
    case FAILURE_MESSAGE:
        response = (Object[]) message.obj;
        if (response != null && response.length >= 4) {
            onFailure((Integer) response[0], (Header[]) response[1],
                    (byte[]) response[2], (Throwable) response[3]);
        } else { /* [...] */ }
        break;
    // [...]
} #\urlbox#
\end{minted}

In the next example, instead of a \code{switch},
an \code{if} statement is used to guard the cast (in line 6).

%https://lgtm.com/projects/g/FenixEdu/fenixedu-academic/snapshot/dist-29270029-1524814812150/files/src/main/java/org/fenixedu/academic/ui/renderers/student/curriculum/StudentCurricularPlanRenderer.java?sort=name&dir=ASC&mode=heatmap#L853
\def\urlvar{http://bit.ly/FenixEdu_fenixedu_academic_2SUNOUJ}
\begin{minted}[highlightlines=6]{java}
for (final IEnrolment enrolment : dismissal.getSourceIEnrolments()) {
    if (enrolment.isExternalEnrolment()) {
        generateExternalEnrolmentRow(mainTable, (ExternalEnrolment) enrolment,
                level + 1, true);
    } else {
        generateEnrolmentRow(mainTable, (Enrolment) enrolment,
                level + 1, false, true, true);
    }
} #\urlbox#
\end{minted}

In the next example,
the parameter \code{args} is cast to \code{Object[]} (line 13).
The ``type tag'' is given by the fact that the cast is executed in a \code{catch} block,
and that \code{value} is an instance of \code{Closure} (line 9).
The \code{args} parameter flows into two methods,
\code{invokeMethod(String name, Object args)} and
\code{call(Object... args)}.
Thus, \code{args} is treated as an \code{Object} or \code{Object[]} depending on the ``type tag'', resembling a union type.

%https://lgtm.com/projects/g/groovy/groovy-core/snapshot/dist-45390050-1524814812150/files/src/main/groovy/util/Expando.java?sort=name&dir=ASC&mode=heatmap#L103
\def\urlvar{http://bit.ly/groovy_groovy_core_2SGzK16}
\begin{minted}[highlightlines=13]{java}
public Object invokeMethod(String name, Object args) {
    try {
        return super.invokeMethod(name, args);
    }
    catch (GroovyRuntimeException e) {
        // br should get a "native" property match first.
        // getProperty includes such fall-back logic
        Object value = this.getProperty(name);
        if (value instanceof Closure) {
            Closure closure = (Closure) value;
            closure = (Closure) closure.clone();
            closure.setDelegate(this);
            return closure.call((Object[]) args);
        } else {
            throw e;
        }
    }
} #\urlbox#
\end{minted}

In the \variant{GuardByClassLiteral} variant, a cast uses an application-specific guard, but the guard depends on a class literal.
In the following example,
a cast is performed to the \code{field} variable (line 22),
based whether the runtime class of the variable is actually \code{Short.class}.

%https://lgtm.com/projects/g/elastic/elasticsearch/snapshot/dist-1916470085-1524814812150/files/server/src/main/java/org/elasticsearch/common/lucene/Lucene.java?sort=name&dir=ASC&mode=heatmap#L478
\def\urlvar{http://bit.ly/elastic_elasticsearch_2SSgsFV}
\begin{minted}[highlightlines=22]{java}
Class type = field.getClass();
if (type == String.class) {
    out.writeByte((byte) 1);
    out.writeString((String) field);
} else if (type == Integer.class) {
    out.writeByte((byte) 2);
    out.writeInt((Integer) field);
} else if (type == Long.class) {
    out.writeByte((byte) 3);
    out.writeLong((Long) field);
} else if (type == Float.class) {
    out.writeByte((byte) 4);
    out.writeFloat((Float) field);
} else if (type == Double.class) {
    out.writeByte((byte) 5);
    out.writeDouble((Double) field);
} else if (type == Byte.class) {
    out.writeByte((byte) 6);
    out.writeByte((Byte) field);
} else if (type == Short.class) {
    out.writeByte((byte) 7);
    out.writeShort((Short) field);
} else if (type == Boolean.class) {
    out.writeByte((byte) 8);
    out.writeBoolean((Boolean) field);
} else if (type == BytesRef.class) {
    out.writeByte((byte) 9);
    out.writeBytesRef((BytesRef) field);
} else {
    throw new IOException("Can't handle sort field value of type ["+type+"]");
} #\urlbox#
\end{minted}

Similar to the previous example, the next snippet contains several type cases.
Each type case is guarded by an \code{equals} comparison between a class literal and the \code{clazz} parameter.
The cast is applied to the type parameter \code{T} only if the guard succeeds.

%https://lgtm.com/projects/g/OPCFoundation/UA-Java-Legacy/snapshot/dist-1804064538-1524814812150/files/src/main/java/org/opcfoundation/ua/encoding/binary/BinaryDecoder.java?sort=name&dir=ASC&mode=heatmap#L214
\def\urlvar{http://bit.ly/OPCFoundation_UA_Java_Legacy_2Fb2xmZ}
\begin{minted}[highlightlines=9]{java}
@Override
@SuppressWarnings("unchecked")
public <T> T get(String fieldName, Class<T> clazz) throws DecodingException {
    if (clazz.equals(Boolean.class)) {
        return (T) getBoolean(fieldName);
    }
    // [...]
    if (clazz.equals(ExtensionObject.class)) {
        return (T) getExtensionObject(fieldName);
    }
    // [...]
} #\urlbox#
\end{minted}

In the following listing,
a cast is applied to the result of the \code{getObject} method (line 2).
The target type of the cast, \code{MyKey}, corresponds to the class literal argument, \code{MyKey.class}.
Essentially, \code{getObject} is using the \code{isInstance} method%
\footnote{\url{https://docs.oracle.com/javase/8/docs/api/java/lang/Class.html\#isInstance-java.lang.Object-}}
of the class \code{java.lang.Class} to check whether an object is from a certain type.

%https://lgtm.com/projects/g/smartdevicelink/sdl_android/snapshot/dist-2037360569-1524814812150/files/sdl_android/src/main/java/com/smartdevicelink/proxy/rpc/GetVehicleDataResponse.java?sort=name&dir=ASC&mode=heatmap#L236
\def\urlvar{http://bit.ly/smartdevicelink_sdl_android_2EjJiaq}
\begin{minted}[highlightlines=2]{java}
public MyKey getMyKey() {
    return (MyKey) getObject(MyKey.class, KEY_MY_KEY);
} #\urlbox#
\end{minted}

The following snippet shows an instance of the \variant{GuardByClassLiteral} variant. 
In this case, the cast is guaranteed to succeed because the class literal used as argument to the recursive call (\code{Integer.class}) determines that the method returns an \code{int} value.
%https://lgtm.com/projects/g/apache/karaf/snapshot/dist-13960098-1524814812150/files/shell/core/src/main/java/org/apache/karaf/shell/support/converter/DefaultConverter.java?sort=name&dir=ASC&mode=heatmap#L117
\def\urlvar{http://bit.ly/apache_karaf_2HE55gE}
\begin{minted}[highlightlines=4]{java}
public Object convertToNumber(Number value, Class toType) throws Exception {
    toType = unwrap(toType);
    if (AtomicInteger.class == toType) {
        return new AtomicInteger((Integer)convertToNumber(value,Integer.class));
    } else if (AtomicLong.class == toType) {
        return new AtomicLong((Long) convertToNumber(value, Long.class));
    } else if (Integer.class == toType) {
        return value.intValue();
    } else if (Short.class == toType) {
        return value.shortValue();
    } else if (Long.class == toType) {
        return value.longValue();
    } else if (Float.class == toType) {
        return value.floatValue();
    } else if (Double.class == toType) {
        return value.doubleValue();
    } else if (Byte.class == toType) {
        return value.byteValue();
    } else if (BigInteger.class == toType) {
        return new BigInteger(value.toString());
    } else if (BigDecimal.class == toType) {
        return new BigDecimal(value.toString());
    } else {
        throw new Exception("Unable to convert number "+value+" to "+toType);
    }
} #\urlbox#
\end{minted}


\detection{}
When implementing the pattern,
care must be taken with complex operands that the value of the operand is
not changed between the guard and the cast, possibly even by another thread.
For instance, in some situations the operand expression is a method invocation.
The value returned by the method should be the same for both the
\code{instanceof} and the cast, thus the method should be a pure method.
Typically, this problem is avoided by using an effectively final local variable in both the guard and the cast operand.

The Query~\ref{lst:ql:GuardByInstanceOfCast} detects the \variant{GuardByInstanceOf} variant.
It is decoupled in two \ql{} classes.
The \code{ControlByInstanceOfCast} class checks that the cast---\emph{to a variable}---is control-dependant on an \code{instanceof} on the same variable.
Then, the \code{GuardByInstanceOfCast} class checks that the value tested by the \code{instanceof} is the same to be cast.
That is, it checks that there is no assignment to the variable between the \code{instanceof} and the cast.

\begin{listing}
\begin{minted}{\qllexer}
class ControlByInstanceOfCast extends #\qlref{VarCast}# { #\qlbox#
	InstanceOfExpr iof;
	private ConditionBlock cb;
	ControlByInstanceOfCast() {
		iof = cb.getCondition() and
		cb.controls(getBasicBlock(), true) and
		var.getAnAccess() = iof.getExpr()
	}
	InstanceOfExpr getIof() { result = iof }
}

class GuardByInstanceOfCast extends ControlByInstanceOfCast {
	GuardByInstanceOfCast() {
		forall (VariableUpdate def | defUsePair(def, getExpr()) |
		  defUsePair(def, iof.getExpr())
		)
	}
}
\end{minted}
\caption{Query for the \variant{GuardByInstanceOf} variant.}
\label{lst:ql:GuardByInstanceOfCast}
\end{listing}

The implementation of \variant{GuardByTypeTag} variant is application-specific,
and thus its automatic detection in \ql{} is impractical.
Nevertheless, the Query~\ref{lst:casts:detect:switchtypetag} detect the special case when a cast is applied to a field in an object inside a \code{switch} statement.
The expression to be \code{switch}ed is another field in the same object.

\begin{listing}
\begin{minted}{\qllexer}
class SwitchFieldTypeTagCast extends #\qlref{Cast}# { #\qlbox#
	FieldAccess tagAccess;
	FieldAccess castAccess;
	Variable v;
	SwitchFieldTypeTagCast() {
		tagAccess = this.(SwitchedExpr).getSwitchStmt().getExpr() and
		castAccess = getExprOrDef() and
		v.getAnAccess() = tagAccess.getQualifier() and
		v.getAnAccess() = castAccess.getQualifier()
	}
}
\end{minted}
\caption{Detection of a cast inside a \code{switch} statement}
\label{lst:casts:detect:switchtypetag}
\end{listing}

Similar to the previous case,
the Query~\ref{lst:casts:detect:isarraytypetag} detects when a cast is guarded by a call to the \code{Class.isArray} method.
This query detects \emph{only} the case when the variable to be cast and
the \code{getClass} invocation are in the same method.

\begin{listing}
\begin{minted}{\qllexer}
class ControlByIsArrayCast extends #\qlref{VarCast}# { #\qlbox#
	ConditionBlock cb;
	MethodAccess iama;
	ControlByIsArrayCast() {
		exists (VariableAssign def, GetClassMethodAccess gcls |
			gcls.getQualifier() = var.getAnAccess() and
			def.getSource() = gcls and
			defUsePair(def, iama.getQualifier().(VarAccess) )
		) and
		iama.getMethod() instanceof IsArrayClassMethod and
		(
			(cb.getCondition()=iama and cb.controls(getBasicBlock(), true)) or
			(cb.getCondition().(LogNotExpr).getExpr() = iama and
				cb.controls(getBasicBlock(), false)
			)
		)
	}
}
\end{minted}
\caption{Detection of a cast guarded by the \code{Class.isArray} method.}
\label{lst:casts:detect:isarraytypetag}
\end{listing}

The following query detects the \variant{GuardByClassLiteral} variant.
Similar to the previous case,
this query \emph{does not} detect the case when the variable to be cast and the \code{Class} object are passed as parameters.
To detect such case would require an inter-procedural (global) data flow analysis.
Such analysis does not scale easily.

\begin{listing}
\begin{minted}{\qllexer}
class GuardByClassLiteral extends #\qlref{VarCast}# { #\qlbox#
	TypeLiteral tl;
	GetClassMethodAccess gcma;
	GuardByClassLiteral() {
		gcma.getQualifier() = getVar().getAnAccess() and
		#\qlref{isSubtype}#(tl.getTypeName().getType(), getTargetType()) and (
			#\qlref{controlByEqualityTest}#(tl, gcma, this) or
			#\qlref{controlByEqualsMethod}#(tl, gcma, this)
		)
	}
}
\end{minted}
\caption{Query for the \variant{GuardByClassLiteral} variant.}
\end{listing}


\issues{}
Having only a single case---that is, a single guard and cast---is common.
In the 742 instances of \thisp{} that used \code{instanceof}, 511
(69\%) had only one case.

The \thisp{} pattern can be seen as an \adhoc{} alternative to a
\code{typecase} or pattern matching~\citep{milnerProposalStandardML1984} as a
language construct.
In \kotlin{}, flow-sensitive typing is used so that immutable values can be
used at a subtype when a type guard on the value is successful.%
\footnote{\url{https://kotlinlang.org/docs/reference/typecasts.html\#smart-casts}}
This feature eliminates much of the need for the guarded casts.
Pattern matching can be seen in several other languages, \eg, \ml{}, \scala{}, \csharp{}, and \haskell{}.
For instance, in \scala{} the pattern matching construct is achieved using the \code{match} keyword.
In this example,%
\footnote{Adapted from \url{https://docs.scala-lang.org/tour/pattern-matching.html}}
a different action is taken according to the runtime type of the parameter \code{notification} (line 10).

\begin{minted}[highlightlines=10]{scala}
abstract class Notification #\scalabox#
case class Email(sender: String, title: String, body: String)
	extends Notification
case class SMS(caller: String, message: String)
	extends Notification
case class VoiceRecording(contactName: String, link: String)
	extends Notification

def showNotification(notification: Notification): String = {
	notification match {
		case Email(email, title, _) => s"Email from $email titled: $title"
		case SMS(number, message) => s"SMS from $number! Message: $message"
		case VoiceRecording(name, link) => s"Recording from $name! Link: $link"
	}
}
\end{minted}

Alternatives to the \thisp{} pattern would be to use the visitor pattern or to use virtual dispatch on the match scrutinee.
However, both of these alternatives might be difficult to implement when the scrutinee is defined in a library or in third-party code.
There is an ongoing proposal%
\footnote{\url{https://openjdk.java.net/jeps/305}}$^{,}$%
\footnote{\url{https://cr.openjdk.java.net/~briangoetz/amber/pattern-match.html}}%
~\citep{jep305} to add pattern matching to the \java{} language.
The proposal explores changing the \code{instanceof} operator in order to support pattern matching.
\java{} 12 extends the \code{switch} statement to be used as either a statement or an expression%
\footnote{\url{https://openjdk.java.net/jeps/325}}$^{,}$%
\footnote{\url{https://openjdk.java.net/jeps/354}}~\citep{jep325,jep354}.
This enhancement aims to ease the transition to a \code{switch} expression that supports pattern matching.

The \variant{GuardByClassLiteral} variant may be used instead of the \code{instanceof} operator when the developer wants to match exactly the runtime class of an object.
The \code{instanceof} operator%
\footnote{\url{https://docs.oracle.com/javase/specs/jls/se8/html/jls-15.html\#jls-15.20.2}}
returns \code{true} if the expression could be cast to the specified type,
whereas using a class literal comparison returns \code{true} if the expression is exactly the runtime class.

In some cases, the \variant{GuardByTypeTag} variant can be replaced by \variant{GuardByInstanceOf}.
However, if the application-specific tag is a numeric value,
the \variant{GuardByTypeTag} could perform better than the \variant{GuardByInstanceOf} using \code{instanceof}.
Moreover, there are situation where the \code{instanceof} operator cannot be avoid since the types to be cast are the same.

\end{pattern}
