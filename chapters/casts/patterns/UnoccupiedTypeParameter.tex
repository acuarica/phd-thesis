\begin{pattern}{UnoccupiedTypeParameter}
This pattern occurs when a generic type changes its type parameter,
but the new type parameter hold no values.

\instances{}
This instance%
\def\urlvar{http://bit.ly/vavr_io_vavr_2SMIfI2}
is used to implement an \code{Either} type.
A value of type \code{Either<L, R>} can be either a value of type \code{L} or of type \code{R}.
In this instance, the receiver---of type \code{Either<L, R>}---is cast to \code{Either<U, R>} (line 9).
There is no subtype relation between \code{L} and \code{U}.
However, the cast succeeds because the programmer ensures
(using the guard \code{isLeft} in line 6)
that no value of type \code{U} is accessible from \code{this}.
Note that this cast does not conform to the \nameref{pat:Typecase}
pattern, despite the guard, because the target type is not a subtype of the
cast operand.
The cast is successful only because of \java{}'s type erasure
implementation.

%https://lgtm.com/projects/g/vavr-io/vavr/snapshot/dist-11000099-1524814812150/files/vavr/src/main/java/io/vavr/control/Either.java?sort=name&dir=ASC&mode=heatmap#L398
\begin{minted}[highlightlines=9]{java}
public interface Either<L, R> extends Value<R>, Serializable {
    // [...]
    @SuppressWarnings("unchecked")
    default <U> Either<U, R> mapLeft(Function<? super L, ? extends U> leftMapper) {
        Objects.requireNonNull(leftMapper, "leftMapper is null");
        if (isLeft()) {
            return Either.left(leftMapper.apply(getLeft()));
        } else {
            return (Either<U, R>) this;
        }
    }
    // [...]
} #\urlbox#
\end{minted}


\issues{}
This pattern is related to the use of \emph{phantom types} in parametrically polymorphic languages~\cite{LeijenMeijer99,cheneyHinzePhantomTypes}.
Phantom types are type parameters used solely for type checking and are not occupied by any value.

This pattern also occurs with empty collections.
For instance, the \java{} standard library implementation of the method \code{Collections.<T>emptyList()} casts a private constant with raw type \code{List} to a \code{List<T>}.
This is safe because the the list is empty and has no elements of type \code{T}.

\scala{} has an unoccupied \code{Nothing} type to handle this situation.
For instance, an (immutable) empty list has \code{List[Nothing]},
which is a subtype of \code{List[T]} for any type \code{T}.

\end{pattern}