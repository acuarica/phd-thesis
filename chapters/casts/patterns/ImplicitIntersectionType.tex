\begin{pattern}{ImplicitIntersectionType}
This pattern occurs when there is a downcast of reference $v$ of type $T$ to a target interface type $I$.
Although $T$ does not implement $I$, 
the cast succeeds because all possible run-time types of $v$ do implement $I$.

\instances{}
For instance, in the following example the method call returns a \code{Number},
which does not implement \code{Comparable}; however,
all values that could be returned by the method are subclasses of \code{Number}
in \code{java.lang} that do implement \code{Comparable}.

%https://lgtm.com/projects/g/senbox-org/snap-desktop/snapshot/dist-5990318-1524814812150/files/snap-ui/src/main/java/org/esa/snap/ui/UIUtils.java?sort=name&dir=ASC&mode=heatmap#L379
\def\urlvar{http://bit.ly/senbox_org_snap_desktop_2FQOt4v}
\begin{minted}[highlightlines=1,linenos=false]{java}
final Comparable max = (Comparable) properties.getMaxValue();
#\urlbox#
\end{minted}

This pattern can be used to implement a dynamic proxy.
In the following example,
\code{pyObjectValue} is a proxy to \code{PyObjectValue}.
Nevertheless, a cast to \code{Proxy} is needed to invoke the \code{setHandler}.

%https://lgtm.com/projects/g/CloudSlang/cloud-slang/snapshot/dist-13290023-1524814812150/files/cloudslang-entities/src/main/java/io/cloudslang/lang/entities/bindings/values/PyObjectValueProxyFactory.java?sort=name&dir=ASC&mode=heatmap#L51
\def\urlvar{http://bit.ly/CloudSlang_cloud_slang_2EkgP4l}
\begin{minted}[highlightlines=4]{java}
PyObjectValueProxyClass proxyClass = getProxyClass(pyObject);
PyObjectValue pyObjectValue = (PyObjectValue) proxyClass.getConstructor()
            .newInstance(proxyClass.getParams());
((Proxy) pyObjectValue).setHandler(
            new PyObjectValueMethodHandler(content, sensitive, pyObject));
#\urlbox#
\end{minted}


\detection{}
The Query~\ref{lst:ql:ImplicitIntersectionTypeCast} detects this pattern.
Usually, in a downcast \code{(T) e}, the class or interface \code{T} is a subtype of {e}'s class or interface.
This query essentially detects whether \code{T} has no subtyping relation with the type of \code{e}.

\begin{listing}
\begin{minted}{java}
class ImplicitIntersectionTypeCast extends #\qlref{Cast}# { #\qlbox#
	ImplicitIntersectionTypeCast() {
		getTargetType() instanceof Interface and
		not isSubtype(getTargetType(), getExpr().getType()) and
		not this instanceof Upcast and
		not getExpr() instanceof NullLiteral and
		#\qlref{notGenericRelated}#(getTargetType()) and
		#\qlref{notGenericRelated}#(getExpr().getType())
	}
}
\end{minted}
\caption{Detection of the \thisp{} pattern.}
\label{lst:ql:ImplicitIntersectionTypeCast}
\end{listing}


\issues{}
The cast could avoided by having the operand type implement the target type
interface or by introducing a more precise interface.
In the first example,
one could imagine an interface \code{ComparableNumber} that extends both
\code{Number} and \code{Comparable}.
\scala{} supports interface types,
allowing the type \code{Number with Comparable} to be used directly.

\end{pattern}