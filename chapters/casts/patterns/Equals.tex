\begin{pattern}{Equals}
This pattern is a common pattern to implement the well-known \code{equals} method (declared in \code{java.lang.Object}).
It is a particularly instance of guarded casts.
A cast expression is guarded by either an \code{instanceof} test or a \code{getClass} comparison (usually to the same target type as the cast);
in an \code{equals}%
\footnote{\url{https://docs.oracle.com/javase/8/docs/api/java/lang/Object.html\#equals-java.lang.Object-}} method implementation.
This is done to check if the argument has same type as the receiver
(\code{this} argument).
Notice that a cast in an \code{equals} method is needed because it
receives an \code{Object} as a parameter.

To detect this pattern,
a cast must be applied to the parameter of the \code{equals} method.
The result value of the cast must be then used in an equality comparison.
We relax the constraint that the target type of the cast must the enclosing class.

\instances{}
This pattern accounts for \nEqualsOutofGuarded\% of guarded casts,
\nEqualsPattern{} instances out of \nGuarded{}.
The following listing shows an example of the \thisp{} pattern.
In this case,
\code{instanceof} is used to guard for the same type as the receiver.

%https://lgtm.com/projects/g/neo4j/neo4j/snapshot/dist-15760049-1519892555006/files/community/kernel/src/main/java/org/neo4j/kernel/impl/api/CountsRecordState.java?sort=name&dir=ASC&mode=heatmap&excluded=false#L182
\def\urlvar{http://bit.ly/neo4j_neo4j_2vJw94J}
\begin{minted}[highlightlines=7]{java}
@Override
public boolean equals(Object obj) {
    if ( this == obj ) {
        return true;
    }
    if ( (obj instanceof Difference) ) {
        Difference that = (Difference) obj;
        return actualFirst == that.actualFirst
          && expectedFirst == that.expectedFirst
          && actualSecond == that.actualSecond 
          && expectedSecond == that.expectedSecond
          && key.equals( that.key );
    }
    return false;
} #\urlbox#
\end{minted}

Alternatively, the following listing shows another example of the \thisp{} pattern.
But in this case,
a \code{getClass} comparison is used to guard for the same type as the receiver (line 4).

%https://lgtm.com/projects/g/neo4j/neo4j/snapshot/dist-15760049-1519892555006/files/community/bolt/src/main/java/org/neo4j/bolt/v1/messaging/infrastructure/ValuePath.java?sort=name&dir=ASC&mode=heatmap&excluded=false#L278
\def\urlvar{http://bit.ly/neo4j_neo4j_2vKP0MW}
\begin{minted}[highlightlines=7]{java}
@Override
public boolean equals( Object o ) {
    if ( this == o ) return true;
    if ( o == null || getClass() != o.getClass() )
        return false;

    ValuePath that = (ValuePath) o;
    return nodes.equals(that.nodes) &&
        relationships.equals(that.relationships);
} #\urlbox#
\end{minted}

In some situations, the type cast is not the same as the enclosing class.
Instead, the type cast is the super class of the enclosing class.
The following example shows this scenario.
This happens, for example, when the Google AutoValue library%
\footnote{\url{https://github.com/google/auto/tree/master/value}}
is used.
AutoValue is a code generator for value classes.

%https://lgtm.com/projects/g/square/sqlbrite/snapshot/3a9916985485ba5922097fe59a18230500f02df4/files/sample/build/generated/source/apt/debug/com/example/sqlbrite/todo/ui/$AutoValue_ListsItem.java?sort=name&dir=ASC&mode=heatmap&showExcluded=false#L52
\def\urlvar{http://bit.ly/square_sqlbrite_2HmHMYE}
\begin{minted}[highlightlines=13]{java}
@AutoValue
abstract class ListsItem implements Parcelable {
    // [...]
}

abstract class $AutoValue_ListsItem extends ListsItem {
    @Override
    public boolean equals(Object o) {
      if (o == this) {
        return true;
      }
      if (o instanceof ListsItem) {
        ListsItem that = (ListsItem) o;
        return (this.id == that.id())
             && (this.name.equals(that.name()))
             && (this.itemCount == that.itemCount());
      }
      return false;
    }
} #\urlbox#
\end{minted}

The following snippet shows a non-trivial implementation of \code{equals}.
The enclosing class of the \code{equals} method is \code{CapReq} (line 1).
However, the cast instance (line 13) is not against the enclosing class,
it is against to the \code{Requirement} class.
Note that the cast using the enclosing class as target type is in line 9.

%https://lgtm.com/projects/g/bndtools/bnd/snapshot/dist-930051-1524814812150/files/biz.aQute.bndlib/src/aQute/bnd/osgi/resource/CapReq.java?sort=name&dir=ASC&mode=heatmap#L73
\def\urlvar{http://bit.ly/bndtools_bnd_2SM5pOw}
\begin{minted}[highlightlines=13]{java}
class CapReq {
    @Override
    public boolean equals(Object obj) {
        if (this == obj)
                return true;
        if (obj == null)
                return false;
        if (obj instanceof CapReq)
                return equalsNative((CapReq) obj);
        if ((mode == MODE.Capability) && (obj instanceof Capability))
                return equalsCap((Capability) obj);
        if ((mode == MODE.Requirement) && (obj instanceof Requirement))
                return equalsReq((Requirement) obj);
        return false;
    }
} #\urlbox#
\end{minted}


\issues{}
The pattern for an \code{equals} method implementation is well-known.
Most \code{equals} methods in our sample are implemented with the same boilerplate structure:
that is, first checking if the parameter is another reference to \code{this},
then checking if the argument is not null,
and finally, checking if the argument is of the right class
(with either an \code{instanceof} test or a \code{getClass} comparison).
Once all checks are performed, a cast follows, and a field-by-field comparison is made.

To avoid this boilerplate, other languages bake in deep equality comparisons,
at least for some types (\eg, \scala{} case classes),
or provide mechanisms to generate the boilerplate code (\eg, \code{deriving Eq}
in \haskell{} or \code{\#[derive(Eq)]} in \rust{}).
\cite{vaziriDeclarativeObjectIdentity2007} propose a declarative approach to avoid boilerplate code when implementing
both the \code{equals} and \code{hashCode} methods.
They manually analyzed several applications, and found there are many issues while implementing \code{equals()} and \code{hashCode()} methods.
It would be interesting to check whether these issues happen in real application code.

There is an exploratory document%
\footnote{\url{http://cr.openjdk.java.net/\~briangoetz/amber/datum.html}}
by Brian Goetz---\java{} Language Architect---addressing these issues from a more general perspective.
It is definitely a starting point towards improving the \java{} language.

This pattern can be seen as a special instance of the \nameref{pat:Typecase} pattern when the guard is an \code{instanceof} test or a \code{getClass} comparison.

\end{pattern}