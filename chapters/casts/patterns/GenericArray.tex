\begin{pattern}{GenericArray}
A cast due to the instantiation of an array with a parameterized base type.
In \java{} these arrays cannot be instantiated,
instead an \code{Object[]} or an array of raw types must be created.
The cast is necessary to use the array at the intended type. 

\instances{}
In the following snippet,
a cast is required when accessing an element in the array (line 4).
The array is created using the raw type \code{List[][]}
and assigned to a variable of using the wildcard type \code{List<?>[][]} (line 1).
It is not possible to simply allocate a \code{List<byte[]>}.

%https://lgtm.com/projects/g/ppiastucki/recast4j/snapshot/dist-4860452-1524814812150/files/detourtilecache/src/main/java/org/recast4j/detour/tilecache/AbstractTileLayersBuilder.java?sort=name&dir=ASC&mode=heatmap#L50
\def\urlvar{http://bit.ly/ppiastucki_recast4j_2EM7zWK}
\begin{minted}[highlightlines=4]{java}
List<?>[][] partialResults = new List[th][tw];
for (...) {
    partialResults[ty][tx] = build(tx, ty, order, cCompatibility);
    layers.addAll((List<byte[]>) partialResults[y][x]);
} #\urlbox#
\end{minted}
% protected abstract List<byte[]> build(int tx, int ty, ByteOrder order, boolean cCompatibility);

Instead of casting individual elements,
the following example shows a cast applied directly when the array is created. 

%https://lgtm.com/projects/g/seven332/Nimingban/snapshot/dist-1506180917302-1524814812150/files/yorozuya/src/main/java/com/hippo/yorozuya/ArrayUtils.java?sort=name&dir=ASC&mode=heatmap#L124
\def\urlvar{http://bit.ly/seven332_Nimingban_2UdBwIL}
\begin{minted}[highlightlines=1,linenos=false]{java}
T[] newArray = (T[]) new Object[growSize(currentSize)];
#\urlbox#
\end{minted}


\detection{}
The following queries detect different variations of the \thisp{} pattern.
The first one detects when a generic cast is applied to the array,
\eg{}, \code{(E[]) new Object[length]}.

\begin{listing}
\begin{minted}{\qllexer}
class OnArrayGenericArrayCast extends #\qlref{Cast}# { #\qlbox#
	OnArrayGenericArrayCast() {
		getTargetType().(Array).getComponentType() instanceof TypeVariable and
		getExpr().getType() instanceof Array
	}
}
\end{minted}
\end{listing}

The following query detects the case when the target type of the cast is a type variable used to get an element from the array.
For instance, \code{(T) events[i]},
where the \code{events} array is defined as \code{EventObject[]} and \code{T} is declared as \code{T extends EventObject}.

\begin{listing}
\begin{minted}{\qllexer}
class TypeVariableGenericArrayCast extends #\qlref{Cast}# { #\qlbox#
	TypeVariableGenericArrayCast() {
		getExprOrDef() instanceof ArrayAccess and
		getTargetType() instanceof TypeVariable
	}
}
\end{minted}
\end{listing}

Our last query for this pattern is similar to the previous one.
But in this case, the component type of the array is either a raw type or a wildcard type (\eg{}, \code{List<?>}).
For instance, a cast \code{(Any<T>) entries[i]} where \code{entries} is defined as \code{Any[] entries = new Any[n]}.
In this example, \code{Any} is a generic type, but the array is using the raw type instead.

\begin{listing}
\begin{minted}{\qllexer}
class OnElementGenericArrayCast extends #\qlref{Cast}# { #\qlbox#
	OnElementGenericArrayCast() {
		(
        getExprOrDef().(ArrayAccess).getArray().getType().(Array)
                .getComponentType() instanceof RawType or
        containsWildcard( getExprOrDef().(ArrayAccess).getArray()
                .getType().(Array).getComponentType() )
        ) and
            getTargetType() instanceof ParameterizedType
	}
}
\end{minted}
\end{listing}


\issues{}
This pattern occurs because generic type parameters are not reified at run time,
but array types are reified.
To create a generic \code{T[]}, for instance,
since the parameter \code{T} is not known statically,
the compiler cannot know the run-time representation of the array.
The \java{} specification just forbids these problematic cases and therefore requires programmers to create arrays of raw types and to use casts.

\end{pattern}
