\begin{pattern}{Redundant}
A redundant cast is a cast that is not necessary for compilation.
The cast could be removed from source code without affecting the application.

To detect the \thisp{} pattern, 
the expression being cast needs to be of the same type as the type being cast to.

\instances{}
The following listing exhibits an instance of the \thisp{} pattern.
A redundant cast is applied to a lambda expression (line 8).
This cast is not needed a \java{} compiler can infer that the lambda expression is of type \code{TransactionCallback<Void>} (defined in line 22).

%https://lgtm.com/projects/g/vladmihalcea/high-performance-java-persistence/snapshot/dist-1813180502-1524814812150/files/core/src/test/java/com/vladmihalcea/book/hpjp/hibernate/schema/flyway/FlywayTest.java#L40
\def\urlvar{http://bit.ly/vladmihalcea_high_performance_java_persistence_2FWXw2e}
\begin{minted}[highlightlines=8]{java}
public class FlywayTest {
    private TransactionTemplate transactionTemplate;
    @Test
    public void test() {
        // [...]
        transactionTemplate.execute(
            (TransactionCallback<Void>) transactionStatus -> {
                Post post = new Post();
                entityManager.persist(post);
                return null;
        });
    }
}
public interface TransactionStatus { /* [...] */ }
@FunctionalInterface
public interface TransactionCallback<T> {	
	T doInTransaction(TransactionStatus status);
}
public class TransactionTemplate {
	<T> T execute(TransactionCallback<T> action) { /* [...] */ }
} #\urlbox#
\end{minted}

The next cast instance is trivially redundant:
both the target type and the static type of the operand \code{count(b)},
are \code{BigDecimal}.

%https://lgtm.com/projects/g/sigmoidanalytics/spork/snapshot/dist-1794190395-1524814812150/files/src/org/apache/pig/builtin/BigDecimalAvg.java?sort=name&dir=ASC&mode=heatmap#L258
\def\urlvar{http://bit.ly/sigmoidanalytics_spork_2SIqWYq}
\begin{minted}[highlightlines=4]{java}
@Override
public void accumulate(Tuple b) throws IOException {
    // [...]
    BigDecimal count = (BigDecimal)count(b);
    // [...]
}
static protected BigDecimal count(Tuple input) throws ExecException {
    // [...]
} #\urlbox#
\end{minted}

In the following cast instance,
a cast is applied to the \code{node.right} field (line 12).
Nevertheless, the \code{right} field of the \code{Node} class is already defined as \code{Node<T>},
rendering the cast redundant.

%https://lgtm.com/projects/g/phishman3579/java-algorithms-implementation/snapshot/dist-1797111073-1524814812150/files/src/com/jwetherell/algorithms/data_structures/ImplicitKeyTreap.java?sort=name&dir=ASC&mode=heatmap#L266
\def\urlvar{http://bit.ly/phishman3579_java_algorithms_implementation_2SGcH6w}
\begin{minted}[highlightlines=12]{java}
public class ImplicitKeyTreap<T> implements IList<T> {
    protected Node<T> root = null; 
    // [...] 
    private int getIndexByValue(T value) {
        final Node<T> node = (Node<T>)root;
        if (value == null || node == null)
            return Integer.MIN_VALUE;
        final Node<T> l = (Node<T>)node.left;
        final Node<T> r = (Node<T>)node.right;
        // [...]
        return i;
    }
    public static class Node<T> {
        private T value = null;
        private int priority;
        private int size;
        private Node<T> parent = null;
        private Node<T> left = null;
        private Node<T> right = null;
        // [...]
    }
} #\urlbox#
\end{minted}

There are cases when code generators insert superfluous casts to \code{null}.
The following cast instance could be removed since in this case the cast to \code{null} is not needed.

%https://lgtm.com/projects/g/togglz/togglz/snapshot/dist-44910020-1524814812150/files/slack/target/generated-sources/groovy-stubs/test/org/togglz/slack/sender/AsyncNotifierSpecIT.java?sort=name&dir=ASC&mode=heatmap#L17
\def\urlvar{http://bit.ly/togglz_togglz_2SGncXB}
\begin{minted}[highlightlines=2]{java}
public groovy.lang.MetaClass getMetaClass() {
    return (groovy.lang.MetaClass) null;
} #\urlbox#
\end{minted}


\issues{}
Redundant casts are generally upcasts or casts involving erased type parameters.
This pattern arises often in generated code.
It may also appear due to code refactoring that change a type and therefore make the cast redundant.
    
\end{pattern}
