\begin{pattern}{SelectOverload}
This pattern is used to select the appropriate version of an overloaded method%
\footnote{Using ad-hoc polymorphism~\citep{stracheyFundamentalConceptsProgramming2000}.}
where two or more of its implementations differ \emph{only} in some argument type.

A cast of the \code{null} literal is often used to resolve method overloading ambiguity because the type of \code{null} is a subtype of any reference type.%
\footnote{\url{https://docs.oracle.com/javase/specs/jls/se8/html/jls-4.html\#jls-4.1}}

A cast to \code{null} is often used to select against different versions of a method,
\ie{}, to resolve method overloading ambiguity.
Whenever a \code{null} value needs to be an argument of an a cast is
needed to select the appropriate implementation.
This is because the type of \code{null} has the special type \emph{null}%
\footnote{\url{https://docs.oracle.com/javase/specs/jls/se8/html/jls-4.html\#jls-4.1}}
which can be treated as any reference type.
In this case,
the compiler cannot determine which method implementation to select.

Another use case is to select the appropriate the right argument when calling a method with variable arguments.

\instances{}
The following listing shows an example of the \thisp{} pattern.
In this example, there are three versions of the \code{onSuccess} method,
The cast \code{(String) null} is used to select the appropriate version
(line 7), based on the third parameter.
Overloaded methods that differ only in their argument type (the third one).

% https://lgtm.com/projects/g/loopj/android-async-http/snapshot/dist-1879340034-1518514025554/files/library/src/main/java/com/loopj/android/http/JsonHttpResponseHandler.java?sort=name\&dir=ASC\&mode=heatmap\&excluded=false#L150
\def\urlvar{http://bit.ly/loopj_android_async_http_2FENovD}
\begin{minted}[highlightlines=1]{java}
onSuccess(statusCode, headers, (String) null);
public void onSuccess(
      int statusCode, Header[] headers, JSONObject response) { /* [...] */ }
public void onSuccess(
      int statusCode, Header[] headers, JSONArray response) { /* [...] */ }
public void onSuccess(
      int statusCode, Header[] headers, String responseString) { /* [...] */ }
#\urlbox#
\end{minted}

In the following example \code{actual.data()} returns a boxed \code{Long}.
Because implicit upcasts have precedence over implicit unboxing conversions,
the call is needed to invoke the method that takes a \code{long} (line 3) rather than the method that takes an \code{Object} (line 2).

%https://lgtm.com/projects/g/spullara/redis-protocol/snapshot/dist-41940059-1524814812150/files/client/src/test/java/redis/client/AllCommandsTest.java?sort=name&dir=ASC&mode=heatmap#L366
\def\urlvar{http://bit.ly/spullara_redis_protocol_2FC9Llb}
\begin{minted}[highlightlines=1]{java}
assertEquals(expected, (long) actual.data());
public static void assertEquals(Object expected, Object actual) { /* [...] */ }
public static void assertEquals(long expected, long actual) { /* [...] */ }
#\urlbox#
\end{minted}

The following snippet is similar to the previous example,
but notice how that the cast is applied to a
primitive---\emph{non-reference}---type.

%https://lgtm.com/projects/g/apache/poi/snapshot/dist-1790760597-1524814812150/files/src/testcases/org/apache/poi/hssf/record/chart/TestLegendRecord.java#L50
\def\urlvar{http://bit.ly/apache_poi_2StrlOn}
\begin{minted}[highlightlines=1,linenos=false]{java}
assertEquals((byte) 0x1, record.getSpacing());
#\urlbox#
\end{minted}

In the last example of \thisp,
an upcast of a generic type is performed to select the appropriate overload of the \code{max} method.

%https://lgtm.com/projects/g/groovy/groovy-core/snapshot/dist-45390050-1524814812150/files/src/main/org/codehaus/groovy/runtime/DefaultGroovyMethods.java?sort=name&dir=ASC&mode=heatmap#L6715
\def\urlvar{http://bit.ly/groovy_groovy_core_2HDAkbF}
\begin{minted}[highlightlines=2]{java}
public static <T> T max(Iterator<T> self, Comparator<T> comparator) {
      return max((Iterable<T>)toList(self), comparator);
}
public static <T> List<T> toList(Iterator<T> self) {
      // [...]
}
@Deprecated
public static <T> T max(Collection<T> self, Comparator<T> comparator) {
      // [...]
}
public static <T> T max(Iterable<T> self, Comparator<T> comparator) {
      // [...]
} #\urlbox#
\end{minted}

\issues{}
Casting the \code{null} constant seems rather artificial.
This pattern shows either a lack of expressiveness in \java{} or a bad \api{} design.
Passing \code{null} to a method might better be handled by using overloading with fewer parameters or by using default parameters.
Several other languages support default parameters,
\eg, \scala{}, \csharp{} and \cpp{}.
Adding default parameters might be a partial solution.

In addition, a pure object-oriented language would not distinguish between primitives and objects,
avoiding the need for autoboxing to be visible at the type level.

% TODO: Relate null as theorical point of view in the TAPL book.

\end{pattern}