\begin{pattern}{UseRawType}
A cast is in the \thisp{} pattern when a \emph{raw type} is used rather than a generic type.
Methods of raw types typically return \code{Object} rather than a more specific type.

\instances{}
For example, in the following code,%
\def\urlvar{http://bit.ly/bcgit_bc_java_2SD2HLm}
the collection \code{c} and iterator \c{it} are declared to be of the raw types \code{Collection} and \code{Iterator} rather than as parameterized types.
The call to \c{next} on line 4 must be cast to a more specific type because static type information was lost by the use of raw types.

%https://lgtm.com/projects/g/bcgit/bc-java/snapshot/dist-20740003-1524814812150/files/pkix/src/test/java/org/bouncycastle/cms/test/Rfc4134Test.java?sort=name&dir=ASC&mode=heatmap#L268
\begin{minted}[highlightlines=4]{java}
Collection c = recipients.getRecipients();
assertTrue(c.size() >= 1 && c.size() <= 2);
Iterator it = c.iterator();
verifyRecipient((RecipientInformation)it.next(), privKey);
\end{minted}

\done{Nate: Cast of Object because of AccessController. PrivAct<T>}
\done{Luis: Not necessarily mistakes, maybe requirement is jdk 1.2}
The following snippet%
\def\urlvar{http://bit.ly/robovm_robovm_2FAI5x5}

%https://lgtm.com/projects/g/robovm/robovm/snapshot/dist-39650108-1524814812150/files/rt/external/apache-xml/src/main/java/org/apache/xml/dtm/SecuritySupport12.java#L59
\begin{minted}[highlightlines=3]{java}
class SecuritySupport12 extends SecuritySupport {
    ClassLoader getSystemClassLoader() {
        return (ClassLoader)
            AccessController.doPrivileged(new PrivilegedAction() {
                public Object run() {
                    ClassLoader cl = null;
                    try {
                        cl = ClassLoader.getSystemClassLoader();
                    } catch (SecurityException ex) {}
                    return cl;
                }
            });
    }
}
public final class AccessController {
    public static <T> T doPrivileged(PrivilegedAction<T> action) {
        return action.run();
    }
} #\urlbox#
\end{minted}

The following example%
\def\urlvar{http://bit.ly/fangjie008_tiexue_mcp_parent_2FSZKzm}
uses the raw type of the \code{Comparable} ---generic--- interface.%
\footnote{\url{https://docs.oracle.com/javase/8/docs/api/java/lang/Comparable.html}}

%https://lgtm.com/projects/g/fangjie008/tiexue-mcp-parent/snapshot/dist-1505957596672-1524814812150/files/mcp-core/src/main/java/com/tiexue/mcp/core/dto/McpSettlementDetailDto.java?sort=name&dir=ASC&mode=heatmap#L100
\begin{minted}[highlightlines=6]{java}
public class McpSettlementDetailDto implements Comparable {
    // [...]

    @Override
    public int compareTo(Object o){
        McpSettlementDetailDto mcpSettlementDetailDto=(McpSettlementDetailDto)o;
        Integer newConsume=(int)mcpSettlementDetailDto.getConsume();
        Integer temp=((int)this.consume);
        return temp.compareTo(newConsume);
    }
} #\urlbox#
\end{minted}

\discussion{}
Raw types exist in \java{} to support legacy code.
Best practice would be to rewrite the code to use generics,
but this is not always feasible or cost effective.

Casts among generic types and between raw types and generic types are unchecked at run time,
although other casts are typically inserted by the compiler to ensure type safety dynamically.
When these inserted casts fail, the reported location of the failure may not match the programmer's expectation.
Indeed, this is similar to the problem of \emph{blame} in gradually typed languages~\cite{wadlerWellTypedProgramsCan2009}.
In this setting, when a run-time cast fails the blame should be put on the appropriate programmer-inserted cast,
not on a compiler-inserted cast.

\end{pattern}
