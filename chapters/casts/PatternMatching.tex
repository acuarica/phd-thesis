\begin{pattern}{PatternMatching}
This pattern is composed of a guard (\code{instanceof}) followed by a cast on known subtypes of the static type.
Often there is just one case and the default case, \ie, \code{instanceof} fails, does a no-op or reports an error.
Another common approach is to have several cases,
usually one \emph{per} subtype.

\instances{}
The following listing shows an example of the \thisp{} pattern.\footnote{\url{http://bit.ly/2FzYYHq}}
In this example, there is only a single case.
After the cast is done, a specific operation is performed on the casted type, \ie, invoking the \code{checkBlock} method.

%https://lgtm.com/projects/g/OpenMods/OpenBlocks/snapshot/dist-2040060754-1524814812150/files/build/sources/java/openblocks/common/tileentity/TileEntityImaginary.java?sort=name&dir=ASC&mode=heatmap#L268
\begin{minted}[highlightlines=3]{java}
Item item = helmet.getItem();
if (item instanceof ItemImaginationGlasses)
	return ((ItemImaginationGlasses)item).checkBlock(what, helmet, this);
\end{minted}

In the next case,\footnote{\url{http://bit.ly/2HnNwB7}} more than one type test are performed.
Each cast is corresponded with a type test.

%https://lgtm.com/projects/g/PenguinSquad/Enchiridion/snapshot/dist-19218583-1524814812150/files/build/sources/main/java/joshie/enchiridion/helpers/StackHelper.java?sort=name&dir=ASC&mode=heatmap#L27
\begin{minted}[highlightlines=7]{java}
public static String getStringFromObject(Object object) {
	if (object instanceof Item) {
		return getStringFromStack(new ItemStack((Item) object));
	} else if (object instanceof Block) {
		return getStringFromStack(new ItemStack((Block) object));
	} else if (object instanceof ItemStack) {
		return getStringFromStack((ItemStack) object);
	} else if (object instanceof String) {
		return (String) object;
	} else if (object instanceof List) {
		return getStringFromStack((ItemStack) ((List) object).get(0));
	} else return "";
}
\end{minted}

There are situations when the type test is applied to more than one variable.
In the following example\footnote{\url{http://bit.ly/2FDN9Rd}} a double
type test is performed on the parameters \code{o1} and \code{o2}.

%https://lgtm.com/projects/g/bbossgroups/bboss/snapshot/dist-2025970729-1524814812150/files/bboss-util/src/org/frameworkset/util/ObjectUtils.java?sort=name&dir=ASC&mode=heatmap#L228
\begin{minted}[highlightlines=25]{java}
public static boolean nullSafeEquals(Object o1, Object o2) {
	if (o1 == o2) {
		return true;
	}
	if (o1 == null || o2 == null) {
		return false;
	}
	if (o1.equals(o2)) {
		return true;
	}
	if (o1.getClass().isArray() && o2.getClass().isArray()) {
		if (o1 instanceof Object[] && o2 instanceof Object[]) {
			return Arrays.equals((Object[]) o1, (Object[]) o2);
		}
		if (o1 instanceof boolean[] && o2 instanceof boolean[]) {
			return Arrays.equals((boolean[]) o1, (boolean[]) o2);
		}
		if (o1 instanceof byte[] && o2 instanceof byte[]) {
			return Arrays.equals((byte[]) o1, (byte[]) o2);
		}
		if (o1 instanceof char[] && o2 instanceof char[]) {
			return Arrays.equals((char[]) o1, (char[]) o2);
		}
		if (o1 instanceof double[] && o2 instanceof double[]) {
			return Arrays.equals((double[]) o1, (double[]) o2);
		}
		if (o1 instanceof float[] && o2 instanceof float[]) {
			return Arrays.equals((float[]) o1, (float[]) o2);
		}
		if (o1 instanceof int[] && o2 instanceof int[]) {
			return Arrays.equals((int[]) o1, (int[]) o2);
		}
		if (o1 instanceof long[] && o2 instanceof long[]) {
			return Arrays.equals((long[]) o1, (long[]) o2);
		}
		if (o1 instanceof short[] && o2 instanceof short[]) {
			return Arrays.equals((short[]) o1, (short[]) o2);
		}
	}
	return false;
}
\end{minted}

\detection{}
To detect this pattern, we look for a cast guarded by an \code{instanceof} expression.
The operand expression to both the cast and the \code{instanceof} needs to be the same.
In the case the operand expression is a variable, we also check that there is no assignment between the \code{instanceof} guard and the cast to that variable.

In some situations the operand expression is a method invocation.
We make sure that the value returned is the same for both the \code{instanceof} and the cast.

\discussion{}
The \thisp{} pattern consists of testing the runtime type of a variable against several related types.
It is a technique that allows a developer to take different actions according to the runtime type of an object.
Depending on the --- runtime --- type of an object, different cases, usually one for each type will follow.

The \thisp{} pattern can be seen as an \adhoc{} alternative to pattern matching.
This construct can be seen in several other languages, \eg, \scala{}, \csharp{}, and \haskell{}.
For instance, in \scala{} the pattern matching construct is achieved using the \code{match} keyword.
In this example\footnote{Adapted from \url{https://docs.scala-lang.org/tour/pattern-matching.html}}, a different action is taken according to the runtime type of the parameter \code{notification} (line 9).

\begin{minted}[highlightlines=10]{scala}
abstract class Notification
case class Email(sender: String, title: String, body: String)
	extends Notification
case class SMS(caller: String, message: String)
	extends Notification
case class VoiceRecording(contactName: String, link: String)
	extends Notification

def showNotification(notification: Notification): String = {
	notification match {
		case Email(email, title, _) =>
		s"You got an email from $email with title: $title"
		case SMS(number, message) =>
		s"You got an SMS from $number! Message: $message"
		case VoiceRecording(name, link) =>
		s"Voice Recording from $name! Click the link: $link"
	}
	}
	val someSms = SMS("12345", "Are you there?")
	val someVoiceRecording = VoiceRecording("Tom", "voicerecording.org/id/123")
	
	// prints You got an SMS from 12345! Message: Are you there?
	println(showNotification(someSms))
	
	// Voice Recording from Tom! Click the link: voicerecording.org/id/123	
	println(showNotification(someVoiceRecording))
\end{minted}

As a workaround, alternatives to the \thisp{} pattern can be the visitor pattern or polymorphism.
But in some cases, the chain of \code{instanceof}s is of boxed types.
There is an ongoing proposal\footnote{\url{http://openjdk.java.net/jeps/305}} to add pattern matching to the \java{} language.
Thus no polymorphism can be used.

\related{}
The \nameref{pat:Equals} pattern can be seen as a special case for the \thisp{} pattern, because often the parameter to the \code{equals} method is guarded by an \code{instanceof} expression.
\end{pattern}