
\begin{pattern}{PatternMatching}
This pattern is composed of a guard (\code{instanceof}) followed by a
cast on known subtypes of the static type.
Often there is just one case and the default case, \ie,
\code{instanceof} fails, does a no-op or reports an error.

\instances{}
The following listing shows an example of the \pname{} pattern.
\begin{lstlisting}[style=java,caption={Instance of \pname{} (from \url{http://bit.ly/2ns4EJq}) }]
if (res instanceof Observable) {
	return (Observable) res;
} else if (res instanceof Single) {
	return ((Single) res).toObservable();
} else if (res instanceof Completable) {
	return ((Completable) res).toObservable();
} else {
	return Observable.just(res);
}
\end{lstlisting}

\discussion{}
The \pname{} pattern can be seen as an \adhoc{}
alternative to pattern matching.
This construct can be seen in several other languages, \eg,
\haskell{}, \scala{}, and \cs{}.
There is an ongoing proposal%
\footnote{\url{http://openjdk.java.net/jeps/305}} to add pattern
matching to the \java{} language.

As a workaround, alternatives to the \pname{} pattern can be the visitor pattern or polymorphism.
But in some cases, the chain of \code{instanceof}s is of boxed types.
Thus no polymorphism can be used.

% Maybe this should be called properly instanceof-guarded cast, to be more specific.
% This pattern checks whether a parameter in an overridden method has a more specific type.
% A cast to a variable guarded by an \code{instanceof}.
% A variable is \emph{guarded} by a condition when the condition controls
% that access to the variable, and there is no assignment after the
% condition and before the access to that variable.

The \pname{} pattern consists of testing the runtime type of a variable against several related types.
Based on rule taken from:
It was taken from a \lgtm{} rule\footnote{\url{https://lgtm.com/rules/910065/}}.

It is a technique that allows a developer to take different actions according to the runtime type of an object.
Depending on the --- runtime --- type of an object, different cases, usually one for each type will follow.

\end{pattern}
