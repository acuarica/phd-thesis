
\begin{pattern}{SelectOverload}

This pattern is used to select the appropiate version of an overloaded
method\footnote{Using ad-hoc polymorphism~\cite{stracheyFundamentalConceptsProgramming2000}}
where two or more of its implementations differ \emph{only} in some argument type.

A cast to \code{null} is often used to select against different versions of a method,
\ie{}, to resolve method overloading ambiguity.
Whenever a \code{null} value needs to be an argument of an a cast is needed to select
the appropriate implementation.
This is because the type of \code{null} has the special type
\emph{null}\footnote{\url{https://docs.oracle.com/javase/specs/jls/se8/html/jls-4.html\#jls-4.1}}
which can be treated as any reference type.
In this case, the compiler cannot determine which method implementation to select.

\instances
%https://lgtm.com/projects/g/loopj/android-async-http/snapshot/dist-1879340034-1518514025554/files/library/src/main/java/com/loopj/android/http/JsonHttpResponseHandler.java?sort=name\&dir=ASC\&mode=heatmap\&excluded=false#L150

Listing \ref{lst:selection} \footnote{\url{asdf}}
shows an example of \pname{} pattern.
Another use case is to select the appropriate the right argument when calling a method with variable arguments.

\begin{lstlisting}[style=ql,label=lst:selection,caption=Example of \pname{} pattern.]
onSuccess(statusCode, headers, (String) null);
\end{lstlisting}

In this example, there are three versions of the \texttt{onSuccess} method, as shown in listing \ref{orgaa1b3bd}.
The cast \texttt{(String) null} is used to select the appropriate version (line 7), based on the third parameter.

	\lstset{language=java,label=orgaa1b3bd,caption={Overloaded methods that differ only in their argument type (the third one).},captionpos=b,numbers=none,style=java}
	\begin{lstlisting}
public void onSuccess(
      int statusCode, Header[] headers, JSONObject response) {...}

public void onSuccess(
      int statusCode, Header[] headers, JSONArray response) {...}

public void onSuccess(
      int statusCode, Header[] headers, String responseString) {...}
\end{lstlisting}

\detection

Listing \ref{org9e0faf3} shows how to detect this pattern.
This pattern shows up when a cast is directly applied to the \texttt{null} constant.

\lstset{language=sql,label=org9e0faf3,caption={Detection of the \pname{} pattern.},captionpos=b,numbers=none,style=ql}
\begin{lstlisting}
import java

from CastExpr ce, NullLiteral nl
where ce.getExpr() = nl
select ce
\end{lstlisting}

\discussion

Casting the \code{null} constant seems rather artificial.
This pattern shows either a lack of expressiveness in \java{} or
a bad \api{} design.
Several other languages support default parameters, \eg{},
\scala{}, \cs{} and \cpp{}.
Adding default parameters might be a partial solution.
% \todo{Relate null as theorical point of view in the TAPL book.}

\end{pattern}