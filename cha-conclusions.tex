\chapter{Conclusions}\label{cha:conclusions}

% TODO:
% Matthias: Expand significantly.
% Maybe 5 pages or more.
% What insights did you gain, can you summarize to what degree you answered the RQs?
% What limitations?
% What future work?

In this thesis we have presented our research.
We have devised common usage patterns for the \java{} Unsafe \api.
We discussed several current and future alternatives to improve the
\java{} language.
This work has been published in~\citep{mastrangeloUseYourOwn2015}.
On the other hand, we plan to complement our Unsafe \api{} study with 
our casting study.
We are devising common usage patterns that involve the casting operator.
Having a taxonomy of usage patterns---for both the Unsafe \api{} and casting---can shed light on how \java{} developers circumvent
the static type system's constraints.

\section{The \java{} Unsafe \api{} in the Wild}

\smu{} is an API that was designed for limited use in system-level runtime library code.
The \unsafe{} API is powerful, but dangerous.
The improper use of \unsafe{} undermines \java{}'s safety guarantees.
We studied to what degree \unsafe{} usage has spread into third-party libraries,
to what degree such third-party usage of \unsafe{} can impact existing Java code,
and which \unsafe{} API features such third-party libraries actually use.
We studied the questions and discussions developers have about \unsafe{},
and we identified common usage patterns.
We thereby provided a basis for evolving the \unsafe{} API, the \java{} language, and the \jvm{}
by eliminating unused or abused unsafe features,
and by providing safer alternatives for features that are used in meaningful ways.
We hope this will help to make \unsafe{} safer.


\section{Casting about in the Dark}

We found the rationale behind some patterns is due to the inexpressiveness of \java{}'s type system.
On the other hand,
there are patterns that abuse or misuse it.

Many of the patterns we found should be unsurprising to most object-oriented programmers.
That nearly 45\% of casts are (possibly) unguarded 
suggests that developers use application-specific knowledge that cannot be easily encoded in
the type system to ensure the absence of run-time type errors.

Our study provides insight on the boundary between static and dynamic typing,
which may inform research on both static and dynamic,
as well as gradual type systems~\citep{Siek06gradualtyping}.
Conversely, this research can inform the design of extensions of the \java{} type system to reduce the need for casting.

We are currently working to define static analyses to detect some of these patterns automatically.
With these analyses,
tools can be developed to identify instances of the pattern and to ensure that they are being implemented properly.
