\chapter{Introduction}

In programming language design, the main goal of a \emph{static} type system is to prevent certain kind of errors from happening at run-time.
A type system is formulated as a set of constraints that gives any expression or term in a program a well-defined type.
As~\cite{pierceTypesProgrammingLanguages2002} states:
``A type system can be regarded as calculating a kind of \emph{static} approximation to the run-time behaviors of the terms in a program.''
These constraints are enforced by the \emph{type-checker} either when compiling or linking the program.
Thus, any program not satisfying the constraints stated within a type system is simply rejected by the type-checker.

Besides detecting early errors,
a typechecker can also be an invaluable \emph{maintenance} tool.
For instance, it can assist an IDE to perform refactor analysis,
such as renaming a method or field.
A static type system can be helpful to enforce disciplined programming.
When composing large-scale software,
\emph{modular languages} are built-up of types,
shown in the interfaces of modules.
Along these lines,
type systems can be useful for \emph{documenting} purposes.
Type annotations, \eg{}, in method and fields declarations,
can provide useful hints to the developer.
Since type annotations are meant to be checked every time the program is compiled,
this information cannot be outdated, unlike comments in the source text.

Static type systems can aid to generate more \emph{efficient} machine code, \eg{},
choosing a different representation for integer or real values at run-time.
Furthermore, in statically checked languages, \eg{}, \java{} or \rust{},
many checks are performed at compile-time,
instead of being performed otherwise at run-time.
Compare this to dinamically checked languages,
where all checks need to be performed at runtime,
degrading performance.
Table~\ref{table:introduction:langs} shows where mainstream languages fit in the safe/unsafe and statically/dynamically checked spectrum.

\begin{table*}[ht]
\footnotesize
\centering
\caption{Safe/Unsafe and Statically/Dynamically checked languages}
\label{table:introduction:langs}
\begin{tabular}{|l||l|l|}
    \hline
            & Statically checked    & Dynamically checked \\
    \hline\hline
    Safe    & \haskell{}, \ml{}, \java{}, \csharp{}, \rust{}, \etc{} & \python{}, \lisp{}, \racket{}, \etc{} \\
    Unsafe  & \cc{}, \cpp{}, \etc{} & \\
    \hline
\end{tabular}
\end{table*}


\section{Beyond Static Type Checking}

Nevertheless, often the static approximation provided by a type system is not precise enough.
Being static, the analysis done by the type-checker needs to be conservative:
It is better to reject programs that are valid,
but whose validity cannot be ensured by the type-checker,
rather than accept some invalid programs.
However, there are situations when the developer has more information
about the program that is too complex to explain in terms of typing constraints.
To that end, programming languages often provide \emph{mechanisms} that 
make the typing constraints less strict
to permit more programs to be valid,
at the expense of causing more errors at run-time.
These mechanisms are essentially two:
\emph{Unsafe Intrinsics} and \emph{Reflective Capabilities}.

\subsection*{Unsafe Intrinsics}

Some programming languages provide \emph{unsafe intrinsics},
the ability to perform certain operations \emph{without} being checked by the compiler.
They are \emph{unsafe} because any misuse made by the programmer can compromise the entire system, \eg{},
corrupting data structures without notice, or
crashing the run-time system.
In other words,
all guarantees provided by a static type system are undermined by the inclusion of unsafe intrinsics.

Unsafe intrinsics can be seen in safe languages, \eg{},
\java{}, \csharp{}, \rust{}, or \haskell{}.
Foreign Function Interface (\emph{FFI}), \ie{}, calling native code from within a safe environment is unsafe.
It is so because the run-time system cannot guarantee anything about the native code.
In addition to FFI,
some safe languages offer so-called \emph{unsafe} blocks, \ie{},
making unsafe operations within the language itself, \eg{},
\csharp{}%
\urlnote{https://docs.microsoft.com/en-us/dotnet/csharp/language-reference/language-specification/unsafe-code}
and
\rust{}.%
\urlnote{https://doc.rust-lang.org/book/second-edition/ch19-01-unsafe-rust.html}
For instance,
when using \code{unsafe} blocks in \rust{},
the developer can dereference a raw pointer,
making the application crash.

Other languages instead provide an \api{} to perform unsafe operations,
\eg{}, \haskell{}%
\urlnote{http://hackage.haskell.org/package/base-4.11.1.0/docs/System-IO-Unsafe.html}
and \java{}.
But in the case of \java{}, the \api{} to make unsafe operations,
\smu{}, is unsupported%
\urlnote{http://www.oracle.com/technetwork/java/faq-sun-packages-142232.html}
and undocumented.
For instance,
by invoking the \code{allocateInstance} on an instance of \smu{}
the developer can allocate an object without calling any constructor,
thus, violating \java{}'s type system guarantees.
It was originally intended for internal use within the \jdk{},
but as we shall see later on, it is used outside the \jdk{} as well.

\subsection*{Reflective Capabilities}

Many programming languages provide some sort of \emph{reflective capabilities},
\ie{}, it enables an executing program to examine or ``introspect'' upon itself.
Much of the \java{} Reflection \api{} resides in the \code{java.lang.reflect} package,
allowing the running program to obtain information about classes and objects.
By using reflection, it is possible to dynamically create instances of a class at run-time as well, \eg{}, through the \code{Class} class.
\csharp{} provides analogous classes, \eg{}, the \code{Type} class,
to achieve the same functionality.
When reflection is used,
many checks that were done by the typechecker statically (at compile-time)
now need to be performed dynamically (at run-time).

Programming languages with subtyping such as \java{} or \cpp{} provide a mechanism to \emph{view} an expression as a different type as it was defined, a form of \emph{lightweight} reflection.
This mechanism is often called \emph{casting} and usually takes the form \code{(T) t}.
Casting can be in two directions: \emph{upcast} and \emph{downcast}.
An upcast conversion happens when converting from a reference type $S$ to a reference type $T$, provided that $T$ is a \emph{supertype} of $S$.
An upcast does not require any explicit casting operation nor compiler check.
However, as we shall see later on, there are situations where an upcast requires an explicit casting operation.
On the other hand, a downcast happens when converting from a reference type $S$ to a reference type $T$, provided that $T$ is a \emph{subtype} of $S$.
Unlike upcasts,
downcasts require a run-time check to verify that the conversion is indeed valid.
For instance, \java{} provides the cast operator; while
\cpp{} with run-time type information (RTTI) provides the \code{dynamic\_cast<>} operator.
This implies that downcasts provide the means to bypass the static type system.
By avoiding the type system, downcasts can pose potential threats, because it is like the developer saying to the compiler: \emph{``Trust me here, I know what I'm doing''}.
Being a escape-hatch to the type system,
a cast is often seen as a design flaw or code smell~\citep{tufanoWhenWhyYour2015} in an object-oriented system.

Featherweight \java{}~\citep{igarashiFeatherweightJavaMinimal2001} is a minimal core calculus for \java{}.
The language features it models are ``mutually recursive class definitions,
object creation, field access, method invocation, method override, method recursion through \code{this}, subtyping, and casting.''
In their model, there are three ways a computation may get stuck.
When accessing a field not declared for a class,
by invoking an nonexisting method for a class,
or when casting to a class that is \emph{not} a supertype of the object's \emph{runtime} class.
They proved that on any well-typed program, the first two never happen,
and the third one never happens on well-typed programs \emph{without downcasts}.
From the theoretical point of view,
(down)casts make typing rules unsound.

\section{Research Question}

If static type systems aim to prevent certain kind of errors from happening at run-time,
yet they provide the means to circumvent their constraints,
why exactly does one need to do so?
Are these mechanisms actually used in real-world code?
If yes, then how so?
This triggers our \textbf{main research question}:

\begin{mdframed}
\rightline{\footnotesize \emph{MRQ}}

\centering
For what purpose do developers circumvent static type systems?
\end{mdframed}

We have confidence that this knowledge can be:
\begin{inparaenum}[a)]
\item a reference for current and future language designers
to make informed decisions about programming languages,
\eg{}, the adoption of \emph{Variable Handles} in \java{} 9~\citep{jep193},
or the addition of \emph{Smart Casts} in \lang{Kotlin},\footnote{\url{https://kotlinlang.org/docs/reference/typecasts.html\#smart-casts}}
\item a reference for tool builders, \eg{}, by providing more precise or new refactoring analyses,
\item a guide for researchers to test new language features, \eg{}, \cite{wintherGuardedTypePromotion2011} or to carry out controlled experiments about programming, \eg{}, \cite{stuchlikStaticVsDynamic2011} and
\item a guide for developers for best or better practices.
\end{inparaenum}

To answer our question above,
we empirically studied how the two aforementioned mechanisms---unsafe intrinsics and casting---are used by developers.
Since any kind of language study must be language-specific,
we focus on \java{} given its wide usage and relevance for both
research and industry.%
\footnote{\url{https://www.tiobe.com/tiobe-index/}}
Moreover, we focus on the \java{} Unsafe API to study unsafe intrinsics,
given that the \java{} Native Interface already has been studied
in~\cite{tanSafeJavaNative2006,tanEmpiricalSecurityStudy2008,kondohFindingBugsJava2008,sunNativeGuardProtectingAndroid2014,liFindingBugsExceptional2009}.
Similarly, although casting uses run-time type information like
the \java{}'s reflection \api{},
the reflection \api{} has been studied in
\cite{livshitsImprovingSoftwareSecurity2006,livshitsReflectionAnalysisJava2005,landmanChallengesStaticAnalysis2017}.

To better drive our \emph{main research question},
we propose to answer the following set of sub-questions.
To answer these research sub-questions,
we have devised \emph{usage patterns} for both the Unsafe \api{} and casting.
Usage patterns are \emph{recurrent programming idioms} used by developers to solve a specific issue.
We believe that having usage patterns can help us to better categorize use cases and
thus understand how these mechanisms are used.
These patterns can provide an insight into how the language is being used by developers in real-world applications.
Overall these sub-questions will help us to answer our MRQ:

\subsection*{Unsafe API}

\begin{enumerate}[label=$RQ/U\arabic*:$,leftmargin=3.4\parindent]
\item {\bf \urqA} \urqAdesc{}
\item {\bf \urqB} \urqBdesc{}
\end{enumerate}

These questions have been already answered in our previous published
study on the Unsafe API in \java{}~\citep{mastrangeloUseYourOwn2015}.

\subsection*{Casting}

\begin{enumerate}[label=$RQ/C\arabic*:$,leftmargin=3.4\parindent]
\item {\bf \crqA} \crqAdesc{}
\item {\bf \crqB} \crqBdesc{}
\item {\bf \crqC} \crqCdesc{}
\end{enumerate}

The results of this study have been submitted for publication to \conf{OOPSLA}{19}.


\section{Thesis Outline}

The rest of this thesis is as follows.
In Chapter~\ref{cha:literature-review} we give a review of the literature in empirical studies of programming languages features.
In particular, Sections~\ref{sec:literature-review:unsafe} and~\ref{sec:literature-review:casting} review the \emph{state-of-the-art} of the different aspects related to the two proposed studies.
Chapter~\ref{cha:unsafe} presents a summary of our \unsafe{} study;
while in Chapter \ref{cha:casts} we present our \emph{casting} study.
Finally, Chapter~\ref{cha:conclusions} presents the conclusions for the thesis.

The Appendix~\ref{ap:ql} contains an introduction to \ql{}%
---the language we used to approximate automatic detection of patterns---%
and reference material used in our casting study.
Appendix~\ref{ap:jnif}---although not directly related---%
describes our bytecode analysis library used in some experiments in both Chapters~\ref{cha:unsafe} and~\ref{cha:casts}.
