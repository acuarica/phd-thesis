
\newcommand{\ctag}[1]{\textsl{#1}}
% \newcommand{\cpattern}[1]{\textsc{#1}}

% \newcommand{\patternsection}[1]{\subsubsection*{\textbf{#1}}}
\newcommand{\patternsection}[1]{\noindent\textbf{#1.}}
% \newcommand{\patternsection}[1]{\textbf{#1.}}

\newenvironment{pattern}[1]{
	\newcommand{\desc}{\patternsection{Description}}
	\newcommand{\instances}{\patternsection{Instances}}
	\newcommand{\detection}{\patternsection{Detection}}
	\newcommand{\discussion}{\patternsection{Discussion}}
	\newcommand{\pname}{\textsc{#1}}
	\subsection{#1}
	\desc
}{}


\chapter{Casting Operations in the Wild}
\label{cha:casts}

% DONE: Intro

Casting operations provide the means to escape the static type system.
\emph{But do they pose a problem for developers?}
Several studies
\citep{kechagiaUndocumentedUncheckedExceptions2014,coelhoUnveilingExceptionHandling2015,zhitnitskyTop10Exception2016}
show that \code{ClassCastException} is in top 10 of exceptions being
thrown when analysing stack traces.
%
To illustrate the sort of problem developers have when applying casting
conversions, we performed a simple search for
commits
% and
% issues\footnote{\url{https://github.com/search?l=Java&q=ClassCastException&type=Issues}}
including the term \code{ClassCastException} on \github.
The search returns about
$150K$
% and
% $72K$
results.%
\footnote{\url{https://github.com/search?l=Java&q=ClassCastException&type=Commits}}
% respectively.
We have included here a few source code results as an example%
\footnote{To easily spot what the developer has changed to fix
the \code{ClassCastException}, we present each source code excerpt
using the Git commit \emph{diff} as reported by \github{}.}
% DONE: Are these really the best motivating/illustrative examples?

\textbf{Forgotten Guard.}
The following listing\footnote{\url{https://github.com/jenkinsci/extra-columns-plugin/commit/02d10bd1fcbb2e656da9b1b4ec54208b0cc1cbb2}}
shows a cast that throws \code{ClassCastException} because the developer forgot to include a guard.
In this case, the developer fixed the error by introducing a guard on the cast with \code{instanceof}.

\begin{lstlisting}[style=java]
@@ -41,6 +41,8 @@ public SCMTypeColumn() {
   }
       public String getScmType(@SuppressWarnings("rawtypes") Job job) {
+        if(!(job instanceof AbstractProject<?, ?>))
+            return "";
       AbstractProject<?, ?> project = (AbstractProject<?, ?>) job;
       return project.getScm().getDescriptor().getDisplayName();
   }
\end{lstlisting}

\textbf{Wrong Cast Target.}
In the next example\footnote{\url{https://github.com/GoldenGnu/jeveassets/commit/5f4750bc8cfa7eed8ad01efd8add2cd2cc9bd831}}
the \code{CustomFileFilter} is an inner static class inside \code{JCustomFileFilter}.
Notice the cast happens inside an \code{equals} method, where this idiom is well known.
But the developer has used the outer --- wrong --- class to cast to.

\begin{lstlisting}[style=java]
@@ -156,7 +156,7 @@ public boolean equals(Object obj) {
  if (getClass() != obj.getClass()) {
      return false;
  }
- final JCustomFileChooser other = (JCustomFileChooser) obj;
+ final CustomFileFilter other = (CustomFileFilter) obj;
  if (!Objects.equals(this.extensions, other.extensions)) {
      return false;
  }
\end{lstlisting}

% DONE: Third example, explain better
\textbf{Generic Type Inference Mismatch.}
In the following listing,\footnote{\url{https://github.com/ethereum/ethereumj/commit/224e65b9b4ddcb46198a6f8faf69edc65d34d382}}
the \emph{dynamic} property \code{"peer.p2p.pingInterval"} (lines $5$ and $6$) has type \code{int}.
To fix the error, the developer only changed the type of the literal $5$: from \code{long} to \code{int}.

\begin{lstlisting}[style=java]
@@ -281,7 +281,7 @@ private void startTimers() {
        } catch (Throwable t) {
            logger.error("Unhandled exception", t);
        }
-   }, 2, config.getProperty("peer.p2p.pingInterval", 5L), TimeUnit.SECONDS);
+   }, 2, config.getProperty("peer.p2p.pingInterval", 5), TimeUnit.SECONDS);
}
\end{lstlisting}

Looking at the definition of the \code{getProperty} method below,%
\footnote{\url{https://github.com/ethereum/ethereumj/blob/224e65b9b4ddcb46198a6f8faf69edc65d34d382/ethereumj-core/src/main/java/org/ethereum/config/SystemProperties.java\#L312}}
it obtains a dynamic property given a property name.
If it finds a value, return it.
Otherwise, returns the default value (second argument).
But the return type of \code{getProperty}
is a generic type inferred by the type of the default value, in this case, \code{long}.
The \code{ClassCastException} is then thrown in line $5$, when casting \code{java.lang.Integer} to \code{java.lang.Long}.
To then fix the bug, the developer changed the type of the literal: from \code{long} to \code{int}.


\begin{lstlisting}[style=java]
public <T> T getProperty(String propName, T defaultValue) {
    if (!config.hasPath(propName)) return defaultValue;
    String string = config.getString(propName);
    if (string.trim().isEmpty()) return defaultValue;
    return (T) config.getAnyRef(propName);
}
\end{lstlisting}

% \textbf{Compiler Bug}
% One issue\footnote{\url{https://github.com/mockito/mockito/issues/357}} 
% shows bad things happen when abusing the type system.
% A bug in the \emph{javac} compiler\footnote{\url{https://bugs.openjdk.java.net/browse/JDK-8058199}}
% was causing \code{checkcast}\footnote{\url{https://docs.oracle.com/javase/specs/jvms/se8/html/jvms-6.html\#jvms-6.5.checkcast}}
% instructions to be skipped.
% This bug was fixed in JDK 9, breaking Mockito answer strategies.

This indicates that casts represents a source of errors for developers.
We present here our partial results for the cast study.
First we give an overview of the study in \S\ref{sec:casts:overview},
while \S\ref{sec:casts:stats} gives an estimation of how often a cast operator is used.
Finally, \S\ref{sec:casts:methodology} introduces the methodology we plan to use to discover cast usage patterns.

\section{Overview of our Study}
\label{sec:casts:overview}

We propose to answer the following question:
\emph{How and when do developers need to escape the type system?}
The cast operator in \java{} provides the means to view a reference at a different type as it was declared.
Upcasts conversions are done automatically by the compiler.
% DONE: Removed
% Nevertheless, in some situations a developer is forced to insert upcasts.
In the case of downcasts, a check is inserted at run-time to verify that the conversion is sound, thus escaping the type system.
\emph{Why is so?}
Therefore, we believe we should care about how the casting operations are used in the wild.
Specifically, we want to answer the following research questions:

\begin{enumerate}[label=$CRQ\arabic*:$,ref=$CRQ\arabic*$,leftmargin=3.4\parindent]
\item\label{enum:rq1}{\bf \crqA}
We want to understand to what extent application code actually uses casting operations.
\item\label{enum:rq2}{\bf \crqB}
If casts are actually used in application code, we want to know how and why developers need to escape the type system.
\item\label{enum:rq3}{\bf \crqC}
In addition to understand how and why casts are used, we want to measure how often developers need to resort to certain idioms to solve a particular problem.
\end{enumerate}

To answer the above questions,
we need to determine whether and how casting operations are actually used in
real-world \java{} applications.
To achieve our goal, several elements are needed.

\textbf{Source Code Analysis.}
We have implemented our study using the \ql{} query language:
``a declarative, object-oriented logic programming language for querying complex, potentially recursive data structures encoded in a relational data model''~\citep{avgustinovQLObjectorientedQueries2016}.
\ql{} allows us to analyze programs at the source code level by abstracting the code sources into a Datalog model.
Besides providing structural data for programs, \ie{}, ASTs,
\ql{} has the ability to query static types and perform data-flow analysis.
To run our \ql{} queries, we have used the service provided by Semmle.\footnote{\url{https://lgtm.com/}} 

\textbf{Projects.} 
As a code base representative of the ``real world'',
we have chosen open-source projects hosted in 
\github{},
the world-most popular source code management repository.
% , \ie{},
% \github{},
%\footnote{\url{https://github.com/}},
% \gitlab{},
%\footnote{\url{https://gitlab.com/}},
% \bitbucket{},
%\footnote{\url{https://bitbucket.org/}}.
So far, we have analyzed \nproject{} \java{} projects in \lgtm{}.
We plan to scale up our analysis to the whole \lgtm{} project database.

\textbf{Usage Pattern Detection.}
After all cast instances are found, we analyze this information to discover usage patterns.
\ql{} allows us to automatically categorize cast use cases into patterns.
This methodology is described in section~\ref{sec:casts:methodology}.

% DONE: Remove: Weird statement.
% It is common that a project exhibits more than one pattern
Our list of patterns is not exhaustive.
Due to the nature of the cast operator, some casts were uncategorized as they would need a whole program analysis, \eg{}, including libraries in the analysis.

\section{Is the Cast Operator used?}
\label{sec:casts:stats}

To answer \ref{casts:rq1} (\emph{\crqA}) we want to know how many cast instances are used in a given project.
To this end, we gather the following statistics using \ql{}.
We show them here to give an estimation of the size of the code base being analized.
% DONE: Why only 24 project? Should say this is preliminary?
As mentioned above, these results are preliminary.
We plan to scale up our analysis to the whole \lgtm{} project database.

\begin{center}
\begin{tabular}{lr}
	Description & Value\\
	\hline
	Number of Projects & \nproject \\
	Number of LOC & \nloc{} \\
	Number of Methods & \nmethod \\
	Number of Methods \emph{w/}Cast & \nmethodwithcast \\
    Number of Expressions & \nexpr \\
	\hline
    Number of Cast Expressions & \nCastExpr \\
    Number of Cast Methods & \nCastMethod \\
    Number of \code{equals} Methods & \nEqualsMethod \\
    Number of \code{instanceof} Expressions & \nInstanceOf \\
    Number of type switch & \nTypeSwitch \\
\end{tabular}
\end{center}

The \emph{Number of Methods} and \emph{Number of Methods w/Cast} values includes only methods with a body, \ie{}, not abstract, nor native.
The \emph{Number of Exprs} value show how many expressions there are in the ASTs of all source code analyzed.
Finally, the \emph{Number of Casts} value indicates how many cast expressions (subtype of \code{Expr} as defined by \ql{}) were found.

% DONE: This is why, no?: Better explained
For our study, we are interested in both upcasts and downcasts.
Thus, we \emph{exclude} primitive conversions in our study
(\S$5.1.2$, \S$5.1.3$, \S$5.1.4$, and \S$5.1.13$ from the \java{} Language Specification%
\footnote{\url{https://docs.oracle.com/javase/specs/jls/se7/html/jls-5.html}}
).
The \emph{Number of Casts} value shown above include only reference conversions.
Primitive conversions are always safe (in terms of throwing \code{ClassCastException}.
A primitive conversion happens when both the type of the expression to be casted to and
the type to cast to are primitive types.
Note that with this definition, we include in our study \emph{boxed} types.
Since boxed types are reference types (and therefore not necessarily safe)
we want to include them in our analysis.

We want to know how many cast instances there are across projects.
Thus, we have computed the ratio between methods containing
at least a cast over total number of methods --- with implementation --- in a given project.
The following chart shows this ratio for all analyzed projects:

\includegraphics[width=\columnwidth,height=3.8cm]{java-cast-queries/analysis/stats-methodwcastXproject.pdf}

All projects have less than $10\%$ of methods with at least a cast.
Overall, around a~$\castpercentage{}\%$ of methods contain at least one cast operation. 
This means there is a low density of casts.
Given the fact that generics were introduced \java{} 5, this can explain this low density.

Nevertheless, casts are still used.
We want to understand why there are casts instances (\ref{casts:rq2}) and
how often the use cases that leads to casts are used (\ref{casts:rq3}).
The following sections give an answer to these questions.

% DONE: cut
% The query to gather this statistics is available online.\footnote{\url{https://gitlab.com/acuarica/java-cast-queries/blob/master/ql/stats.ql}}
% The \lang{R} script to further analyze the query results is available online as well.\footnote{\url{https://gitlab.com/acuarica/java-cast-queries/blob/master/analysis/stats.r}}

\section{Finding Casts Usage Patterns}
\label{sec:casts:methodology}

To answer both research questions
\ref{casts:rq2} (\emph{\crqB}) and \ref{casts:rq3} (\emph{\crqC})
we have used the \ql{} query language within the \lgtm{} service to look for cast instances.
%
As mentioned in section \ref{sec:casts:stats}, \ql{} treats primitive conversions as casts.
Thus, a preliminary step is to exclude them as cast instances.
The following \ql{} query shows how to retrieve all relevant cast expressions:

\begin{lstlisting}[style=ql,caption=\ql{} query to retrieve all relevant cast expressions.]
import java
from CastExpr ce where not (
ce.getExpr().getType() instanceof PrimitiveType and
ce.getTypeExpr().getType() instanceof PrimitiveType
) select ce
\end{lstlisting}

\tikzstyle{decision} = [diamond, aspect=2, draw, fill=blue!20, 
    text width=6em, text badly centered, node distance=3cm, inner sep=0pt]
\tikzstyle{block} = [rectangle, draw, fill=blue!20, 
    text width=7em, text centered, rounded corners, minimum height=2em]
\tikzstyle{block2} = [rectangle, draw, fill=blue!20, 
    text width=4.0em, text centered, rounded corners, minimum height=2em]
\tikzstyle{line} = [draw, -latex']
\tikzstyle{cloud} = [draw, ellipse,fill=red!20, node distance=3.1cm,
    minimum height=2.9em]

\begin{figure}
% \begin{wrapfigure}{r}{7.6cm}
\centering
\begin{tikzpicture}[node distance = 1.5cm, auto]
    % Place nodes
    \node [block] (run) {Run Query};
    % \node [cloud, left of=run] (tags) {Tags};
    \node [cloud, right of=run] (patterns) {Patterns};
    \node [block, below of=run] (inspect) {Inspect Casts without Pattern};
    % \node [decision, below of=inspect, node distance=1.6cm] (tag) {New Tag?};
    \node [decision, below of=inspect, node distance=2.0cm] (pattern) {New Pattern?};
    % \node [block2, left of=tag, node distance=3.1cm] (update-tags) {Update Tags};
    \node [block2, right of=pattern, node distance=3.1cm] (update-pattern) {Update Patterns};
    % \node [decision, below of=evaluate] (decide) {is best candidate better?};
    \node [block, below of=pattern, node distance=1.6cm] (stop) {Stop};
    % Draw edges
    \path [line] (run) -- (inspect);
    % \path [line] (inspect) -- (evaluate);
    % \path [line] (inspect) -- (tag);
    \path [line] (inspect) -- (pattern);
    % \path [line] (tag) -- node [near start] {yes} (update-tags);
    \path [line] (pattern) -- node [near start] {yes} (update-pattern);
    % \path [line] (update-tags) -- (tags);
    \path [line] (update-pattern) -- (patterns);
    % \path [line] (tag) -- node {no}(pattern);
    \path [line] (pattern) -- node {no}(stop);
    % \path [line,dashed] (tags) -- (run);
    \path [line,dashed] (patterns) -- (run);
\end{tikzpicture}
\caption{Process to discover cast tags and patterns.} \label{fig:process}
\end{figure}

Figure~\ref{fig:process} depicts our methodology.
We have used this initial result as a starting point for our analysis.
Afterwards, we select a random sample for manual inspection.
We manually inspected the mentioned casts trying to understand
why and how they were used.

By manually inspecting several casts instances,
we observe that certain characteristics appear often, \eg,
a cast in a overridden method, or a cast guarded by an \code{instanceof}.
We then \emph{tag} cast instances based on these observations.
We implement a \ql{} predicate that detects them and proceed
to refine our query with this new tag predicate.
% The table of tags is presented in table~\ref{table:tags}.
After a new tag is added, the query is run again to iterate over the new results.

% DONE: Remove randomly.
% Whenever we observe that those tags do not appear randomly,
Whenever we detect that those tags appear often,
we further inspect the source code to check that is indeed a pattern.
We have formalized the structure of each pattern as a \ql{} predicate based on those tags.
Similarly with tags, after a new pattern is added,
the query is run again to inspect the casts without pattern.
To sum up, our methodology iterates over the results until
no \emph{more} patterns can be detected.
% These patterns are presented in the following section.
The final \ql{} query is available online.%
\footnote{\url{https://gitlab.com/acuarica/java-cast-queries/blob/master/obs.ql}}


% DONE: What about patterns we can't write queries for?
\subsection*{Manual Categorization of Patterns}

Some code patterns might be too difficult to
express in terms of \ql{} queries.
This situation arises when the knowledge to determine
the pattern is outside the source code,
\eg, in configuration files or library call sites.
Thus, in those cases we can only acknowledge that a pattern exists,
but not how recurrent it is.

% \begin{table*}[t!]
% \centering
% \caption{Cast tags used to discover cast patterns.}
% \label{table:tags}
% \input{table-tags.inc}
% \end{table*}


% * Look for selecting overloaded methods w/downcast.
% * MethodHandle.

\section{Methodology}

As for the project selection, I have used the lgtm.com project database.
We can argue that this provide a good filter of projects,
since teams that want their code to be analyzed push their projects onto lgtm.com.
This will filter out for instance student projects from github.
There are also popular projects, e.g., gradle, neo4j, google guava,
that probably were pulled in by the Semmle people.
We need to double check with them, but if that’s the case,
we can make a good argument as for the project selection.

There is a total of 7.559 projects, with a total 10,193,435 casts.
For each cast, I have the path within the project.
But to manually analyze them, I need to get the lgtm.com link.
This is necessary to actually see the code snippet in which the cast appear.
There are 215 projects for which I can’t get the lgtm.com link.
These 215 projects contains 1,162,583 casts.
There are also 516 projects which does not contain any cast.
Therefore the cast population from where make the sampling consists of
9,030,852 casts spread in 6,840 projects.

Now comes the question: What is an appropriate sample size?
Using this online calculator:

https://www.surveysystem.com/sscalc.htm

With standard parameters, Confidence Level=95\% and Confidence Interval=5,
I got a sample size of 384.
This seems sketchy.
My first approach was to increase the sample size arbitrarily,
e.g., 10,000 casts to manually analyze.
This can be too much effort.
But more importantly, how to come up with the patterns taxonomy?
The current list of patterns I have (using QL) does not cover all
existing patterns, i.e.,
when doing manual sample I have discovered new patterns.
After meeting with Gabriele, he suggested using saturation sampling:
0. Start with an empty list of patterns.
1. Perform a manual sample of, let’s say 384 casts.
2. For each new pattern seen, add it to the list of patterns.
3. If a new pattern is detected, go to step 1.


\section{Casts Usage Patterns}

\label{sec:patterns}

Using the methodology described in the above section,
we have devised $\npattern{}$ casts usage patterns.
Overall, these patterns cover around 
$\patpercentage\%$
with uncovered casts of
$\anypercentage\%$.

In this section we present the cast usage patterns we found.
Figure~\ref{fig:patterns} presents an overview of our patterns and their occurrences sort by most frequent.

\begin{figure}[ht!]
\centering
\includegraphics[width=\columnwidth]{java-cast-queries/query-results/table-patterns-5000.pdf}
\caption{Cast Patterns Occurrences} \label{fig:patterns}
\end{figure}

Any denotes all cast instances that were not categorized.
Each pattern is described using the following template:

\begin{itemize}
\item \textbf{Description.}
Tells what is this pattern about.
\item \textbf{Instances.}
Gives one or more concrete examples found in real
code.\footnote{Please notice that the snippets presented here were slightly modified for formatting purposes.}
For each instance presented here,
we provide the link to the original source code.
\item \textbf{Detection.}
Describes briefly how this pattern was detected in terms of the tags
introduced in the previous section.
\item \textbf{Discussion.}
Presents suggestions, flaws, or comments about the pattern.
\end{itemize}

% \includegraphics[width=\columnwidth,page=2]{upset-patterns.pdf}
% \includegraphics[width=\columnwidth,page=2]{upset-tags.pdf}


\begin{pattern}{PatternMatching}
This pattern is composed of a guard (\code{instanceof}) followed by a
cast on known subtypes of the static type.
Often there is just one case and the default case, \ie,
\code{instanceof} fails, does a no-op or reports an error.
Another common approach is to have several cases,
usually one \emph{per} subtype.

\instances{}
The following listing shows an example of the \thisp{} pattern.%
\footnote{\url{http://bit.ly/2FzYYHq}}
In this example, there is only one case

% https://lgtm.com/projects/g/OpenMods/OpenBlocks/snapshot/dist-2040060754-1524814812150/files/build/sources/java/openblocks/common/tileentity/TileEntityImaginary.java?sort=name&dir=ASC&mode=heatmap#L268
\begin{minted}[highlightlines=3]{java}
Item item = helmet.getItem();
if (item instanceof ItemImaginationGlasses)
	return ((ItemImaginationGlasses)item).checkBlock(what, helmet, this);
\end{minted}

Double typecase example
\footnote{\url{http://bit.ly/2FDN9Rd}}

% https://lgtm.com/projects/g/bbossgroups/bboss/snapshot/dist-2025970729-1524814812150/files/bboss-util/src/org/frameworkset/util/ObjectUtils.java?sort=name&dir=ASC&mode=heatmap#L228
\begin{minted}[highlightlines=25]{java}
public static boolean nullSafeEquals(Object o1, Object o2) {
	if (o1 == o2) {
		return true;
	}
	if (o1 == null || o2 == null) {
		return false;
	}
	if (o1.equals(o2)) {
		return true;
	}
	if (o1.getClass().isArray() && o2.getClass().isArray()) {
		if (o1 instanceof Object[] && o2 instanceof Object[]) {
			return Arrays.equals((Object[]) o1, (Object[]) o2);
		}
		if (o1 instanceof boolean[] && o2 instanceof boolean[]) {
			return Arrays.equals((boolean[]) o1, (boolean[]) o2);
		}
		if (o1 instanceof byte[] && o2 instanceof byte[]) {
			return Arrays.equals((byte[]) o1, (byte[]) o2);
		}
		if (o1 instanceof char[] && o2 instanceof char[]) {
			return Arrays.equals((char[]) o1, (char[]) o2);
		}
		if (o1 instanceof double[] && o2 instanceof double[]) {
			return Arrays.equals((double[]) o1, (double[]) o2);
		}
		if (o1 instanceof float[] && o2 instanceof float[]) {
			return Arrays.equals((float[]) o1, (float[]) o2);
		}
		if (o1 instanceof int[] && o2 instanceof int[]) {
			return Arrays.equals((int[]) o1, (int[]) o2);
		}
		if (o1 instanceof long[] && o2 instanceof long[]) {
			return Arrays.equals((long[]) o1, (long[]) o2);
		}
		if (o1 instanceof short[] && o2 instanceof short[]) {
			return Arrays.equals((short[]) o1, (short[]) o2);
		}
	}
	return false;
}
\end{minted}


\detection{}
To detect this pattern, we look

\discussion{}
The \thisp{} pattern can be seen as an \adhoc{}
alternative to pattern matching.
This construct can be seen in several other languages, \eg,
\haskell{}, \scala{}, and \cs{}.
There is an ongoing proposal%
\footnote{\url{http://openjdk.java.net/jeps/305}} to add pattern
matching to the \java{} language.

As a workaround, alternatives to the \thisp{} pattern can be the
visitor pattern or polymorphism.
But in some cases, the chain of \code{instanceof}s is of boxed types.
Thus no polymorphism can be used.

% Maybe this should be called properly instanceof-guarded cast, to be more specific.
% This pattern checks whether a parameter in an overridden method has a more specific type.
% A cast to a variable guarded by an \code{instanceof}.
% A variable is \emph{guarded} by a condition when the condition controls
% that access to the variable, and there is no assignment after the
% condition and before the access to that variable.

The \thisp{} pattern consists of testing the runtime type of a variable against several related types.
Based on rule taken from:
It was taken from a \lgtm{} rule\footnote{\url{https://lgtm.com/rules/910065/}}.

It is a technique that allows a developer to take different actions according to the runtime type of an object.
Depending on the --- runtime --- type of an object, different cases, usually one for each type will follow.

\end{pattern}

\begin{pattern}{Literal}
Cast a numeric literal or constant --- defined as \code{static final} ---
to a primitive type of
\code{byte}, \code{char}, \code{short}

\instances

The following listing shows an example of the \pname{Literal} pattern.%
\footnote{https://lgtm.com/projects/g/kaitoy/pcap4j/snapshot/dist-29675155-1524814812150/files/pcap4j-core/src/main/java/org/pcap4j/packet/namednumber/TcpPort.java?sort=name\&dir=ASC\&mode=heatmap\#L2329}

\begin{lstlisting}[style=java,caption=Literal example]
public static final TcpPort POWERBURST =
    new TcpPort((short)485, "Air Soft Power Burst");
\end{lstlisting}

\discussion

This pattern is related with \ref{pat:Prim}

\end{pattern}
\begin{pattern}{Prim}

Primitive conversion between numerical values.

\instances
    
\end{pattern}
\begin{pattern}{RawTypes}
When a generic method is not used as such.
The expression of this cast is a method invocation,
but the declaration differs from the usage.

\instances

\end{pattern}
\begin{pattern}{Family}
Family polymorphism.

\instances{}

\detection{}

\discussion{}
\cite{ernstFamilyPolymorphism2001}

\related{}

\end{pattern}
\begin{pattern}{LookupById}
This pattern is used to extract values from a heterogenous container.
It looks up an object by a compile-time constant identifier, tag, or name and casts the result to an appropriate type.
It accesses a collection that holds values of different types
(usually implemented as \code{Collection<Object>} or as \code{Map<K, Object>}).
The actual run-time type returned from the lookup is determined by the value of the identifier.

\instances{}
In the example shown below,%
\footnote{\url{http://bit.ly/loopj_android-async-http_2SUzY4E}}
the \code{getAttribute} method returns \code{Object}.
The variable \texttt{context} is of type \code{BasicHttpContext},
which is implemented with \code{HashMap}.

%https://lgtm.com/projects/g/loopj/android-async-http/snapshot/dist-1879340034-1518514025554/files/library/src/main/java/com/loopj/android/http/AsyncHttpClient.java?sort=name&dir=ASC&mode=heatmap&excluded=false#L258
\begin{minted}[highlightlines=1,linenos=false]{java}
AuthState authState = (AuthState) context.getAttribute(ClientContext.TARGET_AUTH_STATE);
\end{minted}

The next snippet%
\footnote{\url{http://bit.ly/skerit_cmusphinx_2HGgL1D}}
shows a call site to the \code{getComponent} method cast to the \code{ActiveListManager} class (line 15).
The \code{getComponent} method in this cast instance uses as argument the \code{PROP\_ACTIVE\_LIST\_MANAGER} constant.
Looking at the definition of this constant (line 3),
we can see there is a companion attribute (\code{@S4Component}) whose argument is the \code{ActiveListManager} class, the target of the cast instance.

% https://lgtm.com/projects/g/skerit/cmusphinx/snapshot/dist-1506204046584-1524814812150/files/sphinx4/src/sphinx4/edu/cmu/sphinx/decoder/search/WordPruningBreadthFirstSearchManager.java?sort=name&dir=ASC&mode=heatmap#L207
\begin{minted}[highlightlines=15]{java}
/** The property that defines the type of active list to use */
@S4Component(type = ActiveListManager.class)
public final static String PROP_ACTIVE_LIST_MANAGER = "activeListManager";

@Override
public void newProperties(PropertySheet ps) throws PropertyException {
    super.newProperties(ps);
    logMath = (LogMath) ps.getComponent(PROP_LOG_MATH);
    logger = ps.getLogger();
    linguist = (Linguist) ps.getComponent(PROP_LINGUIST);
    pruner = (Pruner) ps.getComponent(PROP_PRUNER);
    scorer = (AcousticScorer) ps.getComponent(PROP_SCORER);
    activeListManager = 
            (ActiveListManager) ps.getComponent(PROP_ACTIVE_LIST_MANAGER);
    // [...]
}
\end{minted}

A common version of this pattern is to retrieve a value instantiated from
static resource file, \eg,
an XML, HTML or \java{} properties file.
The file contents are (in theory) known at compile-time and the file is included in the binary distribution
of the application. These files are often built using tools such as GUI
builders.

In the following example from an Android application,%
\footnote{\url{http://bit.ly/pwittchen_NetworkEvents_2HGbrMq}}
a cast is applied to the \code{findViewById} method invocation.
View classes are instantiated by the application framework using an XML resource file.
The \code{findViewById} method looks up the view by its ID.

%https://lgtm.com/projects/g/pwittchen/NetworkEvents/snapshot/dist-2032650416-1524814812150/files/example/src/main/java/com/github/pwittchen/networkevents/app/MainActivity.java?sort=name&dir=ASC&mode=heatmap#L65
\begin{minted}[highlightlines=1,linenos=false]{java}
mobileNetworkType = (TextView) findViewById(R.id.mobile_network_type);
\end{minted}

\discussion{}
This pattern suggests a heterogeneous dictionary.
In our manual inspection,
all dictionary keys and the resulting types are known at
compile time, however
a cast is needed because the dictionary type does not encode the
relationship between key values and the result type.
Casts in this pattern are typically not guarded indicating that the programmer
knows the source of the cast based on the value of the key.

This pattern is often seen in Android applications.
The Butter Knife framework%
\footnote{\url{http://jakewharton.github.io/butterknife/}}
uses annotations to avoid the ``manual'' casting.
Instead, code is generated that casts the result of \code{findViewById} to the
appropriate type.


But in any case a cast is needed given the inexpressiveness of the type system.
As a complementary analysis,
it would be interesting to check whether all call sites to
\code{getAttribute} receives a constant (\code{final static} field).

Notice that this pattern is not guarded by an \code{instanceof}.
However, the cast involved does not fail at runtime.
This means that the source of the cast is known to the programmer.
This raises the following questions:
\begin{itemize}
\item \emph{What kind of analysis is needed to detect the source of the cast?}
\item \emph{Is worth to have it?}
\item \emph{Is better to change API?}
\item \emph{How other --- statically typed --- languages support this kind of idiom?}
\item \emph{Could generative programming a.k.a. templates solve this problem?}
\end{itemize}

This pattern retrieves a stashed a value from an heterogeneous collection (or dictionary).
A cast is needed because the return type depends on the ID to be retrieved.
This scenario is similar to the \nameref{pat:Tag} pattern,
where usually the developer retrieves a stashed value from a superclass field.
Since this pattern casts a value to a known type from a method invocation,
it can be seen as a kind of \nameref{pat:KnownReturnType} pattern.

\end{pattern}
\begin{pattern}{Factory}
Creates an object based on some arguments either to the method call or constructor.
Since the arguments are known at compile-time, cast to the specific type.

Cast factory method result to subtype (special case of family polymorphism).
Usually Logger.getLogger.

The method is declared to return URLConnection but can return a more specific type based on the URL string.
Cast to that.
We should generalize this pattern.

\instances{}

\footnote{\url{http://bit.ly/2HvRlUX}}

\footnote{\url{https://docs.oracle.com/javase/8/docs/api/java/security/KeyPair.html\#getPrivate()}}

%https://lgtm.com/projects/b/connect2id/oauth-2.0-sdk-with-openid-connect-extensions/snapshot/dist-1311020143-1524814812150/files/src/test/java/com/nimbusds/oauth2/sdk/jose/jwk/RemoteJWKSetTest.java?sort=name&dir=ASC&mode=heatmap#L242
\begin{minted}[highlightlines=10]{java}
KeyPairGenerator pairGen = KeyPairGenerator.getInstance("RSA");
pairGen.initialize(1024);
KeyPair keyPair = pairGen.generateKeyPair();
RSAKey rsaJWK1 = new RSAKey.Builder((RSAPublicKey) keyPair.getPublic())
        .privateKey((RSAPrivateKey) keyPair.getPrivate())
        .keyID("1")
        .build();
keyPair = pairGen.generateKeyPair();
RSAKey rsaJWK2 = new RSAKey.Builder((RSAPublicKey) keyPair.getPublic())
        .privateKey((RSAPrivateKey) keyPair.getPrivate())
        .keyID("2")
        .build();
\end{minted}

\footnote{\url{http://bit.ly/2E6KY6T}}

\footnote{\url{https://docs.oracle.com/javase/8/docs/api/java/net/URL.html\#openConnection--}}

%https://lgtm.com/projects/g/apache/hadoop/snapshot/dist-956730001-1524814812150/files/hadoop-yarn-project/hadoop-yarn/hadoop-yarn-server/hadoop-yarn-server-resourcemanager/src/test/java/org/apache/hadoop/yarn/server/resourcemanager/webapp/TestRMWebServicesHttpStaticUserPermissions.java?sort=name&dir=ASC&mode=heatmap#L138
\begin{minted}[highlightlines=2]{java}
URL url = new URL("http://localhost:8088/ws/v1/cluster/apps");
HttpURLConnection conn = (HttpURLConnection) url.openConnection();
\end{minted}

\detection{}

\discussion{}

\related{}

\end{pattern}
\begin{pattern}{TypeTag}
Lookup in a collection using a application-specific type tag or a
\code{java.lang.Class}.

A cast guarded by a test on a field from the same object instead of
using \code{instanceof}.

\instances{}
The following example%
\footnote{\url{http://bit.ly/2Ho8bVL}}
shows an instance of the \thisp{} pattern.
The cast type of the parameter \code{reference} is determined by the value of the parameter \code{referenceType}.

%https://lgtm.com/projects/b/JesusFreke/smali/snapshot/dist-1306230039-1524814812150/files/dexlib2/src/main/java/org/jf/dexlib2/writer/InstructionWriter.java?sort=name&dir=ASC&mode=heatmap#L492
\begin{minted}[highlightlines=8]{java}
private int getReferenceIndex(int referenceType, Reference reference) {
    switch (referenceType) {
        case ReferenceType.FIELD:
            return fieldSection.getItemIndex((FieldRefKey) reference);
        case ReferenceType.METHOD:
            return methodSection.getItemIndex((MethodRefKey) reference);
        case ReferenceType.STRING:
            return stringSection.getItemIndex((StringRef) reference);
        case ReferenceType.TYPE:
            return typeSection.getItemIndex((TypeRef) reference);
        case ReferenceType.METHOD_PROTO:
            return protoSection.getItemIndex((ProtoRefKey) reference);
        default:
            throw new ExceptionWithContext(
                "Unknown reference type: %d",  referenceType);
    }
}
\end{minted}

In the next case%
\footnote{\url{http://bit.ly/2FW5SXU}}
a type test is performed --- through a method call --- before actually applying the cast to the variable \code{props}.
Note that the type test is using the \code{instanceof} operator (line 8).

%https://lgtm.com/projects/g/apache/poi/snapshot/dist-1790760597-1524814812150/files/src/ooxml/java/org/apache/poi/xslf/usermodel/XSLFPropertiesDelegate.java?sort=name&dir=ASC&mode=heatmap#L1367
\begin{minted}[highlightlines=3]{java}
@Override
public CTSolidColorFillProperties getSolidFill() {
    return isSetSolidFill() ? (CTSolidColorFillProperties)props : null;
}

@Override
public boolean isSetSolidFill() {
    return (props instanceof CTSolidColorFillProperties);
}
\end{minted}

\detection{}
The detection of this pattern is similar to the \nameref{pat:PatternMatching} detection, but instead of looking for an \code{instanceof} guarded cast, we look for an application-specific guard.

\discussion{}

\related{}

\end{pattern}
\begin{pattern}{Equals}
This pattern is a common pattern to implement the \code{equals} method (declared in \code{java.lang.Object}).
A cast expression is guarded by either an \code{instanceof} test or a \code{getClass} comparison (usually to the same target type as the cast);
in an \code{equals}%
\footnote{\url{https://docs.oracle.com/javase/8/docs/api/java/lang/Object.html\#equals-java.lang.Object-}} method implementation.
This is done to check if the argument has same type as the receiver
(\code{this} argument).

Notice that a cast in an \code{equals} method is needed because it
receives an \code{Object} as a parameter.

\instances{}
The following listing\footnote{\url{http://bit.ly/2vJw94J}} shows an example of the \pname{} pattern.
In this case, \code{instanceof} is used to guard for the same type as the receiver.

%https://lgtm.com/projects/g/neo4j/neo4j/snapshot/dist-15760049-1519892555006/files/community/kernel/src/main/java/org/neo4j/kernel/impl/api/CountsRecordState.java?sort=name&dir=ASC&mode=heatmap&excluded=false#L182
\begin{minted}[highlightlines=7]{java}
@Override
public boolean equals(Object obj) {
    if ( this == obj ) {
        return true;
    }
    if ( (obj instanceof Difference) ) {
        Difference that = (Difference) obj;
        return actualFirst == that.actualFirst
          && expectedFirst == that.expectedFirst
          && actualSecond == that.actualSecond 
          && expectedSecond == that.expectedSecond
          && key.equals( that.key );
    }
    return false;
}
\end{minted}

Alternatively, the following listing%
\footnote{\url{http://bit.ly/2vKP0MW}}
shows another example of the \thisp{} pattern.
But in this case, a \code{getClass} comparison is used to guard for the same type as the receiver.

%https://lgtm.com/projects/g/neo4j/neo4j/snapshot/dist-15760049-1519892555006/files/community/bolt/src/main/java/org/neo4j/bolt/v1/messaging/infrastructure/ValuePath.java?sort=name&dir=ASC&mode=heatmap&excluded=false#L278
\begin{minted}[highlightlines=7]{java}
@Override
public boolean equals( Object o ) {
    if ( this == o ) return true;
    if ( o == null || getClass() != o.getClass() )
        return false;

    ValuePath that = (ValuePath) o;
    return nodes.equals(that.nodes) &&
        relationships.equals(that.relationships);
}
\end{minted}

In some situations, the type casted to is not same as the enclosing class.
Instead, the type casted to is the super class of the enclosing class.
The following example%
\footnote{\url{http://bit.ly/2HmHMYE}}
shows this scenario.
This usually happens when the Google AutoValue library%
\footnote{\url{https://github.com/google/auto/tree/master/value}}
is used.
The AutoValue is a code generator for value classes.

%https://lgtm.com/projects/g/square/sqlbrite/snapshot/3a9916985485ba5922097fe59a18230500f02df4/files/sample/build/generated/source/apt/debug/com/example/sqlbrite/todo/ui/$AutoValue_ListsItem.java?sort=name&dir=ASC&mode=heatmap&showExcluded=false#L52
\begin{minted}[highlightlines=13]{java}
@AutoValue
abstract class ListsItem implements Parcelable {
    // [...]
}

abstract class $AutoValue_ListsItem extends ListsItem {
    @Override
    public boolean equals(Object o) {
      if (o == this) {
        return true;
      }
      if (o instanceof ListsItem) {
        ListsItem that = (ListsItem) o;
        return (this.id == that.id())
             && (this.name.equals(that.name()))
             && (this.itemCount == that.itemCount());
      }
      return false;
    }
}
\end{minted}

\footnote{\url{http://bit.ly/2SM5pOw}}
%https://lgtm.com/projects/g/bndtools/bnd/snapshot/dist-930051-1524814812150/files/biz.aQute.bndlib/src/aQute/bnd/osgi/resource/CapReq.java?sort=name&dir=ASC&mode=heatmap#L73
\begin{minted}[highlightlines=12]{java}
@Override
public boolean equals(Object obj) {
    if (this == obj)
            return true;
    if (obj == null)
            return false;
    if (obj instanceof CapReq)
            return equalsNative((CapReq) obj);
    if ((mode == MODE.Capability) && (obj instanceof Capability))
            return equalsCap((Capability) obj);
    if ((mode == MODE.Requirement) && (obj instanceof Requirement))
            return equalsReq((Requirement) obj);
    return false;
}
\end{minted}

\detection{}
The detection query looks for a cast expression inside an \code{equals} method implementation.
Moreover, the cast needs to be guarded by either an \code{instanceof} test or a \code{getClass} comparison.
And the type being casted to needs to be either the same as the enclosing class or a superclass of it.

\discussion{}
The pattern for an \code{equals} method implementation is well-known.

We found out that, with respect to cast,
most \code{equals} methods are implemented with the same structure.
Maybe avoid boilerplate code by providing code generation,
like in \haskell{} (with \code{deriving}).

\cite{vaziriDeclarativeObjectIdentity2007} propose a declarative approach to avoid boilerplate code when implementing both the \code{equals} and \code{hashCode} methods.
They manually analyzed several applications, and found there are many issues while implementing \code{equals()} and \code{hashCode()} methods.
It would be interesting to check whether these issues happen in real application code.

There is an exploratory document%
\footnote{\url{http://cr.openjdk.java.net/\~briangoetz/amber/datum.html}}
by Brian Goetz --- \java{} Language Architect --- addressing these issues from a more general perspective.
It is definitely a starting point towards improving the \java{} language.

\related{}
This pattern can be seen as a special instance of the \nameref{pat:PatternMatching} pattern.
\end{pattern}
\begin{pattern}{SelectOverload}
This pattern is used to select the appropriate version of an overloaded method%
\footnote{Using ad-hoc polymorphism~\citep{stracheyFundamentalConceptsProgramming2000}.}
where two or more of its implementations differ \emph{only} in some argument type.

A cast of the \code{null} literal is often used to resolve method overloading ambiguity because the type of \code{null} is a subtype of any reference type.%
\footnote{\url{https://docs.oracle.com/javase/specs/jls/se8/html/jls-4.html\#jls-4.1}}

A cast to \code{null} is often used to select against different versions of a method,
\ie{}, to resolve method overloading ambiguity.
Whenever a \code{null} value needs to be an argument of an a cast is
needed to select the appropriate implementation.
This is because the type of \code{null} has the special type \emph{null}%
\footnote{\url{https://docs.oracle.com/javase/specs/jls/se8/html/jls-4.html\#jls-4.1}}
which can be treated as any reference type.
In this case,
the compiler cannot determine which method implementation to select.

Another use case is to select the appropriate the right argument when calling a method with variable arguments.

\instances{}
The following listing shows an example of the \thisp{} pattern.
In this example, there are three versions of the \code{onSuccess} method,
The cast \code{(String) null} is used to select the appropriate version
(line 7), based on the third parameter.
Overloaded methods that differ only in their argument type (the third one).

% https://lgtm.com/projects/g/loopj/android-async-http/snapshot/dist-1879340034-1518514025554/files/library/src/main/java/com/loopj/android/http/JsonHttpResponseHandler.java?sort=name\&dir=ASC\&mode=heatmap\&excluded=false#L150
\def\urlvar{http://bit.ly/loopj_android_async_http_2FENovD}
\begin{minted}[highlightlines=1]{java}
onSuccess(statusCode, headers, (String) null);
public void onSuccess(
      int statusCode, Header[] headers, JSONObject response) { /* [...] */ }
public void onSuccess(
      int statusCode, Header[] headers, JSONArray response) { /* [...] */ }
public void onSuccess(
      int statusCode, Header[] headers, String responseString) { /* [...] */ }
#\urlbox#
\end{minted}

In the following example \code{actual.data()} returns a boxed \code{Long}.
Because implicit upcasts have precedence over implicit unboxing conversions,
the call is needed to invoke the method that takes a \code{long} (line 3) rather than the method that takes an \code{Object} (line 2).

%https://lgtm.com/projects/g/spullara/redis-protocol/snapshot/dist-41940059-1524814812150/files/client/src/test/java/redis/client/AllCommandsTest.java?sort=name&dir=ASC&mode=heatmap#L366
\def\urlvar{http://bit.ly/spullara_redis_protocol_2FC9Llb}
\begin{minted}[highlightlines=1]{java}
assertEquals(expected, (long) actual.data());
public static void assertEquals(Object expected, Object actual) { /* [...] */ }
public static void assertEquals(long expected, long actual) { /* [...] */ }
#\urlbox#
\end{minted}

The following snippet is similar to the previous example,
but notice how that the cast is applied to a
primitive---\emph{non-reference}---type.

%https://lgtm.com/projects/g/apache/poi/snapshot/dist-1790760597-1524814812150/files/src/testcases/org/apache/poi/hssf/record/chart/TestLegendRecord.java#L50
\def\urlvar{http://bit.ly/apache_poi_2StrlOn}
\begin{minted}[highlightlines=1,linenos=false]{java}
assertEquals((byte) 0x1, record.getSpacing());
#\urlbox#
\end{minted}

In the last example of \thisp,
an upcast of a generic type is performed to select the appropriate overload of the \code{max} method.

%https://lgtm.com/projects/g/groovy/groovy-core/snapshot/dist-45390050-1524814812150/files/src/main/org/codehaus/groovy/runtime/DefaultGroovyMethods.java?sort=name&dir=ASC&mode=heatmap#L6715
\def\urlvar{http://bit.ly/groovy_groovy_core_2HDAkbF}
\begin{minted}[highlightlines=2]{java}
public static <T> T max(Iterator<T> self, Comparator<T> comparator) {
      return max((Iterable<T>)toList(self), comparator);
}
public static <T> List<T> toList(Iterator<T> self) {
      // [...]
}
@Deprecated
public static <T> T max(Collection<T> self, Comparator<T> comparator) {
      // [...]
}
public static <T> T max(Iterable<T> self, Comparator<T> comparator) {
      // [...]
} #\urlbox#
\end{minted}


\detection{}
The Query~\ref{lst:ql:SelectOverloadCast} detects when a cast is used as an argument of an overloaded method.
A cast returned by this query needs to be either a cast to \code{null} or an upcast.
This is an approximation because the query does not check whether the overloaded method differs only on the type of the argument that is cast.

\begin{listing}
\begin{minted}{java}
class SelectOverloadCast extends #\qlref{Cast}# { #\qlbox#
	SelectOverloadCast() {
		(getExpr() instanceof NullLiteral or this instanceof #\qlref{Upcast}#) and
		this instanceof #\qlref{OverloadedArgument}#
	}
	Callable getOverload() {
		result = this.(#\qlref{OverloadedArgument}#).getAnOverload()
	}
}
\end{minted}
\caption{Query to detect the \thisp{} pattern.}
\label{lst:ql:SelectOverloadCast}
\end{listing}

\issues{}
Casting the \code{null} constant seems rather artificial.
This pattern shows either a lack of expressiveness in \java{} or a bad \api{} design.
Passing \code{null} to a method might better be handled by using overloading with fewer parameters or by using default parameters.
Several other languages support default parameters,
\eg, \scala{}, \csharp{} and \cpp{}.
Adding default parameters might be a partial solution.

In addition, a pure object-oriented language would not distinguish between primitives and objects,
avoiding the need for autoboxing to be visible at the type level.

\cite{oostvogelsStaticTypingComplex2018a} propose an extension to \typescript{} to express constraints between properties,
which can then be mapped onto optional parameters.

Both the \nameref{pat:AccessSuperclassField} and this pattern are used to select class members.
While this pattern is used to select the appropriate overloaded method,
the \nameref{pat:AccessSuperclassField} is used to select a field in a superclass.

\end{pattern}

\begin{pattern}{Redundant}
A cast that is not necessary for compilation.

\instances{}
The following example%
\footnote{\url{http://bit.ly/2FWXw2e}}

%https://lgtm.com/projects/g/vladmihalcea/high-performance-java-persistence/snapshot/dist-1813180502-1524814812150/files/core/src/test/java/com/vladmihalcea/book/hpjp/hibernate/schema/flyway/FlywayTest.java#L40
\begin{minted}[highlightlines=1]{java}
transactionTemplate.execute((TransactionCallback<Void>) transactionStatus -> {
    Post post = new Post();
    entityManager.persist(post);
    return null;
});
\end{minted}

\detection{}

\discussion{}

\related{}
    
\end{pattern}
\input{java-cast-queries/ql/KnownLibraryMethod.tex}
\begin{pattern}{NewDynamicInstance}
Dynamically creation of object by means of reflection.
These are the casts that can not be avoidable.

The \code{newInstance} method family declared in the
\code{Class}\footnote{\url{https://docs.oracle.com/javase/8/docs/api/java/lang/Class.html\#newInstance--}},
\code{Array}\footnote{\url{https://docs.oracle.com/javase/8/docs/api/java/lang/reflect/Array.html\#newInstance-java.lang.Class-int-}}\(^{,}\)
\footnote{\url{https://docs.oracle.com/javase/8/docs/api/java/lang/reflect/Array.html\#newInstance-java.lang.Class-int...-}} and
\code{Constructor}\footnote{\url{https://docs.oracle.com/javase/8/docs/api/java/lang/reflect/Constructor.html\#newInstance-java.lang.Object...-}}
classes creates an object or array by means of reflection.

This pattern consists of casting the result of these methods to the appropriate target type.

\instances{}

%https://lgtm.com/projects/g/apache/hadoop/snapshot/6bedbef6c5f2d937a6cbc268300ce2a39609d06c/files/hadoop-hdfs-project/hadoop-hdfs/src/main/java/org/apache/hadoop/hdfs/server/namenode/FSNamesystem.java?sort=name\&dir=ASC\&mode=heatmap\&showExcluded=false#L1039

The following example shows a cast from the \code{Class.newInstance()}
method.
% \footnote{\url{d}}

\begin{lstlisting}[style=java,caption={The \pname{} pattern using the \texttt{Class} class.}]
logger = (AuditLogger) Class.forName(className).newInstance();
\end{lstlisting}

%https://lgtm.com/projects/g/neo4j/neo4j/snapshot/27aaa67633e4d26446e38125d04fbbd27f938b75/files/community/collections/src/main/java/org/neo4j/helpers/collection/Iterables.java?sort=name\&dir=ASC\&mode=heatmap\&showExcluded=false#L403
The following example shows how to dynamically create an array.
%\footnote{\url{d}}

\begin{lstlisting}[style=java,caption={Example of the \pname{} pattern using the \texttt{Array} class.}]
return list.toArray( (T[]) Array.newInstance( componentType, list.size()));
\end{lstlisting}

%https://lgtm.com/projects/g/gradle/gradle/snapshot/209c3175e75af6ac30cb66c02eda15b0f8b6a616/files/subprojects/internal-integ-testing/src/main/groovy/org/gradle/integtests/fixtures/executer/OutputScrapingExecutionFailure.java?sort=name\&dir=ASC\&mode=heatmap\&showExcluded=false#L174

Whenever a constructor other than the default constructor is needed,
the \code{newInstance} method declared in the \code{Constructor} class
should be used to select the appropriate constructor,
as shown in the following example.
%\footnote{\url{d}}

\begin{lstlisting}[style=java,caption=Example of the \pname{} pattern using the \code{Constructor} class.]
return (Exception) Class
                       .forName(className)
                       .getConstructor(String.class)
                       .newInstance(message);
\end{lstlisting}

\detection{}
This detection query looks for casts,
where the expression being cast is a call site to methods mentioned above.

\discussion{}
The cast here is needed because of the dynamic essence of reflection.
This pattern is unguarded, that is,
the application programmer knows what is the target type being created.

\related{}
Reflection.

\end{pattern}
\begin{pattern}{Clone}
A cast to a \code{clone} method.

\instances

\end{pattern}
\begin{pattern}{ImplicitIntersectionType}
Cast a reference $v$ of type --- class or interface --- $T$ to an
interface type $I$ whether $T$ does not implement $I$.
The cast succeeds at runtime because all possible runtime types of $v$
actually implement the interface $I$.
For instance, in \code{(Comparable)(Number)4}, \code{Number} does not
implement the \code{Comparable} interface, but class \code{Integer} does.

\instances

\begin{lstlisting}[style=java,caption=From \url{http://bit.ly/2FQOt4v}]
final Comparable max = (Comparable) properties.getMaxValue();
\end{lstlisting}
\end{pattern}
\begin{pattern}{Deserialization}
This pattern is used to deserialize an object at run-time.

\instances{}
The following example%
\footnote{\url{http://bit.ly/internetarchive_heritrix3_2SF4j7k}}
shows how the \thisp{} pattern is used to create objects from a file system (line 19).

%https://lgtm.com/projects/g/internetarchive/heritrix3/snapshot/dist-12140105-1524814812150/files/engine/src/test/java/org/archive/crawler/datamodel/CrawlURITest.java?sort=name&dir=ASC&mode=heatmap#L83
\begin{minted}[highlightlines=19]{java}
final public void testSerialization()
        throws IOException, ClassNotFoundException {
    File serialize = new File(getTmpDir(),
            this.getClass().getName() + ".serialize");
    try {
        FileOutputStream fos = new FileOutputStream(serialize);
        ObjectOutputStream oos = new ObjectOutputStream(fos);
        oos.writeObject(this.seed);
        oos.reset();
        oos.writeObject(this.seed);
        oos.reset();
        oos.writeObject(this.seed);
        oos.close();
        // Read in the object.
        FileInputStream fis = new FileInputStream(serialize);
        ObjectInputStream ois = new ObjectInputStream(fis);
        CrawlURI deserializedCuri = (CrawlURI)ois.readObject();
        deserializedCuri = (CrawlURI)ois.readObject();
        deserializedCuri = (CrawlURI)ois.readObject();
        assertEquals("Deserialized not equal to original",
                this.seed.toString(), deserializedCuri.toString());
        String host = this.seed.getUURI().getHost();
        assertTrue("Deserialized host not null",
                host != null && host.length() >= 0);
    } finally {
        serialize.delete();
    }
}
\end{minted}

\detection{}
This pattern is characterized for a cast to the \code{readObject} method on a \code{ObjectInputStream} object.

\discussion{}
From a language design perspective,
the \thisp{} pattern is one of the most difficult patterns to avoid.
It is difficult to avoid because a compiler cannot verify at compile-time that a certain byte stream can be deserialized into an object of a given type.

\related{}
Both this pattern and the \nameref{pat:NewDynamicInstance} pattern create objects by using reflection.
\nameref{pat:StaticResource}

\end{pattern}
\begin{pattern}{StaticResource}
A cast to a method access to that reads a static resource file, \eg,
XML, HTML or \java{} properties file.
The file is static since its contents are known at compile-time.
Usually this file is build with a third-party tool, like a GUI designer.

\instances{}
In the following example,%
\footnote{\url{http://bit.ly/pwittchen_NetworkEvents_2HGbrMq}}
a cast is applied to a \code{findViewById} method invocation.
The \code{findViewById} method looks up for the given ID in a XML resource file to retrieve the specified view. 

%https://lgtm.com/projects/g/pwittchen/NetworkEvents/snapshot/dist-2032650416-1524814812150/files/example/src/main/java/com/github/pwittchen/networkevents/app/MainActivity.java?sort=name&dir=ASC&mode=heatmap#L65
\begin{minted}[highlightlines=6]{java}
@Override
protected void onCreate(Bundle savedInstanceState) {
    super.onCreate(savedInstanceState);
    setContentView(R.layout.activity_main);
    connectivityStatus = (TextView) findViewById(R.id.connectivity_status);
    mobileNetworkType = (TextView) findViewById(R.id.mobile_network_type);
    accessPoints = (ListView) findViewById(R.id.access_points);
    busWrapper = getOttoBusWrapper(new Bus());
    networkEvents = new NetworkEvents(getApplicationContext(), busWrapper)
        .enableInternetCheck()
        .enableWifiScan();
}
\end{minted}

The next listing,%
\footnote{\url{http://bit.ly/pentaho_pentaho-kettle_2TswNSf}}
shows a cast to a GUI component (\code{XulListbox}) using the \code{getElementById} method (lines 12 and 13).
In this case the developer is using the XUL language.%
\footnote{\url{https://developer.mozilla.org/en-US/docs/Mozilla/Tech/XUL}}

%https://lgtm.com/projects/g/pentaho/pentaho-kettle/snapshot/dist-1815472020-1524814812150/files/ui/src/main/java/org/pentaho/di/ui/repository/controllers/RepositoriesController.java?sort=name&dir=ASC&mode=heatmap#L115
\begin{minted}[highlightlines=12-13]{java}
private void createBindings() {
    loginDialog = (XulDialog) document
                    .getElementById( "repository-login-dialog" );
    repositoryEditButton = (XulButton) document
                    .getElementById( "repository-edit" );
    repositoryRemoveButton = (XulButton) document
                    .getElementById( "repository-remove" );
    username = (XulTextbox) document
                    .getElementById( "user-name" );
    userPassword = (XulTextbox) document
                    .getElementById( "user-password" );
    availableRepositories = (XulListbox) document
                    .getElementById( "available-repository-list" );
    showAtStartup = (XulCheckbox) document
                    .getElementById( "show-login-dialog-at-startup" );
    okButton = (XulButton) document
                    .getElementById( "repository-login-dialog_accept" );
    cancelButton = (XulButton) document
                    .getElementById( "repository-login-dialog_cancel" );
    // [...]
}
\end{minted}


\detection{}
To detect this pattern, we need to identify well-known frameworks that use static resource.
Using our methodology,
we have identified the Android API and Mozilla XUL language.

\discussion{}
%
\done{Nate: Could be determined at compile-time.}
%
These casts could be solved by using code generation,
or partial classes like in \csharp{}.
Since the contents of the resource file are known at compile-time,
code generation could be used to generate the corresponding \java{} code.

This is a pattern most often seen when using the Android platform.
The Butter Knife framework%
\footnote{\url{http://jakewharton.github.io/butterknife/}}
make use of annotations to avoid the ``manual'' casting.
Instead, code is generated the cast the result of \code{findViewById}.
%
\done{Luis: Mention frameworks to avoid this cast using annotation.}

\related{}
This pattern is similar to \nameref{pat:LookupById},
since both use a key or ID to look up in a collection and cast the result.
However, the difference is how the content value was generated.
In the \nameref{pat:LookupById} pattern,
the developer ensures in another class the return value,
whereas in the \thisp{} pattern the content is given by a static resource file.
%
\done{Nate: Why not LookupById?}
%

\end{pattern}
\begin{pattern}{ObjectAsArray}
In this pattern an array is used as an untyped object.
A cast is applied to a constant array slot, \eg, \code{(String) array[1]}.

\instances{}
The following example%
\footnote{\url{http://bit.ly/datanucleus_datanucleus-core_2S1L5Zf}}
shows an instance of the \thisp{} pattern.
The variable \code{currentState} contains an \code{Object[]} with a fixed
schema.%
\footnote{\url{http://www.datanucleus.org/javadocs/core/5.0/org/datanucleus/enhancement/Detachable.html}}
A cast is performed of a constant array slot, \code{(BitSet) currentState[3]} on line 5.

%https://lgtm.com/projects/g/datanucleus/datanucleus-core/snapshot/dist-14100061-1524814812150/files/src/main/java/org/datanucleus/state/StateManagerImpl.java?sort=name&dir=ASC&mode=heatmap#L3324
\begin{minted}[highlightlines=5]{java}
        BitSet theLoadedFields = (BitSet)currentState[2];
        for (int i = 0; i < this.loadedFields.length; i++) {
            this.loadedFields[i] = theLoadedFields.get(i);
        }
        BitSet theModifiedFields = (BitSet)currentState[3];
        for (int i = 0; i < dirtyFields.length; i++) {
            dirtyFields[i] = theModifiedFields.get(i);
        }
        setVersion(currentState[1]);
\end{minted}

\discussion{}
This pattern usually suggests an abuse of the type system.
Using an object with statically typed fields might be a better alternative.

\end{pattern}

\begin{pattern}{ReflectiveAccessibility}
This pattern accesses a field of an object by means of reflection.
Typically reflection is used because the field is private and therefore
inaccessible at compile time and the developer cannot change the field
declaration itself.
In this case, the method \code{Field::setAccessible(true)} is invoked on the field
before getting the value of the field.
The cast is needed because \code{Field::get} returns an \code{Object}.

\instances{}
The following two snippets show how this pattern is used:

%https://lgtm.com/projects/g/loopj/android-async-http/snapshot/dist-1879340034-1529316783166/files/library/src/main/java/com/loopj/android/http/AsyncHttpClient.java?sort=name&dir=ASC&mode=heatmap&showExcluded=false#L445
\def\urlvar{http://bit.ly/loopj_android_async_http_2SOISRr}
\begin{minted}[highlightlines=2]{java}
f.setAccessible(true);
HttpEntity wrapped = (HttpEntity) f.get(entity);
#\urlbox#
\end{minted}

%https://lgtm.com/projects/g/joel-costigliola/assertj-db/snapshot/dist-890344-1524814812150/files/src/test/java/org/assertj/db/navigation/ToChange_ChangeOfModification_Integer_Test.java?sort=name&dir=ASC&mode=heatmap#L199
\def\urlvar{http://bit.ly/joel_costigliola_assertj_db_2Ip1Rho}
\begin{minted}[highlightlines=5]{java}
Field fieldPosition=ChangesOutputter.class.getDeclaredField("changesPosition");
fieldPosition.setAccessible(true);
ChangesOutputter changesDisplayBis = output(changes);
PositionWithChanges<ChangesAssert, ChangeAssert> positionBis = 
            (PositionWithChanges) fieldPosition.get(changesDisplayBis);
#\urlbox#
\end{minted}


\detection{}
The Query~\ref{lst:ql:ReflectiveAccessibilityCast} detects the \thisp{} pattern.

\begin{listing}
\begin{minted}{java}	
class ReflectiveAccessibilityCast extends #\qlref{Cast}# { #\qlbox#
	Variable fieldVariable;
    #\qlref{ReflectiveMethodAccess}# reflectiveMethodAccess;
	#\qlref{SetAccessibleTrueMethodAccess}# satma;
	ReflectiveAccessibilityCast() {
		reflectiveMethodAccess = getExprOrDef() and
		fieldVariable.getAnAccess() = 
                getExprOrDef().(MethodAccess).getQualifier().(VarAccess) and
		fieldVariable.getAnAccess() = satma.getQualifier()
	}
}
\end{minted}
\caption{Detection of the \thisp{} pattern.}
\label{lst:ql:ReflectiveAccessibilityCast}
\end{listing}


\issues{}
Using reflection to access a field is a common workaround to tight access
  control restrictions. However, it should generally be regarded as a code
  smell.

As with \nameref{pat:Deserialization}, this pattern is necessary because
a library method can return values of many different types at run time,
and so is declared to return \code{Object}.


\end{pattern}

\begin{pattern}{ExceptionSoftening}
We can throw CheckedExceptions even on methods that don't declare them (via Exception softening).

\instances

\end{pattern}

\section{Limitations}