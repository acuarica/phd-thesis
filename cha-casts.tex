\chapter{Casting about in the Dark}\label{cha:casts}

The main goal of a \emph{static} type system
is to prevent certain kinds of errors from happening at run time.
A type system is formulated as a set of constraints that gives any expression or term in a program a well-defined type.
Any program not satisfying the constraints specified by the type system is
simply rejected by the compiler.

Nevertheless, often a static type system is insufficiently precise.
The type checker is necessarily conservative:
it must not accept invalid programs,
but it may reject programs that are valid but whose validity cannot be ensured at compile time.
However, there are situations when the developer has more information
about the program than can be encoded---or encoded easily---into the types.
To that end, programming languages often provide mechanisms to make the typing constraints less strict,
allowing more valid programs at the expense of more errors at run time.

A common mechanism for relaxing the static typing constraints in object-oriented languages is \emph{casting}.
In programming languages with subtyping%
---or \emph{subtype polymorphism}~\citep{cardelliUnderstandingTypesData1985}---
such as \java{}, \csharp{} or \cpp{},
casting allows an expression to be viewed at a different type than the one at which it was defined.
Casts are checked dynamically, \ie, at run-time, to ensure that the object
being cast is an instance of the desired type.

We aim to understand why developers use casts.
Why is the static type system insufficient,
requiring an escape hatch into dynamic type checking?
Specifically, we attempt to answer the following three research questions:

\begin{enumerate}[label=RQ/C\arabic*:,ref=RQ/C\arabic*,leftmargin=3.4\parindent]
\item\label{casts:rq1}{\bf \crqA} \crqAdesc{}
\item\label{casts:rq2}{\bf \crqB} \crqBdesc{}
\item\label{casts:rq3}{\bf \crqC} \crqCdesc{}
\end{enumerate}

To answer these research questions, we devise
\emph{usage patterns}.
Usage patterns are \emph{recurrent programming idioms} used by developers to solve a specific issue.
Usage patterns enable the categorization of different kinds of cast usages and
thus provide insights into how the language is being used by developers in real-world applications.
Our cast usage patterns can be:
\begin{inparaenum}[(1)]
\item a reference for current and future language designers
to make more informed decisions about programming languages,
\eg{},
the addition of \emph{smart casts} in \lang{Kotlin},\footnote{\url{https://kotlinlang.org/docs/reference/typecasts.html\#smart-casts}}
\item a reference for tool builders, \eg{}, by providing more precise or new
  refactoring or code smell analyses,
\item a guide for researchers to test new language features, \eg{},~\cite{wintherGuardedTypePromotion2011} or to carry out controlled
  experiments about programming, \eg{},~\cite{stuchlikStaticVsDynamic2011}, and
\item a guide for developers for best or better practices.
\end{inparaenum}
To answer our research questions,
we empirically study how casts are used by developers.

\section*{Outline}

Section~\ref{sec:casts:casts} provides an introduction to casts in \java{},
while Section~\ref{sec:casts:issues} illustrates the sort of problems developers have when applying casting conversions.
In Section~\ref{sec:casts:methodology} we introduce the methodology we used to analyze casts and to devise cast usage patterns.
Sections~\ref{sec:casts:overview} and~\ref{sec:casts:patterns} present the cast usage patterns and answers our research questions.
Finally, Section~\ref{sec:casts:discussion} discusses the patterns we found,
while Section~\ref{sec:casts:conclusions} concludes.

\newcommand{\npattern}{0}
\newcommand{\ngroup}{10}
\newcommand{\nprim}{218}
\newcommand{\nbrokenlinks}{180}


\input{chapters/casts/sec-casts-casts}
\section{Issues Developers have Applying the Cast Operator}
\label{sec:casts:issues}

\emph{Do cast operations pose a problem for developers?}
Several studies~\citep{kechagiaUndocumentedUncheckedExceptions2014,coelhoUnveilingExceptionHandling2015,zhitnitskyTop10Exception2016}
suggest that in \java{},
the \code{ClassCastException} is in the top ten of exceptions being thrown when analysing stack traces.
These studies have analysed the exceptions thrown in stack traces.
The exceptions come from third-party libraries and the Android \api{},
indicating a misuse of such \api{}s.
\code{ClassCastException} is in the top 10 of exceptions thrown,
thus it represents a problem for developers.

To illustrate the sort of problems developers have when applying casting conversions,
we performed a search for commits and issues including the term
\code{ClassCastException} within projects using \java{} on \github{},
the largest host of source code in the world~\citep{gousiosLeanGHTorrentGitHub2014}.
Our searches returned about 171K commits%
\footnote{\url{https://github.com/search?l=Java&q=ClassCastException&type=Commits}}
and 73K issues,%
\footnote{\url{https://github.com/search?l=Java&q=ClassCastException&type=Issues}}
respectively, at the time of this writing.
At first glance, these results indicate that \code{ClassCastException} indeed
represents a source for problems for developers.

Typical classes of bugs encountered when using a cast are using the wrong cast target type,
or using the wrong operand, or failing to guard a cast.
We present a few examples we found.
Each example presented here contains the link to the commit in \github{}.
Instead of presenting long \github{} URLs,
we have used the URL shortening service
\href{https://bitly.com/}{\bitly} for easier reading.
Each \bitly{} link was customized to include the project name.

The following snippet shows a cast applied to the variable \code{job} (in line 3) that throws \code{ClassCastException} because the developer forgot to include a guard.
In this case, the developer fixed the error by introducing an \code{instanceof} guard to the cast (lines 1 and 2).

%https://github.com/jenkinsci/extra-columns-plugin/commit/02d10bd1fcbb2e656da9b1b4ec54208b0cc1cbb2
\def\urlvar{http://bit.ly/jenkinsci_extra_columns_plugin_2vviBuc}
\begin{listing}
\begin{minted}[highlightlines=3]{java}
if(!(job instanceof AbstractProject<?, ?>))
  return "";
AbstractProject<?, ?> project = (AbstractProject<?, ?>) job;
#\urlbox#
\end{minted}
\caption{Cast throws \code{ClassCastException} because of a forgotten guard.}
\end{listing}

In the next example the developer made a mistake by choosing a wrong class for the cast target,
\ie, \code{JCustomFileChooser} instead of \code{CustomFile\-Filter} (line 9).
The \code{CustomFileFilter} is an inner static class inside the \code{JCustom\-FileFilter} class.
There is no subclass relationship between these two classes.
The cast happens inside an \code{equals} method---where this idiom is well known---within the \code{CustomFileFilter} class.
But the developer picked the wrong class, the outer class (\code{JCustomFileFilter}), instead of the inner class (\code{CustomFileFilter}).

%https://github.com/GoldenGnu/jeveassets/commit/5f4750bc8cfa7eed8ad01efd8add2cd2cc9bd831
\def\urlvar{http://bit.ly/GoldenGnu_jeveassets_2vsLbMr}
\begin{listing}
\begin{minted}[highlightlines=9]{java}
public final class JCustomFileChooser extends JFileChooser {
  /* [...] */
  public static class CustomFileFilter extends FileFilter {
    /* [...] */
    public boolean equals(Object obj) {
      if (getClass() != obj.getClass()) {
          return false;
      }
      final JCustomFileChooser other = (JCustomFileChooser) obj;
      if (!Objects.equals(this.extensions, other.extensions)) {
          return false;
      }
    }
  }
} #\urlbox#
\end{minted} 
\caption{Cast throws \code{ClassCastException} because of wrong cast target.}
\end{listing}

More subtle, however, is the interaction between casting and generics.
For example, the following call to the \code{getProperty} method (line 1),%
throws a \code{Class\-CastException}.
The method definition is shown in line 3.%
%https://github.com/ethereum/ethereumj/blob/224e65b9b4ddcb46198a6f8faf69edc65d34d382/ethereumj-core/src/main/java/org/ethereum/config/SystemProperties.java\#L312
\footnote{\url{http://bit.ly/ethereum_ethereumj_getProperty_2vwQIBH}}

%https://github.com/ethereum/ethereumj/commit/224e65b9b4ddcb46198a6f8faf69edc65d34d382
\def\urlvar{http://bit.ly/ethereum_ethereumj_2vw4If8}
\begin{listing}
\begin{minted}[highlightlines=1]{java}
config.getProperty("peer.p2p.pingInterval", 5L)

public <T> T getProperty(String propName, T defaultValue) {
    if (!config.hasPath(propName)) return defaultValue;
    String string = config.getString(propName);
    if (string.trim().isEmpty()) return defaultValue;
    return (T) config.getAnyRef(propName);
} #\urlbox#
\end{minted}
\caption{Cast throws \code{ClassCastException} because of generic inference.}
\end{listing}

The first argument to the method is the name of a property,
used to lookup a value in a table.
The second argument is a default value to use if the property is not in the table.
If the lookup is successful,
the method casts the value found to type \code{T}.
In the call, the given property
\code{"peer.p2p.pingInterval"} is in the table and mapped to an \code{Integer}.
However, \java{} uses the type of the \code{defaultValue} argument 
to instantiate the type parameter \code{T}.
In this case, \code{Long}---autoboxed from \code{5L} of type \code{long}---is used as the type parameter \code{T}.

Note, however, that the cast inside \code{getProperty}, which in this context
should cast from \code{Integer} to \code{Long}, \emph{does not fail}.
This is because the \java{} compiler erases the type
parameters like \code{T} and so dynamic type tests are not performed on them.
Instead, the compiler inserts a cast where the return value
of \code{getProperty} is used later with type \code{Long}.
It is this cast that fails at run time and that is reported at run time.

The fix for this bug is to change the default value argument from \code{5L}
to just \code{5}.
This causes the call's return type is inferred to be
\code{Integer}, and the compiler-inserted cast succeeds.

As these examples show, problems with casts are not always obvious.
In this thesis we aim to uncover the many different ways in which developers use casts
by manually analysing a large sample of cast usages in open source software.
\section{Finding Casts Usage Patterns}
\label{sec:casts:methodology}





\textbf{Usage Pattern Detection.}
After all cast instances are found, we analyze this information to discover usage patterns.
\ql{} allows us to automatically categorize cast use cases into patterns.
This methodology is described in section~\ref{sec:casts:methodology}.

Our list of patterns is not exhaustive.
Due to the nature of the cast operator, some casts were uncategorized as they would need a whole program analysis, \eg{}, including libraries in the analysis.






To answer both research questions
\ref{casts:rq2} (\emph{\crqB}) and \ref{casts:rq3} (\emph{\crqC})
we have used the \ql{} query language within the \lgtm{} service to look for cast instances.
%
As mentioned in section \ref{sec:casts:stats}, \ql{} treats primitive conversions as casts.
Thus, a preliminary step is to exclude them as cast instances.
The following \ql{} query shows how to retrieve all relevant cast expressions:

\begin{lstlisting}[style=ql,caption=\ql{} query to retrieve all relevant cast expressions.]
import java
from CastExpr ce where not (
ce.getExpr().getType() instanceof PrimitiveType and
ce.getTypeExpr().getType() instanceof PrimitiveType
) select ce
\end{lstlisting}

\tikzstyle{decision} = [diamond, aspect=2, draw, fill=blue!20, 
    text width=6em, text badly centered, node distance=3cm, inner sep=0pt]
\tikzstyle{block} = [rectangle, draw, fill=blue!20, 
    text width=7em, text centered, rounded corners, minimum height=2em]
\tikzstyle{block2} = [rectangle, draw, fill=blue!20, 
    text width=4.0em, text centered, rounded corners, minimum height=2em]
\tikzstyle{line} = [draw, -latex']
\tikzstyle{cloud} = [draw, ellipse,fill=red!20, node distance=3.1cm,
    minimum height=2.9em]

\begin{figure}
% \begin{wrapfigure}{r}{7.6cm}
\centering
\begin{tikzpicture}[node distance = 1.5cm, auto]
    % Place nodes
    \node [block] (run) {Run Query};
    % \node [cloud, left of=run] (tags) {Tags};
    \node [cloud, right of=run] (patterns) {Patterns};
    \node [block, below of=run] (inspect) {Inspect Casts without Pattern};
    % \node [decision, below of=inspect, node distance=1.6cm] (tag) {New Tag?};
    \node [decision, below of=inspect, node distance=2.0cm] (pattern) {New Pattern?};
    % \node [block2, left of=tag, node distance=3.1cm] (update-tags) {Update Tags};
    \node [block2, right of=pattern, node distance=3.1cm] (update-pattern) {Update Patterns};
    % \node [decision, below of=evaluate] (decide) {is best candidate better?};
    \node [block, below of=pattern, node distance=1.6cm] (stop) {Stop};
    % Draw edges
    \path [line] (run) -- (inspect);
    % \path [line] (inspect) -- (evaluate);
    % \path [line] (inspect) -- (tag);
    \path [line] (inspect) -- (pattern);
    % \path [line] (tag) -- node [near start] {yes} (update-tags);
    \path [line] (pattern) -- node [near start] {yes} (update-pattern);
    % \path [line] (update-tags) -- (tags);
    \path [line] (update-pattern) -- (patterns);
    % \path [line] (tag) -- node {no}(pattern);
    \path [line] (pattern) -- node {no}(stop);
    % \path [line,dashed] (tags) -- (run);
    \path [line,dashed] (patterns) -- (run);
\end{tikzpicture}
\caption{Process to discover cast tags and patterns.} \label{fig:process}
\end{figure}

Figure~\ref{fig:process} depicts our methodology.
We have used this initial result as a starting point for our analysis.
Afterwards, we select a random sample for manual inspection.
We manually inspected the mentioned casts trying to understand
why and how they were used.

By manually inspecting several casts instances,
we observe that certain characteristics appear often, \eg,
a cast in a overridden method, or a cast guarded by an \code{instanceof}.
We then \emph{tag} cast instances based on these observations.
We implement a \ql{} predicate that detects them and proceed
to refine our query with this new tag predicate.
% The table of tags is presented in table~\ref{table:tags}.
After a new tag is added, the query is run again to iterate over the new results.

% DONE: Remove randomly.
% Whenever we observe that those tags do not appear randomly,
Whenever we detect that those tags appear often,
we further inspect the source code to check that is indeed a pattern.
We have formalized the structure of each pattern as a \ql{} predicate based on those tags.
Similarly with tags, after a new pattern is added,
the query is run again to inspect the casts without pattern.
To sum up, our methodology iterates over the results until
no \emph{more} patterns can be detected.
% These patterns are presented in the following section.
The final \ql{} query is available online.%
\footnote{\url{https://gitlab.com/acuarica/java-cast-queries/blob/master/obs.ql}}


% DONE: What about patterns we can't write queries for?
\subsection*{Manual Categorization of Patterns}

Some code patterns might be too difficult to
express in terms of \ql{} queries.
This situation arises when the knowledge to determine
the pattern is outside the source code,
\eg, in configuration files or library call sites.
Thus, in those cases we can only acknowledge that a pattern exists,
but not how recurrent it is.

\subsection{Methodology}

As for the project selection, I have used the lgtm.com project database.
We can argue that this provide a good filter of projects,
since teams that want their code to be analyzed push their projects onto lgtm.com.
This will filter out for instance student projects from github.
There are also popular projects, e.g., gradle, neo4j, google guava,
that probably were pulled in by the Semmle people.
We need to double check with them, but if that’s the case,
we can make a good argument as for the project selection.

There is a total of $7.559$ projects, with a total 10,193,435 casts.
For each cast, I have the path within the project.
But to manually analyze them, I need to get the lgtm.com link.
This is necessary to actually see the code snippet in which the cast appear.
There are 215 projects for which I can’t get the lgtm.com link.
These 215 projects contains 1,162,583 casts.
There are also 516 projects which does not contain any cast.
Therefore the cast population from where make the sampling consists of
9,030,852 casts spread in 6,840 projects.

Now comes the question: What is an appropriate sample size?
Using this online calculator:

https://www.surveysystem.com/sscalc.htm

With standard parameters, Confidence Level=95\% and Confidence Interval=5,
I got a sample size of 384.
This seems sketchy.
My first approach was to increase the sample size arbitrarily,
e.g., 10,000 casts to manually analyze.
This can be too much effort.
But more importantly, how to come up with the patterns taxonomy?
The current list of patterns I have (using QL) does not cover all
existing patterns, i.e.,
when doing manual sample I have discovered new patterns.
After meeting with Gabriele, he suggested using saturation sampling:
0. Start with an empty list of patterns.
1. Perform a manual sample of, let’s say 384 casts.
2. For each new pattern seen, add it to the list of patterns.
3. If a new pattern is detected, go to step 1.


\section{Overview of the Sampled Casts}
\label{sec:casts:overview}

\begin{wraptable}{r}{0.41\textwidth}
\scriptsize
\centering
\caption{Statistics on Sampled Casts}
\label{table:casts:guarded}
\begin{tabular}{|l|r|r|}
  \hline
  All sampled casts & \nSize{} & 100\% \\
  \hline
  Reference casts & \nReference{} & \pReference\% \\
  Primitive casts & \nPrimitivePattern{} & \pPrimitivePattern\% \\
%  Bug & \nBug{} & \pBug\% \\
%  Source code inaccessible & \nBrokenLink{} & \pBrokenLink\% \\
  \hline
  Upcasts & \nUpcast{} & \pUpcast\% \\
  Downcasts & \nDowncast{} & \pDowncast\% \\
  \hline
  Boxing casts & \nToRemoveBoxingSubpattern{} & \pToRemoveBoxingSubpattern\% \\
  Unboxing casts & \nToRemoveUnboxingSubpattern{} & \pToRemoveUnboxingSubpattern\% \\
  \hline
  Guarded by \code{instanceof} & \nTypecaseGuardByInstanceOfSubpattern{} & \pTypecaseGuardByInstanceOfSubpattern\% \\
  Guarded by \code{getClass} & \nTypecaseGuardByClassLiteralSubpattern{} & \pTypecaseGuardByClassLiteralSubpattern\% \\
  Guarded by type tag & \nTypecaseGuardByTypeTagSubpattern{} & \pTypecaseGuardByTypeTagSubpattern\% \\
  Unguarded or possibly unguarded & \nUnguarded{} & \pUnguarded\% \\
  \hline
\end{tabular}
\end{wraptable}

The casts we sampled
are summarized in Table~\ref{table:casts:guarded}.
In our sample of \nSize{} casts,
we found \nPrimitivePattern{} primitive conversions.
Moreover, we found \nBrokenLink{} links that were not accessible during our
analysis, making manual code inspection impossible.
We also found one cast that was clearly a bug, a downcast using the wrong cast
operand.
The remaining \nReference{} casts are either reference upcasts, downcasts, boxing
casts, or unboxing casts.

Casts can be classified as either \emph{guarded} or \emph{unguarded} casts.
A guard is a conditional expression on which the cast is control dependent,
which, if successful, ensures the cast will not fail.
Guards are typically implemented using the \code{instanceof} operator or using
a test of the source value's class (retrieved using the
\code{Object::getClass} method) against a subtype of the cast target type.
Guards can also be implemented in an application-specific manner, for instance
by associating a ``type tag'' with the source value that can be used to
distinguish the run-time type.

Of the \nReference{} analyzed reference casts,
we found that \nGuarded{} (\pGuarded\%) were guarded by a
guard in the same method as the cast and \nUnguarded{} (\pUnguarded\%)
were either unguarded or had a guard in another method.
In the latter case, which we refer to as \emph{possibly unguarded},
determining by manual inspection if a guard is actually
present is often infeasible. The possibly unguarded casts are cases where the application developer
has some reason for believing the cast will succeed, but it is not immediately
apparent in the source code.

As we describe in the next section, nearly all guarded casts fit into just a
few patterns. Unguarded or possibly unguarded casts account for most of the
patterns.

\newcommand{\npattern}{0}
\newcommand{\ngroup}{10}
\newcommand{\nprim}{218}
\newcommand{\nbrokenlinks}{180}


\newcommand{\castpatternsection}[1]{\noindent\textbf{#1.}}
\newcommand{\pname}[1]{\textsc{#1}}
\newcommand{\group}[1]{

\

\

{\noindent\Large Patterns for \textsc{#1}}

\

% \begin{figure}[ht!]
% \centering
% \includegraphics[width=\textwidth]{"analysis/table-patterns-5000-by-group-#1"}
% \caption{#1 Cast Pattern Occurrences}
% \label{fig:group-patterns:#1}
% \end{figure}

}

\newenvironment{pattern}[1]{
	\newcommand{\desc}{\castpatternsection{Description}}
	\newcommand{\instances}{\castpatternsection{Instances}}
	\newcommand{\detection}{\castpatternsection{Detection}}
	\newcommand{\discussion}{\castpatternsection{Discussion}}
	\newcommand{\related}{\castpatternsection{Related Patterns}}
    \newcommand{\thisp}{\textsc{#1}}
    \subsection{\pname{#1}}
    \label{pat:#1}
	\desc
}{}


\section{Cast Usage Patterns}
\label{sec:casts:patterns}

\done{Describe where patterns come from. Sample -> Invent patterns -> repeat.?}
%
Using the methodology described in the above section,
we have devised \nPattern{} cast usage patterns.
We have excluded casts that represent primitive numeric type conversions,
as they do not represent any pattern.
However, during our manual analysis we found nprim{} primitive conversions.
Moreover, we found \nBrokenLink{} links that were not accessible during our analysis.

In this section we present the cast usage patterns we found.
To ease the patterns presentation,
%
\done{Nate: Overview of categories.}
\done{Nate: Explain why each pattern belongs to a certain category.}
\done{Nate: Try to lump together patterns into fewer categories.}
%
we have organized them into ngroup{} categories according to their purpose.
Table~\ref{table:casts:patterns} presents each pattern and the groups they belong to.
Moreover, we are interested in the scope of the cast instance,
\ie, \emph{does it appear in application source code, test code, or generated code?}
%
\done{Matthias: Add a paragraph describing and analyzing what one sees in that figure.}
%
Figure~\ref{fig:patterns} shows our patterns and their occurrences sorted by frequency.
The column on the right corresponds to the group the pattern belongs to.

\input{chapters/casts/table-cast-patterns}

1 & \nameref{pat:Typecase} & \cmark & \xmark & \xmark & \xmark \\
2 & \nameref{pat:LookupById} & \xmark & \cmark & \xmark & \xmark \\
3 & \nameref{pat:Factory} & \xmark & \cmark & \xmark & \xmark \\
4 & \nameref{pat:Family} & \cmark & \xmark & \xmark & \xmark \\
5 & \nameref{pat:UseRawType} & \xmark & \xmark & \cmark & \xmark \\
6 & \nameref{pat:KnownReturnType} & \xmark & \cmark & \cmark & \xmark \\
7 & \nameref{pat:Redundant} & \xmark & \xmark & \cmark & \xmark \\
8 & \nameref{pat:SelectOverload} & \cmark & \xmark & \xmark & \xmark \\
9 & \nameref{pat:Deserialization} & \xmark & \cmark & \xmark & \xmark \\
10 & \nameref{pat:VariableSupertype} & \xmark & \xmark & \cmark & \xmark \\
11 & \nameref{pat:SoleSubclassImplementation} & \cmark & \xmark & \xmark & \xmark \\
12 & \nameref{pat:NewDynamicInstance} & \xmark & \cmark & \xmark & \xmark \\
13 & \nameref{pat:ObjectAsArray} & \xmark & \xmark & \cmark & \xmark \\
14 & \nameref{pat:ImplicitIntersectionType} & \cmark & \xmark & \xmark & \xmark \\
15 & \nameref{pat:CovariantReturnType} & \cmark & \xmark & \xmark & \xmark \\
16 & \nameref{pat:RemoveWildcard} & \cmark & \xmark & \xmark & \cmark \\
17 & \nameref{pat:OperandStack} & \cmark & \xmark & \xmark & \xmark \\
18 & \nameref{pat:ReflectiveAccessibility} & \cmark & \xmark & \xmark & \xmark \\
19 & \nameref{pat:FluentAPI} & \cmark & \xmark & \xmark & \xmark \\
20 & \nameref{pat:CovariantGeneric} & \cmark & \xmark & \xmark & \cmark \\
21 & \nameref{pat:Composite} & \cmark & \xmark & \xmark & \xmark \\
22 & \nameref{pat:GenericArray} & \cmark & \xmark & \xmark & \cmark \\
23 & \nameref{pat:AccessSuperclassField} & \xmark & \xmark & \cmark & \xmark \\
24 & \nameref{pat:UnoccupiedTypeParameter} & \cmark & \xmark & \xmark & \cmark \\


\begin{figure}[ht!]
\centering
\includegraphics[width=\textwidth]{analysis/table-patterns.pdf}
\caption{Cast Pattern Occurrences} \label{fig:patterns}
\end{figure}

\discuss{Matthias: Start a new subsection here.}
\done{Matthias: When describing each pattern, you may want to repeat the number of occurrences within the description {\scriptsize (maybe as part of the section title, or just below)}}
Each pattern is described using the following template:

\begin{itemize}
\item \textbf{Description.}
Tells what the pattern is about.
It gives a general overview of the structure of the pattern.
\item \textbf{Instances.}
Gives one or more concrete examples found in real code.
%
\done{Nate: What's the orange (hightlight color) mean? The line with the cast being inspected. Explain here}
%
Each example contains a highlighted line which shows the cast instance being inspected.
Please notice that the snippets presented here were slightly
modified for formatting purposes.
Moreover, to facilitate some snippet presentations,
we remove irrelevant code and replace it with the
comment \code{// [...]}.
For each instance presented here, we provide the link to the source code repository in \lgtm{}.
We provide the link in case the reader wants to do further inspection
of the presented snippet.
%
\done{Nate: List project name too.}
%
Instead of presenting \lgtm{} long URLs,
we have used the URL shortening service
\href{https://bitly.com/}{\bitly} for easier reading.
%
\done{Luis: Links were customized to include the project names.}
%
Each \bitly{} link was customized to include the project name.
\item \textbf{Detection.}
Describes briefly how this pattern was detected in terms of the tags introduced in the previous section.
\item \textbf{Discussion.}
Presents suggestions, flaws, or comments about the pattern.
\item \textbf{Related Patterns.}
%
\done{Nate: Remove "?"}
%
How the pattern being described relates to other patterns.
\end{itemize}

\group{Language Designers}

\done{Matthias: Say what it means to be guarded.}
The patterns in this category are guarded casts.
A guarded cast is a cast such that before the cast is applied,
some condition --- the \emph{guard} -- needs to be verified.
The condition to be verified guarantees that the cast will not fail at runtime (unless there is a bug in the application), \ie,
the cast will not throw a \code{ClassCastException}.
Some kind of guards ensure that the cast will not fail at the language-level,
while others only can guarantee it at the application-level.

\group{Tool Builders1}

\done{Matthias: In these paragraphs, can you also enumerate the patterns by name, so readers see what's coming?}
\done{Matthias: ???}
Creational patterns are cast instances that are determined by how the value being cast is created.

\group{Developers}

These patterns can be avoidable either directly (by removing them)
or through some minor refactor.
Some of these patterns in this category also are 
These are code smells.

\begin{pattern}{Typecase}
The \thisp{} pattern consists of dispatching to different cases
depending on the run-time type of the source value.
The run-time type is tested against known subtypes of the operand type,
with each test followed by a cast to that type.
The guard may be implemented in one of three ways: an \code{instanceof} operator,
a comparison of the runtime class against a class literal,
or an application-specific type tag.

When implementing the pattern,
care must be taken with complex operands that the value of the operand is
not changed between the guard and the cast, possibly even by another thread.
For instance, in some situations the operand expression is a method invocation.
The value returned by the method should be the same for both the
\code{instanceof} and the cast, thus the method should be a pure method.
Typically, this problem is avoided by using an effectively final local variable in both the guard and the
cast operand.

\instances{}
\thisp{} is by far the most common pattern.
The following listing shows an example of the \thisp{} pattern.%
\footnote{\url{http://bit.ly/PenguinSquad_Enchiridion_2HnNwB7}}

%https://lgtm.com/projects/g/PenguinSquad/Enchiridion/snapshot/dist-19218583-1524814812150/files/build/sources/main/java/joshie/enchiridion/helpers/StackHelper.java?sort=name&dir=ASC&mode=heatmap#L27
\begin{minted}[highlightlines=6]{java}
	if (object instanceof Item) {
		return getStringFromStack(new ItemStack((Item) object));
	} else if (object instanceof Block) {
		return getStringFromStack(new ItemStack((Block) object));
	} else if (object instanceof ItemStack) {
		return getStringFromStack((ItemStack) object);
	} else if (object instanceof String) {
		return (String) object;
	} else if (object instanceof List) {
		return getStringFromStack((ItemStack) ((List) object).get(0));
	} else return "";
\end{minted}

Often there is just one case and the default case, \ie, when the guard
fails, performs a no-op or reports an error.

A particularly common instance of \thisp{} (usually with just one case) is in \code{equals} methods (223
instances out of 1,296, or 17\%).%
\footnote{\url{https://docs.oracle.com/javase/8/docs/api/java/lang/Object.html\#equals-java.lang.Object-}}
Here the pattern is used to check if the argument (of type \code{Object}) has the same type as \code{this}
in order to compare the fields of the argument with the receiver.

Another common scenario%
\footnote{\url{http://bit.ly/codefollower_Tomcat-Research_2SGDUG5}}
is when several cases are used to re-\code{throw} an exception of the right type, as shown below.
The cast instance is applied to a variable of type \code{Throwable}
(line 13).
Nevertheless, the enclosing method is only allowed to throw \code{NamingException} by the \code{throws} declaration (line 3).
Since an exception of type \code{Throwable} is checked,
a cast to \code{VirtualMachineError} (subclass of \code{Error}) is needed.

%https://lgtm.com/projects/g/codefollower/Tomcat-Research/snapshot/dist-13734061-1524814812150/files/java/org/apache/naming/factory/DataSourceLinkFactory.java?sort=name&dir=ASC&mode=heatmap#L85
\begin{minted}[highlightlines=13]{java}
protected Object wrapDataSource(
			Object datasource, String username, String password)
			throws NamingException {
	try {
		// [...]
	}catch (Exception x) {
		if (x instanceof InvocationTargetException) {
			Throwable cause = x.getCause();
			if (cause instanceof ThreadDeath) {
				throw (ThreadDeath) cause;
			}
			if (cause instanceof VirtualMachineError) {
				throw (VirtualMachineError) cause;
			}
			if (cause instanceof Exception) {
				x = (Exception) cause;
			}
		}
		if (x instanceof NamingException) throw (NamingException)x;
		else {
			// [...]
		}
	}
}
\end{minted}

The next example%
\footnote{\url{http://bit.ly/kiegroup_jbpm_2ENCL8a}}
shows that
\thisp{} can also be used to filter elements by type within a stream.
The cast is applied to stream operations (line 1) over the \code{caseAssignments} collection.
The \code{instanceof} is performed in line 1 as well.

%https://lgtm.com/projects/g/kiegroup/jbpm/snapshot/dist-1810050-1524814812150/files/jbpm-case-mgmt/jbpm-case-mgmt-impl/src/main/java/org/jbpm/casemgmt/impl/wih/StartCaseWorkItemHandler.java?sort=name&dir=ASC&mode=heatmap#L212
\begin{minted}[highlightlines=1]{java}
user = (User) caseAssignments.stream().filter(oe -> oe instanceof User)
                                      .findFirst()
                                      .orElseThrow(() -> new IllegalArgumentException());
\end{minted}

Rather than using an \code{instanceof},
in the following example%
\footnote{\url{http://bit.ly/JesusFreke_smali_2Ho8bVL}}
the target type of the parameter \code{reference} is determined by the value
of the parameter \code{referenceType},
which acts as a \emph{type tag} for \code{reference}.

%https://lgtm.com/projects/b/JesusFreke/smali/snapshot/dist-1306230039-1524814812150/files/dexlib2/src/main/java/org/jf/dexlib2/writer/InstructionWriter.java?sort=name&dir=ASC&mode=heatmap#L492
\begin{minted}[highlightlines=4]{java}
switch (referenceType) {
    case ReferenceType.FIELD: return fieldSection.getItemIndex((FieldRefKey) reference);
    case ReferenceType.METHOD: return methodSection.getItemIndex((MethodRefKey) reference);
    case ReferenceType.STRING: return stringSection.getItemIndex((StringRef) reference);
    case ReferenceType.TYPE: return typeSection.getItemIndex((TypeRef) reference);
    case ReferenceType.METHOD_PROTO: return protoSection.getItemIndex((ProtoRefKey) reference);
    default: throw new ExceptionWithContext("Unknown reference type: %d",  referenceType);
}
\end{minted}

In some cases, the target types of the casts are the same in every branch.
In the following snippet,%
\footnote{\url{http://bit.ly/loopj_android-async-http_2IpIULk}}
the cast is applied to the \code{message.obj} field to (line 11),
according to the value of the tag \code{message.what} field (line 1).
However, a similar cast is applied in the first branch (line 3).
In both branches \code{message.obj} is of type \code{Object[]},
  but with different lengths.
  The casts in the calls to \code{onSuccess} and
  \code{onFailure} (lines 5, 13--14) are instances of the
  \nameref{pat:ObjectAsArray} pattern.

%https://lgtm.com/projects/g/loopj/android-async-http/snapshot/dist-1879340034-1549372228293/files/library/src/main/java/com/loopj/android/http/AsyncHttpResponseHandler.java?sort=name&dir=ASC&mode=heatmap#L359
\begin{minted}[highlightlines=9]{java}
switch (message.what) {
    case SUCCESS_MESSAGE:
        response = (Object[]) message.obj;
        if (response != null && response.length >= 3) {
            onSuccess((Integer) response[0], (Header[]) response[1], (byte[]) response[2]);
        } else { ... }
        break;
    case FAILURE_MESSAGE:
        response = (Object[]) message.obj;
        if (response != null && response.length >= 4) {
            onFailure((Integer) response[0], (Header[]) response[1],
                    (byte[]) response[2], (Throwable) response[3]);
        } else { ... }
        break;
    ...
}
\end{minted}

% There is an exploratory document%
% \footnote{\url{http://cr.openjdk.java.net/\~briangoetz/amber/datum.html}}
% by \java{} Language Architect Brian Goetz addressing these issues from a more general perspective.
% It is definitely a starting point towards improving the \java{} language.

\discussion{}
Having only a single case---that is, a single guard and cast---is common.
In the 742 instances of \thisp{} that used \code{instanceof}, 511
(69\%) had only one case.

The \thisp{} pattern can be seen as an \adhoc{} alternative to a
\code{typecase} or pattern matching~\citep{milnerProposalStandardML1984} as a
language construct.
In \kotlin{}, flow-sensitive typing is used so that immutable values can be
used at a subtype when a type guard on the value is successful.%
\footnote{\url{https://kotlinlang.org/docs/reference/typecasts.html\#smart-casts}}
This feature eliminates much of the need for the guarded casts.
Pattern matching can be seen in several other languages, \eg, \ml{}, \scala{}, \csharp{}, and \haskell{}.
For instance, in \scala{} the pattern matching construct is achieved using the \code{match} keyword.
In this example,%
\footnote{Adapted from \url{https://docs.scala-lang.org/tour/pattern-matching.html}}
a different action is taken according to the runtime type of the parameter \code{notification} (line 9).

\begin{minted}[highlightlines=10]{scala}
abstract class Notification
case class Email(sender: String, title: String, body: String)
	extends Notification
case class SMS(caller: String, message: String)
	extends Notification
case class VoiceRecording(contactName: String, link: String)
	extends Notification

def showNotification(notification: Notification): String = {
	notification match {
		case Email(email, title, _) =>
		s"You got an email from $email with title: $title"
		case SMS(number, message) =>
		s"You got an SMS from $number! Message: $message"
		case VoiceRecording(name, link) =>
		s"Voice Recording from $name! Click the link: $link"
	}
	}
	val someSms = SMS("12345", "Are you there?")
	val someVoiceRecording = VoiceRecording("Tom", "voicerecording.org/id/123")
	
	// prints You got an SMS from 12345! Message: Are you there?
	println(showNotification(someSms))
	
	// Voice Recording from Tom! Click the link: voicerecording.org/id/123	
	println(showNotification(someVoiceRecording))
\end{minted}

Alternatives to the \thisp{} pattern would be to use the visitor pattern or to
use virtual dispatch on the match scrutinee.
However, both of these
alternatives might be difficult to implement when the scrutinee is defined in
a library or in third-party code.
There is an ongoing proposal%
\footnote{\url{http://openjdk.java.net/jeps/305}}$^{,}$%
\footnote{\url{https://cr.openjdk.java.net/~briangoetz/amber/pattern-match.html}}
to add pattern matching to the \java{} language.
The proposal explores changing the \code{instanceof} operator in order to support pattern matching.
%
\done{Java 12 will (maybe) add matching.}
%
\java{} 12 extends the \code{switch} statement to be used as either a statement or an expression.%
\footnote{\url{https://openjdk.java.net/jeps/325}}
This enhancement aims to ease the transition to a \code{switch} expression that supports pattern matching.

The pattern for an \code{equals} method implementation is well-known.
Most \code{equals} methods in our sample are implemented with the same
boilerplate structure:
that is, first checking if the parameter is another reference to \code{this},
then checking if the argument is not null,
and finally, checking if the argument is of the right class
(with either an \code{instanceof} test or a \code{getClass} comparison).
Once all checks are performed, a cast follows, and a field-by-field comparison is made.

To avoid this boilerplate, other languages bake in deep equality comparisons, at least for some types
(\eg, \scala{} case classes),
or provide mechanisms to generate the boilerplate code (\eg, \code{deriving Eq}
in \haskell{} or \code{\#[derive(Eq)]} in \rust{}).
\cite{vaziriDeclarativeObjectIdentity2007} propose a declarative approach to avoid boilerplate code when implementing
both the \code{equals} and \code{hashCode} methods.

\end{pattern}

\begin{pattern}{LookupById}
This pattern is used to extract values from a heterogenous container.
It looks up an object by a compile-time constant identifier, tag, or name and casts the result to an appropriate type.
It accesses a collection that holds values of different types
(usually implemented as \code{Collection<Object>} or as \code{Map<K, Object>}).
The actual run-time type returned from the lookup is determined by the value of the identifier.

\instances{}
In the example shown below,%
\footnote{\url{http://bit.ly/loopj_android-async-http_2SUzY4E}}
the \code{getAttribute} method returns \code{Object}.
The variable \texttt{context} is of type \code{BasicHttpContext},
which is implemented with \code{HashMap}.

%https://lgtm.com/projects/g/loopj/android-async-http/snapshot/dist-1879340034-1518514025554/files/library/src/main/java/com/loopj/android/http/AsyncHttpClient.java?sort=name&dir=ASC&mode=heatmap&excluded=false#L258
\begin{minted}[highlightlines=1,linenos=false]{java}
AuthState authState = (AuthState) context.getAttribute(ClientContext.TARGET_AUTH_STATE);
\end{minted}

The next snippet%
\footnote{\url{http://bit.ly/skerit_cmusphinx_2HGgL1D}}
shows a call site to the \code{getComponent} method cast to the \code{ActiveListManager} class (line 15).
The \code{getComponent} method in this cast instance uses as argument the \code{PROP\_ACTIVE\_LIST\_MANAGER} constant.
Looking at the definition of this constant (line 3),
we can see there is a companion attribute (\code{@S4Component}) whose argument is the \code{ActiveListManager} class, the target of the cast instance.

% https://lgtm.com/projects/g/skerit/cmusphinx/snapshot/dist-1506204046584-1524814812150/files/sphinx4/src/sphinx4/edu/cmu/sphinx/decoder/search/WordPruningBreadthFirstSearchManager.java?sort=name&dir=ASC&mode=heatmap#L207
\begin{minted}[highlightlines=15]{java}
/** The property that defines the type of active list to use */
@S4Component(type = ActiveListManager.class)
public final static String PROP_ACTIVE_LIST_MANAGER = "activeListManager";

@Override
public void newProperties(PropertySheet ps) throws PropertyException {
    super.newProperties(ps);
    logMath = (LogMath) ps.getComponent(PROP_LOG_MATH);
    logger = ps.getLogger();
    linguist = (Linguist) ps.getComponent(PROP_LINGUIST);
    pruner = (Pruner) ps.getComponent(PROP_PRUNER);
    scorer = (AcousticScorer) ps.getComponent(PROP_SCORER);
    activeListManager = 
            (ActiveListManager) ps.getComponent(PROP_ACTIVE_LIST_MANAGER);
    // [...]
}
\end{minted}

A common version of this pattern is to retrieve a value instantiated from
static resource file, \eg,
an XML, HTML or \java{} properties file.
The file contents are (in theory) known at compile-time and the file is included in the binary distribution
of the application. These files are often built using tools such as GUI
builders.

In the following example from an Android application,%
\footnote{\url{http://bit.ly/pwittchen_NetworkEvents_2HGbrMq}}
a cast is applied to the \code{findViewById} method invocation.
View classes are instantiated by the application framework using an XML resource file.
The \code{findViewById} method looks up the view by its ID.

%https://lgtm.com/projects/g/pwittchen/NetworkEvents/snapshot/dist-2032650416-1524814812150/files/example/src/main/java/com/github/pwittchen/networkevents/app/MainActivity.java?sort=name&dir=ASC&mode=heatmap#L65
\begin{minted}[highlightlines=1,linenos=false]{java}
mobileNetworkType = (TextView) findViewById(R.id.mobile_network_type);
\end{minted}

\discussion{}
This pattern suggests a heterogeneous dictionary.
In our manual inspection,
all dictionary keys and the resulting types are known at
compile time, however
a cast is needed because the dictionary type does not encode the
relationship between key values and the result type.
Casts in this pattern are typically not guarded indicating that the programmer
knows the source of the cast based on the value of the key.

This pattern is often seen in Android applications.
The Butter Knife framework%
\footnote{\url{http://jakewharton.github.io/butterknife/}}
uses annotations to avoid the ``manual'' casting.
Instead, code is generated that casts the result of \code{findViewById} to the
appropriate type.


But in any case a cast is needed given the inexpressiveness of the type system.
As a complementary analysis,
it would be interesting to check whether all call sites to
\code{getAttribute} receives a constant (\code{final static} field).

Notice that this pattern is not guarded by an \code{instanceof}.
However, the cast involved does not fail at runtime.
This means that the source of the cast is known to the programmer.
This raises the following questions:
\begin{itemize}
\item \emph{What kind of analysis is needed to detect the source of the cast?}
\item \emph{Is worth to have it?}
\item \emph{Is better to change API?}
\item \emph{How other --- statically typed --- languages support this kind of idiom?}
\item \emph{Could generative programming a.k.a. templates solve this problem?}
\end{itemize}

This pattern retrieves a stashed a value from an heterogeneous collection (or dictionary).
A cast is needed because the return type depends on the ID to be retrieved.
This scenario is similar to the \nameref{pat:Tag} pattern,
where usually the developer retrieves a stashed value from a superclass field.
Since this pattern casts a value to a known type from a method invocation,
it can be seen as a kind of \nameref{pat:KnownReturnType} pattern.

\end{pattern}
\begin{pattern}{Factory}
Creates an object based on some arguments either to the method call or constructor.
Since the arguments are known at compile-time, cast to the specific type.

Cast factory method result to subtype (special case of family polymorphism).
Usually Logger.getLogger.

The method is declared to return URLConnection but can return a more specific type based on the URL string.
Cast to that.
We should generalize this pattern.

\instances{}

\footnote{\url{http://bit.ly/2HvRlUX}}

\footnote{\url{https://docs.oracle.com/javase/8/docs/api/java/security/KeyPair.html\#getPrivate()}}

%https://lgtm.com/projects/b/connect2id/oauth-2.0-sdk-with-openid-connect-extensions/snapshot/dist-1311020143-1524814812150/files/src/test/java/com/nimbusds/oauth2/sdk/jose/jwk/RemoteJWKSetTest.java?sort=name&dir=ASC&mode=heatmap#L242
\begin{minted}[highlightlines=10]{java}
KeyPairGenerator pairGen = KeyPairGenerator.getInstance("RSA");
pairGen.initialize(1024);
KeyPair keyPair = pairGen.generateKeyPair();
RSAKey rsaJWK1 = new RSAKey.Builder((RSAPublicKey) keyPair.getPublic())
        .privateKey((RSAPrivateKey) keyPair.getPrivate())
        .keyID("1")
        .build();
keyPair = pairGen.generateKeyPair();
RSAKey rsaJWK2 = new RSAKey.Builder((RSAPublicKey) keyPair.getPublic())
        .privateKey((RSAPrivateKey) keyPair.getPrivate())
        .keyID("2")
        .build();
\end{minted}

\footnote{\url{http://bit.ly/2E6KY6T}}

\footnote{\url{https://docs.oracle.com/javase/8/docs/api/java/net/URL.html\#openConnection--}}

%https://lgtm.com/projects/g/apache/hadoop/snapshot/dist-956730001-1524814812150/files/hadoop-yarn-project/hadoop-yarn/hadoop-yarn-server/hadoop-yarn-server-resourcemanager/src/test/java/org/apache/hadoop/yarn/server/resourcemanager/webapp/TestRMWebServicesHttpStaticUserPermissions.java?sort=name&dir=ASC&mode=heatmap#L138
\begin{minted}[highlightlines=2]{java}
URL url = new URL("http://localhost:8088/ws/v1/cluster/apps");
HttpURLConnection conn = (HttpURLConnection) url.openConnection();
\end{minted}

\detection{}

\discussion{}

\related{}

\end{pattern}
\begin{pattern}{Family}
Family polymorphism.

\instances{}

\detection{}

\discussion{}
\cite{ernstFamilyPolymorphism2001}

\related{}

\end{pattern}
\begin{pattern}{UseRawType}
A cast is in the \thisp{} pattern when a \emph{raw type} is used rather than a generic type.
Methods of raw types typically return \code{Object} rather than a more specific type.

\instances{}
For example, in the following code,
the collection \code{c} and iterator \c{it} are declared to be of the raw types \code{Collection} and \code{Iterator} rather than as parameterized types.
The call to \code{next} on line 4 must be cast to a more specific type because static type information was lost by the use of raw types.

%https://lgtm.com/projects/g/bcgit/bc-java/snapshot/dist-20740003-1524814812150/files/pkix/src/test/java/org/bouncycastle/cms/test/Rfc4134Test.java?sort=name&dir=ASC&mode=heatmap#L268
\def\urlvar{http://bit.ly/bcgit_bc_java_2SD2HLm}
\begin{minted}[highlightlines=4]{java}
Collection c = recipients.getRecipients();
assertTrue(c.size() >= 1 && c.size() <= 2);
Iterator it = c.iterator();
verifyRecipient((RecipientInformation)it.next(), privKey);
#\urlbox#
\end{minted}

The following example uses the \code{Comparable} interface (line 1).
This interface is generic,%
\footnote{\url{https://docs.oracle.com/javase/8/docs/api/java/lang/Comparable.html}}
but in this case the developer is using its raw type.
Therefore a cast is needed in line 5.

%https://lgtm.com/projects/g/fangjie008/tiexue-mcp-parent/snapshot/dist-1505957596672-1524814812150/files/mcp-core/src/main/java/com/tiexue/mcp/core/dto/McpSettlementDetailDto.java?sort=name&dir=ASC&mode=heatmap#L100
\def\urlvar{http://bit.ly/fangjie008_tiexue_mcp_parent_2FSZKzm}
\begin{minted}[highlightlines=5]{java}
public class McpSettlementDetailDto implements Comparable {
    // [...]
    @Override
    public int compareTo(Object o){
        McpSettlementDetailDto mcpSettlementDetailDto=(McpSettlementDetailDto)o;
        Integer newConsume=(int)mcpSettlementDetailDto.getConsume();
        Integer temp=((int)this.consume);
        return temp.compareTo(newConsume);
    }
} #\urlbox#
\end{minted}


In the following snippet,
a cast is applied to the result of the \code{doPrivileged} method in lines 3 and 4.
This method takes a \code{PrivilegedAction<T>},
but the cast is needed because it is invoked with a raw type, \eg, \code{new PrivilegedAction()}.
Inspecting further the source code application,
we found that it might be a requirement to be compatible with the JDK 1.2.
Generics were added to \java{} 5.
Thus, this cast might be still necessary.

%https://lgtm.com/projects/g/robovm/robovm/snapshot/dist-39650108-1524814812150/files/rt/external/apache-xml/src/main/java/org/apache/xml/dtm/SecuritySupport12.java#L59
\def\urlvar{http://bit.ly/robovm_robovm_2FAI5x5}
\begin{minted}[highlightlines=3-4]{java}
class SecuritySupport12 extends SecuritySupport {
    ClassLoader getSystemClassLoader() {
        return (ClassLoader)
            AccessController.doPrivileged(new PrivilegedAction() {
                public Object run() {
                    ClassLoader cl = null;
                    try {
                        cl = ClassLoader.getSystemClassLoader();
                    } catch (SecurityException ex) {}
                    return cl;
                }
            });
    }
}
public final class AccessController {
    public static <T> T doPrivileged(PrivilegedAction<T> action) {
        return action.run();
    }
}
public interface PrivilegedAction<T> {
    public T run();
} #\urlbox#
\end{minted}


\issues{}
Raw types exist in \java{} to support legacy code.
Best practice would be to rewrite the code to use generics,
but this is not always feasible or cost effective.

Casts among generic types and between raw types and generic types are unchecked at run time,
although other casts are typically inserted by the compiler to ensure type safety dynamically.
When these inserted casts fail, the reported location of the failure may not match the programmer's expectation.
Indeed, this is similar to the problem of \emph{blame} in gradually typed languages~\citep{wadlerWellTypedProgramsCan2009}.
In this setting, when a run-time cast fails the blame should be put on the appropriate programmer-inserted cast,
not on a compiler-inserted cast.

\end{pattern}

\begin{pattern}{KnownReturnType}
There are cases when a method's return type is less specific than the
actual return type value.
This is often to hide implementation details, but may also be because the
  method overrides another method with a less-specific type
  and the return type is not changed covariantly.

This pattern is used to cast from the method's return type to
the \emph{known} actual return type.
This pattern is characterized by a method that always returns a
value of the same type, a subtype of the declared return type,
regardless of the context or the arguments to the method call.

\instances{}
\def\urlvar{http://bit.ly/apache_kylin_2SIjooO}


%https://lgtm.com/projects/g/apache/kylin/snapshot/dist-45150010-1524814812150/files/storage-hbase/src/main/java/org/apache/kylin/storage/hbase/steps/HBaseMRSteps.java?sort=name&dir=ASC&mode=heatmap#L211
\begin{minted}[highlightlines=1]{java}
final List<CubeSegment> mergingSegments = ((CubeInstance) seg.getRealization())
            .getMergingSegments((CubeSegment) seg);

public class CubeSegment implements IBuildable, ISegment, Serializable {
    // [...]
    private CubeInstance cubeInstance;
    // [...]
    public IRealization getRealization() {
        return cubeInstance;
    }
}

public class CubeInstance
            extends RootPersistentEntity implements IRealization, IBuildable {
    // [...]
}
\end{minted}


\def\urlvar{http://bit.ly/eclipse_pdt_2Ekeu9v}

%https://lgtm.com/projects/g/eclipse/pdt/snapshot/dist-19313119-1524814812150/files/plugins/org.eclipse.php.debug.core/src/org/eclipse/php/internal/debug/core/zend/communication/DebugConnection.java?sort=name&dir=ASC&mode=heatmap#L795
\begin{minted}[highlightlines=1]{java}
debugTarget = (PHPDebugTarget) createDebugTarget(this, launch, URL,
        requestPort, process, runWithDebug, stopAtFirstLine, project);

protected IDebugTarget createDebugTarget(DebugConnection thread,
        ILaunch launch, String url, int requestPort, PHPProcess process,
        boolean runWithDebug, boolean stopAtFirstLine, IProject project)
        throws CoreException {
    return new PHPDebugTarget(thread, launch, url, requestPort, process,
            runWithDebug, stopAtFirstLine, project);
}

public class PHPDebugTarget extends PHPDebugElement
        implements IPHPDebugTarget, IBreakpointManagerListener, IStepFilters {
}
\end{minted}

\def\urlvar{http://bit.ly/apache_activemq_2EnSivc}

%https://lgtm.com/projects/g/apache/activemq/snapshot/dist-11730660-1524814812150/files/activemq-client/src/main/java/org/apache/activemq/ActiveMQConnectionFactory.java?sort=name&dir=ASC&mode=heatmap#L235
\begin{minted}[highlightlines=1]{java}
throw (IllegalArgumentException)
        new IllegalArgumentException("Invalid broker URI: " + brokerURL)
        .initCause(e);
\end{minted}


TODO: Also Logger

Logger here
\def\urlvar{http://bit.ly/skylot_jadx_2HIoR9X}

%https://lgtm.com/projects/g/skylot/jadx/snapshot/dist-41240110-1524814812150/files/jadx-gui/src/main/java/jadx/gui/utils/LogCollector.java?sort=name&dir=ASC&mode=heatmap#L22
\begin{minted}[highlightlines=1]{java}
Logger rootLogger = (Logger) LoggerFactory.getLogger(Logger.ROOT_LOGGER_NAME);
\end{minted}


\discussion{}
This pattern usually indicates an abstraction violation: the caller 
needs to know the method implementation to know the correct target type.

  \nameref{pat:CovariantReturnType} can be considered a special case of this
  pattern where the return type is known to vary with the receiver type.
  Like that pattern, associated types~\cite{chakravartyAssociatedTypeSynonyms2005}
  in languages like \haskell{} or \rust{} 
  could be used to avoid the cast.

\end{pattern}




\begin{pattern}{Redundant}
A cast that is not necessary for compilation.

\instances{}
The following example%
\footnote{\url{http://bit.ly/2FWXw2e}}

%https://lgtm.com/projects/g/vladmihalcea/high-performance-java-persistence/snapshot/dist-1813180502-1524814812150/files/core/src/test/java/com/vladmihalcea/book/hpjp/hibernate/schema/flyway/FlywayTest.java#L40
\begin{minted}[highlightlines=1]{java}
transactionTemplate.execute((TransactionCallback<Void>) transactionStatus -> {
    Post post = new Post();
    entityManager.persist(post);
    return null;
});
\end{minted}

\detection{}

\discussion{}

\related{}
    
\end{pattern}
\begin{pattern}{SelectOverload}
This pattern is used to select the appropriate version of an overloaded method%
\footnote{Using ad-hoc polymorphism~\citep{stracheyFundamentalConceptsProgramming2000}.}
where two or more of its implementations differ \emph{only} in some argument type.

A cast of the \code{null} literal is often used to resolve method overloading ambiguity because the type of \code{null} is a subtype of any reference type.%
\footnote{\url{https://docs.oracle.com/javase/specs/jls/se8/html/jls-4.html\#jls-4.1}}

A cast to \code{null} is often used to select against different versions of a method,
\ie{}, to resolve method overloading ambiguity.
Whenever a \code{null} value needs to be an argument of an a cast is
needed to select the appropriate implementation.
This is because the type of \code{null} has the special type \emph{null}%
\footnote{\url{https://docs.oracle.com/javase/specs/jls/se8/html/jls-4.html\#jls-4.1}}
which can be treated as any reference type.
In this case,
the compiler cannot determine which method implementation to select.

Another use case is to select the appropriate the right argument when calling a method with variable arguments.

\instances{}
The following listing shows an example of the \thisp{} pattern.
In this example, there are three versions of the \code{onSuccess} method,
The cast \code{(String) null} is used to select the appropriate version
(line 7), based on the third parameter.
Overloaded methods that differ only in their argument type (the third one).

% https://lgtm.com/projects/g/loopj/android-async-http/snapshot/dist-1879340034-1518514025554/files/library/src/main/java/com/loopj/android/http/JsonHttpResponseHandler.java?sort=name\&dir=ASC\&mode=heatmap\&excluded=false#L150
\def\urlvar{http://bit.ly/loopj_android_async_http_2FENovD}
\begin{minted}[highlightlines=1]{java}
onSuccess(statusCode, headers, (String) null);
public void onSuccess(
      int statusCode, Header[] headers, JSONObject response) { /* [...] */ }
public void onSuccess(
      int statusCode, Header[] headers, JSONArray response) { /* [...] */ }
public void onSuccess(
      int statusCode, Header[] headers, String responseString) { /* [...] */ }
#\urlbox#
\end{minted}

In the following example \code{actual.data()} returns a boxed \code{Long}.
Because implicit upcasts have precedence over implicit unboxing conversions,
the call is needed to invoke the method that takes a \code{long} (line 3) rather than the method that takes an \code{Object} (line 2).

%https://lgtm.com/projects/g/spullara/redis-protocol/snapshot/dist-41940059-1524814812150/files/client/src/test/java/redis/client/AllCommandsTest.java?sort=name&dir=ASC&mode=heatmap#L366
\def\urlvar{http://bit.ly/spullara_redis_protocol_2FC9Llb}
\begin{minted}[highlightlines=1]{java}
assertEquals(expected, (long) actual.data());
public static void assertEquals(Object expected, Object actual) { /* [...] */ }
public static void assertEquals(long expected, long actual) { /* [...] */ }
#\urlbox#
\end{minted}

The following snippet is similar to the previous example,
but notice how that the cast is applied to a
primitive---\emph{non-reference}---type.

%https://lgtm.com/projects/g/apache/poi/snapshot/dist-1790760597-1524814812150/files/src/testcases/org/apache/poi/hssf/record/chart/TestLegendRecord.java#L50
\def\urlvar{http://bit.ly/apache_poi_2StrlOn}
\begin{minted}[highlightlines=1,linenos=false]{java}
assertEquals((byte) 0x1, record.getSpacing());
#\urlbox#
\end{minted}

In the last example of \thisp,
an upcast of a generic type is performed to select the appropriate overload of the \code{max} method.

%https://lgtm.com/projects/g/groovy/groovy-core/snapshot/dist-45390050-1524814812150/files/src/main/org/codehaus/groovy/runtime/DefaultGroovyMethods.java?sort=name&dir=ASC&mode=heatmap#L6715
\def\urlvar{http://bit.ly/groovy_groovy_core_2HDAkbF}
\begin{minted}[highlightlines=2]{java}
public static <T> T max(Iterator<T> self, Comparator<T> comparator) {
      return max((Iterable<T>)toList(self), comparator);
}
public static <T> List<T> toList(Iterator<T> self) {
      // [...]
}
@Deprecated
public static <T> T max(Collection<T> self, Comparator<T> comparator) {
      // [...]
}
public static <T> T max(Iterable<T> self, Comparator<T> comparator) {
      // [...]
} #\urlbox#
\end{minted}


\detection{}
The Query~\ref{lst:ql:SelectOverloadCast} detects when a cast is used as an argument of an overloaded method.
A cast returned by this query needs to be either a cast to \code{null} or an upcast.
This is an approximation because the query does not check whether the overloaded method differs only on the type of the argument that is cast.

\begin{listing}
\begin{minted}{java}
class SelectOverloadCast extends #\qlref{Cast}# { #\qlbox#
	SelectOverloadCast() {
		(getExpr() instanceof NullLiteral or this instanceof #\qlref{Upcast}#) and
		this instanceof #\qlref{OverloadedArgument}#
	}
	Callable getOverload() {
		result = this.(#\qlref{OverloadedArgument}#).getAnOverload()
	}
}
\end{minted}
\caption{Query to detect the \thisp{} pattern.}
\label{lst:ql:SelectOverloadCast}
\end{listing}

\issues{}
Casting the \code{null} constant seems rather artificial.
This pattern shows either a lack of expressiveness in \java{} or a bad \api{} design.
Passing \code{null} to a method might better be handled by using overloading with fewer parameters or by using default parameters.
Several other languages support default parameters,
\eg, \scala{}, \csharp{} and \cpp{}.
Adding default parameters might be a partial solution.

In addition, a pure object-oriented language would not distinguish between primitives and objects,
avoiding the need for autoboxing to be visible at the type level.

\cite{oostvogelsStaticTypingComplex2018a} propose an extension to \typescript{} to express constraints between properties,
which can then be mapped onto optional parameters.

Both the \nameref{pat:AccessSuperclassField} and this pattern are used to select class members.
While this pattern is used to select the appropriate overloaded method,
the \nameref{pat:AccessSuperclassField} is used to select a field in a superclass.

\end{pattern}

\begin{pattern}{Deserialization}
This pattern is used to deserialize an object at run-time.

\instances{}
The following example%
\footnote{\url{http://bit.ly/internetarchive_heritrix3_2SF4j7k}}
shows how the \thisp{} pattern is used to create objects from a file system (line 19).

%https://lgtm.com/projects/g/internetarchive/heritrix3/snapshot/dist-12140105-1524814812150/files/engine/src/test/java/org/archive/crawler/datamodel/CrawlURITest.java?sort=name&dir=ASC&mode=heatmap#L83
\begin{minted}[highlightlines=19]{java}
final public void testSerialization()
        throws IOException, ClassNotFoundException {
    File serialize = new File(getTmpDir(),
            this.getClass().getName() + ".serialize");
    try {
        FileOutputStream fos = new FileOutputStream(serialize);
        ObjectOutputStream oos = new ObjectOutputStream(fos);
        oos.writeObject(this.seed);
        oos.reset();
        oos.writeObject(this.seed);
        oos.reset();
        oos.writeObject(this.seed);
        oos.close();
        // Read in the object.
        FileInputStream fis = new FileInputStream(serialize);
        ObjectInputStream ois = new ObjectInputStream(fis);
        CrawlURI deserializedCuri = (CrawlURI)ois.readObject();
        deserializedCuri = (CrawlURI)ois.readObject();
        deserializedCuri = (CrawlURI)ois.readObject();
        assertEquals("Deserialized not equal to original",
                this.seed.toString(), deserializedCuri.toString());
        String host = this.seed.getUURI().getHost();
        assertTrue("Deserialized host not null",
                host != null && host.length() >= 0);
    } finally {
        serialize.delete();
    }
}
\end{minted}

\detection{}
This pattern is characterized for a cast to the \code{readObject} method on a \code{ObjectInputStream} object.

\discussion{}
From a language design perspective,
the \thisp{} pattern is one of the most difficult patterns to avoid.
It is difficult to avoid because a compiler cannot verify at compile-time that a certain byte stream can be deserialized into an object of a given type.

\related{}
Both this pattern and the \nameref{pat:NewDynamicInstance} pattern create objects by using reflection.
\nameref{pat:StaticResource}

\end{pattern}
\begin{pattern}{VariableSupertype}
This pattern occurs when a cast is applied to a variable (local variable,
parameter, or field),
that has usually been assigned just once and
is declared with a proper supertype of the value assigned into it.
The type of the value being assigned to can be determined locally
either within the enclosing method or class.

To detect this pattern, a cast needs to be applied to a variable whose
value can be determined simply by looking at
the enclosing method or class.


\instances{}
The following snippet%
\def\urlvar{http://bit.ly/apache_cxf_2SNoUXj}
shows an example of the \thisp{} pattern (line 4).
The \code{samlTokenRenewer} variable is being cast to the \code{SAMLTokenRenewer} class.
The variable is declared with type \code{TokenRenewer} (superclass of \code{SAMLTokenRenewer}) in line 1.
However, the variable is being initialized with the expression \code{new SAMLTokenRenewer()}. 
Thus, the cast instance could be trivially avoided by changing the declaration of the \code{samlTokenRenewer} variable to \code{SAMLTokenRenewer} instead of \code{TokenRenewer}.

%https://lgtm.com/projects/g/apache/cxf/snapshot/dist-2126650689-1524814812150/files/services/sts/sts-core/src/test/java/org/apache/cxf/sts/token/renewer/SAMLTokenRenewerTest.java?sort=name&dir=ASC&mode=heatmap#L465
\begin{minted}[highlightlines=4]{java}
TokenRenewer samlTokenRenewer = new SAMLTokenRenewer();
samlTokenRenewer.setVerifyProofOfPossession(false);
samlTokenRenewer.setAllowRenewalAfterExpiry(true);
((SAMLTokenRenewer)samlTokenRenewer).setMaxExpiry(1L);
#\urlbox#
\end{minted}

The following listing%
\def\urlvar{http://bit.ly/facebookarchive_hadoop_20_2FuDeO7}
shows an example of the \thisp{} pattern.
We can see that the field \code{uncompressedDirectBuf} is being cast to the \code{java.nio.ByteBuffer} class (line $13$) but it is declared as \code{java.nio.Buffer} (line $3$).
Nevertheless, the field is assigned only once in the constructor (line $7$)
with a value of type \code{java.nio.ByteBuffer}.
The value assigned is returned by the method
\code{ByteBuffer.allocateDirect}.%
\footnote{\url{https://docs.oracle.com/javase/7/docs/api/java/nio/ByteBuffer.html\#allocateDirect(int)}}
Inspecting the enclosing class, there is no other assignment to the
\code{uncompressedDirectBuf} field,
thus making possible to declare it as \code{final}.
Therefore, the cast pattern in line $13$ will always succeed.
Any other similar use of the \code{uncompressedDirectBuf} field needs to be cast as well.

% https://lgtm.com/projects/g/facebookarchive/hadoop-20/snapshot/dist-1802091768-1524814812150/files/src/core/org/apache/hadoop/io/compress/snappy/SnappyCompressor.java?sort=name&dir=ASC&mode=heatmap#L134
\begin{minted}[highlightlines=13]{java}
public class SnappyCompressor implements Compressor {
    // [...]
    private Buffer uncompressedDirectBuf = null;
    // [...]
    public SnappyCompressor(int directBufferSize) {
        // [...]
        uncompressedDirectBuf = ByteBuffer.allocateDirect(directBufferSize);
        // [...]
    }
    // [...]
    synchronized void setInputFromSavedData() {
        // [...]
        ((ByteBuffer) uncompressedDirectBuf).put(userBuf, userBufOff,
            uncompressedDirectBufLen);
        // [...]
    }
    // [...]
} #\urlbox#
\end{minted}

In the next cast instance,%
\def\urlvar{http://bit.ly/oblac_jodd_2UKxm6H}
the parameter \code{k1} is cast to the \code{Comparable} class (line 7).
\code{k1} is declared as \code{E} (line 5), an unbounded type parameter (line 1).
The developer likely designed the class so that
\code{E} must be \code{Comparable} only if \code{comparator} is \code{null},
providing an API with two ways to compare list elements.

%https://lgtm.com/projects/g/oblac/jodd/snapshot/dist-12050004-1524814812150/files/jodd-core/src/main/java/jodd/util/collection/SortedArrayList.java?sort=name&dir=ASC&mode=heatmap#L154
\begin{minted}[highlightlines=7]{java}
public class SortedArrayList<E> extends ArrayList<E> {
    protected final Comparator<E> comparator;
    // [...]
    @SuppressWarnings( {"unchecked"})
    protected int compare(final E k1, final E k2) {
            if (comparator == null) {
                    return ((Comparable) k1).compareTo(k2);
            }
            return comparator.compare(k1, k2);
    }
} #\urlbox#
\end{minted}

In the next example,%
\def\urlvar{http://bit.ly/tarzanek_luke_2OhDT6O}
the \code{ir} field is cast to \code{DirectoryReader} (line 11).
The \code{ir} field is declared as \code{IndexReader} (superclass of \code{DirectoryReader}) in line 1.
The cast to \code{ir} is performed using the value of the expression \code{readers.get(0)} (line 10).
But \code{readers} is defined as \code{ArrayList<DirectoryReader>} (line 3),
making the cast superfluous if an extra variable of type \code{DirectoryReader} had been used.

%https://lgtm.com/projects/g/tarzanek/luke/snapshot/dist-1794550833-1524814812150/files/src/org/getopt/luke/Luke.java?sort=name&dir=ASC&mode=heatmap#L970
\begin{minted}[highlightlines=11]{java}
private IndexReader ir = null;  
// [...]
ArrayList<DirectoryReader> readers = new ArrayList<DirectoryReader>();
for (Directory dd : dirs) {
    DirectoryReader reader;        
    reader = DirectoryReader.open(dd);
    readers.add(reader);
}
if (readers.size() == 1) {
    ir = readers.get(0);
    dir = ((DirectoryReader)ir).directory();
} else {
    ir = new MultiReader(
            (IndexReader[])readers.toArray(new IndexReader[readers.size()]));
} #\urlbox#
\end{minted}


\discussion{}
In most the cases this can be considered as a bad practice or code smell.
This is because by only changing the declaration of the variable
to a more specific type type, the cast can be simply eliminated.

This pattern sometimes related to the \nameref{pat:Redundant} pattern.
Although \thisp{} is not redundant,
by only changing the declaration of the variable to a more specific type,
the cast becomes redundant.

\end{pattern}

\begin{pattern}{SoleSubclassImplementation}
The \thisp{} occurs when an interface or abstract class has only one implementing subclass.
Casting the interface to this class must succeed because it cannot possibly be an instance of another class.

\instances{}
In the following example the \code{jobId}
variable is cast to the sole implementation (\code{JobIdImpl}).

%https://lgtm.com/projects/g/ow2-proactive/scheduling/snapshot/dist-6910096-1524814812150/files/scheduler/scheduler-api/src/main/java/org/ow2/proactive/scheduler/job/JobIdImpl.java?sort=name&dir=ASC&mode=heatmap#L118
\def\urlvar{http://bit.ly/ow2_proactive_scheduling_2Ulcjfs}
\begin{minted}[highlightlines=1,linenos=false]{java}
return Longs.compare(id, ((JobIdImpl) jobId).id);
#\urlbox#
\end{minted}

Similar to the previous example,
the variable \code{user} is cast to the known implementation (\code{UserImpl}).

%https://lgtm.com/projects/g/Javacord/Javacord/snapshot/dist-1791982360-1524814812150/files/src/main/java/de/btobastian/javacord/entities/message/impl/ImplMessage.java?sort=name&dir=ASC&mode=heatmap#L681
\def\urlvar{http://bit.ly/Javacord_Javacord_2GwGjuV}
\begin{minted}[highlightlines=2]{java}
for (User user : api.getUsers()) {
    if (channelId.equals(((ImplUser) user).getUserChannelId())) {
        return user;
    }
} #\urlbox#
\end{minted}


\detection{}
The following query returns all casts such that the type---class or interface---of the expression being cast has only one subtype.
The \emph{transitive closure} symbol \code{+} indicates that \code{getASubtype} may be followed one or more times.

\begin{listing}
\begin{minted}{\qllexer}
class SoleSubclassImplementation extends #\qlref{Cast}# { #\qlbox#
	SoleSubclassImplementation() {
		count(RefType rt | 
				rt = getExpr().getType() and rt.fromSource() | 
				rt.getASubtype+() ) = 1
	}
}
\end{minted}
\caption{Detection of the \thisp{} pattern.}
\label{lst:ql:SoleSubclassImplementationCast}
\end{listing}


\issues{}
This pattern occurs when there is high cohesion between super and subclass.
In some cases, the cast instance appears in a generated class.
This mechanism allows the developer to extend this class to add custom code.
Therefore, this high cohesion is acceptable.
The developer assumes that there is no other implementation of the base class,
otherwise the cast instance fails.

\end{pattern}

\begin{pattern}{NewDynamicInstance}
Dynamically creation of object by means of reflection.
These are the casts that can not be avoidable.

The \code{newInstance} method family declared in the
\code{Class}\footnote{\url{https://docs.oracle.com/javase/8/docs/api/java/lang/Class.html\#newInstance--}},
\code{Array}\footnote{\url{https://docs.oracle.com/javase/8/docs/api/java/lang/reflect/Array.html\#newInstance-java.lang.Class-int-}}\(^{,}\)
\footnote{\url{https://docs.oracle.com/javase/8/docs/api/java/lang/reflect/Array.html\#newInstance-java.lang.Class-int...-}} and
\code{Constructor}\footnote{\url{https://docs.oracle.com/javase/8/docs/api/java/lang/reflect/Constructor.html\#newInstance-java.lang.Object...-}}
classes creates an object or array by means of reflection.

This pattern consists of casting the result of these methods to the appropriate target type.

\instances{}

%https://lgtm.com/projects/g/apache/hadoop/snapshot/6bedbef6c5f2d937a6cbc268300ce2a39609d06c/files/hadoop-hdfs-project/hadoop-hdfs/src/main/java/org/apache/hadoop/hdfs/server/namenode/FSNamesystem.java?sort=name\&dir=ASC\&mode=heatmap\&showExcluded=false#L1039

The following example shows a cast from the \code{Class.newInstance()}
method.
% \footnote{\url{d}}

\begin{lstlisting}[style=java,caption={The \pname{} pattern using the \texttt{Class} class.}]
logger = (AuditLogger) Class.forName(className).newInstance();
\end{lstlisting}

%https://lgtm.com/projects/g/neo4j/neo4j/snapshot/27aaa67633e4d26446e38125d04fbbd27f938b75/files/community/collections/src/main/java/org/neo4j/helpers/collection/Iterables.java?sort=name\&dir=ASC\&mode=heatmap\&showExcluded=false#L403
The following example shows how to dynamically create an array.
%\footnote{\url{d}}

\begin{lstlisting}[style=java,caption={Example of the \pname{} pattern using the \texttt{Array} class.}]
return list.toArray( (T[]) Array.newInstance( componentType, list.size()));
\end{lstlisting}

%https://lgtm.com/projects/g/gradle/gradle/snapshot/209c3175e75af6ac30cb66c02eda15b0f8b6a616/files/subprojects/internal-integ-testing/src/main/groovy/org/gradle/integtests/fixtures/executer/OutputScrapingExecutionFailure.java?sort=name\&dir=ASC\&mode=heatmap\&showExcluded=false#L174

Whenever a constructor other than the default constructor is needed,
the \code{newInstance} method declared in the \code{Constructor} class
should be used to select the appropriate constructor,
as shown in the following example.
%\footnote{\url{d}}

\begin{lstlisting}[style=java,caption=Example of the \pname{} pattern using the \code{Constructor} class.]
return (Exception) Class
                       .forName(className)
                       .getConstructor(String.class)
                       .newInstance(message);
\end{lstlisting}

\detection{}
This detection query looks for casts,
where the expression being cast is a call site to methods mentioned above.

\discussion{}
The cast here is needed because of the dynamic essence of reflection.
This pattern is unguarded, that is,
the application programmer knows what is the target type being created.

\related{}
Reflection.

\end{pattern}
\begin{pattern}{ObjectAsArray}
In this pattern an array is used as an untyped object.
A cast is applied to a constant array slot, \eg, \code{(String) array[1]}.

\instances{}
The following example%
\footnote{\url{http://bit.ly/datanucleus_datanucleus-core_2S1L5Zf}}
shows an instance of the \thisp{} pattern.
The variable \code{currentState} contains an \code{Object[]} with a fixed
schema.%
\footnote{\url{http://www.datanucleus.org/javadocs/core/5.0/org/datanucleus/enhancement/Detachable.html}}
A cast is performed of a constant array slot, \code{(BitSet) currentState[3]} on line 5.

%https://lgtm.com/projects/g/datanucleus/datanucleus-core/snapshot/dist-14100061-1524814812150/files/src/main/java/org/datanucleus/state/StateManagerImpl.java?sort=name&dir=ASC&mode=heatmap#L3324
\begin{minted}[highlightlines=5]{java}
        BitSet theLoadedFields = (BitSet)currentState[2];
        for (int i = 0; i < this.loadedFields.length; i++) {
            this.loadedFields[i] = theLoadedFields.get(i);
        }
        BitSet theModifiedFields = (BitSet)currentState[3];
        for (int i = 0; i < dirtyFields.length; i++) {
            dirtyFields[i] = theModifiedFields.get(i);
        }
        setVersion(currentState[1]);
\end{minted}

\discussion{}
This pattern usually suggests an abuse of the type system.
Using an object with statically typed fields might be a better alternative.

\end{pattern}

\begin{pattern}{ImplicitIntersectionType}
Cast a reference $v$ of type --- class or interface --- $T$ to an
interface type $I$ whether $T$ does not implement $I$.
The cast succeeds at runtime because all possible runtime types of $v$
actually implement the interface $I$.
For instance, in \code{(Comparable)(Number)4}, \code{Number} does not
implement the \code{Comparable} interface, but class \code{Integer} does.

\instances

\begin{lstlisting}[style=java,caption=From \url{http://bit.ly/2FQOt4v}]
final Comparable max = (Comparable) properties.getMaxValue();
\end{lstlisting}
\end{pattern}
\begin{pattern}{CovariantReturnType}
The \thisp{} pattern is used to cast a call to a method that returns
an instance of a type that is covariant with the receiver type.
Commonly the method returns a instance of the receiver type itself.

\instances{}
A common instance of this pattern is for calls to the \code{clone} method of \code{java.lang.Object} (\nCovariantReturnTypeCloneSubpattern{} instances),
which returns an object of the same type as the receiver,
but whose static type is \code{Object}.
The following snippet shows a cast to the \code{clone} method.

%https://lgtm.com/projects/g/aws/aws-sdk-java/snapshot/dist-6950065-1524814812150/files/aws-java-sdk-elasticache/src/main/java/com/amazonaws/services/elasticache/model/ListTagsForResourceResult.java?sort=name&dir=ASC&mode=heatmap#L156
\def\urlvar{http://bit.ly/aws_aws_sdk_java_2GvHhYt}
\begin{minted}[highlightlines=4]{java}
@Override
public ListTagsForResourceResult clone() {
    try {
        return (ListTagsForResourceResult) super.clone();
    } catch (CloneNotSupportedException e) {
        throw new IllegalStateException(/* [...] */);
    }
} #\urlbox#
\end{minted}

In the following example,
the \code{unmarshall} method overrides a superclass method with a covariant return type.
A cast is used on the call to the superclass method to change the type of the return value to match the more precise return type.

%https://lgtm.com/projects/g/aws-amplify/aws-sdk-android/snapshot/dist-2970378-1524814812150/files/aws-android-sdk-autoscaling/src/main/java/com/amazonaws/services/autoscaling/model/transform/ResourceContentionExceptionUnmarshaller.java?sort=name&dir=ASC&mode=heatmap#L39
\def\urlvar{http://bit.ly/aws_amplify_aws_sdk_android_2FVWl13}
\begin{minted}[highlightlines=13]{java}
public class ResourceContentionExceptionUnmarshaller
                    extends StandardErrorUnmarshaller {
    public ResourceContentionExceptionUnmarshaller() {
        super(ResourceContentionException.class);
    }
    public AmazonServiceException unmarshall(Node node) throws Exception {
        // Bail out if this isn't the right error code that this
        // marshaller understands.
        String errorCode = parseErrorCode(node);
        if (errorCode == null || !errorCode.equals("ResourceContention"))
            return null;
        ResourceContentionException e =
                        (ResourceContentionException) super.unmarshall(node);
        return e;
    }
} #\urlbox#
\end{minted}

The \code{initCause} method---from the \code{java.lang.Throwable} class---has return type \code{Throwable}.
Nevertheless, this method returns the receiver (after setting the cause exception).
Therefore a cast is needed to recover the original exception type,
as shown in the following example.
This use case resembles the \nameref{pat:FluentAPI} pattern.

%https://lgtm.com/projects/g/apache/activemq/snapshot/dist-11730660-1524814812150/files/activemq-client/src/main/java/org/apache/activemq/ActiveMQConnectionFactory.java?sort=name&dir=ASC&mode=heatmap#L235
\def\urlvar{http://bit.ly/apache_activemq_2EnSivc}
\begin{minted}[highlightlines=1]{java}
throw (IllegalArgumentException)
        new IllegalArgumentException("Invalid broker URI: " + brokerURL)
        .initCause(e);
#\urlbox#
\end{minted}


\detection{}
The Query~\ref{lst:ql:CovariantReturnTypeCast} approximates the detection of the \thisp{} pattern when a cast is applied to a method in a \code{super} class, \eg{},
in the first two examples shown above.

\begin{listing}
\begin{minted}{\qllexer}
class CovariantReturnTypeCast extends #\qlref{Cast}# { #\qlbox#
	Method m;
	MethodAccess ma;
	CovariantReturnTypeCast() {
		getExpr() = ma and ma.isOwnMethodAccess() and
		getEnclosingCallable() = m and m.overrides(ma.getMethod())
	}
}
\end{minted}
\caption{\thisp{} detection query.}
\label{lst:ql:CovariantReturnTypeCast}
\end{listing}


\issues{}
The situation of returning \code{this} could be avoided if \java{} supported self types~\citep{bruceChallengingTypingIssues2003}.
More generally, associated types~\citep{chakravartyAssociatedTypeSynonyms2005} can provide a statically typed solution,
for instance in the second example above.

\end{pattern}

\begin{pattern}{RemoveWildcard}
A cast is in the \thisp{} pattern when a \emph{wildcard type} is used rather than a generic type.


\instances{}
In the following example,%
\def\urlvar{http://bit.ly/eclipse_jetty_project_2WMI0Ld}
\code{unit} is declared as \code{Unit<?>},
but to actually be able to use it a cast to a concrete type is needed.

%https://lgtm.com/projects/g/eclipse/jetty.project/snapshot/dist-1793550978-1524814812150/files/jetty-util/src/main/java/org/eclipse/jetty/util/PathWatcher.java?sort=name&dir=ASC&mode=heatmap#L752
\begin{minted}[highlightlines=1,linenos=false]{java}
copy.setUnitOfMeasure( (Unit<Length>) unit );
#\urlbox#
\end{minted}


\detection{}
The following query detects the \thisp{} pattern.
The query checks that the type of the cast operand is a wildcard,
or a parameterized type containing a wildcard.

\begin{listing}
\begin{minted}{\qllexer}
predicate containsWildcard(Type t) { #\qlbox#
	t instanceof Wildcard or
	containsWildcard( t.(ParameterizedType).getATypeArgument() )
}

class RemoveWildcardCast extends #\qlref{Cast}# {
	RemoveWildcardCast() {
		containsWildcard(getExpr().getType())
	}
}
\end{minted}
\caption{Detection of the \thisp{} pattern.}
\label{lst:ql:RemoveWildcardCast}
\end{listing}


\issues{}
Wildcard types are a form of existential type and consequently can limit
access to members of a generic type.
Casts are used to restore access at a particular type.

Since this pattern is an unchecked cast,
the discussion about compiler-inserted casts and \emph{blame} is similar to the \nameref{pat:UseRawType} pattern.

\end{pattern}

\begin{pattern}{OperandStack}
The \thisp{} pattern consists of
multiple cases, dispatched depending on some application-specific control state, with
  casts of the top elements of stack-like collection in each case.
An application invariant ensures that if the application is in a given state
then the 
  top elements of the stack should be of known run-time types.

\instances{}
The following example,%
\footnote{\url{http://bit.ly/fabioz_Pydev_2HF6nrF}}
shows a cast whose value is on top of a stack (line 2).
In this case, the code is transforming a parse tree into an abstract syntax
tree. The casts in the switch case are guarded by the parse tree node type and
its arity. 

%https://lgtm.com/projects/g/fabioz/Pydev/snapshot/dist-20832102-1524814812150/files/plugins/org.python.pydev.parser/src/org/python/pydev/parser/grammar27/TreeBuilder27.java?sort=name&dir=ASC&mode=heatmap#L231
\begin{minted}[highlightlines=2]{java}
case JJTASSERT_STMT:
    exprType msg = arity == 2 ? ((exprType) stack.popNode()) : null;
    test = (exprType) stack.popNode();
    return new Assert(test, msg);
\end{minted}

Similar to the previous example,
in this case%
\footnote{\url{http://bit.ly/Sable_soot_2MZLZ3m}}
a guarded cast is performed on a stack of grammar symbols.
The code was generated using an LR parser
generator. The guard 
ensures that the parser has already matched a given prefix of the
input and so the top of the stack should contain the expected symbols.

%https://lgtm.com/projects/g/Sable/soot/snapshot/dist-1791462132-1524814812150/files/src/main/generated/jastadd/soot/JastAddJ/JastAddJavaParser.java?sort=name&dir=ASC&mode=heatmap#L1036
\begin{minted}[highlightlines=4]{java}
case 40: // qualified_name_decl = name_decl.n DOT.DOT IDENTIFIER.i
{
    final Symbol _symbol_n = _symbols[offset + 1];
    final IdUse n = (IdUse) _symbol_n.value;
    final Symbol DOT = _symbols[offset + 2];
    final Symbol i = _symbols[offset + 3];
    return new IdUse(n.getID() + "." + ((String)i.value));
}
\end{minted}

\discussion{}
This pattern is usually seen when implementing grammar-related operations,
such as parsers or interpreters.

  Similar to \nameref{pat:Typecase}, multiple cases are evaluated
  with casts to different types, depending on application-specific guards.
  However, unlike 
  \nameref{pat:Typecase}, 
  the success of the casts is ensured not 
  by a type-tag-like value, but by application-specific state
  (\eg, the current parser state or the state of an evaluator)
  and proper use of the stack.

\end{pattern}

\begin{pattern}{ReflectiveAccessibility}
This pattern accesses a field of an object by means of reflection.
Typically reflection is used because the field is private and therefore
inaccessible at compile time and the developer cannot change the field
declaration itself.
In this case, the method \code{Field::setAccessible(true)} is invoked on the field
before getting the value of the field.
The cast is needed because \code{Field::get} returns an \code{Object}.

\instances{}
The following two snippets show how this pattern is used:

%https://lgtm.com/projects/g/loopj/android-async-http/snapshot/dist-1879340034-1529316783166/files/library/src/main/java/com/loopj/android/http/AsyncHttpClient.java?sort=name&dir=ASC&mode=heatmap&showExcluded=false#L445
\def\urlvar{http://bit.ly/loopj_android_async_http_2SOISRr}
\begin{minted}[highlightlines=2]{java}
f.setAccessible(true);
HttpEntity wrapped = (HttpEntity) f.get(entity);
#\urlbox#
\end{minted}

%https://lgtm.com/projects/g/joel-costigliola/assertj-db/snapshot/dist-890344-1524814812150/files/src/test/java/org/assertj/db/navigation/ToChange_ChangeOfModification_Integer_Test.java?sort=name&dir=ASC&mode=heatmap#L199
\def\urlvar{http://bit.ly/joel_costigliola_assertj_db_2Ip1Rho}
\begin{minted}[highlightlines=5]{java}
Field fieldPosition=ChangesOutputter.class.getDeclaredField("changesPosition");
fieldPosition.setAccessible(true);
ChangesOutputter changesDisplayBis = output(changes);
PositionWithChanges<ChangesAssert, ChangeAssert> positionBis = 
            (PositionWithChanges) fieldPosition.get(changesDisplayBis);
#\urlbox#
\end{minted}


\detection{}
The Query~\ref{lst:ql:ReflectiveAccessibilityCast} detects the \thisp{} pattern.

\begin{listing}
\begin{minted}{java}	
class ReflectiveAccessibilityCast extends #\qlref{Cast}# { #\qlbox#
	Variable fieldVariable;
    #\qlref{ReflectiveMethodAccess}# reflectiveMethodAccess;
	#\qlref{SetAccessibleTrueMethodAccess}# satma;
	ReflectiveAccessibilityCast() {
		reflectiveMethodAccess = getExprOrDef() and
		fieldVariable.getAnAccess() = 
                getExprOrDef().(MethodAccess).getQualifier().(VarAccess) and
		fieldVariable.getAnAccess() = satma.getQualifier()
	}
}
\end{minted}
\caption{Detection of the \thisp{} pattern.}
\label{lst:ql:ReflectiveAccessibilityCast}
\end{listing}


\issues{}
Using reflection to access a field is a common workaround to tight access
  control restrictions. However, it should generally be regarded as a code
  smell.

As with \nameref{pat:Deserialization}, this pattern is necessary because
a library method can return values of many different types at run time,
and so is declared to return \code{Object}.


\end{pattern}

\begin{pattern}{FluentAPI}
A fluent \api{} is an \api{} that allows the developer to operate on the same
object using method chaining.
This pattern is exhibited when the receiver (\code{this} reference) is cast to a type parameter
  which is itself bounded by the self type.


\instances{}
In the following snippet,%
\def\urlvar{http://bit.ly/HanSolo_Medusa_2TyBObH}
the receiver (\code{this}) is cast to a type parameter (\code{B}) (line 5).
This allows subclasses to reuse the methods in the base class without overriding them just to change the return type.

%https://lgtm.com/projects/g/HanSolo/Medusa/snapshot/dist-1798710811-1524814812150/files/src/main/java/eu/hansolo/medusa/ClockBuilder.java?sort=name&dir=ASC&mode=heatmap#L293
\begin{minted}[highlightlines=5]{java}
public class ClockBuilder <B extends ClockBuilder<B>> {
    // [...]
    public final B alarms(final Alarm... ALARMS) {
        properties.put("alarmsArray", new SimpleObjectProperty<>(ALARMS));
        return (B) this;
    }
} #\urlbox#
\end{minted}

\issues{}
  This pattern is concerned with a particular implementation of fluent \api{}s
  where recursive generics are used to mimic self
  types~\cite{bruceChallengingTypingIssues2003}.
  Other implementations of fluent \api{}s simply return \code{this} without a
  cast, but these are less extensible.

\end{pattern}

\begin{pattern}{CovariantGeneric}
The \thisp{} pattern occurs when a cast is used to use an invariant generic type as if it were covariant.

\instances{}
In the following snippet,
an upcast is performed to ensure that the inferred type of the call to \code{singletonList} is a supertype of the type that would be otherwise inferred.
The \code{singletonList} method has the signature \code{<T> List<T> singletonList(T o)}.%
\footnote{\url{https://docs.oracle.com/javase/8/docs/api/java/util/Collections.html}}
If \code{curframe} were passed in without the cast,
the type of the list would be inferred to be \code{List<FrameBuilder>},
which is not a subtype of the method return type \code{List<Framedata>},
causing a compilation error.
With the cast, the list type is inferred to be the same as the return type.

%https://lgtm.com/projects/g/arpruss/raspberryjammod/snapshot/dist-1796220064-1524814812150/files/build/sources/java/org/java_websocket/drafts/Draft_10.java?sort=name&dir=ASC&mode=heatmap#L157
\def\urlvar{http://bit.ly/arpruss_raspberryjammod_2USL7Ai}
\begin{minted}[highlightlines=5]{java}
@Override
public List<Framedata> createFrames(String text, boolean mask) {
    FrameBuilder curframe = new FramedataImpl1();
    /* [...] */
    return Collections.singletonList( (Framedata) curframe );
}
public interface FrameBuilder extends Framedata { /* [...] */ }
public class Collections {
    public static <T> List<T> singletonList(T o) { /* [...] */ }
} #\urlbox#
\end{minted}

Similar to the previous example, in the following case,
an upcast is performed to change the return type of the
\code{Matcher<T> equalTo(T)} method.

%https://lgtm.com/projects/g/jfaster/mango/snapshot/dist-1793930711-1524814812150/files/src/test/java/org/jfaster/mango/operator/UpdateOperatorTest.java?sort=name&dir=ASC&mode=heatmap#L125
\def\urlvar{http://bit.ly/jfaster_mango_2EhXzUW}
\begin{minted}[highlightlines=5]{java}
@Test
public void testUpdateReturnBoolean() throws Exception {
    /* [...] */
    List<Object> args = boundSql.getArgs();
    assertThat(args.get(0), equalTo((Object) "ash"));
}
public static <T> Matcher<T> equalTo(T operand) {
    // [...]
} #\urlbox#
\end{minted}

Instead of an upcast, in this example,
a cast to \code{null} is performed to change the return type.
This use case resembles the \nameref{pat:SelectOverload} pattern.

%https://lgtm.com/projects/g/EngineHub/WorldGuard/snapshot/dist-1795351250-1524814812150/files/worldguard-legacy/src/test/java/com/sk89q/worldguard/protection/FlagValueCalculatorTest.java?sort=name&dir=ASC&mode=heatmap#L1024
\def\urlvar{http://bit.ly/EngineHub_WorldGuard_2IVUOx1}
\begin{minted}[highlightlines=1]{java}
assertThat(result.queryValue(memberOne, DefaultFlag.BUILD), is((State) null));
public static <T> Matcher<T> is(T value) {
    // [...]
} #\urlbox#
\end{minted}

Another common version of this pattern for type \code{S} a subtype of \code{T},
is to cast a generic type like \code{List<S>} to a raw type (\code{List}),
which can then be assigned to a variable of \code{List<T>}.
%https://lgtm.com/projects/g/spockframework/spock/snapshot/dist-7950040-1524814812150/files/spock-core/src/main/java/org/spockframework/compiler/WhereBlockRewriter.java?sort=name&dir=ASC&mode=heatmap#L360
\def\urlvar{http://bit.ly/spockframework_spock_2UYEsF5}
\begin{minted}[highlightlines=2]{java}
private final List<VariableExpression> dataProcessorVars = new ArrayList<>();
new ArrayExpression(ClassHelper.OBJECT_TYPE, (List) dataProcessorVars);
public class ArrayExpression extends Expression {
    public ArrayExpression(ClassNode elementType, List<Expression> exprs) {}
} #\urlbox#
\end{minted}


\detection{}
The Query~\ref{lst:ql:CovariantGenericCast} detects when a cast is used to select the return type of a generic method.
This query \emph{does not} detect a cast belonging to this pattern when the raw type is used, \eg{}, last example shown above.

\begin{listing}
\begin{minted}{\qllexer}
class CovariantGenericCast extends #\qlref{Cast}# { #\qlbox#
	Argument arg;
	Call call;
	Callable m;
	CovariantGenericCast() {
		this = arg and
		call = arg.getCall() and
		arg.getCall().getCallee() = m and
		(
			m.getReturnType().(ParameterizedType).getATypeArgument() =
					m.getParameterType(arg.getPosition()).(TypeVariable) or
			m.getReturnType().(TypeVariable) =
					m.getParameterType(arg.getPosition()).(TypeVariable)
		)
	}
}
\end{minted}
\caption{Query to detect the \thisp{} pattern.}
\label{lst:ql:CovariantGenericCast}
\end{listing}


\issues{}
In some cases, this pattern could be avoided using explicit type parameter,
\eg{}, \code{Collections.<Framedata>singletonList(curframe)}.
From \java{} 8 this cast is unnecessary due to better type inference.%
\footnote{\url{https://docs.oracle.com/javase/specs/jls/se8/html/jls-18.html\#jls-18.5}}

\cite{altidorTamingWildcardsCombining2011} define a type system that infers definition-site variance to \java{}.
This can reduce the need for this pattern.

\end{pattern}
\begin{pattern}{Composite}
The \thisp{} pattern is characterized by a cast to another element of a
composite data structure, typically a tree, where the target type is known because of its
position in the data structure.

\instances{}
The following example%
\footnote{\url{http://bit.ly/flyingsaucerproject_flyingsaucer_2N2nYbY}}
shows a cast from a \code{Box} to a \code{TableSectionBox}.
The programmer reasons that the cast will succeed because the 
the source of the cast is a sibling of another \code{TableSectionBox}.

%https://lgtm.com/projects/g/flyingsaucerproject/flyingsaucer/snapshot/dist-26624048-1524814812150/files/flying-saucer-core/src/main/java/org/xhtmlrenderer/newtable/TableBox.java#L711
\begin{minted}[highlightlines=3]{java}
public class TableBox extends BlockBox {
    protected TableSectionBox sectionAbove(TableSectionBox section, boolean skipEmptySections) {
        TableSectionBox prevSection=(TableSectionBox)section.getPreviousSibling();
    }
}

public abstract class Box implements Styleable {
    // [...]
    public Box getPreviousSibling() {
        // [...]
    }
}
\end{minted}

\discussion{}
The pattern is typical of hierarchical data structures such as abstract syntax
trees, document models, or UI layouts. Based on the grammar of 
the data structure, the types of adjacent objects in the structure can be known.
The cast succeeds if the data structure is well-formed.

More precise typing of the links in the the data structure could 
eliminate the need for the casts. For example, in the above example,
the sibling of a \code{TableSectionBox} might be declared to have type
\code{TableSectionBox}. However, this may require the programmer to override
methods to refine return types covariantly.
Language features available in other languages like generalized algebraic data types
  (GADTs)~\cite{gadts} or self types~\cite{bruceChallengingTypingIssues2003,scalaIndependentlyExtensible} could also be 
used to provide a more precise typing.

The pattern can be thought of as a more dynamic variant of the
\nameref{pat:Family} pattern. Rather than
reasoning that the cast will succeed because of the source type's relative position in the 
class hierarchy, the cast will succeed because of the source value's position
in a composite data structure.

\end{pattern}

\begin{pattern}{GenericArray}

\instances{}
The following snippet%
\footnote{\url{http://bit.ly/ppiastucki_recast4j_2EM7zWK}}

%https://lgtm.com/projects/g/ppiastucki/recast4j/snapshot/dist-4860452-1524814812150/files/detourtilecache/src/main/java/org/recast4j/detour/tilecache/AbstractTileLayersBuilder.java?sort=name&dir=ASC&mode=heatmap#L50
\begin{minted}[highlightlines=1]{java}
List<?>[][] partialResults = new List[th][tw];
partialResults[ty][tx] = build(tx, ty, order, cCompatibility);
layers.addAll((List<byte[]>) partialResults[y][x]);

protected abstract List<byte[]> build(int tx, int ty, ByteOrder order, boolean cCompatibility);

\end{minted}

\detection{}

\discussion{}

\related{}

\end{pattern}
\begin{pattern}{AccessPrivateField}
Perform an upcast to access a private field defined within the same class.
Notice that there is only one instance of this pattern,
and from generated code.

\instances{}

\footnote{\url{http://bit.ly/FenixEdu_fenixedu-academic_2SQxlkC}}

%https://lgtm.com/projects/g/FenixEdu/fenixedu-academic/snapshot/dist-29270029-1524814812150/files/target/generated-test-sources/dml-maven-plugin/org/fenixedu/academic/domain/residence/StudentsPerformanceReport_Base.java?sort=name&dir=ASC&mode=heatmap#L12
\begin{minted}[highlightlines=4]{java}
public abstract class StudentsPerformanceReport_Base extends QueueJobWithFile {
    // [...]
    public ExecutionSemester getValue(StudentsPerformanceReport o1) {
        return ((StudentsPerformanceReport_Base)o1).executionSemester.get();
    }
    private OwnedVBox<ExecutionSemester> executionSemester;
}
 
public class StudentsPerformanceReport extends StudentsPerformanceReport_Base {
    // [...]
}
\end{minted}

\detection{}

\discussion{}

\related{}

\end{pattern}
\begin{pattern}{UnoccupiedTypeParameter}

\instances{}

Either this fluent \api{}
\footnote{\url{http://bit.ly/vavr-io-vavr-2SMIfI2}}

%https://lgtm.com/projects/g/vavr-io/vavr/snapshot/dist-11000099-1524814812150/files/vavr/src/main/java/io/vavr/control/Either.java?sort=name&dir=ASC&mode=heatmap#L398
\begin{listing}[h!]
\caption{Either fluent \api{}  \url{http://bit.ly/vavr-io-vavr-2SMIfI2}  }
\begin{minted}[highlightlines=9]{java}
public interface Either<L, R> extends Value<R>, Serializable {
    // [...]
    @SuppressWarnings("unchecked")
    default <U> Either<U, R> mapLeft(Function<? super L, ? extends U> leftMapper) {
        Objects.requireNonNull(leftMapper, "leftMapper is null");
        if (isLeft()) {
            return Either.left(leftMapper.apply(getLeft()));
        } else {
            return (Either<U, R>) this;
        }
    }
    // [...]
}
\end{minted}
\end{listing}

\detection{}

\discussion{}

Scala has a \code{Nothing} type
Option[String] :> Option[Nothing]

\related{}

\end{pattern}

\section{Discussion}\label{sec:casts:discussion}

\begin{table*}[t!]
\scriptsize
\centering
\caption{Categorization of Cast Usage Patterns}
\label{table:casts:categories}
\begin{tabularx}{\linewidth}{|X||c|c|c|c|c|c|c||c|}
\hdr \hline
      \multicolumn{1}{|c||}{\textbf{Pattern}}
    & \textbf{Guard}
    & \textbf{Lang}
    & \textbf{Tools}
    & \textbf{Auto}
    & \textbf{Refactor}
    & \textbf{Generics}
    & \textbf{Boxing}
    & \textbf{\ql{}}
    \\ \hline
\row 1 & \nameref{pat:Typecase} & \cmark & \cmark & \cmark &  &  &  \\
\alt 2 & \nameref{pat:Stash} &  &  & \cmark & \cmark &  &  \\
\row 3 & \nameref{pat:Factory} &  &  & \cmark &  &  &  \\
\alt 4 & \nameref{pat:Family} &  & \cmark &  &  &  &  \\
\row 5 & \nameref{pat:UseRawType} &  &  &  &  & \cmark & \cmark \\
\alt 6 & \nameref{pat:Equals} & \cmark &  &  & \cmark &  &  \\
\row 7 & \nameref{pat:Redundant} &  &  &  &  & \cmark &  \\
\alt 8 & \nameref{pat:CovariantReturnType} &  & \cmark &  &  &  &  \\
\row 9 & \nameref{pat:SelectOverload} &  & \cmark &  &  &  &  \\
\alt 10 & \nameref{pat:KnownReturnType} &  &  & \cmark &  & \cmark &  \\
\row 11 & \nameref{pat:Deserialization} &  &  & \cmark &  &  &  \\
\alt 12 & \nameref{pat:VariableSupertype} &  &  &  &  & \cmark &  \\
\row 13 & \nameref{pat:NewDynamicInstance} &  &  & \cmark &  &  &  \\
\alt 14 & \nameref{pat:SoleSubclassImplementation} &  & \cmark &  &  &  &  \\
\row 15 & \nameref{pat:ObjectAsArray} &  &  &  &  & \cmark &  \\
\alt 16 & \nameref{pat:ImplicitIntersectionType} &  & \cmark &  &  &  &  \\
\row 17 & \nameref{pat:RemoveWildcard} &  & \cmark &  &  &  & \cmark \\
\alt 18 & \nameref{pat:OperandStack} & \cmark & \cmark &  &  &  &  \\
\row 19 & \nameref{pat:ReflectiveAccessibility} &  & \cmark &  &  &  &  \\
\alt 20 & \nameref{pat:FluentAPI} &  & \cmark &  &  &  & \cmark \\
\row 21 & \nameref{pat:CovariantGeneric} &  & \cmark &  &  &  & \cmark \\
\alt 22 & \nameref{pat:Composite} &  & \cmark &  &  &  &  \\
\row 23 & \nameref{pat:GenericArray} &  & \cmark &  &  &  & \cmark \\
\alt 24 & \nameref{pat:AccessSuperclassField} &  &  &  &  & \cmark &  \\
\row 25 & \nameref{pat:UnoccupiedTypeParameter} &  & \cmark &  &  &  & \cmark \\

\hline
\end{tabularx}
\end{table*}


\newcommand{\gh}[1]{\emph{\textbf{#1}}}

There are common aspects shared by several patterns.
Table~\ref{table:casts:categories} presents a summary of the patterns and their different aspects.
The table consists of the following columns:
The \gh{Pattern} column indicates the name of the pattern.
\gh{Guarded} The patterns here are guarded casts.
A guarded cast is a cast such that before the cast is applied,
some condition---the \emph{guard}---needs to be verified.
The condition to be verified guarantees that the cast will not fail at runtime (unless there is a bug in the application), \ie,
the cast will not throw a \code{ClassCastException}.
Some kind of guards ensure that the cast will not fail at the language-level,
while others only can guarantee it at the application-level.
\gh{Lang} These casts could be ameliorated if there is enough language support by changing the type system.
\gh{Tools} The casts in this group could be checked with new analysis or compiler tools.
\gh{Auto} These casts are related to generated or boilerplate code.
\gh{Refactor} The casts with this aspect can be simply removed by the developer,
can be removed with little refactoring,
or suggest a code smell in the source code.
\gh{Generics} The casts in this category are related to generics or reified generics.
\gh{Boxing} These casts are related with explicit boxing/unboxing operations, \ie{},
explicit converting values of primitive types to boxed types back and forth.
\gh{\ql{}}
A bullet (\exis{}) in the \ql{} columns indicates that a pattern is partially detected in \ql{};
a check mark (\tick{}) indicates that we have provided a \ql{} query for automatic detection; and
a cross mark (\xmark{}) indicates that is unfeasible or impractical to detect this pattern in \ql{}.

Many programming languages provide features to ameliorate the more common use cases of casts.
For instance,
\kotlin{}'s smart casts couple together the \code{instanceof} operator and cast operation on value, 
providing direct support for the \nameref{pat:Typecase} and \nameref{pat:Equals} patterns.
More generally, ML-style pattern matching subsumes this pattern.
Smart casts do not apply directly to the \nameref{pat:OperandStack} pattern,
since it is dispatched depending on some application-specific control state.

Other language features that might at least partially obviate the need for some of the patterns are
intersection types (\cf{} \nameref{pat:ImplicitIntersectionType}),
and self types or associated types (\cf{} 
\nameref{pat:Factory},
\nameref{pat:KnownReturnType},
\nameref{pat:Deserialization},
\nameref{pat:CovariantReturnType},
\nameref{pat:FluentAPI}).
Virtual classes~\citep{gbeta, scalaIndependentlyExtensible} and languages that support family polymorphism~\citep{ernstFamilyPolymorphism2001}
would help with casts in the \nameref{pat:Family} pattern.

Some cast can be automatically generated.
The \variant{StaticResource} variant in \nameref{pat:Stash} could be generated by a GUI editor,
given that it is most seen in Android applications.
The \nameref{pat:Equals} pattern is composed of boilerplate code.
For instance, \scala{}' solves this issue by introducing \emph{case classes},
which among other features, provide equality out of the box.

Our study also suggests analyses could be performed to improve code quality and eliminate some cast usages,
for instance finding opportunities to use generics instead (\cf{} \nameref{pat:UseRawType}),
removing redundant casts (\cf{} \nameref{pat:Redundant}),
or locating and removing code smells (\cf{}
\nameref{pat:KnownReturnType},
\nameref{pat:VariableSupertype}, and
\nameref{pat:ObjectAsArray}).

The
\nameref{pat:RemoveWildcard},
\nameref{pat:GenericArray} and
\nameref{pat:CovariantGeneric}
patterns are used to workaround the erasure of generic type parameters in \java{};
while the \nameref{pat:UnoccupiedTypeParameter} pattern is used to take advantage of it.
Reified generics or definition-site, rather than use-site,
variance annotations~\citep{altidorTamingWildcardsCombining2011}
would reduce the need for these patterns.
There is an ongoing proposal%
\footnote{\url{https://openjdk.java.net/jeps/300}}~\citep{jep300}
to enhance \java{} with this feature.

The
\nameref{pat:UseRawType},
\nameref{pat:CovariantGeneric}, and
\nameref{pat:GenericArray}
patterns use boxing/unboxing because of the interplay between primitive types and generics.
The JEP 218 Generics over Primitive Types%
\footnote{\url{https://openjdk.java.net/jeps/218}}~\citep{jep218}
could ameliorate the situation in this aspect.
On the other hand, 
the \nameref{pat:SelectOverload} pattern uses boxing/unboxing to select the appropriate method,
while the \nameref{pat:ReflectiveAccessibility} pattern
uses unboxing when the field being accessed is of a primitive type.

To detect patterns like
\nameref{pat:Typecase},
\nameref{pat:Equals}, and
\nameref{pat:Redundant},
only a local analysis (within a method) is needed.
Usually generic related patterns, \eg{},
\nameref{pat:UseRawType},
\nameref{pat:RemoveWildcard},


On the other hand,
patterns like \nameref{pat:VariableSupertype} and
\nameref{pat:UnoccupiedTypeParameter} require a global analysis.

There are patterns that depend exclusively on known methods, \eg{},
\nameref{pat:NewDynamicInstance}, and
\nameref{pat:ReflectiveAccessibility}.
These patterns are easily detectable.
Although a pattern like \nameref{pat:Deserialization} depends on a well-known method---\code{readObject}---an application could use others deserialization mechanisms.

Some other patterns are inherently complex to detect, \eg{},
\nameref{pat:Stash},
\nameref{pat:Family},
\nameref{pat:OperandStack}, and
\nameref{pat:Composite}.
Recognition of these patterns would require to take into account many different variants,
which makes automatic detection impractical.
Manually inspection would be better suited in these cases. 

Patterns like \nameref{pat:Factory} and \nameref{pat:KnownReturnType}



\section{Conclusions}

\todo{Matthias: You need at least one section to close this chapter. Some kind of discussion/reflection, and conclusion.}
Discussion.