\chapter{Literature Review}
\label{cha:literature-review}

Understanding how developers use language features
and \api{}s is a broad topic.
There is plenty of research in the computer science literature about
empirical studies of programs which involves multiple \emph{dimensions}
directly related to our plan.
Over the last decades,
researchers always have been interested in understanding what
kind of programs developers write.
The motivation behind these studies is quite broad,
and has been shifted to the needs of researchers,
together with the evolution of computer science itself.

For instance, to measure the advantages between compilation and interpretation in \basic{},
\cite{hammondBASICEvaluationProcessing1977} studied a representative dataset of programs.
\cite{knuthEmpiricalStudyFORTRAN1971} started to study \fortran{} programs.
By knowing what kind of programs arise in practice,
a compiler optimizer can focus in those cases,
and therefore can be more effective.
Adding to Knuth's work,%
~\cite{shenEmpiricalStudyFortran1990} conducted an empirical study for
parallelizing compilers.
Similar works have been done for
\cobol{}~\cite{salvadoriStaticProfileCOBOL1975,chevanceStaticProfileDynamic1978},
\pascal{}~\cite{cookContextualAnalysisPascal1982},
and \apl{}~\cite{saalPropertiesAPLPrograms1975,saalEmpiricalStudyAPL1977} programs.
\cite{millerEmpiricalStudyReliability1990,millerFuzzRevisitedReexamination1995,forresterEmpiricalStudyRobustness2000}
studied the reliability of programs using \emph{fuzz} testing.
\cite{dieckmannStudyAllocationBehavior1999} studied the memory allocating
behavior in the SPECjvm98 benchmarks.%
\footnote{\url{https://www.spec.org/jvm98/}}

But there is more than empirical studies at the source code level.
A machine instruction set is effectively another kind of language.
Therefore, its design can be affected by how compilers generate machine code.
Several studies targeted the \jvm{}~\cite{collbergEmpiricalStudyJava2007,odonoghueBigramAnalysisJava2002,antonioliAnalysisJavaClass1998};
while~\cite{cookEmpiricalAnalysisLilith1989} did a similar study for \lilith{} in the past.

The importance of conducting empirical studies of programs gave rise to the International Conference on Mining Software Repositories%
\footnote{\url{http://www.msrconf.org/}}
in 2004.

When conducting empirical studies about programs,
multiple dimensions are involved.
The first one is \emph{What to analyze?}
Benchmarks and corpora are used as a source of programs to analyze.
Another aspect is how to select good candidates projects from a large-base software repository.
This is presented in Section~\ref{sec:literature-review:benchmarks}.
After the selection of programs to analyze is set,
comes the question \emph{how to analyze them?}
An overview of what tools are available to extract information from software repositories is given in Section~\ref{sec:literature-review:mining}.
With this infrastructure, \emph{what questions do researchers ask?}
In Section~\ref{sec:literature-review:largescale},
we give an overview of large-scale empirical studies that show what kind of questions researchers ask.
This chapter ends by presenting the related work more specific to the Unsafe API and Casting in Sections~\ref{sec:literature-review:unsafe} and \ref{sec:literature-review:casting} respectively.


\section{Benchmarks and Corpora}
\label{sec:literature-review:benchmarks}

Benchmarks are crucial to properly evaluate and measure product development.
This is key for both research and industry.
One popular benchmark suite for \java{} is the DaCapo Benchmark~\citep{blackburnDaCapoBenchmarksJava2006}.
This suite has been already cited in more than thousand publications, showing how important is to have reliable benchmark suites.
The SPECjvm2008\footnote{\url{https://www.spec.org/jvm2008/}}
(Java Virtual Machine Benchmark)
and
SPECjbb2000\footnote{\url{https://www.spec.org/jbb2000/}}
(Java Business Benchmark)
are another popular \java{} benchmark suite.

Another suite has been developed by~\cite{temperoQualitasCorpusCurated2010}.
They provide a corpus of curated open source systems to facilitate empirical studies on source code.
On top of Qualitas Corpus,~\cite{dietrichXCorpusExecutableCorpus2017} provide an executable corpus of \java{} programs.
This allows any researcher to experiment with both static and dynamic analysis.

For any benchmark or corpus to be useful and reliable,
it must faithfully represent real world code.
For instance,
DaCapo applications were selected to be diverse real applications and
ease of use, but they ``excluded GUI applications since they are difficult
to benchmark systematically.''
Along these lines, \cite{allamanisMiningSourceCode2013} go one step further and provide a large-scale (14,807) curated corpus of open source \java{} projects.

With the advent of cloud computing,
several source code management (SCM) hosting services have emerged, \eg{},
\github{}, \gitlab{}, \bitbucket{}, and \sourceforge{}.
These services allow the developer to work with different SCMs, \eg,
Git, Mercurial, Subversion to host their open source projects.
These projects are usually taken as a representation of
real-world applications.
Thus, while not curated corpora, these hosting services are
commonly used to conduct empirical studies.

Another dimension to consider when analyzing large codebases, is how relevant the repositories are.
\cite{lopesDeJaVuMapCode2017} conducted a study to measure code duplication in \github{}.
They found out that much of the code there is actually duplicated.
This raises a flag when considering which projects to analyze when mining software repositories.

\cite{baxterCloneDetectionUsing1998} propose a clone detection algorithm using Abstract Syntax Trees,
while \cite{riegerVisualDetectionDuplicated} propose a visual detection for clones.
\cite{yuanCMCDCountMatrix2011,chenReplicationReproductionCode} instead propose Count Matrix-based approach to detect code clones.

\cite{nagappanDiversitySoftwareEngineering2013} have developed the Software Projects Sampling (SPS) tool.
SPS tries to find a maximal set of projects based on representativeness and diversity.
Diversity dimensions considered include total lines of code,
project age, activity, number of contributors, total code churn,
and number of commits.

\section{Tools for Mining Software Repositories}
\label{sec:literature-review:mining}

When talking about mining software repositories,
we refer to extracting any kind of information from large-scale codebase repositories. 
Usually doing so requires several engineering but challenging tasks.
The most common being downloading, storing, parsing, analyzing and
properly extracting information from different kinds of artifacts.
In this scenario, there are several tools that allows a researcher or developer to query information about software repositories.

\cite{urmaProgrammingLanguageEvolution2012} evaluated seven source code
query languages\footnote{\url{https://wiki.openjdk.java.net/display/Compiler/Java+Corpus+Tools}}:
\emph{Java Tools Language} \citep{cohenJTLJavaTools},
\emph{Browse-By-Query}\footnote{\url{http://browsebyquery.sourceforge.net/}},
\emph{SOUL} \citep{derooverSOULToolSuite2011},
\emph{JQuery} \citep{volderJqueryGenericCode2006},
\emph{.QL} \citep{moorKeynoteAddressQL2007},
\emph{Jackpot}\footnote{\url{http://wiki.netbeans.org/Jackpot}}, and
\emph{PMD}\footnote{\url{https://pmd.github.io/}}.
They have implemented---whenever possible---four use cases using the tools mentioned above.
They concluded that only \emph{SOUL} and \emph{.QL} have the minimal features to implement all their use cases.

\cite{dyerBoaLanguageInfrastructure2013,dyerDeclarativeVisitorsEase2013} built \boa{}, both a domain-specific language and an online platform\footnote{\url{http://boa.cs.iastate.edu/}}. 
It is used to query software repositories on two popular hosting services, \github{} and \sourceforge{}.
The same authors of \boa{} conducted a study on
how new \java{} features, \eg,
\emph{Assertions},
\emph{Enhanced-For Loop},
\emph{Extends Wildcard},
were adopted by developers over time~\citep{dyerMiningBillionsAST2014}.
This study is based \sourceforge{} data.
The current problem with \sourceforge{} is that is outdated.

To this end, \cite{gousiosGHTorentDatasetTool2013} provides an offline mirror of \github{} that allows researchers to query any kind of that data.
Later on, \cite{gousiosLeanGHTorrentGitHub2014} published the dataset construction process of \github{}.

Similar to \boa{}, \lgtm{}\footnote{\url{https://lgtm.com/}} is a platform to query software projects properties.
It works by querying repositories from \github{}.
But it does not work at a large-scale, \ie{}, \lgtm{} allows the user to query just a few projects.
Unlike \boa{}, \lgtm{} is based on \ql{}---before named \emph{.QL}---,
an object-oriented domain-specific language to query recursive data structures based on Datalog~\citep{avgustinovQLObjectorientedQueries2016}.
Another static analysis framework based on Datalog is \doop{}~\citep{bravenboerStrictlyDeclarativeSpecification}.
Since \ql{} and \doop{} are based on Datalog,
both are well-suited to perform points-to analysis and data-flow analysis.
However, scaling such analysis to a large-scale study remains an open problem.

On top of \boa{},~\cite{tiwariCandoiaPlatformBuilding2017} built \candoia{}%
\footnote{\url{http://candoia.github.io/}}.
Although it is not a mining software repository \perse{},
it eases the creation of mining applications. 

Another tool to analyze large software repositories is presented in~\cite{brandauerSpencerInteractiveHeap2017}.
In this case, the analysis is dynamic, based on program traces. 
At the time of this writing, the service\footnote{\url{http://www.spencer-t.racing/datasets}} was unavailable for testing. 

\cite{bajracharyaSourcererInternetscaleSoftware2009} provide a tool to query large code bases by extracting the source code into a relational model.
Sourcegraph\footnote{\url{https://sourcegraph.com}} is a tool that allows regular expression and diff searches.
It integrates with source repositories to ease navigate software projects.

\cite{posnettTHEXMiningMetapatterns2010} have extended
\asm{}~\citep{brunetonASMCodeManipulation2002,kuleshovUsingASMFramework2007}
to detect meta-patterns, \ie,
purely structural patterns of object-oriented interaction.
\cite{huDynamicAnalysisDesign2008} used both dynamic and static analysis to discover design patterns, while \cite{arcelliDesignPatternDetection2008} used only dynamic.

Trying to unify analysis and transformation tools,
\cite{vinjuHowMakeBridge2006} and~\cite{klintRASCALDomainSpecific2009} built \rascal,
a DSL that aims to bring them together by querying the AST of a program.

As its name suggests,
JavaParser\footnote{\url{http://javaparser.org/}}
is a parser for \java{}.
The main issue with JavaParser is that it lacks the ability to perform symbol resolution integrated with the project dependencies.


\section{Large-scale Codebase Empirical Studies}
\label{sec:literature-review:largescale}

In the same direction as our plan,
\cite{callauHowWhyDevelopers2013} performed an empirical study to assess
how much the dynamic and reflective features of \smalltalk{} are actually
used in practice.
Analogously, \cite{richardsAnalysisDynamicBehavior2010,richardsEvalThatMen2011,weiEmpiricalStudyDynamic2016}
conducted a similar study, but in this case targeting \javascript's dynamic
behavior and in particular the \code{eval} function.
Also, for \javascript{}, \cite{madsenStringAnalysisDynamic2014} analyzed
how fields are accessed via strings,
while~\cite{jangEmpiricalStudyPrivacyviolating2010}
analyzed privacy violations.
Similar empirical studies were done for
\php{}~\citep{hillsEmpiricalStudyPHP2013,dahseExperienceReportEmpirical2015,doyleEmpiricalStudyEvolution2011}
and \swift{}~\citep{reboucasEmpiricalStudyUsage2016}.
\cite{PINTO201559} conducted a large-scale study on how concurrency is used in \java{}

Going one step forward, \cite{rayLargescaleStudyProgramming2017} studied the correlation between programming languages and defects. 
One important note is that they choose relevant projects by popularity,
measured by how many times was \emph{starred} in \github{}.
We argue that it is more important to analyze projects that are \emph{representative}, not \emph{popular}.

\cite{gorlaCheckingAppBehavior2014} mined a large set of Android applications, clustering applications by their description topics and identifying outliers in each cluster with respect to their API usage.
\cite{grechanikEmpiricalInvestigationLargescale2010} also mined large scale software repositories to obtain several statistics on how source code is actually written.

For \java{},~\cite{dietrichContractsWildStudy2017a} conducted a study
about how programmers use contracts in \mavencentral{}\footnote{\url{http://central.sonatype.org/}}.
\cite{dietrichBrokenPromisesEmpirical2014} have studied how
\api{} changes impact \java{} programs.
They have used the Qualitas Corpus~\citep{temperoQualitasCorpusCurated2010} mentioned above for their study.

\cite{tufanoWhenWhyYour2015,tufanoWhenWhyYour2017} studied when code
smells are introduced in source code.
\cite{palombaLandfillOpenDataset2015}
contribute a dataset of five types of code smells together with a systematic procedure for validating code smell datasets.
\cite{palombaDetectingBadSmells2013} propose to detect code smells using change history information.

\cite{nagappanEmpiricalStudyGoto2015} conducted a study on how the
\code{goto} statement is used in \cc{}.
They used \github{} as a data source for \cc{} programs.
They concluded that \code{goto} statements are most used for
\emph{handling errors} and \emph{cleaning up resources}.

\textbf{Static vs. Dynamic Analysis.}
Given the dynamic nature of \javascript, most of the studies mentioned
above for \javascript{} perform dynamic analysis.
However, \cite{callauHowWhyDevelopers2013} uses static analysis to study
a dynamically checked language.
For \java, most empirical studies use static analysis.
This is due the fact of the availability of input data.
Finding valid input data for test cases is not a trivial task,
even less to make it scale.
For \javascript, having a big corpus of web-sites generating valid
input data makes more feasible to implement dynamic analysis.

\subsection*{Exceptions}

\cite{keryExaminingProgrammerPractices2016,asaduzzamanHowDevelopersUse2016} focus on exceptions.
They conducted empirical studies on how programmers handle exceptions in \java{} code.
The work done by~\cite{nakshatriAnalysisExceptionHandling2016} categorized them into patterns.
\cite{coelhoUnveilingExceptionHandling2015} used a more dynamic approach by analysing stack traces and code issues in \github{}.

\cite{kechagiaUndocumentedUncheckedExceptions2014} analyzed how undocumented and
unchecked exceptions cause most of the exceptions in
Android applications.

\subsection*{Programming Language Features}

Programming language design has been always a hot topic in computer science literature.
It has been extensively studied in the past decades.
There is a trend in incorporating programming features into mainstream object-oriented languages, \eg,
lambdas in \java{} 8\footnote{\url{https://docs.oracle.com/javase/specs/jls/se8/html/jls-15.html\#jls-15.27}},
\cpp{}11\footnote{\url{http://www.open-std.org/jtc1/sc22/wg21/docs/papers/2006/n1968.pdf}} and
\csharp{} 3.0\footnote{\url{https://msdn.microsoft.com/en-us/library/bb308966.aspx\#csharp3.0overview\_topic7}};
or parametric polymorphism, \ie{}, generics, in \java{} 5.%
\footnote{\url{https://docs.oracle.com/javase/1.5.0/docs/guide/language/generics.html}}\(^{,}\)\footnote{\url{http://www.oracle.com/technetwork/java/javase/generics-tutorial-159168.pdf}}
For instance, \java{} generics were designed to extend
\java's type system to allow
``a type or method to operate on objects of various types while
providing compile-time type safety''
\citep{Gosling:2013:JLS:2462622}.
However, it was later shown~\citep{aminJavaScalaType2016} that 
compile-time type safety was not fully achieved.

\cite{mazinanianUnderstandingUseLambda2017} and \cite{uesbeckEmpiricalStudyImpact2016} studied how developers use lambdas in \java{} and \cpp{} respectively.
The inclusion of generics in \java{} is closely related to collections. 
\cite{parninJavaGenericsAdoption2011,parninAdoptionUseJava2013} studied how generics were adopted by \java{} developers.
They found that the use of generics does not significantly reduce the number of type casts.

\cite{costaEmpiricalStudyUsage2017} have mined \github{} corpus to study the use and performance of collections,
and how these usages can be improved.
They found that in most cases there is an alternative usage that improves performance.

Another study about how a programming language feature is used is done in
\cite{temperoHowJavaPrograms2008}.
They conducted a study on how inheritance is used in \java{} programs.

This kind of studies give an insight of the adoption of lambdas and generics; which can drive future direction for language designers and tool builders, while providing developers with best practices.


\subsection{Unsafe Intrinsics in \java}
\label{sec:literature-review:unsafe}

Oracle provides the \smu{} class for low-level programming,
\eg{}, synchronization primitives, direct memory access methods,
array manipulation and memory usage.
Although \smu{} is not officially documented,
it is being used in both industrial applications
and research projects~\citep{korlandNoninvasiveConcurrencyJava2010,pukallFlexibleDynamicSoftware,gligoricCoDeSeFastDeserialization2011}
outside the JDK, compromising the safety of the \java{} ecosystem.

Software engineer Paul Sandoz from Oracle performed an informal analysis of
\mavencentral{} artifacts and usages in Grepcode~\citep{sandoz-personal-communication}
and conducted a unscientific user survey to study how \unsafe{} is used~\citep{psandoz14}.
The survey consists of 7 questions%
\footnote{\url{http://www.infoq.com/news/2014/02/Unsafe-Survey}} 
that help to understand what pieces of \smu{} should be mainstreamed.
In our work~\citep{mastrangeloUseYourOwn2015} we extend Sandoz' work
by performing a comprehensive study of the \mavencentral{}
software repository to analyze how and when \smu{} is being used.
This study is summarized in Chapter~\ref{cha:unsafe}.

\cite{tanSafeJavaNative2006} propose a safe variant of \jni{}.
\cite{tanEmpiricalSecurityStudy2008} and~\cite{kondohFindingBugsJava2008}
conducted an empirical security study to describe a taxonomy to classify bugs when using \jni{}.
\cite{sunNativeGuardProtectingAndroid2014} develop a method to isolate native components in Android applications.
\cite{liFindingBugsExceptional2009} analyze the discrepancy between how exceptions are handled in native code and \java{}.

\subsection{Casting}\label{sec:literature-review:casting}

Casting operations in \java{}%
\footnote{\url{https://docs.oracle.com/javase/specs/jls/se8/html/jls-15.html\#jls-15.16}}
allows the developer to view a reference at a different type as it was declared.
The related \code{instanceof} operator%
\footnote{\url{https://docs.oracle.com/javase/specs/jls/se8/html/jls-15.html\#jls-15.20.2}}---written \code{e instanceof T}---tests whether a reference \code{e} could be cast to a different type \code{T} without
throwing \code{ClassCastException} at run-time.

\cite{wintherGuardedTypePromotion2011} has implemented a
path sensitive analysis that allows the developer to avoid casting
once a guarded \code{instanceof} is provided.
He proposes four cast categorizations according to their
run-time type safety:
\emph{Guarded Casts}, \emph{Semi-Guarded Casts},
\emph{Unguarded Casts}, and \emph{Safe Casts}.
We refined this categorization to answer
our~\ref{casts:rq2} (\emph{\crqB}).
This is described in Chapter~\ref{cha:casts}.

\cite{tsantalisJDeodorantIdentificationRemoval2008} present an
Eclipse plug-in that identifies type-checking bad smells,
a "variation of an algorithm that should be executed,
depending on the value of an attribute".
They provide refactoring analysis to remove the detected smells
by introducing inheritance and polymorphism.
This refactoring will introduce casts to select
the right type of the object.

\cite{livshitsImprovingSoftwareSecurity2006,livshitsReflectionAnalysisJava2005} ``describes an approach to call graph construction for \java{} programs in the presence of reflection.''
He has devised some common usage patterns for reflection.
Most of the patterns use casts.
We plan to categorize all cast usages,
not only where reflection is used.

\cite{landmanChallengesStaticAnalysis2017} have analyzed the relevance of
static analysis tools with respect to reflection.
They conducted an empirical study to check how often the reflection
\api{} is used in real-world code.
They have devised reflection AST patterns,
which often involve the use of casts.
Finally, they argue that controlled programming experiments on
subjects need to be correlated with real-world use cases,
\eg, \github{} or \mavencentral{}.

\textbf{Controlled Experiments on Subjects.}
There is an extensive literature \perse{} in controlled experiments on subjects to understand several aspects in programming, and programming languages.
For instance,~\cite{solowayEmpiricalStudiesProgramming1984} tried to understand how expert programmers face problem solving.
\cite{buddTheoreticalEmpiricalStudies1980} made a empirical study on how effective is mutation testing.
\cite{precheltEmpiricalComparisonSeven2000} compared how a given---fixed---task was implemented in several programming languages.
\cite{latozaDevelopersAskReachability2010} realize that, in essence, programmers need to answer reachability questions to understand large codebases.
Several authors~\cite{stuchlikStaticVsDynamic2011,mayerEmpiricalStudyInfluence2012,harlinImpactUsingStaticType2017} measure whether using a static-type system improves programmers productivity.
They compare how a static and a dynamic type system impact on productivity.
The common setting for these studies is to have a set of programming problems.
Then, let a group of developers solve them in both a static and dynamic languages.
For this kind of studies to reflect reality, the problems to be solved need to be representative of the real-world code.
Having artificial problems may lead to invalid conclusions.
The work by~\cite{wuHowTypeErrors2017,wuLearningUserFriendly2017} goes towards this direction. 
They have examined programs written by students to understand real debugging conditions. 
Their focus is on ill-typed programs written in \haskell{}.
