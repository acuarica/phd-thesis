\begin{pattern}{KnownReturnType}
There are cases when a method's return type is less specific than the actual return type value.
This is often to hide implementation details,
but may also be because the method overrides another method with a less-specific type and the return type is not changed covariantly.

This pattern is used to cast from the method's return type to the \emph{known} actual return type.
This pattern is characterized by a method that always returns a value of the same type,
a subtype of the declared return type,
regardless of the context or the arguments to the method call.

\instances{}
In the following example,
a cast is performed to a call to the \code{getRealization} method (line 1).
Its implementation returns a value of type \code{CubeInstance} (line 9).

%https://lgtm.com/projects/g/apache/kylin/snapshot/dist-45150010-1524814812150/files/storage-hbase/src/main/java/org/apache/kylin/storage/hbase/steps/HBaseMRSteps.java?sort=name&dir=ASC&mode=heatmap#L211
\def\urlvar{http://bit.ly/apache_kylin_2SIjooO}
\begin{minted}[highlightlines=1]{java}
final List<CubeSegment> mergingSegments = ((CubeInstance) seg.getRealization())
            .getMergingSegments((CubeSegment) seg);
public class CubeSegment implements IBuildable, ISegment, Serializable {
    // [...]
    private CubeInstance cubeInstance;
    // [...]
    public IRealization getRealization() {
        return cubeInstance;
    }
}
public class CubeInstance
            extends RootPersistentEntity implements IRealization, IBuildable {
    // [...]
} #\urlbox#
\end{minted}

In the following example,
a cast is applied to the result of an invocation to the \code{createDebugTarget} method.
This method is known to return a value of type \code{PHPDebugTarget},
which implements \code{IPHPDebugTarget}.

%https://lgtm.com/projects/g/eclipse/pdt/snapshot/dist-19313119-1524814812150/files/plugins/org.eclipse.php.debug.core/src/org/eclipse/php/internal/debug/core/zend/communication/DebugConnection.java?sort=name&dir=ASC&mode=heatmap#L795
\def\urlvar{http://bit.ly/eclipse_pdt_2Ekeu9v}
\begin{minted}[highlightlines=1]{java}
debugTarget = (PHPDebugTarget) createDebugTarget(/* [...] */);

protected IDebugTarget createDebugTarget(/* [...] */) throws CoreException {
    return new PHPDebugTarget(/* [...] */);
} #\urlbox#
\end{minted}

In some situations,
an \api{} method is designed to return an abstract class or interface.
This \api{} allows the developer to then choose which implementation to use at run time.
The following example shows this situation.
The cast is applied to the \code{getLogger} method---with return type \code{org.slf4j.Logger}---in line 4.
But the developer set up the application to use \code{ch.qos.logback.classic.Logger} instead.

%https://lgtm.com/projects/g/skylot/jadx/snapshot/dist-41240110-1524814812150/files/jadx-gui/src/main/java/jadx/gui/utils/LogCollector.java?sort=name&dir=ASC&mode=heatmap#L22
\def\urlvar{http://bit.ly/skylot_jadx_2HIoR9X}
\begin{minted}[highlightlines=4]{java}
import ch.qos.logback.classic.Logger;
import org.slf4j.LoggerFactory;

Logger rootLogger = (Logger) LoggerFactory.getLogger(Logger.ROOT_LOGGER_NAME);
#\urlbox#
\end{minted}


\detection{}
Similar to the \nameref{pat:Factory} pattern,
\thisp{} requires analysis of the method implementation called in the cast expression.
Expressing this kind of analysis in \ql{} becomes impractical.


\issues{}
This pattern usually indicates an abstraction violation:
the caller needs to know the method implementation to know the correct target type.

The \nameref{pat:CovariantReturnType} pattern can be considered a special case of this pattern where the return type is known to vary with the receiver type.
Like that pattern,
associated types~\citep{chakravartyAssociatedTypeSynonyms2005}
in languages like \haskell{} or \rust{} could be used to avoid the cast.

\end{pattern}