\begin{pattern}{CovariantGeneric}
The \thisp{} pattern occurs when a cast is used
to use an invariant generic type as if it were covariant.
It can be implemented by
casting a generic type like \code{List<S>} to a raw type (\code{List}),
which can then be assigned to a variable of \code{List<T>},
where \code{S} is a subtype of \code{T}.


\instances{}
In the following snippet we show an instance of this pattern.
%https://lgtm.com/projects/g/spockframework/spock/snapshot/dist-7950040-1524814812150/files/spock-core/src/main/java/org/spockframework/compiler/WhereBlockRewriter.java?sort=name&dir=ASC&mode=heatmap#L360
\def\urlvar{http://bit.ly/spockframework_spock_2UYEsF5}
\begin{minted}[highlightlines=2]{java}
private final List<VariableExpression> dataProcessorVars = new ArrayList<>();
new ArrayExpression(ClassHelper.OBJECT_TYPE, (List) dataProcessorVars);
public class ArrayExpression extends Expression {
    public ArrayExpression(ClassNode elementType, List<Expression> exprs) {}
} #\urlbox#
\end{minted}


\issues{}
\citet{altidorTamingWildcardsCombining2011} define a type system that adds
definition-site variance to \java{}.
This could reduce the need for this pattern,
although not in the instance above since \code{List} is invariant.
Scala addresses this issue by taking advantage of definition-site variance in
the collections library, for instance by providing a covariant immutable list type.
  % Kotlin addresses this problem by using mixed-site variance.

\end{pattern}
