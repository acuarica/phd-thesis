\begin{pattern}{RemoveWildcard}
A cast is in the \thisp{} pattern when a \emph{wildcard type} is used rather than a generic type.


\instances{}
In the following example,%
\def\urlvar{http://bit.ly/eclipse_jetty_project_2WMI0Ld}
\code{unit} is declared as \code{Unit<?>},
but to actually be able to use it a cast to a concrete type is needed.

%https://lgtm.com/projects/g/eclipse/jetty.project/snapshot/dist-1793550978-1524814812150/files/jetty-util/src/main/java/org/eclipse/jetty/util/PathWatcher.java?sort=name&dir=ASC&mode=heatmap#L752
\begin{minted}[highlightlines=1,linenos=false]{java}
copy.setUnitOfMeasure( (Unit<Length>) unit );
#\urlbox#
\end{minted}


\detection{}
The following query detects the \thisp{} pattern.
The query checks that the type of the cast operand is a wildcard,
or a parameterized type containing a wildcard.

\begin{listing}
\begin{minted}{\qllexer}
predicate containsWildcard(Type t) { #\qlbox#
	t instanceof Wildcard or
	containsWildcard( t.(ParameterizedType).getATypeArgument() )
}

class RemoveWildcardCast extends #\qlref{Cast}# {
	RemoveWildcardCast() {
		containsWildcard(getExpr().getType())
	}
}
\end{minted}
\caption{Detection of the \thisp{} pattern.}
\label{lst:ql:RemoveWildcardCast}
\end{listing}


\issues{}
Wildcard types are a form of existential type and consequently can limit
access to members of a generic type.
Casts are used to restore access at a particular type.

Since this pattern is an unchecked cast,
the discussion about compiler-inserted casts and \emph{blame} is similar to the \nameref{pat:UseRawType} pattern.

\end{pattern}
