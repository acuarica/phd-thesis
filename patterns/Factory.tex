\begin{pattern}{Factory}
Creates an object based on some arguments to a method call.
Since the arguments are known at compile time, cast of the specific type.
In this pattern, the arguments resemble a ``type tag'' descriptor (cf.
\nameref{pat:Typecase}).

This pattern is characterized by a cast of a method call passing one or more arguments.
The method call needs to create an object based on those arguments.
Usually the arguments that determine the run-time type to be returned are known at compile time.

\instances{}
The following snippet
shows an instance of the \thisp{} pattern.
The cast is applied to the result of invoking \code{keyPair.get\-Private}
(line 6).
The variable \code{keyPair} is assigned the result of \code{pairGen.\-generateKeyPair} (line 3).
At the same time, the \code{pairGen} variable is assigned the value returned by \code{KeyPairGenerator.getInstance("RSA")}.
The argument \code{"RSA"} indicates the algorithm to use.
The method%
\footnote{\url{https://docs.oracle.com/javase/8/docs/api/java/security/KeyPair.html\#getPrivate()}}
will return a reference to the private key component,
and this is determined by the algorithm argument described above.

%https://lgtm.com/projects/b/connect2id/oauth-2.0-sdk-with-openid-connect-extensions/snapshot/dist-1311020143-1524814812150/files/src/test/java/com/nimbusds/oauth2/sdk/jose/jwk/RemoteJWKSetTest.java?sort=name&dir=ASC&mode=heatmap#L242
\def\urlvar{http://bit.ly/connect2id_oauth_2_0_sdk_with_2HvRlUX}
\begin{minted}[highlightlines=6]{java}
KeyPairGenerator pairGen = KeyPairGenerator.getInstance("RSA");
pairGen.initialize(1024);
KeyPair keyPair = pairGen.generateKeyPair();
// [...]
RSAKey rsaJWK2 = new RSAKey.Builder((RSAPublicKey) keyPair.getPublic())
        .privateKey((RSAPrivateKey) keyPair.getPrivate())
        .keyID("2")
        .build(); #\urlbox#
\end{minted}

Similar to the above snippet,
the next example shows an instance of this pattern where a cast is performed on the result of the \code{openConnection} method%
\footnote{\url{https://docs.oracle.com/javase/8/docs/api/java/net/URL.html\#openConnection--}}
(line 2).
The method is declared to return \code{URLConnection} but can return a more specific type based on the URL string.
The \code{openConnection} method is applied to the \code{url} variable,
which is assigned in line 1 using the \code{URL} constructor.
The argument to the constructor is an \code{http} URL,
thus the result is cast of \code{HttpURLConnection}.
%https://lgtm.com/projects/g/apache/hadoop/snapshot/dist-956730001-1524814812150/files/hadoop-yarn-project/hadoop-yarn/hadoop-yarn-server/hadoop-yarn-server-resourcemanager/src/test/java/org/apache/hadoop/yarn/server/resourcemanager/webapp/TestRMWebServicesHttpStaticUserPermissions.java?sort=name&dir=ASC&mode=heatmap#L138
\def\urlvar{http://bit.ly/apache_hadoop_2E6KY6T}
\begin{minted}[highlightlines=2]{java}
URL url = new URL("http://localhost:8088/ws/v1/cluster/apps");
HttpURLConnection conn = (HttpURLConnection) url.openConnection();
#\urlbox#
\end{minted}

The following example shows how a cast (line 3) is being determined by the argument to the \code{CertificateFactory.getInstance} method (line 1).
The argument is the string \code{"X.509"},
therefore the method \code{generateCRL} will return a value of type \code{X509CRL}.

%https://lgtm.com/projects/g/bcgit/bc-java/snapshot/dist-20740003-1524814812150/files/prov/src/test/java/org/bouncycastle/jce/provider/test/X509LDAPCertStoreTest.java?sort=name&dir=ASC&mode=heatmap#L241
\def\urlvar{http://bit.ly/bcgit_bc_java_2TEVScM}
\begin{minted}[highlightlines=3]{java}
CertificateFactory cf = CertificateFactory.getInstance("X.509", "BC");
// [...]
X509CRL crl = (X509CRL)cf.generateCRL(new ByteArrayInputStream(directCRL));
#\urlbox#
\end{minted}

In our last example the cast instance (line 2) is applied to the result of \code{parse} method.
The return type of \code{parse} is of type \code{Statement}, but,
since the statement is a \code{SELECT} statement,
the value returned by the \code{parse} method is known to be of type \code{Select}
and the cast should succeed.
%https://lgtm.com/projects/g/JSQLParser/JSqlParser/snapshot/dist-43250114-1524814812150/files/src/test/java/net/sf/jsqlparser/test/select/SelectTest.java?sort=name&dir=ASC&mode=heatmap#L437
\def\urlvar{http://bit.ly/JSQLParser_JSqlParser_2TecMyB}
\begin{minted}[highlightlines=2]{java}
statement = "SELECT * FROM mytable WHERE mytable.col = 9 LIMIT :param_name";
select = (Select) parserManager.parse(new StringReader(statement));
public class Select implements Statement {
        // [...]
}
public class CCJSqlParserManager implements JSqlParser {
    @Override
    public Statement parse(Reader statementReader) throws JSQLParserException {
        // [...]
    }
} #\urlbox#
\end{minted}

In some cases of this pattern, a cast is applied to a method invocation where one of its arguments is a class literal.
The target type of the cast is determined by this class literal,
like in the following snippets.
%https://lgtm.com/projects/g/liferay/liferay-ide/snapshot/dist-2980259-1524814812150/files/tools/plugins/com.liferay.ide.server.ui/src/com/liferay/ide/server/ui/handlers/RedeployHandler.java?sort=name&dir=ASC&mode=heatmap#L64
\def\urlvar{http://bit.ly/liferay_liferay_ide_2FMG0f6}
\begin{minted}[highlightlines=2-3]{java}
final ILiferayServerBehavior liferayServerBehavior =
                (ILiferayServerBehavior) moduleServer.getServer()
                        .loadAdapter( ILiferayServerBehavior.class, null );
#\urlbox#
\end{minted}

%https://lgtm.com/projects/g/robovm/robovm/snapshot/dist-39650108-1524814812150/files/cocoatouch/src/main/java/org/robovm/apple/corevideo/CVBufferMovieTime.java?sort=name&dir=ASC&mode=heatmap#L66
\def\urlvar{http://bit.ly/robovm_robovm_2FMFWvS}
\begin{minted}[highlightlines=1]{java}
CFArray o = (CFArray) CFType.Marshaler.toObject(CFArray.class, handle, flags);
#\urlbox#
\end{minted} 


\detection{}
The detection of this pattern requires to analyse the factory method being called.
This is not always possible in \ql{},
since \ql{} does not analyse project dependencies.

In several instances, to manually determine when a cast belongs to this pattern,
we had to look-up the method implementation in external source code repositories.

\issues{}
In some situations,
the use of this pattern can be seen as breaking the contract between the caller and the callee.
This happens because the caller needs to know how the method is implemented in order to determine the run-time return type.
In \thisp{}, there is a known type hierarchy 
below the return type and the caller casts to a known subtype
in that hierarchy based on the arguments passed into the factory method.

The \nameref{pat:KnownReturnType} pattern is similar to \thisp{},
since both depend on the knowledge that a method returns a more specific type.

This pattern is prevalent in test code \nFactoryPatternTestPerc{}\%.
This is because when testing,
known parameters are given to factory methods.
In these situations, a test method needs to know a more specific type---by using a cast---to properly check for a test condition. 

\end{pattern}
