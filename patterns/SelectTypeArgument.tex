\begin{pattern}{SelectTypeArgument}
This pattern is used to prevent the compiler
from inferring a collection element type that is too precise.
It guides the type checker to provide the right return type of a generic method.


\instances{}
In the following snippet, an upcast is performed to ensure that the
inferred type of the call to \code{singletonList} (line 3)
is \code{List<Framedata>} rather than \code{List<FrameBuilder>}.
Because \code{List<FrameBuilder>} is not a subtype of \code{List<Framedata>},
a compilation error would occur if the cast were omitted.

%https://lgtm.com/projects/g/arpruss/raspberryjammod/snapshot/dist-1796220064-1524814812150/files/build/sources/java/org/java_websocket/drafts/Draft_10.java?sort=name&dir=ASC&mode=heatmap#L157
\def\urlvar{http://bit.ly/arpruss_raspberryjammod_2USL7Ai}
\begin{minted}[highlightlines=3]{java}
public List<Framedata> createFrames( String text, boolean mask ) {
    FrameBuilder curframe = new FramedataImpl1();
    return Collections.singletonList( (Framedata) curframe );
}
public interface FrameBuilder extends Framedata {
} #\urlbox#
\end{minted}

Similar to the previous example, in the following case,
an upcast is performed to change the return type of the
\code{Matcher<T> equalTo(T)} method.

%https://lgtm.com/projects/g/jfaster/mango/snapshot/dist-1793930711-1524814812150/files/src/test/java/org/jfaster/mango/operator/UpdateOperatorTest.java?sort=name&dir=ASC&mode=heatmap#L125
\def\urlvar{http://bit.ly/jfaster_mango_2EhXzUW}
\begin{minted}[highlightlines=5]{java}
@Test
public void testUpdateReturnBoolean() throws Exception {
    /* [...] */
    List<Object> args = boundSql.getArgs();
    assertThat(args.get(0), equalTo((Object) "ash"));
}
public static <T> Matcher<T> equalTo(T operand) {
    // [...]
} #\urlbox#
\end{minted}

Instead of an upcast, in this example,
a cast to \code{null} is performed to change the return type.
This use case resembles the \nameref{pat:SelectOverload} pattern.

%https://lgtm.com/projects/g/EngineHub/WorldGuard/snapshot/dist-1795351250-1524814812150/files/worldguard-legacy/src/test/java/com/sk89q/worldguard/protection/FlagValueCalculatorTest.java?sort=name&dir=ASC&mode=heatmap#L1024
\def\urlvar{http://bit.ly/EngineHub_WorldGuard_2IVUOx1}
\begin{minted}[highlightlines=1]{java}
assertThat(result.queryValue(memberOne, DefaultFlag.BUILD), is((State) null));
public static <T> Matcher<T> is(T value) {
    // [...]
} #\urlbox#
\end{minted}

\detection{}
The Query~\ref{lst:ql:CovariantGenericCast} detects when a cast is used to select the return type of a generic method.

\begin{listing}
\begin{minted}{\qllexer}
class CovariantGenericCast extends #\qlref{Cast}# { #\qlbox#
	Argument arg;
	Call call;
	Callable m;
	CovariantGenericCast() {
		this = arg and
		call = arg.getCall() and
		arg.getCall().getCallee() = m and
		(
			m.getReturnType().(ParameterizedType).getATypeArgument() =
					m.getParameterType(arg.getPosition()).(TypeVariable) or
			m.getReturnType().(TypeVariable) =
					m.getParameterType(arg.getPosition()).(TypeVariable)
		)
	}
}
\end{minted}
\caption{Query to detect the \thisp{} pattern.}
\label{lst:ql:CovariantGenericCast}
\end{listing}


\issues{}
In some cases, instead of casting, this pattern could be avoided using explicit type arguments,
\eg{}, \code{Collections.<Framedata>singletonList(curframe)}.
With \java{} 8 this cast became unnecessary due to better type inference.%
\urlnote{https://docs.oracle.com/javase/specs/jls/se8/html/jls-18.html\#jls-18.5}
\end{pattern}
