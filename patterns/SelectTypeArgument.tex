\begin{pattern}{SelectTypeArgument}
This pattern is used to prevent the compiler
from inferring a collection element type that is too precise.
It guides the type checker to provide the right return type of a generic method.


\instances{}
In the following snippet, an upcast is performed to ensure that the
inferred type of the call to \code{singletonList} (line 3)
is \code{List<Framedata>} rather than \code{List<FrameBuilder>}.
Because \code{List<FrameBuilder>} is not a subtype of \code{List<Framedata>},
a compilation error would occur if the cast were omitted.

%https://lgtm.com/projects/g/arpruss/raspberryjammod/snapshot/dist-1796220064-1524814812150/files/build/sources/java/org/java_websocket/drafts/Draft_10.java?sort=name&dir=ASC&mode=heatmap#L157
\def\urlvar{http://bit.ly/arpruss_raspberryjammod_2USL7Ai}
\begin{minted}[highlightlines=3]{java}
public List<Framedata> createFrames( String text, boolean mask ) {
    FrameBuilder curframe = new FramedataImpl1();
    return Collections.singletonList( (Framedata) curframe );
}
public interface FrameBuilder extends Framedata { } #\urlbox#
\end{minted}


\issues{}
In some cases, instead of casting, this pattern could be avoided using explicit type arguments,
\eg{}, \code{Collections.<Framedata>singletonList(curframe)}.
With \java{} 8 this cast became unnecessary due to better type inference.%
\urlnote{https://docs.oracle.com/javase/specs/jls/se8/html/jls-18.html\#jls-18.5}
\end{pattern}
