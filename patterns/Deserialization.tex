\begin{pattern}{Deserialization}
This pattern is used to deserialize an object at run time.
In its more common form,
this pattern is characterized for a cast to the \code{readObject} method on an \code{ObjectInputStream} object.

\instances{}
The following example shows how the \thisp{} pattern is used to create objects from a file system (line 9).

%https://lgtm.com/projects/g/internetarchive/heritrix3/snapshot/dist-12140105-1524814812150/files/engine/src/test/java/org/archive/crawler/datamodel/CrawlURITest.java?sort=name&dir=ASC&mode=heatmap#L83
\def\urlvar{http://bit.ly/internetarchive_heritrix3_2SF4j7k}
\begin{minted}[highlightlines=5]{java}
FileInputStream fis = new FileInputStream(serialize);
ObjectInputStream ois = new ObjectInputStream(fis);
CrawlURI deserializedCuri = (CrawlURI)ois.readObject();
deserializedCuri = (CrawlURI)ois.readObject();
deserializedCuri = (CrawlURI)ois.readObject();
assertEquals("...", this.seed.toString(), deserializedCuri.toString());
#\urlbox#
\end{minted}


\detection{}
The following query detects \thisp{} with the fact that the \code{readObject} method family is used to deserialize objects.
For other deserialization frameworks,
it would require to analyse external dependencies.

\begin{listing}
\begin{minted}{\qllexer}
class DeserializationCast extends #\qlref{Cast}# { #\qlbox#
	DeserializationCast() {
		getExprOrDef().(MethodAccess).getMethod() instanceof ReadObjectMethod
	}
}
\end{minted}
\caption{Detection of the \thisp{} pattern.}
\label{lst:ql:DeserializationCast}
\end{listing}


\issues{}
The serialization API dates back to \java{} 1.1 in 1997.
Since then, newer serialization APIs have been developed.
For instance, Apache Avro%
\footnote{\url{https://avro.apache.org/docs/current/}}
uses generics and class literals to specify the expected type of an object read. 
In some languages,
type-safe serialization and deserialization boilerplate code can be automatically generated,
for instance in \rust{},
the Serde library\footnote{\url{https://serde.rs/}}
can generate code to serialize most data types
in a variety of formats.

Both this pattern and the \nameref{pat:NewDynamicInstance} pattern create objects by using reflection.
While it might be considered a special case of \nameref{pat:KnownReturnType}, 
\thisp{} differs in that the run-time result type of the \code{readObject} depends on the state of the input stream and can change depending on context.
 
\end{pattern}