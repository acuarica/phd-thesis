\chapter{Automatic Detection of Patterns using \ql{}}\label{ap:ql}

Query all observations defined.
Columns 'patterns', 'tags', and 'show' need to be concatenated since they might have more than one result.
By concatenating the results in each column, we keep one line for observation.
We sort the results by 'obs' to keep the observations in the same order across runs.
This allows us to easily identify observations every time we re-run the query after adding either a tag or a pattern.

\begin{listing}
\begin{minted}{sql}
from Category category, Top obs #\qlbox#
where obs = category.getAnObs()
select
	category,
	concat(obs.(Pattern).getPattern(), "|") as patterns,
	concat(obs.(Tag).getTag(), "|") as tags,
	concat(obs.(Show).show(), " | ") as show,
	obs,
	getLgtmUrl(obs) as url
order by obs
\end{minted}
\caption{Main query to run all observation}
\end{listing}

\begin{listing}
\begin{minted}{java}
abstract class Pattern extends Top {
	abstract string getPattern();
	string getLgtmUrl() { result = getLgtmUrl(this) }
}

abstract class CastTag extends Tag, Cast { }
abstract class CastPattern extends Pattern, Cast { }
\end{minted}
\end{listing}