
\begin{abstract}
The main goal of a static type system is to prevent certain kind of errors from happening at run-time.
A type system is formulated as a set of constraints that gives any expression or term in a program a well-defined type.
Yet mainstream programming languages are endowed with type systems that
provide the means to circumvent their constraints through the \emph{unsafe intrinsics} and \emph{casting} mechanisms.

We want to understand how and when developers circumvent these constraints.
This knowledge can be:
\begin{inparaenum}[a)]
\item a recommendation for current and future language designers
to make informed decisions
\item a reference for tool builders, \eg{},
by providing more precise or new refactoring analyses,
\item a guide for researchers to test new language features,
or to carry out controlled programming experiments, and
\item a guide for developers for better practices.
\end{inparaenum}

We plan to empirically study how these two mechanisms---unsafe intrinsics and casting---are used by \java{} developers to circumvent the static type system.
We have devised usage patterns for both a subset of unsafe intrinsics and casting.
Usage patterns are recurrent programming idioms to solve a specific issue.
We believe that having usage patterns can help us to better categorize use cases and
thus understand how those features are used.

% These patterns can provide an insight on how the language is being used by developers in real-world applications.
\end{abstract}

% item useful to make informed decisions for current \& future language designers, not only \java{},

% Programming languages offer a wide range of features that aim to improve programmers productivity.
% However, to better drive the future evolution of any programming language,
% we believe it is AVOID:paramount to have a thorough understanding of how these features are actually being used in real codebases.

% Understanding how developers make use of language features can be helpful to a AVOID:broad audience besides language designers.
% It can aid tool builders to make more realistic assumptions;
% researchers to improve the state-of-the-art;
% and developers to implement more efficient and effective solutions by providing them best practices.

% In this proposal, we target two specific features, namely, casting and the unsafe API.
% \java{} features, namely, \emph{casting}, \emph{reflection}, \emph{exception handling} and the \emph{unsafe} \api{}.
% We give the rationale behind our decision on why we chose these features.
% We plan to devise language and API usage patterns at large-scale to properly assess this broad audience.
% We hope that having a better understanding on how these features are used, we can make informed decisions for these driving forces.
