\begin{abstract}
The main goal of a static type system is to prevent certain kinds of errors from happening at run time.
A type system is formulated as a set of constraints that gives any expression or term in a program a well-defined type.
Yet mainstream programming languages are endowed with type systems that
provide the means to loosen their static typing constraints through the \emph{unsafe intrinsics} and \emph{reflective capabilities} mechanisms.

We want to understand how and when developers give up these static constraints.
This knowledge can be:
\begin{inparaenum}[a)]
\item a recommendation for current and future language designers
to make informed decisions,
\item a reference for tool builders, \eg{},
by providing more precise or new refactoring analyses,
\item a guide for researchers to test new language features,
or to carry out controlled programming experiments, and
\item a guide for developers for better practices.
\end{inparaenum}

In this dissertation we report two empirical studies to understand how these mechanisms---unsafe intrinsics and reflective capabilities---are used by \java{} developers when the static type system becomes too strict.
We have devised usage patterns for both a subset of unsafe intrinsics and reflective capabilities.
Usage patterns are recurrent programming idioms to solve a specific issue.
We believe that having usage patterns can help us to better categorize use cases and
thus understand how those features are used.

\end{abstract}

