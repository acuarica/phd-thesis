\newcommand{\statartscount}{$959,300$}
\newcommand{\statmavensize}{$1.7$ TB}
\newcommand{\statuniqueartscount}{$106,574$}
\newcommand{\statrepouniquearts}{$86,479$}
\newcommand{\statreposize}{$74$ GB}
\newcommand{\statdeponunsafe}{$25\%$}

\newcommand{\statunsafeuses}{$48,490$}
\newcommand{\statunsafecs}{$48,139$}
\newcommand{\statunsafefieldusages}{$351$}
\newcommand{\statunsafearts}{$817$}

\newcommand{\statartswithpominfo}{$47,127$}

\newcommand{\statartsdepuns}{$21,297$}
\newcommand{\statpercartsdepunsoverpominfo}{47\%}
\newcommand{\statpercartsdepunsovertotal}{25\%}

\newcommand{\statartsdepunsapp}{$19,173$}
\newcommand{\statpercartsdepunsoverpominfoapp}{41\%}
\newcommand{\statpercartsdepunsovertotalapp}{22\%}


\chapter{The \java{} Unsafe \api{} in the Wild}
\label{cha:unsafe}

The \java{} Virtual Machine (\jvm{}) executes \java{} bytecode and
provides other services for programs written in
many programming languages, including \java{}, \scala{}, and \clojure{}.
The \jvm{} was designed to provide strong safety guarantees.
However, many widely used \jvm{} implementations expose an \api{} that
allows the developer to access low-level,
unsafe features of the \jvm{} and underlying hardware,
features that are unavailable in safe \java{} bytecode.
This \api{} is provided through an undocumented%
\footnote{\url{http://www.oracle.com/technetwork/java/faq-sun-packages-142232.html}}
class, \smu{}, in the \java{} reference implementation produced by Oracle.

Other virtual machines provide similar functionality.
For example, the \csharp{} language provides an \code{unsafe} construct
on the .NET platform,%
\footnote{\url{https://msdn.microsoft.com/en-us/en-en/library/chfa2zb8(v=vs.90).aspx}}
and \racket{} provides unsafe operations.%
\footnote{\url{http://docs.racket-lang.org/reference/unsafe.html}}

The operations \smu{} provides can be dangerous,
as they allow developers to circumvent the safety guarantees provided by
the \java{} language and the \jvm{}.
If misused, the consequences can be resource leaks, deadlocks,
data corruption, and even \jvm{} crashes.%
\footnote{\url{https://groups.google.com/d/msg/elasticsearch/Nh-kXI5J6Ek/WXIZKhhGVHkJ}}
\footnote{\url{https://github.com/EsotericSoftware/kryo/issues/219}}
\footnote{\url{https://github.com/dain/snappy/issues/24}}
\footnote{\url{https://netbeans.org/bugzilla/show_bug.cgi?id=229655}}
\footnote{\url{https://netbeans.org/bugzilla/show_bug.cgi?id=244914}}

We believe that \smu{} was introduced to provide better performance and
more capabilities to the writers of the \java{} runtime library.
However, \smu{} is increasingly being used in third-party
frameworks and libraries.
Application developers who rely on \java{}'s safety guarantees have to
trust the implementers of the language runtime environment
(including the core runtime libraries).
Thus the use of \smu{} in the runtime libraries is no more risky than
the use of an unsafe language to implement the \jvm{}.
However, the fact that more and more ``normal'' libraries are using
\smu{} means that application developers have to trust a growing
community of third-party \java{} library developers to not
inadvertently tamper with the fragile internal state of the \jvm{}.

Given that the benefits of safe languages are well known,
and the risks of unsafe languages are obvious,
why exactly does one need unsafe features in third-party libraries?
Are those features used in real-world code?
If yes, how are they used, and what are they used for?

We studied a large repository of \java{} code, \mavencentral{},
to answer these questions.
We have analyzed \statreposize{} of compiled \java{} code,
spread over \statrepouniquearts{} \java{} archives,
to determine how \java{}'s unsafe capabilities are used in real-world
libraries and applications.
We found that $25\%$ of \java{} bytecode archives depend on unsafe
third-party \java{} code, and thus \java{}'s safety
guarantees cannot be trusted.
We identify $14$ different usage patterns of \java{}'s unsafe capabilities,
and we provide supporting evidence for why real-world code needs these capabilities.
Our long-term goal is to provide a strong foundation
to make informed decisions in the future evolution of the \java{} language and virtual machine,
and for the design of new language features to regain safety in \java{}.

We have already published our work on how developers use the Unsafe \api{} in \java{}~\citep{mastrangeloUseYourOwn2015}.
In this thesis we outline the risks of using the \unsafe{} \api{} in Section~\ref{sec:unsafe:background}.
Then we answer \ref{unsafe:rq1} in Section~\ref{sec:unsafe:overview}.
To answer \ref{unsafe:rq2}, first we introduce our methodology and the patterns we found in Sections~\ref{sec:unsafe:methodology} and \ref{sec:unsafe:patterns} respectively, to then present how the patterns we found could be implemented in a safer way in Section~\ref{sec:unsafe:discussion}.

\input{chapters/unsafe/sec-background}

\section{Is Unsafe Used?}
\label{sec:unsafe:overview}

To answer~\ref{unsafe:rq1} (\emph{\urqA})
we need to determine whether and how Unsafe is actually used in real-world third-party \java{} libraries,
and to what degree real-world applications directly and indirectly depend on such unsafe libraries.
To achieve our goal, several elements are needed.

\textbf{Code Repository.}
As a code base representative of the ``real world'',
we have chosen the Maven Central software repository.
% The rationale behind this decision is that a large number of well-known \java{} projects deploy to Maven Central using Apache Maven.
% Besides code written in \java{}, projects written in \lang{Scala} are also deployed to Maven Central using the Scala Build Tool (sbt).
% Moreover, Maven Central is the largest \java{} repository\footnote{\url{http://www.modulecounts.com/}}
% , and it contains projects from the most popular source code management repositories, like \github{} and \sourceforge{}.

\textbf{Artifacts.}
In Maven, an artifact is the output of the build procedure of a project.
% An artifact can be any type of file, ranging from a \emph{.pdf} to a \emph{.zip} file.
% However,
Artifacts are usually \emph{.jar} files,
which archive compiled \java{} bytecode stored in \emph{.class} files.

\textbf{Bytecode Analysis.}
% We examine these kinds of artifacts to analyze how they use \code{sun.misc.\-Unsafe}.
We use a bytecode analysis library to search for method call sites and field accesses of the \code{sun.misc.Unsafe} class.

\textbf{Dependency Analysis.}
We define the impact of an artifact as how many artifacts depend on it,
either directly or indirectly.
This helps us to define the impact of artifacts that use \code{sun.misc.Unsafe},
and thus the impact \code{sun.misc.Unsafe} has on real-world code overall.

% \textbf{Usage Pattern Detection.}
% After all call sites and field accesses are found,
% we analyze this information to discover usage patterns.
% It is common that an artifact exhibits more than one pattern.
% Our list of patterns is not exhaustive.
% We have manually investigated the source code of the 100 highest-impact artifacts using \code{sun.misc.Unsafe} to understand why and how they are using it.




Our analysis found $48,490$ uses of \code{sun.misc.Unsafe} --- $48,139$ call sites and $351$ field accesses --- distributed over $817$ different artifacts.
This initial result shows that Unsafe is indeed used in third-party code.

We use the dependency information to determine the impact of the artifacts that use \code{sun.misc.Unsafe}.
We rank all artifacts according to their impact (the number of artifacts that directly or indirectly depend on them).
High-impact artifacts are important;
a safety violation in them can affect any artifact that directly or indirectly depends on them.
We find that while overall about $1\%$ of artifacts directly use Unsafe,
for the top-ranked $1000$ artifacts, $3\%$ directly use Unsafe.
Thus, Unsafe usage is particularly prevalent in high-impact artifacts, artifacts that can affect many other artifacts.

Moreover, we found that $21,297$ artifacts ($47\%$ of the $47,127$ artifacts with dependency information, or $25\%$ of the $86,479$ artifacts we downloaded) directly or indirectly depend on \code{sun.misc.Unsafe}.
Excluding language artifacts, numbers do not change much:
Instead of $21,297$ artifacts, we found $19,173$ artifacts,
$41\%$ of the artifacts with dependency information, or $22\%$ of artifacts downloaded.
Thus, \code{sun.misc.Unsafe} usage in third-party code indeed impacts a large fraction of projects.

\subsection*{Which Features of \unsafe{} Are Actually Used?}

\begin{figure}[!ht]
\includegraphics[width=0.5\columnwidth]{chapters/unsafe/usage-maven-methods}
\caption{\smu{} method usage on \mavencentral{}}
\label{fig:overview}
\end{figure}

\begin{figure}[!ht]
\includegraphics[width=0.7\columnwidth]{chapters/unsafe/usage-maven-fields}
\caption{\smu{} field usage on \mavencentral{}}
\label{fig:overview-field}
\end{figure}

Figures~\ref{fig:overview} and~\ref{fig:overview-field} show all instance methods and static fields of \smu{}. For each member we show how many call sites or field accesses we found across the artifacts. The class provides $120$ public instance methods and $20$ public fields (version 1.8 update 40). The figure only shows $93$ methods because the $18$ methods in the \smugroup{Heap Get} and \smugroup{Heap Put} groups, and \member{staticFieldBase} are overloaded, and we combine overloaded methods into one bar.

We show two columns, \smugroup{Application} and \smugroup{Language}.
The \smugroup{Language} column corresponds to language implementation artifacts while the \smugroup{Application} column corresponds to the rest of the artifacts.

We categorized the members into groups, based on the functionality they provide:

\begin{itemize}
  
\item The \smugroup{Alloc} group contains only the  \member{allocateInstance} method, 
which allows the developer to allocate a \java{} object without executing a constructor.
This method is used 181 times: 180 in \smugroup{Application} and 1 in \smugroup{Language}.

\item The \smugroup{Array} group contains methods and fields for computing relative addresses of array elements.
The fields were added as a simpler and potentially faster alternative in a more recent version of \unsafe{}.
The value of all fields in this group are constants initialized with the result of a call to either \member{arrayBaseOffset} or \member{arrayIndexScale} in the \smugroup{Array} group.
The figures show that the majority of sites still invoke the methods instead of accessing the corresponding constant fields.

\item The \smugroup{CAS} group contains methods to atomically compare-and-swap a \java{} variable. These operations are implemented using processor-specific atomic instructions. For instance, on \emph{x86} architectures, \member{compareAndSwapInt} is implemented using the \texttt{CMPXCHG} machine instruction. Figure~\ref{fig:overview} shows that these methods represent the most heavily used feature of \unsafe{}.

\item Methods of the \smugroup{Class} group are used to dynamically load and check \java{} classes. They are rarely used, with \member{defineClass} being used the most.

\item 
The methods of the \smugroup{Fence} group provide memory fences to ensure loads and stores are visible to other threads.
These methods are implemented using processor-specific instructions.
These methods were introduced only recently in \java{} 8, which explains their limited use in our data set.
We expect that their use will increase over time and that other operations, such as those in the \smugroup{Ordered Put}, or \smugroup{Volatile Put} groups will decrease as programmers use the lower-level fence operations.

\item The \smugroup{Fetch \& Add} group, like the \smugroup{CAS} group, allows the programmer to atomically update a \java{} variable. This group of methods was also added recently in \java{} 8. We expect their use to increase as programmers replace some calls to methods in the \smugroup{CAS} group with the new functionality.

\item The \smugroup{Heap} group methods are used to directly access memory in the \java{} heap. The \smugroup{Heap Get} and \smugroup{Heap Put} groups allow the developer to load and store a Java variable. These groups are among the most frequently used ones in \unsafe{}.

\item The \smugroup{Misc} group contains the method \member{getLoadAverage}, to get the load average in the operating system run queue assigned to the available processors. It is not used.

\item The \smugroup{Monitor} group contains methods to explicitly manage \java{} monitors.
The \member{tryMonitorEnter} method is never used.

\item The \smugroup{Off-Heap} group provides access to unmanaged memory, enabling explicit memory management. Similarly to the \smugroup{Heap Get} and \smugroup{Heap Put} groups, the \smugroup{Off-Heap Get} and \smugroup{Off-Heap Put} groups allow the developer to load and store values in Off-Heap memory. The usage of these methods is non-negligible, with \member{getByte} and \member{putByte} dominating the rest. The value of the \member{ADDRESS\_SIZE} field is the result of the method \member{addressSize()}.

\item Methods of the \smugroup{Offset} group are used to compute the location of fields within \java{} objects. The offsets are used in calls to many other \smu{} methods, for instance those in the \smugroup{Heap Get}, \smugroup{Heap Put}, and the \smugroup{CAS} groups. The method \member{objectFieldOffset} is the most called method in \smu{} due to its result being used by many other \smu{} methods. The \member{fieldOffset} method is deprecated, and indeed, we found no uses.
The \member{INVALID\_FIELD\_OFFSET} field indicates an invalid field offset; it is never used because code using \member{objectFieldOffset} is not written in a defensive style.
%(given that \unsafe{} is used when performance matters, and extra checks might negatively affect performance).


\item 
The \smugroup{Ordered Put} group has methods to store to a \java{} variable without emitting any memory barrier but guaranteeing no reordering across the store.

\item The \member{park} and \member{unpark} methods are contained in the \smugroup{Park} group. With them, it is possible to block and unblock a thread's execution.

\item The \member{throwException} method is contained in the \smugroup{Throw} group, and allows one to throw checked exceptions without declaring them in the \texttt{throws} clause.

\item Finally, the \smugroup{Volatile Get} and \smugroup{Volatile Put} groups allow the developer to store a value in a \java{} variable with \texttt{volatile} semantics.

\end{itemize}

It is interesting to note that despite our large corpus of code, there are several Unsafe methods that are never actually called. If \unsafe{} is to be used in third-party code, then it might make sense to extract those methods into a separate class to be only used from within the runtime library.

%\subsection{Beyond Maven Central}
%
%While Maven Central is a large repository, we wanted to check whether our results generalize to other common repositories. We thus performed a similar analysis of method usage using the \boa{}~\cite{Dyer-Nguyen-Rajan-Nguyen-13} infrastructure. \boa{} allows the developer to mine ASTs of \java{} projects in SourceForge.
%
%The usage profile of Unsafe methods we obtained from \boa{} was similar in shape, but at a different scale, compared to the one obtained from \mavencentral{}. In \boa{}'s SourceForge data, for instance, the most called method, \texttt{objectFieldOffset}, 
%is called 200 times in $50$ projects.
%This is two orders of magnitude lower than the count we found on \mavencentral{}.
%Although \boa{} enables the mining of source code in a convenient way, 
%the SourceForge data it analyzes probably is not current enough 
%to include the more recent Java code that uses \smu{} more heavily.


\section{Finding \smu{} Usage Patterns} \label{sec:methodology}

We examined the artifacts in the Maven Central software repository to identify usage patterns for \unsafe{}.
This section describes our methodology for identifying these patterns. 

\begin{figure}[h!]
\includegraphics[page=6,width=\columnwidth]{chapters/unsafe/artifacts}
\caption{\artexp{com.lmax}{disruptor} call sites}
\label{fig:appartfingerprint}
\end{figure}

\begin{figure}[h!]
\includegraphics[page=5,width=\columnwidth]{chapters/unsafe/artifacts}
\caption{\artexp{org.scala-lang}{scala-library} call sites}
\label{fig:langartfingerprint}
\end{figure}

Our first step is to visualize how an artifact uses \unsafe{}.
To this end, we count the \unsafe{} call sites and field usages per class in each artifact.
Figures~\ref{fig:appartfingerprint} and \ref{fig:langartfingerprint} show two examples of call sites usages for \artexp{com.lmax}{disruptor} and \artexp{org.scala-lang}{scala-library} respectively.
Each row shows a fully qualified class name and their usage of \smu{}.

After determining the call sites and field usage per artifact, we tried to find a way to group artifacts by how they use \smu{}.
The first issue is to determine which method calls work together to achieve a goal.
These calls might all be located within a single class, be spread across different classes within a package, or be spread across different packages within the whole artifact.
After trying different combinations, we decided to group together calls occurring within a single class and its inner classes.

We cluster classes and their inner classes by \unsafe{} method usage using a dendrogram.
Because a dendrogram can result in different clusters depending on at which height the dendrogram is cut,
we experimented with various clusterings until settling on 31 clusters.
An example of a cluster and its dendrogram is shown in Figure~\ref{fig:classunitfingerprint}.
In the figure we can see classes using methods of the \smugroup{Off-Heap}, \smugroup{Off-Heap Get}, and \smugroup{Off-Heap Put} groups to implement large arrays.

%page=15,

\begin{figure}
\includegraphics[width=\columnwidth]{chapters/unsafe/classunit}
\caption{Classes using off-heap large arrays}
\label{fig:classunitfingerprint}
\end{figure}

Once we had a clustering of the artifacts by method usage, we manually
inspected a sample of artifacts in each cluster to identify patterns.
Some artifacts contained more than one pattern.
For instance the cluster in Figure~\ref{fig:classunitfingerprint} contains
classes that use \unsafe{} to implement large off-heap arrays, but also
contains
calls to methods of the \smugroup{Put Volatile} group used to implement
strongly shared consistent variables.
We tagged each artifact manually inspected with the set of patterns that it exhibits.
%We have inspected each cluster by 
%--- of the most used artifacts ---

\input{chapters/unsafe/sec-patterns}

\section{What is the Unsafe API Used for?}
\label{sec:unsafe:discussion}

In response to \ref{unsafe:rq2} (\emph{\urqB}),
many of the patterns we found indicate that \unsafe{} is used to achieve 
better performance or to implement functionality not otherwise available in the \java{} language or standard library.

However, many of the patterns described can be implemented using APIs
already provided in the \java{} standard library. 
In addition, there are several existing proposals to improve the situation
with \unsafe{} already under development within the \java{} community.
Oracle software engineer Paul~\cite{psandoz14} performed a survey on
the OpenJDK mailing list to
study how Unsafe is
used\footnote{\url{http://www.infoq.com/news/2014/02/Unsafe-Survey}} and
describes several of these proposals.


% alloc probyte lockfree fence park finalfield monitor serialization marshalling throw size largearrays page class

\newcommand{\tick}{\cmark}
\newcommand{\exis}{$\bullet$}

\newcommand{\langalt}{\textbf{Lang}}
\newcommand{\opt}{\textbf{VM}}
\newcommand{\lib}{\textbf{Lib}}
\newcommand{\refl}{\textbf{Ref}}

% \newcommand{\hdr}{\rowcolor{header-color}}
% \newcommand{\alt}{\rowcolor{alt-row-color}}
% \newcommand{\row}{}
%\& \verb+\+textbf\{Unsafe members\}
\begin{table}[t!]
\centering
\caption[Patterns and their alternatives]{Patterns and their alternatives. A bullet (\exis) indicates that an
alternative exists in the \java{} language or API. A check mark (\tick) indicates that there is a proposed alternative for \java{}.}
\label{table:alts}
\begin{tabularx}{\linewidth}{rp{9.25cm}cccccccX}
\hdr \# & \textbf{Pattern}		& \langalt & \opt 	& \lib	& \refl	\\
\alt  1 & \namealloc    		& \tick	& 	& 	&	\\
\row  2 & \nameprobyte 			& 	& \tick	& 	&	\\
\alt  3 & \namelockfree			&	& 	& \exis	&	\\
\row  4 & \namefence 			& 	& 	& \tick &	\\
\alt  5 & \namepark			& 	& 	& \exis &	\\
\row  6 & \namefinalfield		& 	& 	& 	& \exis	\\
\alt  7 & \namemonitor			& \tick	& 	& 	&	\\
\row  8 & \nameserialization		& \tick & 	& \exis & \exis	\\
\alt  9 & \namemarshalling     		& \tick & 	& \exis	&	\\
\row 10 & \namethrow 			& \tick	& 	&   	&	\\
\alt 11 & \namesize 			& \tick	& 	& \tick &	\\
\row 12 & \namelargearrays 		& \tick & 	& \tick &	\\
\alt 13 & \namepage			& \tick	& 	& \tick	&	\\
\row 14 & \nameclass 			& \tick & 	& \tick	&	\\
\hline
\end{tabularx}
\end{table}


A summary of the patterns with existing and proposed alternatives to \unsafe{} is shown in Table~\ref{table:alts}.
The table consists of the following columns:
The \textbf{\em Pattern} column indicates the name of the pattern.
The next three columns indicate whether the pattern could be implemented either as a
language feature (\textbf{\em Lang}),
virtual machine extension (\textbf{\em VM}),
or
library extension (\textbf{\em Lib}).
The \textbf{\em Ref} column indicates that the pattern can be
implemented using reflection.
A bullet (\exis) indicates that an
alternative exists in the \java{} language or API. A check mark (\tick)
indicates that there is a proposed alternative for \java{}.

Many APIs already exist that provide functionality similar to \unsafe{}.
Indeed, these APIs are often implemented using \unsafe{} under the hood, but 
they are designed to be used safely.
They maintain invariants or perform runtime checks
to ensure that their use of \unsafe{} is safe.
Because of this overhead, using \unsafe{}
directly should in principle provide better performance at the cost of safety.
% We are unaware of any studies measuring this overhead, however.

For example,
the \javaclass{java.\-util.\-concurrent} package provides classes 
for safely performing atomic operations on fields and array elements, as well
as several synchronizer classes. These
classes
can be used instead of \unsafe{} to implement
atomic operations or strongly consistent
shared variables.
The standard library class
\javaclass{java.\-util.\-concurrent.\-locks.\-LockSupport} provides
\member{park} and \member{unpark}
methods to be used for implementing locks. 
These methods are just thin wrappers
around the \smu{} methods of the same name and 
could be used to implement the park pattern.
\java{} already supports serialization of objects using the
\javaclass{java.lang.Serializable} and
\javaclass{java.io.Object\-Output\-Stream} API.
The now-deleted JEP 187 Serialization 2.0 proposal%
\footnote{\url{http://mail.openjdk.java.net/pipermail/core-libs-dev/2014-January/024589.html}}
\footnote{\url{http://web.archive.org/web/20140702193924/http://openjdk.java.net/jeps/187}}
addresses some of the issues with \java{} serialization.

Because volatile variable accesses compile to code that issues memory fences, 
strongly consistent variables can be implemented by accessing volatile variables.
However, the fences generated for volatile variables may be stronger (and
therefore less performant) than are needed for a given application.
Indeed, the \unsafe{} \smugroup{Put Ordered}
and \smugroup{Fence} methods were likely introduced
to improve performance versus volatile variables.
% There is currently a proposal for enhanced volatile support in the JVM (JEP 193 Enhanced Volatiles~\cite{jep193}).
The accepted proposal JEP 193 (Enhanced Volatiles~\citep{jep193}) introduces \emph{variable handles}, which allow
atomic operations on fields and array elements.

Many of the patterns can be implemented using the reflection API,
albeit with lower performance than with \unsafe{}~\citep{korlandNoninvasiveConcurrencyJava2010}.
For example,
reflection can be used for accessing object fields to implement serialization.
Similarly, reflection can be used
in combination with
\javaclass{java.nio.Byte\-Buffer} and related classes for
data marshaling.
The reflection API can also be used to write to final fields.
However, this feature of the reflection API 
makes sense only during deserialization or during object construction and may have
unpredictable behavior in other cases.

% \footnote{\url{http://docs.oracle.com/javase/8/docs/api/java/lang/reflect/Field.html\#set(java.lang.Object,\%20java.lang.Object)}}

Writing a final field through reflection may not ensure
the write becomes visible to other threads that might have cached the final
field, and it may not work correctly at all if the VM performs compiler
optimizations such as constant propagation on final fields.

Many patterns use \unsafe{} to use memory more efficiently.
Using \code{struct}s or packed objects can reduce memory overhead by eliminating object headers and other per-object overhead.
\java{} has no native support for \code{struct}s,
but they can be implemented with byte buffers or with \jni{}.\footnote{\url{http://www.oracle.com/technetwork/java/jvmls2013sciam-2013525.pdf}}

The Arrays 2.0 proposal~\citep{arrays20} and
the value types proposal~\citep{valuetypes} address the large arrays pattern.
Project Sumatra~\citep{layouts} proposes features for accessing GPUs
and other accelerators,
one of the use cases for foreign data access.
Related proposals include JEP 191~\citep{jep191},
which proposes a new foreign function interface for \java{},
and Project Panama~\citep{panama}, which supports native data access from the \jvm{}.

A \member{sizeof} feature could be introduced into the language or into the standard library.
A use case for this feature includes cache management implementations.
A higher-level alternative might be to provide an \api{} for memory usage tracking in the \jvm{}.
A page size method could be added to the standard library,
perhaps in the \javaclass{java.nio} package,
which already includes \javaclass{MappedByteBuffer} to access memory-mapped storage.

Other patterns may require \java{} language changes.
For instance, 
the language could be changed to not require methods to declare the exceptions they throw,
obviating the need for \unsafe{} in this case.
Indeed, there is a long-running debate\footnote{\url{http://www.ibm.com/developerworks/library/j-jtp05254/}} about the software-engineering benefits of checked exceptions.
C\#, for instance, does not require that exceptions be declared in method signatures at all.
One alternative not requiring a language change
% , proposed in a \stackoverflow{} discussion,
is to use \java{} generics
instead.
% \footnote{\url{http://stackoverflow.com/questions/11410042}}
Because of type erasure, a checked exception can be coerced unsafely into an unchecked exception and thrown.

Changing the language to support allocation without constructors or non-lexically-scoped monitors is feasible.
However, implementation of these
features must be done carefully to ensure object invariants
are properly maintained.
In particular, supporting arbitrary unconstructed
objects can require type system changes to prevent usage of the object 
before initialization~\citep{qiMaskedTypesSound2009}.
Limiting the scope of this feature to support deserialization only may be a good compromise and
has been suggested in the JEP 187 Serialization 2.0 proposal.

Since \unsafe{} is often used simply for performance reasons,
virtual machine optimizations can reduce the need for \unsafe{}.
For example, the \jvm{}'s runtime compiler can be extended with optimizations for vectorizing byte array accesses,
eliminating the motivation to use \unsafe{} to process byte arrays.
Many patterns use \unsafe{} to use memory more efficiently.
This could be ameliorated with lower GC overhead.
There are proposals for this, for instance JEP 189 Shenandoah:
Low Pause GC~\citep{jep189}.

\section{Conclusions}
\label{sec:unsafe:conclusions}

\smu{} is an API that was designed for limited use in system-level runtime library code.
The \unsafe{} API is powerful, but dangerous.
The improper use of \unsafe{} undermines \java{}'s safety guarantees.
We studied to what degree \unsafe{} usage has spread into third-party libraries,
to what degree such third-party usage of \unsafe{} can impact existing Java code,
and which \unsafe{} API features such third-party libraries actually use.
We studied the questions and discussions developers have about \unsafe{},
and we identified common usage patterns.
We thereby provided a basis for evolving the \unsafe{} API, the \java{} language, and the \jvm{}
by eliminating unused or abused unsafe features,
and by providing safer alternatives for features that are used in meaningful ways.
We hope this will help to make \unsafe{} safer.